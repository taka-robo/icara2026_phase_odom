\section{結論}\label{sec:conclusion}
本稿では,ResNet-18 と DeiT-Ti を対象に,軽量構成での画像分類性能を公平に比較する実験プロトコルを提案した。統一したデータ前処理,学習率スケジューリング,TTA を組み合わせることで,再現性の高いベースラインを構築できることを示した。ResNet-18 はデータ拡張により大きな性能向上を示し,DeiT-Ti は TTA によって精度を伸ばせることから,利用目的に応じて両者を使い分ける指針を得た。

今後の研究では,大規模データセットや蒸留手法,ハードウェア最適化といった要素を組み合わせ,軽量モデルの性能限界と運用上の最適化を探る予定である。また,公開ベンチマークにおける再現性確保のため,コードと設定ファイルをオープンソースとして提供することを計画している。
