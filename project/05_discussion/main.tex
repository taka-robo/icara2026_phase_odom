\section{考察}\label{sec:discussion}
実験結果を踏まえ,軽量 CNN と ViT の特性を考察する。ResNet-18 は初期収束が速く,短時間で実用的な精度を達成するため,限られた学習時間やオンライン更新が求められる環境に適している。一方,DeiT-Ti は最終精度が高く,TTA の恩恵を強く受けるが,推論時間が増加する点に留意する必要がある。

RandAugment に代表されるデータ拡張は CNN において顕著な性能向上をもたらすが,ViT では効果が飽和しやすい傾向が見られた。これは,自己注意機構が入力の多様性に対して比較的ロバストであるためと考えられる。TTA は両モデルで性能改善に寄与するが,特に ViT で顕著であり,推論コストとのトレードオフを踏まえた活用が求められる。

本実験の制約として,CIFAR-10 のみを対象とした点,大規模事前学習や蒸留の効果を評価していない点,ハードウェアコストの詳細な分析が未実施である点が挙げられる。今後はより大規模なデータセット(例: ImageNet-1k)や半教師あり学習との組み合わせを検討し,軽量モデルの性能限界を明らかにしたい。
