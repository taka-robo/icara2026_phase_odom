% !TEX program = xelatex
\documentclass[conference]{IEEEtran}

% 日本語出力と XeLaTeX 対応設定
\usepackage{xeCJK}
\usepackage{fontspec}
\setCJKmainfont{HaranoAjiMincho}

% 数式・図表・参考文献などの標準パッケージ
\usepackage{amsmath, amssymb}
\usepackage{graphicx}
\usepackage{url}
\usepackage{hyperref}

\hypersetup{
  colorlinks=true,
  linkcolor=blue,
  citecolor=blue,
  urlcolor=blue
}

\begin{document}

\title{軽量 CNN と Vision Transformer の公平比較に向けた実験的評価}

\author{\IEEEauthorblockN{著者 太郎\\ }
\IEEEauthorblockA{所属機関\\ 連絡先: author@example.com}}

\maketitle

\begin{abstract}
本稿では,軽量な畳み込みニューラルネットワーク(CNN)と Vision Transformer(ViT)の画像分類性能を,CIFAR-10 を対象に公平に比較する実験プロトコルを提示する。学習率スケジューリング,データ拡張,Test-Time Augmentation (TTA) を統一的に適用し,再現性を重視した評価を実施した。実験の結果,ViT-Ti は TTA により最大で +1.8pt の精度向上を得た一方,ResNet-18 では RandAugment を中心としたデータ拡張で +2.1pt の改善が確認された。これらの知見を通じて,限られた計算資源でも堅牢なベースラインを構築するための指針を示す。
\end{abstract}

\begin{IEEEkeywords}
画像分類, 深層学習, Vision Transformer, Test-Time Augmentation, CIFAR-10
\end{IEEEkeywords}

% !TEX program = xelatex
\documentclass[conference]{IEEEtran}

% 日本語出力と XeLaTeX 対応設定
\usepackage{xeCJK}
\usepackage{fontspec}
\setCJKmainfont{HaranoAjiMincho}

% 数式・図表・参考文献などの標準パッケージ
\usepackage{amsmath, amssymb}
\usepackage{graphicx}
\usepackage{url}
\usepackage{hyperref}

\hypersetup{
  colorlinks=true,
  linkcolor=blue,
  citecolor=blue,
  urlcolor=blue
}

\begin{document}

\title{軽量 CNN と Vision Transformer の公平比較に向けた実験的評価}

\author{\IEEEauthorblockN{著者 太郎\\ }
\IEEEauthorblockA{所属機関\\ 連絡先: author@example.com}}

\maketitle

\begin{abstract}
本稿では,軽量な畳み込みニューラルネットワーク(CNN)と Vision Transformer(ViT)の画像分類性能を,CIFAR-10 を対象に公平に比較する実験プロトコルを提示する。学習率スケジューリング,データ拡張,Test-Time Augmentation (TTA) を統一的に適用し,再現性を重視した評価を実施した。実験の結果,ViT-Ti は TTA により最大で +1.8pt の精度向上を得た一方,ResNet-18 では RandAugment を中心としたデータ拡張で +2.1pt の改善が確認された。これらの知見を通じて,限られた計算資源でも堅牢なベースラインを構築するための指針を示す。
\end{abstract}

\begin{IEEEkeywords}
画像分類, 深層学習, Vision Transformer, Test-Time Augmentation, CIFAR-10
\end{IEEEkeywords}

% !TEX program = xelatex
\documentclass[conference]{IEEEtran}

% 日本語出力と XeLaTeX 対応設定
\usepackage{xeCJK}
\usepackage{fontspec}
\setCJKmainfont{HaranoAjiMincho}

% 数式・図表・参考文献などの標準パッケージ
\usepackage{amsmath, amssymb}
\usepackage{graphicx}
\usepackage{url}
\usepackage{hyperref}

\hypersetup{
  colorlinks=true,
  linkcolor=blue,
  citecolor=blue,
  urlcolor=blue
}

\begin{document}

\title{軽量 CNN と Vision Transformer の公平比較に向けた実験的評価}

\author{\IEEEauthorblockN{著者 太郎\\ }
\IEEEauthorblockA{所属機関\\ 連絡先: author@example.com}}

\maketitle

\begin{abstract}
本稿では,軽量な畳み込みニューラルネットワーク(CNN)と Vision Transformer(ViT)の画像分類性能を,CIFAR-10 を対象に公平に比較する実験プロトコルを提示する。学習率スケジューリング,データ拡張,Test-Time Augmentation (TTA) を統一的に適用し,再現性を重視した評価を実施した。実験の結果,ViT-Ti は TTA により最大で +1.8pt の精度向上を得た一方,ResNet-18 では RandAugment を中心としたデータ拡張で +2.1pt の改善が確認された。これらの知見を通じて,限られた計算資源でも堅牢なベースラインを構築するための指針を示す。
\end{abstract}

\begin{IEEEkeywords}
画像分類, 深層学習, Vision Transformer, Test-Time Augmentation, CIFAR-10
\end{IEEEkeywords}

% !TEX program = xelatex
\documentclass[conference]{IEEEtran}

% 日本語出力と XeLaTeX 対応設定
\usepackage{xeCJK}
\usepackage{fontspec}
\setCJKmainfont{HaranoAjiMincho}

% 数式・図表・参考文献などの標準パッケージ
\usepackage{amsmath, amssymb}
\usepackage{graphicx}
\usepackage{url}
\usepackage{hyperref}

\hypersetup{
  colorlinks=true,
  linkcolor=blue,
  citecolor=blue,
  urlcolor=blue
}

\begin{document}

\title{軽量 CNN と Vision Transformer の公平比較に向けた実験的評価}

\author{\IEEEauthorblockN{著者 太郎\\ }
\IEEEauthorblockA{所属機関\\ 連絡先: author@example.com}}

\maketitle

\begin{abstract}
本稿では,軽量な畳み込みニューラルネットワーク(CNN)と Vision Transformer(ViT)の画像分類性能を,CIFAR-10 を対象に公平に比較する実験プロトコルを提示する。学習率スケジューリング,データ拡張,Test-Time Augmentation (TTA) を統一的に適用し,再現性を重視した評価を実施した。実験の結果,ViT-Ti は TTA により最大で +1.8pt の精度向上を得た一方,ResNet-18 では RandAugment を中心としたデータ拡張で +2.1pt の改善が確認された。これらの知見を通じて,限られた計算資源でも堅牢なベースラインを構築するための指針を示す。
\end{abstract}

\begin{IEEEkeywords}
画像分類, 深層学習, Vision Transformer, Test-Time Augmentation, CIFAR-10
\end{IEEEkeywords}

\input{01_introduction/main}
\input{02_related_work/main}
\input{03_method/main}
\input{04_results/main}
\input{05_discussion/main}
\input{06_conclusion/main}
\input{appendix/main}

\bibliographystyle{ieeetr}
\bibliography{references}

\end{document}

% !TEX program = xelatex
\documentclass[conference]{IEEEtran}

% 日本語出力と XeLaTeX 対応設定
\usepackage{xeCJK}
\usepackage{fontspec}
\setCJKmainfont{HaranoAjiMincho}

% 数式・図表・参考文献などの標準パッケージ
\usepackage{amsmath, amssymb}
\usepackage{graphicx}
\usepackage{url}
\usepackage{hyperref}

\hypersetup{
  colorlinks=true,
  linkcolor=blue,
  citecolor=blue,
  urlcolor=blue
}

\begin{document}

\title{軽量 CNN と Vision Transformer の公平比較に向けた実験的評価}

\author{\IEEEauthorblockN{著者 太郎\\ }
\IEEEauthorblockA{所属機関\\ 連絡先: author@example.com}}

\maketitle

\begin{abstract}
本稿では,軽量な畳み込みニューラルネットワーク(CNN)と Vision Transformer(ViT)の画像分類性能を,CIFAR-10 を対象に公平に比較する実験プロトコルを提示する。学習率スケジューリング,データ拡張,Test-Time Augmentation (TTA) を統一的に適用し,再現性を重視した評価を実施した。実験の結果,ViT-Ti は TTA により最大で +1.8pt の精度向上を得た一方,ResNet-18 では RandAugment を中心としたデータ拡張で +2.1pt の改善が確認された。これらの知見を通じて,限られた計算資源でも堅牢なベースラインを構築するための指針を示す。
\end{abstract}

\begin{IEEEkeywords}
画像分類, 深層学習, Vision Transformer, Test-Time Augmentation, CIFAR-10
\end{IEEEkeywords}

\input{01_introduction/main}
\input{02_related_work/main}
\input{03_method/main}
\input{04_results/main}
\input{05_discussion/main}
\input{06_conclusion/main}
\input{appendix/main}

\bibliographystyle{ieeetr}
\bibliography{references}

\end{document}

% !TEX program = xelatex
\documentclass[conference]{IEEEtran}

% 日本語出力と XeLaTeX 対応設定
\usepackage{xeCJK}
\usepackage{fontspec}
\setCJKmainfont{HaranoAjiMincho}

% 数式・図表・参考文献などの標準パッケージ
\usepackage{amsmath, amssymb}
\usepackage{graphicx}
\usepackage{url}
\usepackage{hyperref}

\hypersetup{
  colorlinks=true,
  linkcolor=blue,
  citecolor=blue,
  urlcolor=blue
}

\begin{document}

\title{軽量 CNN と Vision Transformer の公平比較に向けた実験的評価}

\author{\IEEEauthorblockN{著者 太郎\\ }
\IEEEauthorblockA{所属機関\\ 連絡先: author@example.com}}

\maketitle

\begin{abstract}
本稿では,軽量な畳み込みニューラルネットワーク(CNN)と Vision Transformer(ViT)の画像分類性能を,CIFAR-10 を対象に公平に比較する実験プロトコルを提示する。学習率スケジューリング,データ拡張,Test-Time Augmentation (TTA) を統一的に適用し,再現性を重視した評価を実施した。実験の結果,ViT-Ti は TTA により最大で +1.8pt の精度向上を得た一方,ResNet-18 では RandAugment を中心としたデータ拡張で +2.1pt の改善が確認された。これらの知見を通じて,限られた計算資源でも堅牢なベースラインを構築するための指針を示す。
\end{abstract}

\begin{IEEEkeywords}
画像分類, 深層学習, Vision Transformer, Test-Time Augmentation, CIFAR-10
\end{IEEEkeywords}

\input{01_introduction/main}
\input{02_related_work/main}
\input{03_method/main}
\input{04_results/main}
\input{05_discussion/main}
\input{06_conclusion/main}
\input{appendix/main}

\bibliographystyle{ieeetr}
\bibliography{references}

\end{document}

% !TEX program = xelatex
\documentclass[conference]{IEEEtran}

% 日本語出力と XeLaTeX 対応設定
\usepackage{xeCJK}
\usepackage{fontspec}
\setCJKmainfont{HaranoAjiMincho}

% 数式・図表・参考文献などの標準パッケージ
\usepackage{amsmath, amssymb}
\usepackage{graphicx}
\usepackage{url}
\usepackage{hyperref}

\hypersetup{
  colorlinks=true,
  linkcolor=blue,
  citecolor=blue,
  urlcolor=blue
}

\begin{document}

\title{軽量 CNN と Vision Transformer の公平比較に向けた実験的評価}

\author{\IEEEauthorblockN{著者 太郎\\ }
\IEEEauthorblockA{所属機関\\ 連絡先: author@example.com}}

\maketitle

\begin{abstract}
本稿では,軽量な畳み込みニューラルネットワーク(CNN)と Vision Transformer(ViT)の画像分類性能を,CIFAR-10 を対象に公平に比較する実験プロトコルを提示する。学習率スケジューリング,データ拡張,Test-Time Augmentation (TTA) を統一的に適用し,再現性を重視した評価を実施した。実験の結果,ViT-Ti は TTA により最大で +1.8pt の精度向上を得た一方,ResNet-18 では RandAugment を中心としたデータ拡張で +2.1pt の改善が確認された。これらの知見を通じて,限られた計算資源でも堅牢なベースラインを構築するための指針を示す。
\end{abstract}

\begin{IEEEkeywords}
画像分類, 深層学習, Vision Transformer, Test-Time Augmentation, CIFAR-10
\end{IEEEkeywords}

\input{01_introduction/main}
\input{02_related_work/main}
\input{03_method/main}
\input{04_results/main}
\input{05_discussion/main}
\input{06_conclusion/main}
\input{appendix/main}

\bibliographystyle{ieeetr}
\bibliography{references}

\end{document}

% !TEX program = xelatex
\documentclass[conference]{IEEEtran}

% 日本語出力と XeLaTeX 対応設定
\usepackage{xeCJK}
\usepackage{fontspec}
\setCJKmainfont{HaranoAjiMincho}

% 数式・図表・参考文献などの標準パッケージ
\usepackage{amsmath, amssymb}
\usepackage{graphicx}
\usepackage{url}
\usepackage{hyperref}

\hypersetup{
  colorlinks=true,
  linkcolor=blue,
  citecolor=blue,
  urlcolor=blue
}

\begin{document}

\title{軽量 CNN と Vision Transformer の公平比較に向けた実験的評価}

\author{\IEEEauthorblockN{著者 太郎\\ }
\IEEEauthorblockA{所属機関\\ 連絡先: author@example.com}}

\maketitle

\begin{abstract}
本稿では,軽量な畳み込みニューラルネットワーク(CNN)と Vision Transformer(ViT)の画像分類性能を,CIFAR-10 を対象に公平に比較する実験プロトコルを提示する。学習率スケジューリング,データ拡張,Test-Time Augmentation (TTA) を統一的に適用し,再現性を重視した評価を実施した。実験の結果,ViT-Ti は TTA により最大で +1.8pt の精度向上を得た一方,ResNet-18 では RandAugment を中心としたデータ拡張で +2.1pt の改善が確認された。これらの知見を通じて,限られた計算資源でも堅牢なベースラインを構築するための指針を示す。
\end{abstract}

\begin{IEEEkeywords}
画像分類, 深層学習, Vision Transformer, Test-Time Augmentation, CIFAR-10
\end{IEEEkeywords}

\input{01_introduction/main}
\input{02_related_work/main}
\input{03_method/main}
\input{04_results/main}
\input{05_discussion/main}
\input{06_conclusion/main}
\input{appendix/main}

\bibliographystyle{ieeetr}
\bibliography{references}

\end{document}

% !TEX program = xelatex
\documentclass[conference]{IEEEtran}

% 日本語出力と XeLaTeX 対応設定
\usepackage{xeCJK}
\usepackage{fontspec}
\setCJKmainfont{HaranoAjiMincho}

% 数式・図表・参考文献などの標準パッケージ
\usepackage{amsmath, amssymb}
\usepackage{graphicx}
\usepackage{url}
\usepackage{hyperref}

\hypersetup{
  colorlinks=true,
  linkcolor=blue,
  citecolor=blue,
  urlcolor=blue
}

\begin{document}

\title{軽量 CNN と Vision Transformer の公平比較に向けた実験的評価}

\author{\IEEEauthorblockN{著者 太郎\\ }
\IEEEauthorblockA{所属機関\\ 連絡先: author@example.com}}

\maketitle

\begin{abstract}
本稿では,軽量な畳み込みニューラルネットワーク(CNN)と Vision Transformer(ViT)の画像分類性能を,CIFAR-10 を対象に公平に比較する実験プロトコルを提示する。学習率スケジューリング,データ拡張,Test-Time Augmentation (TTA) を統一的に適用し,再現性を重視した評価を実施した。実験の結果,ViT-Ti は TTA により最大で +1.8pt の精度向上を得た一方,ResNet-18 では RandAugment を中心としたデータ拡張で +2.1pt の改善が確認された。これらの知見を通じて,限られた計算資源でも堅牢なベースラインを構築するための指針を示す。
\end{abstract}

\begin{IEEEkeywords}
画像分類, 深層学習, Vision Transformer, Test-Time Augmentation, CIFAR-10
\end{IEEEkeywords}

\input{01_introduction/main}
\input{02_related_work/main}
\input{03_method/main}
\input{04_results/main}
\input{05_discussion/main}
\input{06_conclusion/main}
\input{appendix/main}

\bibliographystyle{ieeetr}
\bibliography{references}

\end{document}

% !TEX program = xelatex
\documentclass[conference]{IEEEtran}

% 日本語出力と XeLaTeX 対応設定
\usepackage{xeCJK}
\usepackage{fontspec}
\setCJKmainfont{HaranoAjiMincho}

% 数式・図表・参考文献などの標準パッケージ
\usepackage{amsmath, amssymb}
\usepackage{graphicx}
\usepackage{url}
\usepackage{hyperref}

\hypersetup{
  colorlinks=true,
  linkcolor=blue,
  citecolor=blue,
  urlcolor=blue
}

\begin{document}

\title{軽量 CNN と Vision Transformer の公平比較に向けた実験的評価}

\author{\IEEEauthorblockN{著者 太郎\\ }
\IEEEauthorblockA{所属機関\\ 連絡先: author@example.com}}

\maketitle

\begin{abstract}
本稿では,軽量な畳み込みニューラルネットワーク(CNN)と Vision Transformer(ViT)の画像分類性能を,CIFAR-10 を対象に公平に比較する実験プロトコルを提示する。学習率スケジューリング,データ拡張,Test-Time Augmentation (TTA) を統一的に適用し,再現性を重視した評価を実施した。実験の結果,ViT-Ti は TTA により最大で +1.8pt の精度向上を得た一方,ResNet-18 では RandAugment を中心としたデータ拡張で +2.1pt の改善が確認された。これらの知見を通じて,限られた計算資源でも堅牢なベースラインを構築するための指針を示す。
\end{abstract}

\begin{IEEEkeywords}
画像分類, 深層学習, Vision Transformer, Test-Time Augmentation, CIFAR-10
\end{IEEEkeywords}

\input{01_introduction/main}
\input{02_related_work/main}
\input{03_method/main}
\input{04_results/main}
\input{05_discussion/main}
\input{06_conclusion/main}
\input{appendix/main}

\bibliographystyle{ieeetr}
\bibliography{references}

\end{document}


\bibliographystyle{ieeetr}
\bibliography{references}

\end{document}

% !TEX program = xelatex
\documentclass[conference]{IEEEtran}

% 日本語出力と XeLaTeX 対応設定
\usepackage{xeCJK}
\usepackage{fontspec}
\setCJKmainfont{HaranoAjiMincho}

% 数式・図表・参考文献などの標準パッケージ
\usepackage{amsmath, amssymb}
\usepackage{graphicx}
\usepackage{url}
\usepackage{hyperref}

\hypersetup{
  colorlinks=true,
  linkcolor=blue,
  citecolor=blue,
  urlcolor=blue
}

\begin{document}

\title{軽量 CNN と Vision Transformer の公平比較に向けた実験的評価}

\author{\IEEEauthorblockN{著者 太郎\\ }
\IEEEauthorblockA{所属機関\\ 連絡先: author@example.com}}

\maketitle

\begin{abstract}
本稿では,軽量な畳み込みニューラルネットワーク(CNN)と Vision Transformer(ViT)の画像分類性能を,CIFAR-10 を対象に公平に比較する実験プロトコルを提示する。学習率スケジューリング,データ拡張,Test-Time Augmentation (TTA) を統一的に適用し,再現性を重視した評価を実施した。実験の結果,ViT-Ti は TTA により最大で +1.8pt の精度向上を得た一方,ResNet-18 では RandAugment を中心としたデータ拡張で +2.1pt の改善が確認された。これらの知見を通じて,限られた計算資源でも堅牢なベースラインを構築するための指針を示す。
\end{abstract}

\begin{IEEEkeywords}
画像分類, 深層学習, Vision Transformer, Test-Time Augmentation, CIFAR-10
\end{IEEEkeywords}

% !TEX program = xelatex
\documentclass[conference]{IEEEtran}

% 日本語出力と XeLaTeX 対応設定
\usepackage{xeCJK}
\usepackage{fontspec}
\setCJKmainfont{HaranoAjiMincho}

% 数式・図表・参考文献などの標準パッケージ
\usepackage{amsmath, amssymb}
\usepackage{graphicx}
\usepackage{url}
\usepackage{hyperref}

\hypersetup{
  colorlinks=true,
  linkcolor=blue,
  citecolor=blue,
  urlcolor=blue
}

\begin{document}

\title{軽量 CNN と Vision Transformer の公平比較に向けた実験的評価}

\author{\IEEEauthorblockN{著者 太郎\\ }
\IEEEauthorblockA{所属機関\\ 連絡先: author@example.com}}

\maketitle

\begin{abstract}
本稿では,軽量な畳み込みニューラルネットワーク(CNN)と Vision Transformer(ViT)の画像分類性能を,CIFAR-10 を対象に公平に比較する実験プロトコルを提示する。学習率スケジューリング,データ拡張,Test-Time Augmentation (TTA) を統一的に適用し,再現性を重視した評価を実施した。実験の結果,ViT-Ti は TTA により最大で +1.8pt の精度向上を得た一方,ResNet-18 では RandAugment を中心としたデータ拡張で +2.1pt の改善が確認された。これらの知見を通じて,限られた計算資源でも堅牢なベースラインを構築するための指針を示す。
\end{abstract}

\begin{IEEEkeywords}
画像分類, 深層学習, Vision Transformer, Test-Time Augmentation, CIFAR-10
\end{IEEEkeywords}

\input{01_introduction/main}
\input{02_related_work/main}
\input{03_method/main}
\input{04_results/main}
\input{05_discussion/main}
\input{06_conclusion/main}
\input{appendix/main}

\bibliographystyle{ieeetr}
\bibliography{references}

\end{document}

% !TEX program = xelatex
\documentclass[conference]{IEEEtran}

% 日本語出力と XeLaTeX 対応設定
\usepackage{xeCJK}
\usepackage{fontspec}
\setCJKmainfont{HaranoAjiMincho}

% 数式・図表・参考文献などの標準パッケージ
\usepackage{amsmath, amssymb}
\usepackage{graphicx}
\usepackage{url}
\usepackage{hyperref}

\hypersetup{
  colorlinks=true,
  linkcolor=blue,
  citecolor=blue,
  urlcolor=blue
}

\begin{document}

\title{軽量 CNN と Vision Transformer の公平比較に向けた実験的評価}

\author{\IEEEauthorblockN{著者 太郎\\ }
\IEEEauthorblockA{所属機関\\ 連絡先: author@example.com}}

\maketitle

\begin{abstract}
本稿では,軽量な畳み込みニューラルネットワーク(CNN)と Vision Transformer(ViT)の画像分類性能を,CIFAR-10 を対象に公平に比較する実験プロトコルを提示する。学習率スケジューリング,データ拡張,Test-Time Augmentation (TTA) を統一的に適用し,再現性を重視した評価を実施した。実験の結果,ViT-Ti は TTA により最大で +1.8pt の精度向上を得た一方,ResNet-18 では RandAugment を中心としたデータ拡張で +2.1pt の改善が確認された。これらの知見を通じて,限られた計算資源でも堅牢なベースラインを構築するための指針を示す。
\end{abstract}

\begin{IEEEkeywords}
画像分類, 深層学習, Vision Transformer, Test-Time Augmentation, CIFAR-10
\end{IEEEkeywords}

\input{01_introduction/main}
\input{02_related_work/main}
\input{03_method/main}
\input{04_results/main}
\input{05_discussion/main}
\input{06_conclusion/main}
\input{appendix/main}

\bibliographystyle{ieeetr}
\bibliography{references}

\end{document}

% !TEX program = xelatex
\documentclass[conference]{IEEEtran}

% 日本語出力と XeLaTeX 対応設定
\usepackage{xeCJK}
\usepackage{fontspec}
\setCJKmainfont{HaranoAjiMincho}

% 数式・図表・参考文献などの標準パッケージ
\usepackage{amsmath, amssymb}
\usepackage{graphicx}
\usepackage{url}
\usepackage{hyperref}

\hypersetup{
  colorlinks=true,
  linkcolor=blue,
  citecolor=blue,
  urlcolor=blue
}

\begin{document}

\title{軽量 CNN と Vision Transformer の公平比較に向けた実験的評価}

\author{\IEEEauthorblockN{著者 太郎\\ }
\IEEEauthorblockA{所属機関\\ 連絡先: author@example.com}}

\maketitle

\begin{abstract}
本稿では,軽量な畳み込みニューラルネットワーク(CNN)と Vision Transformer(ViT)の画像分類性能を,CIFAR-10 を対象に公平に比較する実験プロトコルを提示する。学習率スケジューリング,データ拡張,Test-Time Augmentation (TTA) を統一的に適用し,再現性を重視した評価を実施した。実験の結果,ViT-Ti は TTA により最大で +1.8pt の精度向上を得た一方,ResNet-18 では RandAugment を中心としたデータ拡張で +2.1pt の改善が確認された。これらの知見を通じて,限られた計算資源でも堅牢なベースラインを構築するための指針を示す。
\end{abstract}

\begin{IEEEkeywords}
画像分類, 深層学習, Vision Transformer, Test-Time Augmentation, CIFAR-10
\end{IEEEkeywords}

\input{01_introduction/main}
\input{02_related_work/main}
\input{03_method/main}
\input{04_results/main}
\input{05_discussion/main}
\input{06_conclusion/main}
\input{appendix/main}

\bibliographystyle{ieeetr}
\bibliography{references}

\end{document}

% !TEX program = xelatex
\documentclass[conference]{IEEEtran}

% 日本語出力と XeLaTeX 対応設定
\usepackage{xeCJK}
\usepackage{fontspec}
\setCJKmainfont{HaranoAjiMincho}

% 数式・図表・参考文献などの標準パッケージ
\usepackage{amsmath, amssymb}
\usepackage{graphicx}
\usepackage{url}
\usepackage{hyperref}

\hypersetup{
  colorlinks=true,
  linkcolor=blue,
  citecolor=blue,
  urlcolor=blue
}

\begin{document}

\title{軽量 CNN と Vision Transformer の公平比較に向けた実験的評価}

\author{\IEEEauthorblockN{著者 太郎\\ }
\IEEEauthorblockA{所属機関\\ 連絡先: author@example.com}}

\maketitle

\begin{abstract}
本稿では,軽量な畳み込みニューラルネットワーク(CNN)と Vision Transformer(ViT)の画像分類性能を,CIFAR-10 を対象に公平に比較する実験プロトコルを提示する。学習率スケジューリング,データ拡張,Test-Time Augmentation (TTA) を統一的に適用し,再現性を重視した評価を実施した。実験の結果,ViT-Ti は TTA により最大で +1.8pt の精度向上を得た一方,ResNet-18 では RandAugment を中心としたデータ拡張で +2.1pt の改善が確認された。これらの知見を通じて,限られた計算資源でも堅牢なベースラインを構築するための指針を示す。
\end{abstract}

\begin{IEEEkeywords}
画像分類, 深層学習, Vision Transformer, Test-Time Augmentation, CIFAR-10
\end{IEEEkeywords}

\input{01_introduction/main}
\input{02_related_work/main}
\input{03_method/main}
\input{04_results/main}
\input{05_discussion/main}
\input{06_conclusion/main}
\input{appendix/main}

\bibliographystyle{ieeetr}
\bibliography{references}

\end{document}

% !TEX program = xelatex
\documentclass[conference]{IEEEtran}

% 日本語出力と XeLaTeX 対応設定
\usepackage{xeCJK}
\usepackage{fontspec}
\setCJKmainfont{HaranoAjiMincho}

% 数式・図表・参考文献などの標準パッケージ
\usepackage{amsmath, amssymb}
\usepackage{graphicx}
\usepackage{url}
\usepackage{hyperref}

\hypersetup{
  colorlinks=true,
  linkcolor=blue,
  citecolor=blue,
  urlcolor=blue
}

\begin{document}

\title{軽量 CNN と Vision Transformer の公平比較に向けた実験的評価}

\author{\IEEEauthorblockN{著者 太郎\\ }
\IEEEauthorblockA{所属機関\\ 連絡先: author@example.com}}

\maketitle

\begin{abstract}
本稿では,軽量な畳み込みニューラルネットワーク(CNN)と Vision Transformer(ViT)の画像分類性能を,CIFAR-10 を対象に公平に比較する実験プロトコルを提示する。学習率スケジューリング,データ拡張,Test-Time Augmentation (TTA) を統一的に適用し,再現性を重視した評価を実施した。実験の結果,ViT-Ti は TTA により最大で +1.8pt の精度向上を得た一方,ResNet-18 では RandAugment を中心としたデータ拡張で +2.1pt の改善が確認された。これらの知見を通じて,限られた計算資源でも堅牢なベースラインを構築するための指針を示す。
\end{abstract}

\begin{IEEEkeywords}
画像分類, 深層学習, Vision Transformer, Test-Time Augmentation, CIFAR-10
\end{IEEEkeywords}

\input{01_introduction/main}
\input{02_related_work/main}
\input{03_method/main}
\input{04_results/main}
\input{05_discussion/main}
\input{06_conclusion/main}
\input{appendix/main}

\bibliographystyle{ieeetr}
\bibliography{references}

\end{document}

% !TEX program = xelatex
\documentclass[conference]{IEEEtran}

% 日本語出力と XeLaTeX 対応設定
\usepackage{xeCJK}
\usepackage{fontspec}
\setCJKmainfont{HaranoAjiMincho}

% 数式・図表・参考文献などの標準パッケージ
\usepackage{amsmath, amssymb}
\usepackage{graphicx}
\usepackage{url}
\usepackage{hyperref}

\hypersetup{
  colorlinks=true,
  linkcolor=blue,
  citecolor=blue,
  urlcolor=blue
}

\begin{document}

\title{軽量 CNN と Vision Transformer の公平比較に向けた実験的評価}

\author{\IEEEauthorblockN{著者 太郎\\ }
\IEEEauthorblockA{所属機関\\ 連絡先: author@example.com}}

\maketitle

\begin{abstract}
本稿では,軽量な畳み込みニューラルネットワーク(CNN)と Vision Transformer(ViT)の画像分類性能を,CIFAR-10 を対象に公平に比較する実験プロトコルを提示する。学習率スケジューリング,データ拡張,Test-Time Augmentation (TTA) を統一的に適用し,再現性を重視した評価を実施した。実験の結果,ViT-Ti は TTA により最大で +1.8pt の精度向上を得た一方,ResNet-18 では RandAugment を中心としたデータ拡張で +2.1pt の改善が確認された。これらの知見を通じて,限られた計算資源でも堅牢なベースラインを構築するための指針を示す。
\end{abstract}

\begin{IEEEkeywords}
画像分類, 深層学習, Vision Transformer, Test-Time Augmentation, CIFAR-10
\end{IEEEkeywords}

\input{01_introduction/main}
\input{02_related_work/main}
\input{03_method/main}
\input{04_results/main}
\input{05_discussion/main}
\input{06_conclusion/main}
\input{appendix/main}

\bibliographystyle{ieeetr}
\bibliography{references}

\end{document}

% !TEX program = xelatex
\documentclass[conference]{IEEEtran}

% 日本語出力と XeLaTeX 対応設定
\usepackage{xeCJK}
\usepackage{fontspec}
\setCJKmainfont{HaranoAjiMincho}

% 数式・図表・参考文献などの標準パッケージ
\usepackage{amsmath, amssymb}
\usepackage{graphicx}
\usepackage{url}
\usepackage{hyperref}

\hypersetup{
  colorlinks=true,
  linkcolor=blue,
  citecolor=blue,
  urlcolor=blue
}

\begin{document}

\title{軽量 CNN と Vision Transformer の公平比較に向けた実験的評価}

\author{\IEEEauthorblockN{著者 太郎\\ }
\IEEEauthorblockA{所属機関\\ 連絡先: author@example.com}}

\maketitle

\begin{abstract}
本稿では,軽量な畳み込みニューラルネットワーク(CNN)と Vision Transformer(ViT)の画像分類性能を,CIFAR-10 を対象に公平に比較する実験プロトコルを提示する。学習率スケジューリング,データ拡張,Test-Time Augmentation (TTA) を統一的に適用し,再現性を重視した評価を実施した。実験の結果,ViT-Ti は TTA により最大で +1.8pt の精度向上を得た一方,ResNet-18 では RandAugment を中心としたデータ拡張で +2.1pt の改善が確認された。これらの知見を通じて,限られた計算資源でも堅牢なベースラインを構築するための指針を示す。
\end{abstract}

\begin{IEEEkeywords}
画像分類, 深層学習, Vision Transformer, Test-Time Augmentation, CIFAR-10
\end{IEEEkeywords}

\input{01_introduction/main}
\input{02_related_work/main}
\input{03_method/main}
\input{04_results/main}
\input{05_discussion/main}
\input{06_conclusion/main}
\input{appendix/main}

\bibliographystyle{ieeetr}
\bibliography{references}

\end{document}


\bibliographystyle{ieeetr}
\bibliography{references}

\end{document}

% !TEX program = xelatex
\documentclass[conference]{IEEEtran}

% 日本語出力と XeLaTeX 対応設定
\usepackage{xeCJK}
\usepackage{fontspec}
\setCJKmainfont{HaranoAjiMincho}

% 数式・図表・参考文献などの標準パッケージ
\usepackage{amsmath, amssymb}
\usepackage{graphicx}
\usepackage{url}
\usepackage{hyperref}

\hypersetup{
  colorlinks=true,
  linkcolor=blue,
  citecolor=blue,
  urlcolor=blue
}

\begin{document}

\title{軽量 CNN と Vision Transformer の公平比較に向けた実験的評価}

\author{\IEEEauthorblockN{著者 太郎\\ }
\IEEEauthorblockA{所属機関\\ 連絡先: author@example.com}}

\maketitle

\begin{abstract}
本稿では,軽量な畳み込みニューラルネットワーク(CNN)と Vision Transformer(ViT)の画像分類性能を,CIFAR-10 を対象に公平に比較する実験プロトコルを提示する。学習率スケジューリング,データ拡張,Test-Time Augmentation (TTA) を統一的に適用し,再現性を重視した評価を実施した。実験の結果,ViT-Ti は TTA により最大で +1.8pt の精度向上を得た一方,ResNet-18 では RandAugment を中心としたデータ拡張で +2.1pt の改善が確認された。これらの知見を通じて,限られた計算資源でも堅牢なベースラインを構築するための指針を示す。
\end{abstract}

\begin{IEEEkeywords}
画像分類, 深層学習, Vision Transformer, Test-Time Augmentation, CIFAR-10
\end{IEEEkeywords}

% !TEX program = xelatex
\documentclass[conference]{IEEEtran}

% 日本語出力と XeLaTeX 対応設定
\usepackage{xeCJK}
\usepackage{fontspec}
\setCJKmainfont{HaranoAjiMincho}

% 数式・図表・参考文献などの標準パッケージ
\usepackage{amsmath, amssymb}
\usepackage{graphicx}
\usepackage{url}
\usepackage{hyperref}

\hypersetup{
  colorlinks=true,
  linkcolor=blue,
  citecolor=blue,
  urlcolor=blue
}

\begin{document}

\title{軽量 CNN と Vision Transformer の公平比較に向けた実験的評価}

\author{\IEEEauthorblockN{著者 太郎\\ }
\IEEEauthorblockA{所属機関\\ 連絡先: author@example.com}}

\maketitle

\begin{abstract}
本稿では,軽量な畳み込みニューラルネットワーク(CNN)と Vision Transformer(ViT)の画像分類性能を,CIFAR-10 を対象に公平に比較する実験プロトコルを提示する。学習率スケジューリング,データ拡張,Test-Time Augmentation (TTA) を統一的に適用し,再現性を重視した評価を実施した。実験の結果,ViT-Ti は TTA により最大で +1.8pt の精度向上を得た一方,ResNet-18 では RandAugment を中心としたデータ拡張で +2.1pt の改善が確認された。これらの知見を通じて,限られた計算資源でも堅牢なベースラインを構築するための指針を示す。
\end{abstract}

\begin{IEEEkeywords}
画像分類, 深層学習, Vision Transformer, Test-Time Augmentation, CIFAR-10
\end{IEEEkeywords}

\input{01_introduction/main}
\input{02_related_work/main}
\input{03_method/main}
\input{04_results/main}
\input{05_discussion/main}
\input{06_conclusion/main}
\input{appendix/main}

\bibliographystyle{ieeetr}
\bibliography{references}

\end{document}

% !TEX program = xelatex
\documentclass[conference]{IEEEtran}

% 日本語出力と XeLaTeX 対応設定
\usepackage{xeCJK}
\usepackage{fontspec}
\setCJKmainfont{HaranoAjiMincho}

% 数式・図表・参考文献などの標準パッケージ
\usepackage{amsmath, amssymb}
\usepackage{graphicx}
\usepackage{url}
\usepackage{hyperref}

\hypersetup{
  colorlinks=true,
  linkcolor=blue,
  citecolor=blue,
  urlcolor=blue
}

\begin{document}

\title{軽量 CNN と Vision Transformer の公平比較に向けた実験的評価}

\author{\IEEEauthorblockN{著者 太郎\\ }
\IEEEauthorblockA{所属機関\\ 連絡先: author@example.com}}

\maketitle

\begin{abstract}
本稿では,軽量な畳み込みニューラルネットワーク(CNN)と Vision Transformer(ViT)の画像分類性能を,CIFAR-10 を対象に公平に比較する実験プロトコルを提示する。学習率スケジューリング,データ拡張,Test-Time Augmentation (TTA) を統一的に適用し,再現性を重視した評価を実施した。実験の結果,ViT-Ti は TTA により最大で +1.8pt の精度向上を得た一方,ResNet-18 では RandAugment を中心としたデータ拡張で +2.1pt の改善が確認された。これらの知見を通じて,限られた計算資源でも堅牢なベースラインを構築するための指針を示す。
\end{abstract}

\begin{IEEEkeywords}
画像分類, 深層学習, Vision Transformer, Test-Time Augmentation, CIFAR-10
\end{IEEEkeywords}

\input{01_introduction/main}
\input{02_related_work/main}
\input{03_method/main}
\input{04_results/main}
\input{05_discussion/main}
\input{06_conclusion/main}
\input{appendix/main}

\bibliographystyle{ieeetr}
\bibliography{references}

\end{document}

% !TEX program = xelatex
\documentclass[conference]{IEEEtran}

% 日本語出力と XeLaTeX 対応設定
\usepackage{xeCJK}
\usepackage{fontspec}
\setCJKmainfont{HaranoAjiMincho}

% 数式・図表・参考文献などの標準パッケージ
\usepackage{amsmath, amssymb}
\usepackage{graphicx}
\usepackage{url}
\usepackage{hyperref}

\hypersetup{
  colorlinks=true,
  linkcolor=blue,
  citecolor=blue,
  urlcolor=blue
}

\begin{document}

\title{軽量 CNN と Vision Transformer の公平比較に向けた実験的評価}

\author{\IEEEauthorblockN{著者 太郎\\ }
\IEEEauthorblockA{所属機関\\ 連絡先: author@example.com}}

\maketitle

\begin{abstract}
本稿では,軽量な畳み込みニューラルネットワーク(CNN)と Vision Transformer(ViT)の画像分類性能を,CIFAR-10 を対象に公平に比較する実験プロトコルを提示する。学習率スケジューリング,データ拡張,Test-Time Augmentation (TTA) を統一的に適用し,再現性を重視した評価を実施した。実験の結果,ViT-Ti は TTA により最大で +1.8pt の精度向上を得た一方,ResNet-18 では RandAugment を中心としたデータ拡張で +2.1pt の改善が確認された。これらの知見を通じて,限られた計算資源でも堅牢なベースラインを構築するための指針を示す。
\end{abstract}

\begin{IEEEkeywords}
画像分類, 深層学習, Vision Transformer, Test-Time Augmentation, CIFAR-10
\end{IEEEkeywords}

\input{01_introduction/main}
\input{02_related_work/main}
\input{03_method/main}
\input{04_results/main}
\input{05_discussion/main}
\input{06_conclusion/main}
\input{appendix/main}

\bibliographystyle{ieeetr}
\bibliography{references}

\end{document}

% !TEX program = xelatex
\documentclass[conference]{IEEEtran}

% 日本語出力と XeLaTeX 対応設定
\usepackage{xeCJK}
\usepackage{fontspec}
\setCJKmainfont{HaranoAjiMincho}

% 数式・図表・参考文献などの標準パッケージ
\usepackage{amsmath, amssymb}
\usepackage{graphicx}
\usepackage{url}
\usepackage{hyperref}

\hypersetup{
  colorlinks=true,
  linkcolor=blue,
  citecolor=blue,
  urlcolor=blue
}

\begin{document}

\title{軽量 CNN と Vision Transformer の公平比較に向けた実験的評価}

\author{\IEEEauthorblockN{著者 太郎\\ }
\IEEEauthorblockA{所属機関\\ 連絡先: author@example.com}}

\maketitle

\begin{abstract}
本稿では,軽量な畳み込みニューラルネットワーク(CNN)と Vision Transformer(ViT)の画像分類性能を,CIFAR-10 を対象に公平に比較する実験プロトコルを提示する。学習率スケジューリング,データ拡張,Test-Time Augmentation (TTA) を統一的に適用し,再現性を重視した評価を実施した。実験の結果,ViT-Ti は TTA により最大で +1.8pt の精度向上を得た一方,ResNet-18 では RandAugment を中心としたデータ拡張で +2.1pt の改善が確認された。これらの知見を通じて,限られた計算資源でも堅牢なベースラインを構築するための指針を示す。
\end{abstract}

\begin{IEEEkeywords}
画像分類, 深層学習, Vision Transformer, Test-Time Augmentation, CIFAR-10
\end{IEEEkeywords}

\input{01_introduction/main}
\input{02_related_work/main}
\input{03_method/main}
\input{04_results/main}
\input{05_discussion/main}
\input{06_conclusion/main}
\input{appendix/main}

\bibliographystyle{ieeetr}
\bibliography{references}

\end{document}

% !TEX program = xelatex
\documentclass[conference]{IEEEtran}

% 日本語出力と XeLaTeX 対応設定
\usepackage{xeCJK}
\usepackage{fontspec}
\setCJKmainfont{HaranoAjiMincho}

% 数式・図表・参考文献などの標準パッケージ
\usepackage{amsmath, amssymb}
\usepackage{graphicx}
\usepackage{url}
\usepackage{hyperref}

\hypersetup{
  colorlinks=true,
  linkcolor=blue,
  citecolor=blue,
  urlcolor=blue
}

\begin{document}

\title{軽量 CNN と Vision Transformer の公平比較に向けた実験的評価}

\author{\IEEEauthorblockN{著者 太郎\\ }
\IEEEauthorblockA{所属機関\\ 連絡先: author@example.com}}

\maketitle

\begin{abstract}
本稿では,軽量な畳み込みニューラルネットワーク(CNN)と Vision Transformer(ViT)の画像分類性能を,CIFAR-10 を対象に公平に比較する実験プロトコルを提示する。学習率スケジューリング,データ拡張,Test-Time Augmentation (TTA) を統一的に適用し,再現性を重視した評価を実施した。実験の結果,ViT-Ti は TTA により最大で +1.8pt の精度向上を得た一方,ResNet-18 では RandAugment を中心としたデータ拡張で +2.1pt の改善が確認された。これらの知見を通じて,限られた計算資源でも堅牢なベースラインを構築するための指針を示す。
\end{abstract}

\begin{IEEEkeywords}
画像分類, 深層学習, Vision Transformer, Test-Time Augmentation, CIFAR-10
\end{IEEEkeywords}

\input{01_introduction/main}
\input{02_related_work/main}
\input{03_method/main}
\input{04_results/main}
\input{05_discussion/main}
\input{06_conclusion/main}
\input{appendix/main}

\bibliographystyle{ieeetr}
\bibliography{references}

\end{document}

% !TEX program = xelatex
\documentclass[conference]{IEEEtran}

% 日本語出力と XeLaTeX 対応設定
\usepackage{xeCJK}
\usepackage{fontspec}
\setCJKmainfont{HaranoAjiMincho}

% 数式・図表・参考文献などの標準パッケージ
\usepackage{amsmath, amssymb}
\usepackage{graphicx}
\usepackage{url}
\usepackage{hyperref}

\hypersetup{
  colorlinks=true,
  linkcolor=blue,
  citecolor=blue,
  urlcolor=blue
}

\begin{document}

\title{軽量 CNN と Vision Transformer の公平比較に向けた実験的評価}

\author{\IEEEauthorblockN{著者 太郎\\ }
\IEEEauthorblockA{所属機関\\ 連絡先: author@example.com}}

\maketitle

\begin{abstract}
本稿では,軽量な畳み込みニューラルネットワーク(CNN)と Vision Transformer(ViT)の画像分類性能を,CIFAR-10 を対象に公平に比較する実験プロトコルを提示する。学習率スケジューリング,データ拡張,Test-Time Augmentation (TTA) を統一的に適用し,再現性を重視した評価を実施した。実験の結果,ViT-Ti は TTA により最大で +1.8pt の精度向上を得た一方,ResNet-18 では RandAugment を中心としたデータ拡張で +2.1pt の改善が確認された。これらの知見を通じて,限られた計算資源でも堅牢なベースラインを構築するための指針を示す。
\end{abstract}

\begin{IEEEkeywords}
画像分類, 深層学習, Vision Transformer, Test-Time Augmentation, CIFAR-10
\end{IEEEkeywords}

\input{01_introduction/main}
\input{02_related_work/main}
\input{03_method/main}
\input{04_results/main}
\input{05_discussion/main}
\input{06_conclusion/main}
\input{appendix/main}

\bibliographystyle{ieeetr}
\bibliography{references}

\end{document}

% !TEX program = xelatex
\documentclass[conference]{IEEEtran}

% 日本語出力と XeLaTeX 対応設定
\usepackage{xeCJK}
\usepackage{fontspec}
\setCJKmainfont{HaranoAjiMincho}

% 数式・図表・参考文献などの標準パッケージ
\usepackage{amsmath, amssymb}
\usepackage{graphicx}
\usepackage{url}
\usepackage{hyperref}

\hypersetup{
  colorlinks=true,
  linkcolor=blue,
  citecolor=blue,
  urlcolor=blue
}

\begin{document}

\title{軽量 CNN と Vision Transformer の公平比較に向けた実験的評価}

\author{\IEEEauthorblockN{著者 太郎\\ }
\IEEEauthorblockA{所属機関\\ 連絡先: author@example.com}}

\maketitle

\begin{abstract}
本稿では,軽量な畳み込みニューラルネットワーク(CNN)と Vision Transformer(ViT)の画像分類性能を,CIFAR-10 を対象に公平に比較する実験プロトコルを提示する。学習率スケジューリング,データ拡張,Test-Time Augmentation (TTA) を統一的に適用し,再現性を重視した評価を実施した。実験の結果,ViT-Ti は TTA により最大で +1.8pt の精度向上を得た一方,ResNet-18 では RandAugment を中心としたデータ拡張で +2.1pt の改善が確認された。これらの知見を通じて,限られた計算資源でも堅牢なベースラインを構築するための指針を示す。
\end{abstract}

\begin{IEEEkeywords}
画像分類, 深層学習, Vision Transformer, Test-Time Augmentation, CIFAR-10
\end{IEEEkeywords}

\input{01_introduction/main}
\input{02_related_work/main}
\input{03_method/main}
\input{04_results/main}
\input{05_discussion/main}
\input{06_conclusion/main}
\input{appendix/main}

\bibliographystyle{ieeetr}
\bibliography{references}

\end{document}


\bibliographystyle{ieeetr}
\bibliography{references}

\end{document}

% !TEX program = xelatex
\documentclass[conference]{IEEEtran}

% 日本語出力と XeLaTeX 対応設定
\usepackage{xeCJK}
\usepackage{fontspec}
\setCJKmainfont{HaranoAjiMincho}

% 数式・図表・参考文献などの標準パッケージ
\usepackage{amsmath, amssymb}
\usepackage{graphicx}
\usepackage{url}
\usepackage{hyperref}

\hypersetup{
  colorlinks=true,
  linkcolor=blue,
  citecolor=blue,
  urlcolor=blue
}

\begin{document}

\title{軽量 CNN と Vision Transformer の公平比較に向けた実験的評価}

\author{\IEEEauthorblockN{著者 太郎\\ }
\IEEEauthorblockA{所属機関\\ 連絡先: author@example.com}}

\maketitle

\begin{abstract}
本稿では,軽量な畳み込みニューラルネットワーク(CNN)と Vision Transformer(ViT)の画像分類性能を,CIFAR-10 を対象に公平に比較する実験プロトコルを提示する。学習率スケジューリング,データ拡張,Test-Time Augmentation (TTA) を統一的に適用し,再現性を重視した評価を実施した。実験の結果,ViT-Ti は TTA により最大で +1.8pt の精度向上を得た一方,ResNet-18 では RandAugment を中心としたデータ拡張で +2.1pt の改善が確認された。これらの知見を通じて,限られた計算資源でも堅牢なベースラインを構築するための指針を示す。
\end{abstract}

\begin{IEEEkeywords}
画像分類, 深層学習, Vision Transformer, Test-Time Augmentation, CIFAR-10
\end{IEEEkeywords}

% !TEX program = xelatex
\documentclass[conference]{IEEEtran}

% 日本語出力と XeLaTeX 対応設定
\usepackage{xeCJK}
\usepackage{fontspec}
\setCJKmainfont{HaranoAjiMincho}

% 数式・図表・参考文献などの標準パッケージ
\usepackage{amsmath, amssymb}
\usepackage{graphicx}
\usepackage{url}
\usepackage{hyperref}

\hypersetup{
  colorlinks=true,
  linkcolor=blue,
  citecolor=blue,
  urlcolor=blue
}

\begin{document}

\title{軽量 CNN と Vision Transformer の公平比較に向けた実験的評価}

\author{\IEEEauthorblockN{著者 太郎\\ }
\IEEEauthorblockA{所属機関\\ 連絡先: author@example.com}}

\maketitle

\begin{abstract}
本稿では,軽量な畳み込みニューラルネットワーク(CNN)と Vision Transformer(ViT)の画像分類性能を,CIFAR-10 を対象に公平に比較する実験プロトコルを提示する。学習率スケジューリング,データ拡張,Test-Time Augmentation (TTA) を統一的に適用し,再現性を重視した評価を実施した。実験の結果,ViT-Ti は TTA により最大で +1.8pt の精度向上を得た一方,ResNet-18 では RandAugment を中心としたデータ拡張で +2.1pt の改善が確認された。これらの知見を通じて,限られた計算資源でも堅牢なベースラインを構築するための指針を示す。
\end{abstract}

\begin{IEEEkeywords}
画像分類, 深層学習, Vision Transformer, Test-Time Augmentation, CIFAR-10
\end{IEEEkeywords}

\input{01_introduction/main}
\input{02_related_work/main}
\input{03_method/main}
\input{04_results/main}
\input{05_discussion/main}
\input{06_conclusion/main}
\input{appendix/main}

\bibliographystyle{ieeetr}
\bibliography{references}

\end{document}

% !TEX program = xelatex
\documentclass[conference]{IEEEtran}

% 日本語出力と XeLaTeX 対応設定
\usepackage{xeCJK}
\usepackage{fontspec}
\setCJKmainfont{HaranoAjiMincho}

% 数式・図表・参考文献などの標準パッケージ
\usepackage{amsmath, amssymb}
\usepackage{graphicx}
\usepackage{url}
\usepackage{hyperref}

\hypersetup{
  colorlinks=true,
  linkcolor=blue,
  citecolor=blue,
  urlcolor=blue
}

\begin{document}

\title{軽量 CNN と Vision Transformer の公平比較に向けた実験的評価}

\author{\IEEEauthorblockN{著者 太郎\\ }
\IEEEauthorblockA{所属機関\\ 連絡先: author@example.com}}

\maketitle

\begin{abstract}
本稿では,軽量な畳み込みニューラルネットワーク(CNN)と Vision Transformer(ViT)の画像分類性能を,CIFAR-10 を対象に公平に比較する実験プロトコルを提示する。学習率スケジューリング,データ拡張,Test-Time Augmentation (TTA) を統一的に適用し,再現性を重視した評価を実施した。実験の結果,ViT-Ti は TTA により最大で +1.8pt の精度向上を得た一方,ResNet-18 では RandAugment を中心としたデータ拡張で +2.1pt の改善が確認された。これらの知見を通じて,限られた計算資源でも堅牢なベースラインを構築するための指針を示す。
\end{abstract}

\begin{IEEEkeywords}
画像分類, 深層学習, Vision Transformer, Test-Time Augmentation, CIFAR-10
\end{IEEEkeywords}

\input{01_introduction/main}
\input{02_related_work/main}
\input{03_method/main}
\input{04_results/main}
\input{05_discussion/main}
\input{06_conclusion/main}
\input{appendix/main}

\bibliographystyle{ieeetr}
\bibliography{references}

\end{document}

% !TEX program = xelatex
\documentclass[conference]{IEEEtran}

% 日本語出力と XeLaTeX 対応設定
\usepackage{xeCJK}
\usepackage{fontspec}
\setCJKmainfont{HaranoAjiMincho}

% 数式・図表・参考文献などの標準パッケージ
\usepackage{amsmath, amssymb}
\usepackage{graphicx}
\usepackage{url}
\usepackage{hyperref}

\hypersetup{
  colorlinks=true,
  linkcolor=blue,
  citecolor=blue,
  urlcolor=blue
}

\begin{document}

\title{軽量 CNN と Vision Transformer の公平比較に向けた実験的評価}

\author{\IEEEauthorblockN{著者 太郎\\ }
\IEEEauthorblockA{所属機関\\ 連絡先: author@example.com}}

\maketitle

\begin{abstract}
本稿では,軽量な畳み込みニューラルネットワーク(CNN)と Vision Transformer(ViT)の画像分類性能を,CIFAR-10 を対象に公平に比較する実験プロトコルを提示する。学習率スケジューリング,データ拡張,Test-Time Augmentation (TTA) を統一的に適用し,再現性を重視した評価を実施した。実験の結果,ViT-Ti は TTA により最大で +1.8pt の精度向上を得た一方,ResNet-18 では RandAugment を中心としたデータ拡張で +2.1pt の改善が確認された。これらの知見を通じて,限られた計算資源でも堅牢なベースラインを構築するための指針を示す。
\end{abstract}

\begin{IEEEkeywords}
画像分類, 深層学習, Vision Transformer, Test-Time Augmentation, CIFAR-10
\end{IEEEkeywords}

\input{01_introduction/main}
\input{02_related_work/main}
\input{03_method/main}
\input{04_results/main}
\input{05_discussion/main}
\input{06_conclusion/main}
\input{appendix/main}

\bibliographystyle{ieeetr}
\bibliography{references}

\end{document}

% !TEX program = xelatex
\documentclass[conference]{IEEEtran}

% 日本語出力と XeLaTeX 対応設定
\usepackage{xeCJK}
\usepackage{fontspec}
\setCJKmainfont{HaranoAjiMincho}

% 数式・図表・参考文献などの標準パッケージ
\usepackage{amsmath, amssymb}
\usepackage{graphicx}
\usepackage{url}
\usepackage{hyperref}

\hypersetup{
  colorlinks=true,
  linkcolor=blue,
  citecolor=blue,
  urlcolor=blue
}

\begin{document}

\title{軽量 CNN と Vision Transformer の公平比較に向けた実験的評価}

\author{\IEEEauthorblockN{著者 太郎\\ }
\IEEEauthorblockA{所属機関\\ 連絡先: author@example.com}}

\maketitle

\begin{abstract}
本稿では,軽量な畳み込みニューラルネットワーク(CNN)と Vision Transformer(ViT)の画像分類性能を,CIFAR-10 を対象に公平に比較する実験プロトコルを提示する。学習率スケジューリング,データ拡張,Test-Time Augmentation (TTA) を統一的に適用し,再現性を重視した評価を実施した。実験の結果,ViT-Ti は TTA により最大で +1.8pt の精度向上を得た一方,ResNet-18 では RandAugment を中心としたデータ拡張で +2.1pt の改善が確認された。これらの知見を通じて,限られた計算資源でも堅牢なベースラインを構築するための指針を示す。
\end{abstract}

\begin{IEEEkeywords}
画像分類, 深層学習, Vision Transformer, Test-Time Augmentation, CIFAR-10
\end{IEEEkeywords}

\input{01_introduction/main}
\input{02_related_work/main}
\input{03_method/main}
\input{04_results/main}
\input{05_discussion/main}
\input{06_conclusion/main}
\input{appendix/main}

\bibliographystyle{ieeetr}
\bibliography{references}

\end{document}

% !TEX program = xelatex
\documentclass[conference]{IEEEtran}

% 日本語出力と XeLaTeX 対応設定
\usepackage{xeCJK}
\usepackage{fontspec}
\setCJKmainfont{HaranoAjiMincho}

% 数式・図表・参考文献などの標準パッケージ
\usepackage{amsmath, amssymb}
\usepackage{graphicx}
\usepackage{url}
\usepackage{hyperref}

\hypersetup{
  colorlinks=true,
  linkcolor=blue,
  citecolor=blue,
  urlcolor=blue
}

\begin{document}

\title{軽量 CNN と Vision Transformer の公平比較に向けた実験的評価}

\author{\IEEEauthorblockN{著者 太郎\\ }
\IEEEauthorblockA{所属機関\\ 連絡先: author@example.com}}

\maketitle

\begin{abstract}
本稿では,軽量な畳み込みニューラルネットワーク(CNN)と Vision Transformer(ViT)の画像分類性能を,CIFAR-10 を対象に公平に比較する実験プロトコルを提示する。学習率スケジューリング,データ拡張,Test-Time Augmentation (TTA) を統一的に適用し,再現性を重視した評価を実施した。実験の結果,ViT-Ti は TTA により最大で +1.8pt の精度向上を得た一方,ResNet-18 では RandAugment を中心としたデータ拡張で +2.1pt の改善が確認された。これらの知見を通じて,限られた計算資源でも堅牢なベースラインを構築するための指針を示す。
\end{abstract}

\begin{IEEEkeywords}
画像分類, 深層学習, Vision Transformer, Test-Time Augmentation, CIFAR-10
\end{IEEEkeywords}

\input{01_introduction/main}
\input{02_related_work/main}
\input{03_method/main}
\input{04_results/main}
\input{05_discussion/main}
\input{06_conclusion/main}
\input{appendix/main}

\bibliographystyle{ieeetr}
\bibliography{references}

\end{document}

% !TEX program = xelatex
\documentclass[conference]{IEEEtran}

% 日本語出力と XeLaTeX 対応設定
\usepackage{xeCJK}
\usepackage{fontspec}
\setCJKmainfont{HaranoAjiMincho}

% 数式・図表・参考文献などの標準パッケージ
\usepackage{amsmath, amssymb}
\usepackage{graphicx}
\usepackage{url}
\usepackage{hyperref}

\hypersetup{
  colorlinks=true,
  linkcolor=blue,
  citecolor=blue,
  urlcolor=blue
}

\begin{document}

\title{軽量 CNN と Vision Transformer の公平比較に向けた実験的評価}

\author{\IEEEauthorblockN{著者 太郎\\ }
\IEEEauthorblockA{所属機関\\ 連絡先: author@example.com}}

\maketitle

\begin{abstract}
本稿では,軽量な畳み込みニューラルネットワーク(CNN)と Vision Transformer(ViT)の画像分類性能を,CIFAR-10 を対象に公平に比較する実験プロトコルを提示する。学習率スケジューリング,データ拡張,Test-Time Augmentation (TTA) を統一的に適用し,再現性を重視した評価を実施した。実験の結果,ViT-Ti は TTA により最大で +1.8pt の精度向上を得た一方,ResNet-18 では RandAugment を中心としたデータ拡張で +2.1pt の改善が確認された。これらの知見を通じて,限られた計算資源でも堅牢なベースラインを構築するための指針を示す。
\end{abstract}

\begin{IEEEkeywords}
画像分類, 深層学習, Vision Transformer, Test-Time Augmentation, CIFAR-10
\end{IEEEkeywords}

\input{01_introduction/main}
\input{02_related_work/main}
\input{03_method/main}
\input{04_results/main}
\input{05_discussion/main}
\input{06_conclusion/main}
\input{appendix/main}

\bibliographystyle{ieeetr}
\bibliography{references}

\end{document}

% !TEX program = xelatex
\documentclass[conference]{IEEEtran}

% 日本語出力と XeLaTeX 対応設定
\usepackage{xeCJK}
\usepackage{fontspec}
\setCJKmainfont{HaranoAjiMincho}

% 数式・図表・参考文献などの標準パッケージ
\usepackage{amsmath, amssymb}
\usepackage{graphicx}
\usepackage{url}
\usepackage{hyperref}

\hypersetup{
  colorlinks=true,
  linkcolor=blue,
  citecolor=blue,
  urlcolor=blue
}

\begin{document}

\title{軽量 CNN と Vision Transformer の公平比較に向けた実験的評価}

\author{\IEEEauthorblockN{著者 太郎\\ }
\IEEEauthorblockA{所属機関\\ 連絡先: author@example.com}}

\maketitle

\begin{abstract}
本稿では,軽量な畳み込みニューラルネットワーク(CNN)と Vision Transformer(ViT)の画像分類性能を,CIFAR-10 を対象に公平に比較する実験プロトコルを提示する。学習率スケジューリング,データ拡張,Test-Time Augmentation (TTA) を統一的に適用し,再現性を重視した評価を実施した。実験の結果,ViT-Ti は TTA により最大で +1.8pt の精度向上を得た一方,ResNet-18 では RandAugment を中心としたデータ拡張で +2.1pt の改善が確認された。これらの知見を通じて,限られた計算資源でも堅牢なベースラインを構築するための指針を示す。
\end{abstract}

\begin{IEEEkeywords}
画像分類, 深層学習, Vision Transformer, Test-Time Augmentation, CIFAR-10
\end{IEEEkeywords}

\input{01_introduction/main}
\input{02_related_work/main}
\input{03_method/main}
\input{04_results/main}
\input{05_discussion/main}
\input{06_conclusion/main}
\input{appendix/main}

\bibliographystyle{ieeetr}
\bibliography{references}

\end{document}


\bibliographystyle{ieeetr}
\bibliography{references}

\end{document}

% !TEX program = xelatex
\documentclass[conference]{IEEEtran}

% 日本語出力と XeLaTeX 対応設定
\usepackage{xeCJK}
\usepackage{fontspec}
\setCJKmainfont{HaranoAjiMincho}

% 数式・図表・参考文献などの標準パッケージ
\usepackage{amsmath, amssymb}
\usepackage{graphicx}
\usepackage{url}
\usepackage{hyperref}

\hypersetup{
  colorlinks=true,
  linkcolor=blue,
  citecolor=blue,
  urlcolor=blue
}

\begin{document}

\title{軽量 CNN と Vision Transformer の公平比較に向けた実験的評価}

\author{\IEEEauthorblockN{著者 太郎\\ }
\IEEEauthorblockA{所属機関\\ 連絡先: author@example.com}}

\maketitle

\begin{abstract}
本稿では,軽量な畳み込みニューラルネットワーク(CNN)と Vision Transformer(ViT)の画像分類性能を,CIFAR-10 を対象に公平に比較する実験プロトコルを提示する。学習率スケジューリング,データ拡張,Test-Time Augmentation (TTA) を統一的に適用し,再現性を重視した評価を実施した。実験の結果,ViT-Ti は TTA により最大で +1.8pt の精度向上を得た一方,ResNet-18 では RandAugment を中心としたデータ拡張で +2.1pt の改善が確認された。これらの知見を通じて,限られた計算資源でも堅牢なベースラインを構築するための指針を示す。
\end{abstract}

\begin{IEEEkeywords}
画像分類, 深層学習, Vision Transformer, Test-Time Augmentation, CIFAR-10
\end{IEEEkeywords}

% !TEX program = xelatex
\documentclass[conference]{IEEEtran}

% 日本語出力と XeLaTeX 対応設定
\usepackage{xeCJK}
\usepackage{fontspec}
\setCJKmainfont{HaranoAjiMincho}

% 数式・図表・参考文献などの標準パッケージ
\usepackage{amsmath, amssymb}
\usepackage{graphicx}
\usepackage{url}
\usepackage{hyperref}

\hypersetup{
  colorlinks=true,
  linkcolor=blue,
  citecolor=blue,
  urlcolor=blue
}

\begin{document}

\title{軽量 CNN と Vision Transformer の公平比較に向けた実験的評価}

\author{\IEEEauthorblockN{著者 太郎\\ }
\IEEEauthorblockA{所属機関\\ 連絡先: author@example.com}}

\maketitle

\begin{abstract}
本稿では,軽量な畳み込みニューラルネットワーク(CNN)と Vision Transformer(ViT)の画像分類性能を,CIFAR-10 を対象に公平に比較する実験プロトコルを提示する。学習率スケジューリング,データ拡張,Test-Time Augmentation (TTA) を統一的に適用し,再現性を重視した評価を実施した。実験の結果,ViT-Ti は TTA により最大で +1.8pt の精度向上を得た一方,ResNet-18 では RandAugment を中心としたデータ拡張で +2.1pt の改善が確認された。これらの知見を通じて,限られた計算資源でも堅牢なベースラインを構築するための指針を示す。
\end{abstract}

\begin{IEEEkeywords}
画像分類, 深層学習, Vision Transformer, Test-Time Augmentation, CIFAR-10
\end{IEEEkeywords}

\input{01_introduction/main}
\input{02_related_work/main}
\input{03_method/main}
\input{04_results/main}
\input{05_discussion/main}
\input{06_conclusion/main}
\input{appendix/main}

\bibliographystyle{ieeetr}
\bibliography{references}

\end{document}

% !TEX program = xelatex
\documentclass[conference]{IEEEtran}

% 日本語出力と XeLaTeX 対応設定
\usepackage{xeCJK}
\usepackage{fontspec}
\setCJKmainfont{HaranoAjiMincho}

% 数式・図表・参考文献などの標準パッケージ
\usepackage{amsmath, amssymb}
\usepackage{graphicx}
\usepackage{url}
\usepackage{hyperref}

\hypersetup{
  colorlinks=true,
  linkcolor=blue,
  citecolor=blue,
  urlcolor=blue
}

\begin{document}

\title{軽量 CNN と Vision Transformer の公平比較に向けた実験的評価}

\author{\IEEEauthorblockN{著者 太郎\\ }
\IEEEauthorblockA{所属機関\\ 連絡先: author@example.com}}

\maketitle

\begin{abstract}
本稿では,軽量な畳み込みニューラルネットワーク(CNN)と Vision Transformer(ViT)の画像分類性能を,CIFAR-10 を対象に公平に比較する実験プロトコルを提示する。学習率スケジューリング,データ拡張,Test-Time Augmentation (TTA) を統一的に適用し,再現性を重視した評価を実施した。実験の結果,ViT-Ti は TTA により最大で +1.8pt の精度向上を得た一方,ResNet-18 では RandAugment を中心としたデータ拡張で +2.1pt の改善が確認された。これらの知見を通じて,限られた計算資源でも堅牢なベースラインを構築するための指針を示す。
\end{abstract}

\begin{IEEEkeywords}
画像分類, 深層学習, Vision Transformer, Test-Time Augmentation, CIFAR-10
\end{IEEEkeywords}

\input{01_introduction/main}
\input{02_related_work/main}
\input{03_method/main}
\input{04_results/main}
\input{05_discussion/main}
\input{06_conclusion/main}
\input{appendix/main}

\bibliographystyle{ieeetr}
\bibliography{references}

\end{document}

% !TEX program = xelatex
\documentclass[conference]{IEEEtran}

% 日本語出力と XeLaTeX 対応設定
\usepackage{xeCJK}
\usepackage{fontspec}
\setCJKmainfont{HaranoAjiMincho}

% 数式・図表・参考文献などの標準パッケージ
\usepackage{amsmath, amssymb}
\usepackage{graphicx}
\usepackage{url}
\usepackage{hyperref}

\hypersetup{
  colorlinks=true,
  linkcolor=blue,
  citecolor=blue,
  urlcolor=blue
}

\begin{document}

\title{軽量 CNN と Vision Transformer の公平比較に向けた実験的評価}

\author{\IEEEauthorblockN{著者 太郎\\ }
\IEEEauthorblockA{所属機関\\ 連絡先: author@example.com}}

\maketitle

\begin{abstract}
本稿では,軽量な畳み込みニューラルネットワーク(CNN)と Vision Transformer(ViT)の画像分類性能を,CIFAR-10 を対象に公平に比較する実験プロトコルを提示する。学習率スケジューリング,データ拡張,Test-Time Augmentation (TTA) を統一的に適用し,再現性を重視した評価を実施した。実験の結果,ViT-Ti は TTA により最大で +1.8pt の精度向上を得た一方,ResNet-18 では RandAugment を中心としたデータ拡張で +2.1pt の改善が確認された。これらの知見を通じて,限られた計算資源でも堅牢なベースラインを構築するための指針を示す。
\end{abstract}

\begin{IEEEkeywords}
画像分類, 深層学習, Vision Transformer, Test-Time Augmentation, CIFAR-10
\end{IEEEkeywords}

\input{01_introduction/main}
\input{02_related_work/main}
\input{03_method/main}
\input{04_results/main}
\input{05_discussion/main}
\input{06_conclusion/main}
\input{appendix/main}

\bibliographystyle{ieeetr}
\bibliography{references}

\end{document}

% !TEX program = xelatex
\documentclass[conference]{IEEEtran}

% 日本語出力と XeLaTeX 対応設定
\usepackage{xeCJK}
\usepackage{fontspec}
\setCJKmainfont{HaranoAjiMincho}

% 数式・図表・参考文献などの標準パッケージ
\usepackage{amsmath, amssymb}
\usepackage{graphicx}
\usepackage{url}
\usepackage{hyperref}

\hypersetup{
  colorlinks=true,
  linkcolor=blue,
  citecolor=blue,
  urlcolor=blue
}

\begin{document}

\title{軽量 CNN と Vision Transformer の公平比較に向けた実験的評価}

\author{\IEEEauthorblockN{著者 太郎\\ }
\IEEEauthorblockA{所属機関\\ 連絡先: author@example.com}}

\maketitle

\begin{abstract}
本稿では,軽量な畳み込みニューラルネットワーク(CNN)と Vision Transformer(ViT)の画像分類性能を,CIFAR-10 を対象に公平に比較する実験プロトコルを提示する。学習率スケジューリング,データ拡張,Test-Time Augmentation (TTA) を統一的に適用し,再現性を重視した評価を実施した。実験の結果,ViT-Ti は TTA により最大で +1.8pt の精度向上を得た一方,ResNet-18 では RandAugment を中心としたデータ拡張で +2.1pt の改善が確認された。これらの知見を通じて,限られた計算資源でも堅牢なベースラインを構築するための指針を示す。
\end{abstract}

\begin{IEEEkeywords}
画像分類, 深層学習, Vision Transformer, Test-Time Augmentation, CIFAR-10
\end{IEEEkeywords}

\input{01_introduction/main}
\input{02_related_work/main}
\input{03_method/main}
\input{04_results/main}
\input{05_discussion/main}
\input{06_conclusion/main}
\input{appendix/main}

\bibliographystyle{ieeetr}
\bibliography{references}

\end{document}

% !TEX program = xelatex
\documentclass[conference]{IEEEtran}

% 日本語出力と XeLaTeX 対応設定
\usepackage{xeCJK}
\usepackage{fontspec}
\setCJKmainfont{HaranoAjiMincho}

% 数式・図表・参考文献などの標準パッケージ
\usepackage{amsmath, amssymb}
\usepackage{graphicx}
\usepackage{url}
\usepackage{hyperref}

\hypersetup{
  colorlinks=true,
  linkcolor=blue,
  citecolor=blue,
  urlcolor=blue
}

\begin{document}

\title{軽量 CNN と Vision Transformer の公平比較に向けた実験的評価}

\author{\IEEEauthorblockN{著者 太郎\\ }
\IEEEauthorblockA{所属機関\\ 連絡先: author@example.com}}

\maketitle

\begin{abstract}
本稿では,軽量な畳み込みニューラルネットワーク(CNN)と Vision Transformer(ViT)の画像分類性能を,CIFAR-10 を対象に公平に比較する実験プロトコルを提示する。学習率スケジューリング,データ拡張,Test-Time Augmentation (TTA) を統一的に適用し,再現性を重視した評価を実施した。実験の結果,ViT-Ti は TTA により最大で +1.8pt の精度向上を得た一方,ResNet-18 では RandAugment を中心としたデータ拡張で +2.1pt の改善が確認された。これらの知見を通じて,限られた計算資源でも堅牢なベースラインを構築するための指針を示す。
\end{abstract}

\begin{IEEEkeywords}
画像分類, 深層学習, Vision Transformer, Test-Time Augmentation, CIFAR-10
\end{IEEEkeywords}

\input{01_introduction/main}
\input{02_related_work/main}
\input{03_method/main}
\input{04_results/main}
\input{05_discussion/main}
\input{06_conclusion/main}
\input{appendix/main}

\bibliographystyle{ieeetr}
\bibliography{references}

\end{document}

% !TEX program = xelatex
\documentclass[conference]{IEEEtran}

% 日本語出力と XeLaTeX 対応設定
\usepackage{xeCJK}
\usepackage{fontspec}
\setCJKmainfont{HaranoAjiMincho}

% 数式・図表・参考文献などの標準パッケージ
\usepackage{amsmath, amssymb}
\usepackage{graphicx}
\usepackage{url}
\usepackage{hyperref}

\hypersetup{
  colorlinks=true,
  linkcolor=blue,
  citecolor=blue,
  urlcolor=blue
}

\begin{document}

\title{軽量 CNN と Vision Transformer の公平比較に向けた実験的評価}

\author{\IEEEauthorblockN{著者 太郎\\ }
\IEEEauthorblockA{所属機関\\ 連絡先: author@example.com}}

\maketitle

\begin{abstract}
本稿では,軽量な畳み込みニューラルネットワーク(CNN)と Vision Transformer(ViT)の画像分類性能を,CIFAR-10 を対象に公平に比較する実験プロトコルを提示する。学習率スケジューリング,データ拡張,Test-Time Augmentation (TTA) を統一的に適用し,再現性を重視した評価を実施した。実験の結果,ViT-Ti は TTA により最大で +1.8pt の精度向上を得た一方,ResNet-18 では RandAugment を中心としたデータ拡張で +2.1pt の改善が確認された。これらの知見を通じて,限られた計算資源でも堅牢なベースラインを構築するための指針を示す。
\end{abstract}

\begin{IEEEkeywords}
画像分類, 深層学習, Vision Transformer, Test-Time Augmentation, CIFAR-10
\end{IEEEkeywords}

\input{01_introduction/main}
\input{02_related_work/main}
\input{03_method/main}
\input{04_results/main}
\input{05_discussion/main}
\input{06_conclusion/main}
\input{appendix/main}

\bibliographystyle{ieeetr}
\bibliography{references}

\end{document}

% !TEX program = xelatex
\documentclass[conference]{IEEEtran}

% 日本語出力と XeLaTeX 対応設定
\usepackage{xeCJK}
\usepackage{fontspec}
\setCJKmainfont{HaranoAjiMincho}

% 数式・図表・参考文献などの標準パッケージ
\usepackage{amsmath, amssymb}
\usepackage{graphicx}
\usepackage{url}
\usepackage{hyperref}

\hypersetup{
  colorlinks=true,
  linkcolor=blue,
  citecolor=blue,
  urlcolor=blue
}

\begin{document}

\title{軽量 CNN と Vision Transformer の公平比較に向けた実験的評価}

\author{\IEEEauthorblockN{著者 太郎\\ }
\IEEEauthorblockA{所属機関\\ 連絡先: author@example.com}}

\maketitle

\begin{abstract}
本稿では,軽量な畳み込みニューラルネットワーク(CNN)と Vision Transformer(ViT)の画像分類性能を,CIFAR-10 を対象に公平に比較する実験プロトコルを提示する。学習率スケジューリング,データ拡張,Test-Time Augmentation (TTA) を統一的に適用し,再現性を重視した評価を実施した。実験の結果,ViT-Ti は TTA により最大で +1.8pt の精度向上を得た一方,ResNet-18 では RandAugment を中心としたデータ拡張で +2.1pt の改善が確認された。これらの知見を通じて,限られた計算資源でも堅牢なベースラインを構築するための指針を示す。
\end{abstract}

\begin{IEEEkeywords}
画像分類, 深層学習, Vision Transformer, Test-Time Augmentation, CIFAR-10
\end{IEEEkeywords}

\input{01_introduction/main}
\input{02_related_work/main}
\input{03_method/main}
\input{04_results/main}
\input{05_discussion/main}
\input{06_conclusion/main}
\input{appendix/main}

\bibliographystyle{ieeetr}
\bibliography{references}

\end{document}


\bibliographystyle{ieeetr}
\bibliography{references}

\end{document}

% !TEX program = xelatex
\documentclass[conference]{IEEEtran}

% 日本語出力と XeLaTeX 対応設定
\usepackage{xeCJK}
\usepackage{fontspec}
\setCJKmainfont{HaranoAjiMincho}

% 数式・図表・参考文献などの標準パッケージ
\usepackage{amsmath, amssymb}
\usepackage{graphicx}
\usepackage{url}
\usepackage{hyperref}

\hypersetup{
  colorlinks=true,
  linkcolor=blue,
  citecolor=blue,
  urlcolor=blue
}

\begin{document}

\title{軽量 CNN と Vision Transformer の公平比較に向けた実験的評価}

\author{\IEEEauthorblockN{著者 太郎\\ }
\IEEEauthorblockA{所属機関\\ 連絡先: author@example.com}}

\maketitle

\begin{abstract}
本稿では,軽量な畳み込みニューラルネットワーク(CNN)と Vision Transformer(ViT)の画像分類性能を,CIFAR-10 を対象に公平に比較する実験プロトコルを提示する。学習率スケジューリング,データ拡張,Test-Time Augmentation (TTA) を統一的に適用し,再現性を重視した評価を実施した。実験の結果,ViT-Ti は TTA により最大で +1.8pt の精度向上を得た一方,ResNet-18 では RandAugment を中心としたデータ拡張で +2.1pt の改善が確認された。これらの知見を通じて,限られた計算資源でも堅牢なベースラインを構築するための指針を示す。
\end{abstract}

\begin{IEEEkeywords}
画像分類, 深層学習, Vision Transformer, Test-Time Augmentation, CIFAR-10
\end{IEEEkeywords}

% !TEX program = xelatex
\documentclass[conference]{IEEEtran}

% 日本語出力と XeLaTeX 対応設定
\usepackage{xeCJK}
\usepackage{fontspec}
\setCJKmainfont{HaranoAjiMincho}

% 数式・図表・参考文献などの標準パッケージ
\usepackage{amsmath, amssymb}
\usepackage{graphicx}
\usepackage{url}
\usepackage{hyperref}

\hypersetup{
  colorlinks=true,
  linkcolor=blue,
  citecolor=blue,
  urlcolor=blue
}

\begin{document}

\title{軽量 CNN と Vision Transformer の公平比較に向けた実験的評価}

\author{\IEEEauthorblockN{著者 太郎\\ }
\IEEEauthorblockA{所属機関\\ 連絡先: author@example.com}}

\maketitle

\begin{abstract}
本稿では,軽量な畳み込みニューラルネットワーク(CNN)と Vision Transformer(ViT)の画像分類性能を,CIFAR-10 を対象に公平に比較する実験プロトコルを提示する。学習率スケジューリング,データ拡張,Test-Time Augmentation (TTA) を統一的に適用し,再現性を重視した評価を実施した。実験の結果,ViT-Ti は TTA により最大で +1.8pt の精度向上を得た一方,ResNet-18 では RandAugment を中心としたデータ拡張で +2.1pt の改善が確認された。これらの知見を通じて,限られた計算資源でも堅牢なベースラインを構築するための指針を示す。
\end{abstract}

\begin{IEEEkeywords}
画像分類, 深層学習, Vision Transformer, Test-Time Augmentation, CIFAR-10
\end{IEEEkeywords}

\input{01_introduction/main}
\input{02_related_work/main}
\input{03_method/main}
\input{04_results/main}
\input{05_discussion/main}
\input{06_conclusion/main}
\input{appendix/main}

\bibliographystyle{ieeetr}
\bibliography{references}

\end{document}

% !TEX program = xelatex
\documentclass[conference]{IEEEtran}

% 日本語出力と XeLaTeX 対応設定
\usepackage{xeCJK}
\usepackage{fontspec}
\setCJKmainfont{HaranoAjiMincho}

% 数式・図表・参考文献などの標準パッケージ
\usepackage{amsmath, amssymb}
\usepackage{graphicx}
\usepackage{url}
\usepackage{hyperref}

\hypersetup{
  colorlinks=true,
  linkcolor=blue,
  citecolor=blue,
  urlcolor=blue
}

\begin{document}

\title{軽量 CNN と Vision Transformer の公平比較に向けた実験的評価}

\author{\IEEEauthorblockN{著者 太郎\\ }
\IEEEauthorblockA{所属機関\\ 連絡先: author@example.com}}

\maketitle

\begin{abstract}
本稿では,軽量な畳み込みニューラルネットワーク(CNN)と Vision Transformer(ViT)の画像分類性能を,CIFAR-10 を対象に公平に比較する実験プロトコルを提示する。学習率スケジューリング,データ拡張,Test-Time Augmentation (TTA) を統一的に適用し,再現性を重視した評価を実施した。実験の結果,ViT-Ti は TTA により最大で +1.8pt の精度向上を得た一方,ResNet-18 では RandAugment を中心としたデータ拡張で +2.1pt の改善が確認された。これらの知見を通じて,限られた計算資源でも堅牢なベースラインを構築するための指針を示す。
\end{abstract}

\begin{IEEEkeywords}
画像分類, 深層学習, Vision Transformer, Test-Time Augmentation, CIFAR-10
\end{IEEEkeywords}

\input{01_introduction/main}
\input{02_related_work/main}
\input{03_method/main}
\input{04_results/main}
\input{05_discussion/main}
\input{06_conclusion/main}
\input{appendix/main}

\bibliographystyle{ieeetr}
\bibliography{references}

\end{document}

% !TEX program = xelatex
\documentclass[conference]{IEEEtran}

% 日本語出力と XeLaTeX 対応設定
\usepackage{xeCJK}
\usepackage{fontspec}
\setCJKmainfont{HaranoAjiMincho}

% 数式・図表・参考文献などの標準パッケージ
\usepackage{amsmath, amssymb}
\usepackage{graphicx}
\usepackage{url}
\usepackage{hyperref}

\hypersetup{
  colorlinks=true,
  linkcolor=blue,
  citecolor=blue,
  urlcolor=blue
}

\begin{document}

\title{軽量 CNN と Vision Transformer の公平比較に向けた実験的評価}

\author{\IEEEauthorblockN{著者 太郎\\ }
\IEEEauthorblockA{所属機関\\ 連絡先: author@example.com}}

\maketitle

\begin{abstract}
本稿では,軽量な畳み込みニューラルネットワーク(CNN)と Vision Transformer(ViT)の画像分類性能を,CIFAR-10 を対象に公平に比較する実験プロトコルを提示する。学習率スケジューリング,データ拡張,Test-Time Augmentation (TTA) を統一的に適用し,再現性を重視した評価を実施した。実験の結果,ViT-Ti は TTA により最大で +1.8pt の精度向上を得た一方,ResNet-18 では RandAugment を中心としたデータ拡張で +2.1pt の改善が確認された。これらの知見を通じて,限られた計算資源でも堅牢なベースラインを構築するための指針を示す。
\end{abstract}

\begin{IEEEkeywords}
画像分類, 深層学習, Vision Transformer, Test-Time Augmentation, CIFAR-10
\end{IEEEkeywords}

\input{01_introduction/main}
\input{02_related_work/main}
\input{03_method/main}
\input{04_results/main}
\input{05_discussion/main}
\input{06_conclusion/main}
\input{appendix/main}

\bibliographystyle{ieeetr}
\bibliography{references}

\end{document}

% !TEX program = xelatex
\documentclass[conference]{IEEEtran}

% 日本語出力と XeLaTeX 対応設定
\usepackage{xeCJK}
\usepackage{fontspec}
\setCJKmainfont{HaranoAjiMincho}

% 数式・図表・参考文献などの標準パッケージ
\usepackage{amsmath, amssymb}
\usepackage{graphicx}
\usepackage{url}
\usepackage{hyperref}

\hypersetup{
  colorlinks=true,
  linkcolor=blue,
  citecolor=blue,
  urlcolor=blue
}

\begin{document}

\title{軽量 CNN と Vision Transformer の公平比較に向けた実験的評価}

\author{\IEEEauthorblockN{著者 太郎\\ }
\IEEEauthorblockA{所属機関\\ 連絡先: author@example.com}}

\maketitle

\begin{abstract}
本稿では,軽量な畳み込みニューラルネットワーク(CNN)と Vision Transformer(ViT)の画像分類性能を,CIFAR-10 を対象に公平に比較する実験プロトコルを提示する。学習率スケジューリング,データ拡張,Test-Time Augmentation (TTA) を統一的に適用し,再現性を重視した評価を実施した。実験の結果,ViT-Ti は TTA により最大で +1.8pt の精度向上を得た一方,ResNet-18 では RandAugment を中心としたデータ拡張で +2.1pt の改善が確認された。これらの知見を通じて,限られた計算資源でも堅牢なベースラインを構築するための指針を示す。
\end{abstract}

\begin{IEEEkeywords}
画像分類, 深層学習, Vision Transformer, Test-Time Augmentation, CIFAR-10
\end{IEEEkeywords}

\input{01_introduction/main}
\input{02_related_work/main}
\input{03_method/main}
\input{04_results/main}
\input{05_discussion/main}
\input{06_conclusion/main}
\input{appendix/main}

\bibliographystyle{ieeetr}
\bibliography{references}

\end{document}

% !TEX program = xelatex
\documentclass[conference]{IEEEtran}

% 日本語出力と XeLaTeX 対応設定
\usepackage{xeCJK}
\usepackage{fontspec}
\setCJKmainfont{HaranoAjiMincho}

% 数式・図表・参考文献などの標準パッケージ
\usepackage{amsmath, amssymb}
\usepackage{graphicx}
\usepackage{url}
\usepackage{hyperref}

\hypersetup{
  colorlinks=true,
  linkcolor=blue,
  citecolor=blue,
  urlcolor=blue
}

\begin{document}

\title{軽量 CNN と Vision Transformer の公平比較に向けた実験的評価}

\author{\IEEEauthorblockN{著者 太郎\\ }
\IEEEauthorblockA{所属機関\\ 連絡先: author@example.com}}

\maketitle

\begin{abstract}
本稿では,軽量な畳み込みニューラルネットワーク(CNN)と Vision Transformer(ViT)の画像分類性能を,CIFAR-10 を対象に公平に比較する実験プロトコルを提示する。学習率スケジューリング,データ拡張,Test-Time Augmentation (TTA) を統一的に適用し,再現性を重視した評価を実施した。実験の結果,ViT-Ti は TTA により最大で +1.8pt の精度向上を得た一方,ResNet-18 では RandAugment を中心としたデータ拡張で +2.1pt の改善が確認された。これらの知見を通じて,限られた計算資源でも堅牢なベースラインを構築するための指針を示す。
\end{abstract}

\begin{IEEEkeywords}
画像分類, 深層学習, Vision Transformer, Test-Time Augmentation, CIFAR-10
\end{IEEEkeywords}

\input{01_introduction/main}
\input{02_related_work/main}
\input{03_method/main}
\input{04_results/main}
\input{05_discussion/main}
\input{06_conclusion/main}
\input{appendix/main}

\bibliographystyle{ieeetr}
\bibliography{references}

\end{document}

% !TEX program = xelatex
\documentclass[conference]{IEEEtran}

% 日本語出力と XeLaTeX 対応設定
\usepackage{xeCJK}
\usepackage{fontspec}
\setCJKmainfont{HaranoAjiMincho}

% 数式・図表・参考文献などの標準パッケージ
\usepackage{amsmath, amssymb}
\usepackage{graphicx}
\usepackage{url}
\usepackage{hyperref}

\hypersetup{
  colorlinks=true,
  linkcolor=blue,
  citecolor=blue,
  urlcolor=blue
}

\begin{document}

\title{軽量 CNN と Vision Transformer の公平比較に向けた実験的評価}

\author{\IEEEauthorblockN{著者 太郎\\ }
\IEEEauthorblockA{所属機関\\ 連絡先: author@example.com}}

\maketitle

\begin{abstract}
本稿では,軽量な畳み込みニューラルネットワーク(CNN)と Vision Transformer(ViT)の画像分類性能を,CIFAR-10 を対象に公平に比較する実験プロトコルを提示する。学習率スケジューリング,データ拡張,Test-Time Augmentation (TTA) を統一的に適用し,再現性を重視した評価を実施した。実験の結果,ViT-Ti は TTA により最大で +1.8pt の精度向上を得た一方,ResNet-18 では RandAugment を中心としたデータ拡張で +2.1pt の改善が確認された。これらの知見を通じて,限られた計算資源でも堅牢なベースラインを構築するための指針を示す。
\end{abstract}

\begin{IEEEkeywords}
画像分類, 深層学習, Vision Transformer, Test-Time Augmentation, CIFAR-10
\end{IEEEkeywords}

\input{01_introduction/main}
\input{02_related_work/main}
\input{03_method/main}
\input{04_results/main}
\input{05_discussion/main}
\input{06_conclusion/main}
\input{appendix/main}

\bibliographystyle{ieeetr}
\bibliography{references}

\end{document}

% !TEX program = xelatex
\documentclass[conference]{IEEEtran}

% 日本語出力と XeLaTeX 対応設定
\usepackage{xeCJK}
\usepackage{fontspec}
\setCJKmainfont{HaranoAjiMincho}

% 数式・図表・参考文献などの標準パッケージ
\usepackage{amsmath, amssymb}
\usepackage{graphicx}
\usepackage{url}
\usepackage{hyperref}

\hypersetup{
  colorlinks=true,
  linkcolor=blue,
  citecolor=blue,
  urlcolor=blue
}

\begin{document}

\title{軽量 CNN と Vision Transformer の公平比較に向けた実験的評価}

\author{\IEEEauthorblockN{著者 太郎\\ }
\IEEEauthorblockA{所属機関\\ 連絡先: author@example.com}}

\maketitle

\begin{abstract}
本稿では,軽量な畳み込みニューラルネットワーク(CNN)と Vision Transformer(ViT)の画像分類性能を,CIFAR-10 を対象に公平に比較する実験プロトコルを提示する。学習率スケジューリング,データ拡張,Test-Time Augmentation (TTA) を統一的に適用し,再現性を重視した評価を実施した。実験の結果,ViT-Ti は TTA により最大で +1.8pt の精度向上を得た一方,ResNet-18 では RandAugment を中心としたデータ拡張で +2.1pt の改善が確認された。これらの知見を通じて,限られた計算資源でも堅牢なベースラインを構築するための指針を示す。
\end{abstract}

\begin{IEEEkeywords}
画像分類, 深層学習, Vision Transformer, Test-Time Augmentation, CIFAR-10
\end{IEEEkeywords}

\input{01_introduction/main}
\input{02_related_work/main}
\input{03_method/main}
\input{04_results/main}
\input{05_discussion/main}
\input{06_conclusion/main}
\input{appendix/main}

\bibliographystyle{ieeetr}
\bibliography{references}

\end{document}


\bibliographystyle{ieeetr}
\bibliography{references}

\end{document}

% !TEX program = xelatex
\documentclass[conference]{IEEEtran}

% 日本語出力と XeLaTeX 対応設定
\usepackage{xeCJK}
\usepackage{fontspec}
\setCJKmainfont{HaranoAjiMincho}

% 数式・図表・参考文献などの標準パッケージ
\usepackage{amsmath, amssymb}
\usepackage{graphicx}
\usepackage{url}
\usepackage{hyperref}

\hypersetup{
  colorlinks=true,
  linkcolor=blue,
  citecolor=blue,
  urlcolor=blue
}

\begin{document}

\title{軽量 CNN と Vision Transformer の公平比較に向けた実験的評価}

\author{\IEEEauthorblockN{著者 太郎\\ }
\IEEEauthorblockA{所属機関\\ 連絡先: author@example.com}}

\maketitle

\begin{abstract}
本稿では,軽量な畳み込みニューラルネットワーク(CNN)と Vision Transformer(ViT)の画像分類性能を,CIFAR-10 を対象に公平に比較する実験プロトコルを提示する。学習率スケジューリング,データ拡張,Test-Time Augmentation (TTA) を統一的に適用し,再現性を重視した評価を実施した。実験の結果,ViT-Ti は TTA により最大で +1.8pt の精度向上を得た一方,ResNet-18 では RandAugment を中心としたデータ拡張で +2.1pt の改善が確認された。これらの知見を通じて,限られた計算資源でも堅牢なベースラインを構築するための指針を示す。
\end{abstract}

\begin{IEEEkeywords}
画像分類, 深層学習, Vision Transformer, Test-Time Augmentation, CIFAR-10
\end{IEEEkeywords}

% !TEX program = xelatex
\documentclass[conference]{IEEEtran}

% 日本語出力と XeLaTeX 対応設定
\usepackage{xeCJK}
\usepackage{fontspec}
\setCJKmainfont{HaranoAjiMincho}

% 数式・図表・参考文献などの標準パッケージ
\usepackage{amsmath, amssymb}
\usepackage{graphicx}
\usepackage{url}
\usepackage{hyperref}

\hypersetup{
  colorlinks=true,
  linkcolor=blue,
  citecolor=blue,
  urlcolor=blue
}

\begin{document}

\title{軽量 CNN と Vision Transformer の公平比較に向けた実験的評価}

\author{\IEEEauthorblockN{著者 太郎\\ }
\IEEEauthorblockA{所属機関\\ 連絡先: author@example.com}}

\maketitle

\begin{abstract}
本稿では,軽量な畳み込みニューラルネットワーク(CNN)と Vision Transformer(ViT)の画像分類性能を,CIFAR-10 を対象に公平に比較する実験プロトコルを提示する。学習率スケジューリング,データ拡張,Test-Time Augmentation (TTA) を統一的に適用し,再現性を重視した評価を実施した。実験の結果,ViT-Ti は TTA により最大で +1.8pt の精度向上を得た一方,ResNet-18 では RandAugment を中心としたデータ拡張で +2.1pt の改善が確認された。これらの知見を通じて,限られた計算資源でも堅牢なベースラインを構築するための指針を示す。
\end{abstract}

\begin{IEEEkeywords}
画像分類, 深層学習, Vision Transformer, Test-Time Augmentation, CIFAR-10
\end{IEEEkeywords}

\input{01_introduction/main}
\input{02_related_work/main}
\input{03_method/main}
\input{04_results/main}
\input{05_discussion/main}
\input{06_conclusion/main}
\input{appendix/main}

\bibliographystyle{ieeetr}
\bibliography{references}

\end{document}

% !TEX program = xelatex
\documentclass[conference]{IEEEtran}

% 日本語出力と XeLaTeX 対応設定
\usepackage{xeCJK}
\usepackage{fontspec}
\setCJKmainfont{HaranoAjiMincho}

% 数式・図表・参考文献などの標準パッケージ
\usepackage{amsmath, amssymb}
\usepackage{graphicx}
\usepackage{url}
\usepackage{hyperref}

\hypersetup{
  colorlinks=true,
  linkcolor=blue,
  citecolor=blue,
  urlcolor=blue
}

\begin{document}

\title{軽量 CNN と Vision Transformer の公平比較に向けた実験的評価}

\author{\IEEEauthorblockN{著者 太郎\\ }
\IEEEauthorblockA{所属機関\\ 連絡先: author@example.com}}

\maketitle

\begin{abstract}
本稿では,軽量な畳み込みニューラルネットワーク(CNN)と Vision Transformer(ViT)の画像分類性能を,CIFAR-10 を対象に公平に比較する実験プロトコルを提示する。学習率スケジューリング,データ拡張,Test-Time Augmentation (TTA) を統一的に適用し,再現性を重視した評価を実施した。実験の結果,ViT-Ti は TTA により最大で +1.8pt の精度向上を得た一方,ResNet-18 では RandAugment を中心としたデータ拡張で +2.1pt の改善が確認された。これらの知見を通じて,限られた計算資源でも堅牢なベースラインを構築するための指針を示す。
\end{abstract}

\begin{IEEEkeywords}
画像分類, 深層学習, Vision Transformer, Test-Time Augmentation, CIFAR-10
\end{IEEEkeywords}

\input{01_introduction/main}
\input{02_related_work/main}
\input{03_method/main}
\input{04_results/main}
\input{05_discussion/main}
\input{06_conclusion/main}
\input{appendix/main}

\bibliographystyle{ieeetr}
\bibliography{references}

\end{document}

% !TEX program = xelatex
\documentclass[conference]{IEEEtran}

% 日本語出力と XeLaTeX 対応設定
\usepackage{xeCJK}
\usepackage{fontspec}
\setCJKmainfont{HaranoAjiMincho}

% 数式・図表・参考文献などの標準パッケージ
\usepackage{amsmath, amssymb}
\usepackage{graphicx}
\usepackage{url}
\usepackage{hyperref}

\hypersetup{
  colorlinks=true,
  linkcolor=blue,
  citecolor=blue,
  urlcolor=blue
}

\begin{document}

\title{軽量 CNN と Vision Transformer の公平比較に向けた実験的評価}

\author{\IEEEauthorblockN{著者 太郎\\ }
\IEEEauthorblockA{所属機関\\ 連絡先: author@example.com}}

\maketitle

\begin{abstract}
本稿では,軽量な畳み込みニューラルネットワーク(CNN)と Vision Transformer(ViT)の画像分類性能を,CIFAR-10 を対象に公平に比較する実験プロトコルを提示する。学習率スケジューリング,データ拡張,Test-Time Augmentation (TTA) を統一的に適用し,再現性を重視した評価を実施した。実験の結果,ViT-Ti は TTA により最大で +1.8pt の精度向上を得た一方,ResNet-18 では RandAugment を中心としたデータ拡張で +2.1pt の改善が確認された。これらの知見を通じて,限られた計算資源でも堅牢なベースラインを構築するための指針を示す。
\end{abstract}

\begin{IEEEkeywords}
画像分類, 深層学習, Vision Transformer, Test-Time Augmentation, CIFAR-10
\end{IEEEkeywords}

\input{01_introduction/main}
\input{02_related_work/main}
\input{03_method/main}
\input{04_results/main}
\input{05_discussion/main}
\input{06_conclusion/main}
\input{appendix/main}

\bibliographystyle{ieeetr}
\bibliography{references}

\end{document}

% !TEX program = xelatex
\documentclass[conference]{IEEEtran}

% 日本語出力と XeLaTeX 対応設定
\usepackage{xeCJK}
\usepackage{fontspec}
\setCJKmainfont{HaranoAjiMincho}

% 数式・図表・参考文献などの標準パッケージ
\usepackage{amsmath, amssymb}
\usepackage{graphicx}
\usepackage{url}
\usepackage{hyperref}

\hypersetup{
  colorlinks=true,
  linkcolor=blue,
  citecolor=blue,
  urlcolor=blue
}

\begin{document}

\title{軽量 CNN と Vision Transformer の公平比較に向けた実験的評価}

\author{\IEEEauthorblockN{著者 太郎\\ }
\IEEEauthorblockA{所属機関\\ 連絡先: author@example.com}}

\maketitle

\begin{abstract}
本稿では,軽量な畳み込みニューラルネットワーク(CNN)と Vision Transformer(ViT)の画像分類性能を,CIFAR-10 を対象に公平に比較する実験プロトコルを提示する。学習率スケジューリング,データ拡張,Test-Time Augmentation (TTA) を統一的に適用し,再現性を重視した評価を実施した。実験の結果,ViT-Ti は TTA により最大で +1.8pt の精度向上を得た一方,ResNet-18 では RandAugment を中心としたデータ拡張で +2.1pt の改善が確認された。これらの知見を通じて,限られた計算資源でも堅牢なベースラインを構築するための指針を示す。
\end{abstract}

\begin{IEEEkeywords}
画像分類, 深層学習, Vision Transformer, Test-Time Augmentation, CIFAR-10
\end{IEEEkeywords}

\input{01_introduction/main}
\input{02_related_work/main}
\input{03_method/main}
\input{04_results/main}
\input{05_discussion/main}
\input{06_conclusion/main}
\input{appendix/main}

\bibliographystyle{ieeetr}
\bibliography{references}

\end{document}

% !TEX program = xelatex
\documentclass[conference]{IEEEtran}

% 日本語出力と XeLaTeX 対応設定
\usepackage{xeCJK}
\usepackage{fontspec}
\setCJKmainfont{HaranoAjiMincho}

% 数式・図表・参考文献などの標準パッケージ
\usepackage{amsmath, amssymb}
\usepackage{graphicx}
\usepackage{url}
\usepackage{hyperref}

\hypersetup{
  colorlinks=true,
  linkcolor=blue,
  citecolor=blue,
  urlcolor=blue
}

\begin{document}

\title{軽量 CNN と Vision Transformer の公平比較に向けた実験的評価}

\author{\IEEEauthorblockN{著者 太郎\\ }
\IEEEauthorblockA{所属機関\\ 連絡先: author@example.com}}

\maketitle

\begin{abstract}
本稿では,軽量な畳み込みニューラルネットワーク(CNN)と Vision Transformer(ViT)の画像分類性能を,CIFAR-10 を対象に公平に比較する実験プロトコルを提示する。学習率スケジューリング,データ拡張,Test-Time Augmentation (TTA) を統一的に適用し,再現性を重視した評価を実施した。実験の結果,ViT-Ti は TTA により最大で +1.8pt の精度向上を得た一方,ResNet-18 では RandAugment を中心としたデータ拡張で +2.1pt の改善が確認された。これらの知見を通じて,限られた計算資源でも堅牢なベースラインを構築するための指針を示す。
\end{abstract}

\begin{IEEEkeywords}
画像分類, 深層学習, Vision Transformer, Test-Time Augmentation, CIFAR-10
\end{IEEEkeywords}

\input{01_introduction/main}
\input{02_related_work/main}
\input{03_method/main}
\input{04_results/main}
\input{05_discussion/main}
\input{06_conclusion/main}
\input{appendix/main}

\bibliographystyle{ieeetr}
\bibliography{references}

\end{document}

% !TEX program = xelatex
\documentclass[conference]{IEEEtran}

% 日本語出力と XeLaTeX 対応設定
\usepackage{xeCJK}
\usepackage{fontspec}
\setCJKmainfont{HaranoAjiMincho}

% 数式・図表・参考文献などの標準パッケージ
\usepackage{amsmath, amssymb}
\usepackage{graphicx}
\usepackage{url}
\usepackage{hyperref}

\hypersetup{
  colorlinks=true,
  linkcolor=blue,
  citecolor=blue,
  urlcolor=blue
}

\begin{document}

\title{軽量 CNN と Vision Transformer の公平比較に向けた実験的評価}

\author{\IEEEauthorblockN{著者 太郎\\ }
\IEEEauthorblockA{所属機関\\ 連絡先: author@example.com}}

\maketitle

\begin{abstract}
本稿では,軽量な畳み込みニューラルネットワーク(CNN)と Vision Transformer(ViT)の画像分類性能を,CIFAR-10 を対象に公平に比較する実験プロトコルを提示する。学習率スケジューリング,データ拡張,Test-Time Augmentation (TTA) を統一的に適用し,再現性を重視した評価を実施した。実験の結果,ViT-Ti は TTA により最大で +1.8pt の精度向上を得た一方,ResNet-18 では RandAugment を中心としたデータ拡張で +2.1pt の改善が確認された。これらの知見を通じて,限られた計算資源でも堅牢なベースラインを構築するための指針を示す。
\end{abstract}

\begin{IEEEkeywords}
画像分類, 深層学習, Vision Transformer, Test-Time Augmentation, CIFAR-10
\end{IEEEkeywords}

\input{01_introduction/main}
\input{02_related_work/main}
\input{03_method/main}
\input{04_results/main}
\input{05_discussion/main}
\input{06_conclusion/main}
\input{appendix/main}

\bibliographystyle{ieeetr}
\bibliography{references}

\end{document}

% !TEX program = xelatex
\documentclass[conference]{IEEEtran}

% 日本語出力と XeLaTeX 対応設定
\usepackage{xeCJK}
\usepackage{fontspec}
\setCJKmainfont{HaranoAjiMincho}

% 数式・図表・参考文献などの標準パッケージ
\usepackage{amsmath, amssymb}
\usepackage{graphicx}
\usepackage{url}
\usepackage{hyperref}

\hypersetup{
  colorlinks=true,
  linkcolor=blue,
  citecolor=blue,
  urlcolor=blue
}

\begin{document}

\title{軽量 CNN と Vision Transformer の公平比較に向けた実験的評価}

\author{\IEEEauthorblockN{著者 太郎\\ }
\IEEEauthorblockA{所属機関\\ 連絡先: author@example.com}}

\maketitle

\begin{abstract}
本稿では,軽量な畳み込みニューラルネットワーク(CNN)と Vision Transformer(ViT)の画像分類性能を,CIFAR-10 を対象に公平に比較する実験プロトコルを提示する。学習率スケジューリング,データ拡張,Test-Time Augmentation (TTA) を統一的に適用し,再現性を重視した評価を実施した。実験の結果,ViT-Ti は TTA により最大で +1.8pt の精度向上を得た一方,ResNet-18 では RandAugment を中心としたデータ拡張で +2.1pt の改善が確認された。これらの知見を通じて,限られた計算資源でも堅牢なベースラインを構築するための指針を示す。
\end{abstract}

\begin{IEEEkeywords}
画像分類, 深層学習, Vision Transformer, Test-Time Augmentation, CIFAR-10
\end{IEEEkeywords}

\input{01_introduction/main}
\input{02_related_work/main}
\input{03_method/main}
\input{04_results/main}
\input{05_discussion/main}
\input{06_conclusion/main}
\input{appendix/main}

\bibliographystyle{ieeetr}
\bibliography{references}

\end{document}


\bibliographystyle{ieeetr}
\bibliography{references}

\end{document}


\bibliographystyle{ieeetr}
\bibliography{references}

\end{document}

% !TEX program = xelatex
\documentclass[conference]{IEEEtran}

% 日本語出力と XeLaTeX 対応設定
\usepackage{xeCJK}
\usepackage{fontspec}
\setCJKmainfont{HaranoAjiMincho}

% 数式・図表・参考文献などの標準パッケージ
\usepackage{amsmath, amssymb}
\usepackage{graphicx}
\usepackage{url}
\usepackage{hyperref}

\hypersetup{
  colorlinks=true,
  linkcolor=blue,
  citecolor=blue,
  urlcolor=blue
}

\begin{document}

\title{軽量 CNN と Vision Transformer の公平比較に向けた実験的評価}

\author{\IEEEauthorblockN{著者 太郎\\ }
\IEEEauthorblockA{所属機関\\ 連絡先: author@example.com}}

\maketitle

\begin{abstract}
本稿では,軽量な畳み込みニューラルネットワーク(CNN)と Vision Transformer(ViT)の画像分類性能を,CIFAR-10 を対象に公平に比較する実験プロトコルを提示する。学習率スケジューリング,データ拡張,Test-Time Augmentation (TTA) を統一的に適用し,再現性を重視した評価を実施した。実験の結果,ViT-Ti は TTA により最大で +1.8pt の精度向上を得た一方,ResNet-18 では RandAugment を中心としたデータ拡張で +2.1pt の改善が確認された。これらの知見を通じて,限られた計算資源でも堅牢なベースラインを構築するための指針を示す。
\end{abstract}

\begin{IEEEkeywords}
画像分類, 深層学習, Vision Transformer, Test-Time Augmentation, CIFAR-10
\end{IEEEkeywords}

% !TEX program = xelatex
\documentclass[conference]{IEEEtran}

% 日本語出力と XeLaTeX 対応設定
\usepackage{xeCJK}
\usepackage{fontspec}
\setCJKmainfont{HaranoAjiMincho}

% 数式・図表・参考文献などの標準パッケージ
\usepackage{amsmath, amssymb}
\usepackage{graphicx}
\usepackage{url}
\usepackage{hyperref}

\hypersetup{
  colorlinks=true,
  linkcolor=blue,
  citecolor=blue,
  urlcolor=blue
}

\begin{document}

\title{軽量 CNN と Vision Transformer の公平比較に向けた実験的評価}

\author{\IEEEauthorblockN{著者 太郎\\ }
\IEEEauthorblockA{所属機関\\ 連絡先: author@example.com}}

\maketitle

\begin{abstract}
本稿では,軽量な畳み込みニューラルネットワーク(CNN)と Vision Transformer(ViT)の画像分類性能を,CIFAR-10 を対象に公平に比較する実験プロトコルを提示する。学習率スケジューリング,データ拡張,Test-Time Augmentation (TTA) を統一的に適用し,再現性を重視した評価を実施した。実験の結果,ViT-Ti は TTA により最大で +1.8pt の精度向上を得た一方,ResNet-18 では RandAugment を中心としたデータ拡張で +2.1pt の改善が確認された。これらの知見を通じて,限られた計算資源でも堅牢なベースラインを構築するための指針を示す。
\end{abstract}

\begin{IEEEkeywords}
画像分類, 深層学習, Vision Transformer, Test-Time Augmentation, CIFAR-10
\end{IEEEkeywords}

% !TEX program = xelatex
\documentclass[conference]{IEEEtran}

% 日本語出力と XeLaTeX 対応設定
\usepackage{xeCJK}
\usepackage{fontspec}
\setCJKmainfont{HaranoAjiMincho}

% 数式・図表・参考文献などの標準パッケージ
\usepackage{amsmath, amssymb}
\usepackage{graphicx}
\usepackage{url}
\usepackage{hyperref}

\hypersetup{
  colorlinks=true,
  linkcolor=blue,
  citecolor=blue,
  urlcolor=blue
}

\begin{document}

\title{軽量 CNN と Vision Transformer の公平比較に向けた実験的評価}

\author{\IEEEauthorblockN{著者 太郎\\ }
\IEEEauthorblockA{所属機関\\ 連絡先: author@example.com}}

\maketitle

\begin{abstract}
本稿では,軽量な畳み込みニューラルネットワーク(CNN)と Vision Transformer(ViT)の画像分類性能を,CIFAR-10 を対象に公平に比較する実験プロトコルを提示する。学習率スケジューリング,データ拡張,Test-Time Augmentation (TTA) を統一的に適用し,再現性を重視した評価を実施した。実験の結果,ViT-Ti は TTA により最大で +1.8pt の精度向上を得た一方,ResNet-18 では RandAugment を中心としたデータ拡張で +2.1pt の改善が確認された。これらの知見を通じて,限られた計算資源でも堅牢なベースラインを構築するための指針を示す。
\end{abstract}

\begin{IEEEkeywords}
画像分類, 深層学習, Vision Transformer, Test-Time Augmentation, CIFAR-10
\end{IEEEkeywords}

\input{01_introduction/main}
\input{02_related_work/main}
\input{03_method/main}
\input{04_results/main}
\input{05_discussion/main}
\input{06_conclusion/main}
\input{appendix/main}

\bibliographystyle{ieeetr}
\bibliography{references}

\end{document}

% !TEX program = xelatex
\documentclass[conference]{IEEEtran}

% 日本語出力と XeLaTeX 対応設定
\usepackage{xeCJK}
\usepackage{fontspec}
\setCJKmainfont{HaranoAjiMincho}

% 数式・図表・参考文献などの標準パッケージ
\usepackage{amsmath, amssymb}
\usepackage{graphicx}
\usepackage{url}
\usepackage{hyperref}

\hypersetup{
  colorlinks=true,
  linkcolor=blue,
  citecolor=blue,
  urlcolor=blue
}

\begin{document}

\title{軽量 CNN と Vision Transformer の公平比較に向けた実験的評価}

\author{\IEEEauthorblockN{著者 太郎\\ }
\IEEEauthorblockA{所属機関\\ 連絡先: author@example.com}}

\maketitle

\begin{abstract}
本稿では,軽量な畳み込みニューラルネットワーク(CNN)と Vision Transformer(ViT)の画像分類性能を,CIFAR-10 を対象に公平に比較する実験プロトコルを提示する。学習率スケジューリング,データ拡張,Test-Time Augmentation (TTA) を統一的に適用し,再現性を重視した評価を実施した。実験の結果,ViT-Ti は TTA により最大で +1.8pt の精度向上を得た一方,ResNet-18 では RandAugment を中心としたデータ拡張で +2.1pt の改善が確認された。これらの知見を通じて,限られた計算資源でも堅牢なベースラインを構築するための指針を示す。
\end{abstract}

\begin{IEEEkeywords}
画像分類, 深層学習, Vision Transformer, Test-Time Augmentation, CIFAR-10
\end{IEEEkeywords}

\input{01_introduction/main}
\input{02_related_work/main}
\input{03_method/main}
\input{04_results/main}
\input{05_discussion/main}
\input{06_conclusion/main}
\input{appendix/main}

\bibliographystyle{ieeetr}
\bibliography{references}

\end{document}

% !TEX program = xelatex
\documentclass[conference]{IEEEtran}

% 日本語出力と XeLaTeX 対応設定
\usepackage{xeCJK}
\usepackage{fontspec}
\setCJKmainfont{HaranoAjiMincho}

% 数式・図表・参考文献などの標準パッケージ
\usepackage{amsmath, amssymb}
\usepackage{graphicx}
\usepackage{url}
\usepackage{hyperref}

\hypersetup{
  colorlinks=true,
  linkcolor=blue,
  citecolor=blue,
  urlcolor=blue
}

\begin{document}

\title{軽量 CNN と Vision Transformer の公平比較に向けた実験的評価}

\author{\IEEEauthorblockN{著者 太郎\\ }
\IEEEauthorblockA{所属機関\\ 連絡先: author@example.com}}

\maketitle

\begin{abstract}
本稿では,軽量な畳み込みニューラルネットワーク(CNN)と Vision Transformer(ViT)の画像分類性能を,CIFAR-10 を対象に公平に比較する実験プロトコルを提示する。学習率スケジューリング,データ拡張,Test-Time Augmentation (TTA) を統一的に適用し,再現性を重視した評価を実施した。実験の結果,ViT-Ti は TTA により最大で +1.8pt の精度向上を得た一方,ResNet-18 では RandAugment を中心としたデータ拡張で +2.1pt の改善が確認された。これらの知見を通じて,限られた計算資源でも堅牢なベースラインを構築するための指針を示す。
\end{abstract}

\begin{IEEEkeywords}
画像分類, 深層学習, Vision Transformer, Test-Time Augmentation, CIFAR-10
\end{IEEEkeywords}

\input{01_introduction/main}
\input{02_related_work/main}
\input{03_method/main}
\input{04_results/main}
\input{05_discussion/main}
\input{06_conclusion/main}
\input{appendix/main}

\bibliographystyle{ieeetr}
\bibliography{references}

\end{document}

% !TEX program = xelatex
\documentclass[conference]{IEEEtran}

% 日本語出力と XeLaTeX 対応設定
\usepackage{xeCJK}
\usepackage{fontspec}
\setCJKmainfont{HaranoAjiMincho}

% 数式・図表・参考文献などの標準パッケージ
\usepackage{amsmath, amssymb}
\usepackage{graphicx}
\usepackage{url}
\usepackage{hyperref}

\hypersetup{
  colorlinks=true,
  linkcolor=blue,
  citecolor=blue,
  urlcolor=blue
}

\begin{document}

\title{軽量 CNN と Vision Transformer の公平比較に向けた実験的評価}

\author{\IEEEauthorblockN{著者 太郎\\ }
\IEEEauthorblockA{所属機関\\ 連絡先: author@example.com}}

\maketitle

\begin{abstract}
本稿では,軽量な畳み込みニューラルネットワーク(CNN)と Vision Transformer(ViT)の画像分類性能を,CIFAR-10 を対象に公平に比較する実験プロトコルを提示する。学習率スケジューリング,データ拡張,Test-Time Augmentation (TTA) を統一的に適用し,再現性を重視した評価を実施した。実験の結果,ViT-Ti は TTA により最大で +1.8pt の精度向上を得た一方,ResNet-18 では RandAugment を中心としたデータ拡張で +2.1pt の改善が確認された。これらの知見を通じて,限られた計算資源でも堅牢なベースラインを構築するための指針を示す。
\end{abstract}

\begin{IEEEkeywords}
画像分類, 深層学習, Vision Transformer, Test-Time Augmentation, CIFAR-10
\end{IEEEkeywords}

\input{01_introduction/main}
\input{02_related_work/main}
\input{03_method/main}
\input{04_results/main}
\input{05_discussion/main}
\input{06_conclusion/main}
\input{appendix/main}

\bibliographystyle{ieeetr}
\bibliography{references}

\end{document}

% !TEX program = xelatex
\documentclass[conference]{IEEEtran}

% 日本語出力と XeLaTeX 対応設定
\usepackage{xeCJK}
\usepackage{fontspec}
\setCJKmainfont{HaranoAjiMincho}

% 数式・図表・参考文献などの標準パッケージ
\usepackage{amsmath, amssymb}
\usepackage{graphicx}
\usepackage{url}
\usepackage{hyperref}

\hypersetup{
  colorlinks=true,
  linkcolor=blue,
  citecolor=blue,
  urlcolor=blue
}

\begin{document}

\title{軽量 CNN と Vision Transformer の公平比較に向けた実験的評価}

\author{\IEEEauthorblockN{著者 太郎\\ }
\IEEEauthorblockA{所属機関\\ 連絡先: author@example.com}}

\maketitle

\begin{abstract}
本稿では,軽量な畳み込みニューラルネットワーク(CNN)と Vision Transformer(ViT)の画像分類性能を,CIFAR-10 を対象に公平に比較する実験プロトコルを提示する。学習率スケジューリング,データ拡張,Test-Time Augmentation (TTA) を統一的に適用し,再現性を重視した評価を実施した。実験の結果,ViT-Ti は TTA により最大で +1.8pt の精度向上を得た一方,ResNet-18 では RandAugment を中心としたデータ拡張で +2.1pt の改善が確認された。これらの知見を通じて,限られた計算資源でも堅牢なベースラインを構築するための指針を示す。
\end{abstract}

\begin{IEEEkeywords}
画像分類, 深層学習, Vision Transformer, Test-Time Augmentation, CIFAR-10
\end{IEEEkeywords}

\input{01_introduction/main}
\input{02_related_work/main}
\input{03_method/main}
\input{04_results/main}
\input{05_discussion/main}
\input{06_conclusion/main}
\input{appendix/main}

\bibliographystyle{ieeetr}
\bibliography{references}

\end{document}

% !TEX program = xelatex
\documentclass[conference]{IEEEtran}

% 日本語出力と XeLaTeX 対応設定
\usepackage{xeCJK}
\usepackage{fontspec}
\setCJKmainfont{HaranoAjiMincho}

% 数式・図表・参考文献などの標準パッケージ
\usepackage{amsmath, amssymb}
\usepackage{graphicx}
\usepackage{url}
\usepackage{hyperref}

\hypersetup{
  colorlinks=true,
  linkcolor=blue,
  citecolor=blue,
  urlcolor=blue
}

\begin{document}

\title{軽量 CNN と Vision Transformer の公平比較に向けた実験的評価}

\author{\IEEEauthorblockN{著者 太郎\\ }
\IEEEauthorblockA{所属機関\\ 連絡先: author@example.com}}

\maketitle

\begin{abstract}
本稿では,軽量な畳み込みニューラルネットワーク(CNN)と Vision Transformer(ViT)の画像分類性能を,CIFAR-10 を対象に公平に比較する実験プロトコルを提示する。学習率スケジューリング,データ拡張,Test-Time Augmentation (TTA) を統一的に適用し,再現性を重視した評価を実施した。実験の結果,ViT-Ti は TTA により最大で +1.8pt の精度向上を得た一方,ResNet-18 では RandAugment を中心としたデータ拡張で +2.1pt の改善が確認された。これらの知見を通じて,限られた計算資源でも堅牢なベースラインを構築するための指針を示す。
\end{abstract}

\begin{IEEEkeywords}
画像分類, 深層学習, Vision Transformer, Test-Time Augmentation, CIFAR-10
\end{IEEEkeywords}

\input{01_introduction/main}
\input{02_related_work/main}
\input{03_method/main}
\input{04_results/main}
\input{05_discussion/main}
\input{06_conclusion/main}
\input{appendix/main}

\bibliographystyle{ieeetr}
\bibliography{references}

\end{document}

% !TEX program = xelatex
\documentclass[conference]{IEEEtran}

% 日本語出力と XeLaTeX 対応設定
\usepackage{xeCJK}
\usepackage{fontspec}
\setCJKmainfont{HaranoAjiMincho}

% 数式・図表・参考文献などの標準パッケージ
\usepackage{amsmath, amssymb}
\usepackage{graphicx}
\usepackage{url}
\usepackage{hyperref}

\hypersetup{
  colorlinks=true,
  linkcolor=blue,
  citecolor=blue,
  urlcolor=blue
}

\begin{document}

\title{軽量 CNN と Vision Transformer の公平比較に向けた実験的評価}

\author{\IEEEauthorblockN{著者 太郎\\ }
\IEEEauthorblockA{所属機関\\ 連絡先: author@example.com}}

\maketitle

\begin{abstract}
本稿では,軽量な畳み込みニューラルネットワーク(CNN)と Vision Transformer(ViT)の画像分類性能を,CIFAR-10 を対象に公平に比較する実験プロトコルを提示する。学習率スケジューリング,データ拡張,Test-Time Augmentation (TTA) を統一的に適用し,再現性を重視した評価を実施した。実験の結果,ViT-Ti は TTA により最大で +1.8pt の精度向上を得た一方,ResNet-18 では RandAugment を中心としたデータ拡張で +2.1pt の改善が確認された。これらの知見を通じて,限られた計算資源でも堅牢なベースラインを構築するための指針を示す。
\end{abstract}

\begin{IEEEkeywords}
画像分類, 深層学習, Vision Transformer, Test-Time Augmentation, CIFAR-10
\end{IEEEkeywords}

\input{01_introduction/main}
\input{02_related_work/main}
\input{03_method/main}
\input{04_results/main}
\input{05_discussion/main}
\input{06_conclusion/main}
\input{appendix/main}

\bibliographystyle{ieeetr}
\bibliography{references}

\end{document}


\bibliographystyle{ieeetr}
\bibliography{references}

\end{document}

% !TEX program = xelatex
\documentclass[conference]{IEEEtran}

% 日本語出力と XeLaTeX 対応設定
\usepackage{xeCJK}
\usepackage{fontspec}
\setCJKmainfont{HaranoAjiMincho}

% 数式・図表・参考文献などの標準パッケージ
\usepackage{amsmath, amssymb}
\usepackage{graphicx}
\usepackage{url}
\usepackage{hyperref}

\hypersetup{
  colorlinks=true,
  linkcolor=blue,
  citecolor=blue,
  urlcolor=blue
}

\begin{document}

\title{軽量 CNN と Vision Transformer の公平比較に向けた実験的評価}

\author{\IEEEauthorblockN{著者 太郎\\ }
\IEEEauthorblockA{所属機関\\ 連絡先: author@example.com}}

\maketitle

\begin{abstract}
本稿では,軽量な畳み込みニューラルネットワーク(CNN)と Vision Transformer(ViT)の画像分類性能を,CIFAR-10 を対象に公平に比較する実験プロトコルを提示する。学習率スケジューリング,データ拡張,Test-Time Augmentation (TTA) を統一的に適用し,再現性を重視した評価を実施した。実験の結果,ViT-Ti は TTA により最大で +1.8pt の精度向上を得た一方,ResNet-18 では RandAugment を中心としたデータ拡張で +2.1pt の改善が確認された。これらの知見を通じて,限られた計算資源でも堅牢なベースラインを構築するための指針を示す。
\end{abstract}

\begin{IEEEkeywords}
画像分類, 深層学習, Vision Transformer, Test-Time Augmentation, CIFAR-10
\end{IEEEkeywords}

% !TEX program = xelatex
\documentclass[conference]{IEEEtran}

% 日本語出力と XeLaTeX 対応設定
\usepackage{xeCJK}
\usepackage{fontspec}
\setCJKmainfont{HaranoAjiMincho}

% 数式・図表・参考文献などの標準パッケージ
\usepackage{amsmath, amssymb}
\usepackage{graphicx}
\usepackage{url}
\usepackage{hyperref}

\hypersetup{
  colorlinks=true,
  linkcolor=blue,
  citecolor=blue,
  urlcolor=blue
}

\begin{document}

\title{軽量 CNN と Vision Transformer の公平比較に向けた実験的評価}

\author{\IEEEauthorblockN{著者 太郎\\ }
\IEEEauthorblockA{所属機関\\ 連絡先: author@example.com}}

\maketitle

\begin{abstract}
本稿では,軽量な畳み込みニューラルネットワーク(CNN)と Vision Transformer(ViT)の画像分類性能を,CIFAR-10 を対象に公平に比較する実験プロトコルを提示する。学習率スケジューリング,データ拡張,Test-Time Augmentation (TTA) を統一的に適用し,再現性を重視した評価を実施した。実験の結果,ViT-Ti は TTA により最大で +1.8pt の精度向上を得た一方,ResNet-18 では RandAugment を中心としたデータ拡張で +2.1pt の改善が確認された。これらの知見を通じて,限られた計算資源でも堅牢なベースラインを構築するための指針を示す。
\end{abstract}

\begin{IEEEkeywords}
画像分類, 深層学習, Vision Transformer, Test-Time Augmentation, CIFAR-10
\end{IEEEkeywords}

\input{01_introduction/main}
\input{02_related_work/main}
\input{03_method/main}
\input{04_results/main}
\input{05_discussion/main}
\input{06_conclusion/main}
\input{appendix/main}

\bibliographystyle{ieeetr}
\bibliography{references}

\end{document}

% !TEX program = xelatex
\documentclass[conference]{IEEEtran}

% 日本語出力と XeLaTeX 対応設定
\usepackage{xeCJK}
\usepackage{fontspec}
\setCJKmainfont{HaranoAjiMincho}

% 数式・図表・参考文献などの標準パッケージ
\usepackage{amsmath, amssymb}
\usepackage{graphicx}
\usepackage{url}
\usepackage{hyperref}

\hypersetup{
  colorlinks=true,
  linkcolor=blue,
  citecolor=blue,
  urlcolor=blue
}

\begin{document}

\title{軽量 CNN と Vision Transformer の公平比較に向けた実験的評価}

\author{\IEEEauthorblockN{著者 太郎\\ }
\IEEEauthorblockA{所属機関\\ 連絡先: author@example.com}}

\maketitle

\begin{abstract}
本稿では,軽量な畳み込みニューラルネットワーク(CNN)と Vision Transformer(ViT)の画像分類性能を,CIFAR-10 を対象に公平に比較する実験プロトコルを提示する。学習率スケジューリング,データ拡張,Test-Time Augmentation (TTA) を統一的に適用し,再現性を重視した評価を実施した。実験の結果,ViT-Ti は TTA により最大で +1.8pt の精度向上を得た一方,ResNet-18 では RandAugment を中心としたデータ拡張で +2.1pt の改善が確認された。これらの知見を通じて,限られた計算資源でも堅牢なベースラインを構築するための指針を示す。
\end{abstract}

\begin{IEEEkeywords}
画像分類, 深層学習, Vision Transformer, Test-Time Augmentation, CIFAR-10
\end{IEEEkeywords}

\input{01_introduction/main}
\input{02_related_work/main}
\input{03_method/main}
\input{04_results/main}
\input{05_discussion/main}
\input{06_conclusion/main}
\input{appendix/main}

\bibliographystyle{ieeetr}
\bibliography{references}

\end{document}

% !TEX program = xelatex
\documentclass[conference]{IEEEtran}

% 日本語出力と XeLaTeX 対応設定
\usepackage{xeCJK}
\usepackage{fontspec}
\setCJKmainfont{HaranoAjiMincho}

% 数式・図表・参考文献などの標準パッケージ
\usepackage{amsmath, amssymb}
\usepackage{graphicx}
\usepackage{url}
\usepackage{hyperref}

\hypersetup{
  colorlinks=true,
  linkcolor=blue,
  citecolor=blue,
  urlcolor=blue
}

\begin{document}

\title{軽量 CNN と Vision Transformer の公平比較に向けた実験的評価}

\author{\IEEEauthorblockN{著者 太郎\\ }
\IEEEauthorblockA{所属機関\\ 連絡先: author@example.com}}

\maketitle

\begin{abstract}
本稿では,軽量な畳み込みニューラルネットワーク(CNN)と Vision Transformer(ViT)の画像分類性能を,CIFAR-10 を対象に公平に比較する実験プロトコルを提示する。学習率スケジューリング,データ拡張,Test-Time Augmentation (TTA) を統一的に適用し,再現性を重視した評価を実施した。実験の結果,ViT-Ti は TTA により最大で +1.8pt の精度向上を得た一方,ResNet-18 では RandAugment を中心としたデータ拡張で +2.1pt の改善が確認された。これらの知見を通じて,限られた計算資源でも堅牢なベースラインを構築するための指針を示す。
\end{abstract}

\begin{IEEEkeywords}
画像分類, 深層学習, Vision Transformer, Test-Time Augmentation, CIFAR-10
\end{IEEEkeywords}

\input{01_introduction/main}
\input{02_related_work/main}
\input{03_method/main}
\input{04_results/main}
\input{05_discussion/main}
\input{06_conclusion/main}
\input{appendix/main}

\bibliographystyle{ieeetr}
\bibliography{references}

\end{document}

% !TEX program = xelatex
\documentclass[conference]{IEEEtran}

% 日本語出力と XeLaTeX 対応設定
\usepackage{xeCJK}
\usepackage{fontspec}
\setCJKmainfont{HaranoAjiMincho}

% 数式・図表・参考文献などの標準パッケージ
\usepackage{amsmath, amssymb}
\usepackage{graphicx}
\usepackage{url}
\usepackage{hyperref}

\hypersetup{
  colorlinks=true,
  linkcolor=blue,
  citecolor=blue,
  urlcolor=blue
}

\begin{document}

\title{軽量 CNN と Vision Transformer の公平比較に向けた実験的評価}

\author{\IEEEauthorblockN{著者 太郎\\ }
\IEEEauthorblockA{所属機関\\ 連絡先: author@example.com}}

\maketitle

\begin{abstract}
本稿では,軽量な畳み込みニューラルネットワーク(CNN)と Vision Transformer(ViT)の画像分類性能を,CIFAR-10 を対象に公平に比較する実験プロトコルを提示する。学習率スケジューリング,データ拡張,Test-Time Augmentation (TTA) を統一的に適用し,再現性を重視した評価を実施した。実験の結果,ViT-Ti は TTA により最大で +1.8pt の精度向上を得た一方,ResNet-18 では RandAugment を中心としたデータ拡張で +2.1pt の改善が確認された。これらの知見を通じて,限られた計算資源でも堅牢なベースラインを構築するための指針を示す。
\end{abstract}

\begin{IEEEkeywords}
画像分類, 深層学習, Vision Transformer, Test-Time Augmentation, CIFAR-10
\end{IEEEkeywords}

\input{01_introduction/main}
\input{02_related_work/main}
\input{03_method/main}
\input{04_results/main}
\input{05_discussion/main}
\input{06_conclusion/main}
\input{appendix/main}

\bibliographystyle{ieeetr}
\bibliography{references}

\end{document}

% !TEX program = xelatex
\documentclass[conference]{IEEEtran}

% 日本語出力と XeLaTeX 対応設定
\usepackage{xeCJK}
\usepackage{fontspec}
\setCJKmainfont{HaranoAjiMincho}

% 数式・図表・参考文献などの標準パッケージ
\usepackage{amsmath, amssymb}
\usepackage{graphicx}
\usepackage{url}
\usepackage{hyperref}

\hypersetup{
  colorlinks=true,
  linkcolor=blue,
  citecolor=blue,
  urlcolor=blue
}

\begin{document}

\title{軽量 CNN と Vision Transformer の公平比較に向けた実験的評価}

\author{\IEEEauthorblockN{著者 太郎\\ }
\IEEEauthorblockA{所属機関\\ 連絡先: author@example.com}}

\maketitle

\begin{abstract}
本稿では,軽量な畳み込みニューラルネットワーク(CNN)と Vision Transformer(ViT)の画像分類性能を,CIFAR-10 を対象に公平に比較する実験プロトコルを提示する。学習率スケジューリング,データ拡張,Test-Time Augmentation (TTA) を統一的に適用し,再現性を重視した評価を実施した。実験の結果,ViT-Ti は TTA により最大で +1.8pt の精度向上を得た一方,ResNet-18 では RandAugment を中心としたデータ拡張で +2.1pt の改善が確認された。これらの知見を通じて,限られた計算資源でも堅牢なベースラインを構築するための指針を示す。
\end{abstract}

\begin{IEEEkeywords}
画像分類, 深層学習, Vision Transformer, Test-Time Augmentation, CIFAR-10
\end{IEEEkeywords}

\input{01_introduction/main}
\input{02_related_work/main}
\input{03_method/main}
\input{04_results/main}
\input{05_discussion/main}
\input{06_conclusion/main}
\input{appendix/main}

\bibliographystyle{ieeetr}
\bibliography{references}

\end{document}

% !TEX program = xelatex
\documentclass[conference]{IEEEtran}

% 日本語出力と XeLaTeX 対応設定
\usepackage{xeCJK}
\usepackage{fontspec}
\setCJKmainfont{HaranoAjiMincho}

% 数式・図表・参考文献などの標準パッケージ
\usepackage{amsmath, amssymb}
\usepackage{graphicx}
\usepackage{url}
\usepackage{hyperref}

\hypersetup{
  colorlinks=true,
  linkcolor=blue,
  citecolor=blue,
  urlcolor=blue
}

\begin{document}

\title{軽量 CNN と Vision Transformer の公平比較に向けた実験的評価}

\author{\IEEEauthorblockN{著者 太郎\\ }
\IEEEauthorblockA{所属機関\\ 連絡先: author@example.com}}

\maketitle

\begin{abstract}
本稿では,軽量な畳み込みニューラルネットワーク(CNN)と Vision Transformer(ViT)の画像分類性能を,CIFAR-10 を対象に公平に比較する実験プロトコルを提示する。学習率スケジューリング,データ拡張,Test-Time Augmentation (TTA) を統一的に適用し,再現性を重視した評価を実施した。実験の結果,ViT-Ti は TTA により最大で +1.8pt の精度向上を得た一方,ResNet-18 では RandAugment を中心としたデータ拡張で +2.1pt の改善が確認された。これらの知見を通じて,限られた計算資源でも堅牢なベースラインを構築するための指針を示す。
\end{abstract}

\begin{IEEEkeywords}
画像分類, 深層学習, Vision Transformer, Test-Time Augmentation, CIFAR-10
\end{IEEEkeywords}

\input{01_introduction/main}
\input{02_related_work/main}
\input{03_method/main}
\input{04_results/main}
\input{05_discussion/main}
\input{06_conclusion/main}
\input{appendix/main}

\bibliographystyle{ieeetr}
\bibliography{references}

\end{document}

% !TEX program = xelatex
\documentclass[conference]{IEEEtran}

% 日本語出力と XeLaTeX 対応設定
\usepackage{xeCJK}
\usepackage{fontspec}
\setCJKmainfont{HaranoAjiMincho}

% 数式・図表・参考文献などの標準パッケージ
\usepackage{amsmath, amssymb}
\usepackage{graphicx}
\usepackage{url}
\usepackage{hyperref}

\hypersetup{
  colorlinks=true,
  linkcolor=blue,
  citecolor=blue,
  urlcolor=blue
}

\begin{document}

\title{軽量 CNN と Vision Transformer の公平比較に向けた実験的評価}

\author{\IEEEauthorblockN{著者 太郎\\ }
\IEEEauthorblockA{所属機関\\ 連絡先: author@example.com}}

\maketitle

\begin{abstract}
本稿では,軽量な畳み込みニューラルネットワーク(CNN)と Vision Transformer(ViT)の画像分類性能を,CIFAR-10 を対象に公平に比較する実験プロトコルを提示する。学習率スケジューリング,データ拡張,Test-Time Augmentation (TTA) を統一的に適用し,再現性を重視した評価を実施した。実験の結果,ViT-Ti は TTA により最大で +1.8pt の精度向上を得た一方,ResNet-18 では RandAugment を中心としたデータ拡張で +2.1pt の改善が確認された。これらの知見を通じて,限られた計算資源でも堅牢なベースラインを構築するための指針を示す。
\end{abstract}

\begin{IEEEkeywords}
画像分類, 深層学習, Vision Transformer, Test-Time Augmentation, CIFAR-10
\end{IEEEkeywords}

\input{01_introduction/main}
\input{02_related_work/main}
\input{03_method/main}
\input{04_results/main}
\input{05_discussion/main}
\input{06_conclusion/main}
\input{appendix/main}

\bibliographystyle{ieeetr}
\bibliography{references}

\end{document}


\bibliographystyle{ieeetr}
\bibliography{references}

\end{document}

% !TEX program = xelatex
\documentclass[conference]{IEEEtran}

% 日本語出力と XeLaTeX 対応設定
\usepackage{xeCJK}
\usepackage{fontspec}
\setCJKmainfont{HaranoAjiMincho}

% 数式・図表・参考文献などの標準パッケージ
\usepackage{amsmath, amssymb}
\usepackage{graphicx}
\usepackage{url}
\usepackage{hyperref}

\hypersetup{
  colorlinks=true,
  linkcolor=blue,
  citecolor=blue,
  urlcolor=blue
}

\begin{document}

\title{軽量 CNN と Vision Transformer の公平比較に向けた実験的評価}

\author{\IEEEauthorblockN{著者 太郎\\ }
\IEEEauthorblockA{所属機関\\ 連絡先: author@example.com}}

\maketitle

\begin{abstract}
本稿では,軽量な畳み込みニューラルネットワーク(CNN)と Vision Transformer(ViT)の画像分類性能を,CIFAR-10 を対象に公平に比較する実験プロトコルを提示する。学習率スケジューリング,データ拡張,Test-Time Augmentation (TTA) を統一的に適用し,再現性を重視した評価を実施した。実験の結果,ViT-Ti は TTA により最大で +1.8pt の精度向上を得た一方,ResNet-18 では RandAugment を中心としたデータ拡張で +2.1pt の改善が確認された。これらの知見を通じて,限られた計算資源でも堅牢なベースラインを構築するための指針を示す。
\end{abstract}

\begin{IEEEkeywords}
画像分類, 深層学習, Vision Transformer, Test-Time Augmentation, CIFAR-10
\end{IEEEkeywords}

% !TEX program = xelatex
\documentclass[conference]{IEEEtran}

% 日本語出力と XeLaTeX 対応設定
\usepackage{xeCJK}
\usepackage{fontspec}
\setCJKmainfont{HaranoAjiMincho}

% 数式・図表・参考文献などの標準パッケージ
\usepackage{amsmath, amssymb}
\usepackage{graphicx}
\usepackage{url}
\usepackage{hyperref}

\hypersetup{
  colorlinks=true,
  linkcolor=blue,
  citecolor=blue,
  urlcolor=blue
}

\begin{document}

\title{軽量 CNN と Vision Transformer の公平比較に向けた実験的評価}

\author{\IEEEauthorblockN{著者 太郎\\ }
\IEEEauthorblockA{所属機関\\ 連絡先: author@example.com}}

\maketitle

\begin{abstract}
本稿では,軽量な畳み込みニューラルネットワーク(CNN)と Vision Transformer(ViT)の画像分類性能を,CIFAR-10 を対象に公平に比較する実験プロトコルを提示する。学習率スケジューリング,データ拡張,Test-Time Augmentation (TTA) を統一的に適用し,再現性を重視した評価を実施した。実験の結果,ViT-Ti は TTA により最大で +1.8pt の精度向上を得た一方,ResNet-18 では RandAugment を中心としたデータ拡張で +2.1pt の改善が確認された。これらの知見を通じて,限られた計算資源でも堅牢なベースラインを構築するための指針を示す。
\end{abstract}

\begin{IEEEkeywords}
画像分類, 深層学習, Vision Transformer, Test-Time Augmentation, CIFAR-10
\end{IEEEkeywords}

\input{01_introduction/main}
\input{02_related_work/main}
\input{03_method/main}
\input{04_results/main}
\input{05_discussion/main}
\input{06_conclusion/main}
\input{appendix/main}

\bibliographystyle{ieeetr}
\bibliography{references}

\end{document}

% !TEX program = xelatex
\documentclass[conference]{IEEEtran}

% 日本語出力と XeLaTeX 対応設定
\usepackage{xeCJK}
\usepackage{fontspec}
\setCJKmainfont{HaranoAjiMincho}

% 数式・図表・参考文献などの標準パッケージ
\usepackage{amsmath, amssymb}
\usepackage{graphicx}
\usepackage{url}
\usepackage{hyperref}

\hypersetup{
  colorlinks=true,
  linkcolor=blue,
  citecolor=blue,
  urlcolor=blue
}

\begin{document}

\title{軽量 CNN と Vision Transformer の公平比較に向けた実験的評価}

\author{\IEEEauthorblockN{著者 太郎\\ }
\IEEEauthorblockA{所属機関\\ 連絡先: author@example.com}}

\maketitle

\begin{abstract}
本稿では,軽量な畳み込みニューラルネットワーク(CNN)と Vision Transformer(ViT)の画像分類性能を,CIFAR-10 を対象に公平に比較する実験プロトコルを提示する。学習率スケジューリング,データ拡張,Test-Time Augmentation (TTA) を統一的に適用し,再現性を重視した評価を実施した。実験の結果,ViT-Ti は TTA により最大で +1.8pt の精度向上を得た一方,ResNet-18 では RandAugment を中心としたデータ拡張で +2.1pt の改善が確認された。これらの知見を通じて,限られた計算資源でも堅牢なベースラインを構築するための指針を示す。
\end{abstract}

\begin{IEEEkeywords}
画像分類, 深層学習, Vision Transformer, Test-Time Augmentation, CIFAR-10
\end{IEEEkeywords}

\input{01_introduction/main}
\input{02_related_work/main}
\input{03_method/main}
\input{04_results/main}
\input{05_discussion/main}
\input{06_conclusion/main}
\input{appendix/main}

\bibliographystyle{ieeetr}
\bibliography{references}

\end{document}

% !TEX program = xelatex
\documentclass[conference]{IEEEtran}

% 日本語出力と XeLaTeX 対応設定
\usepackage{xeCJK}
\usepackage{fontspec}
\setCJKmainfont{HaranoAjiMincho}

% 数式・図表・参考文献などの標準パッケージ
\usepackage{amsmath, amssymb}
\usepackage{graphicx}
\usepackage{url}
\usepackage{hyperref}

\hypersetup{
  colorlinks=true,
  linkcolor=blue,
  citecolor=blue,
  urlcolor=blue
}

\begin{document}

\title{軽量 CNN と Vision Transformer の公平比較に向けた実験的評価}

\author{\IEEEauthorblockN{著者 太郎\\ }
\IEEEauthorblockA{所属機関\\ 連絡先: author@example.com}}

\maketitle

\begin{abstract}
本稿では,軽量な畳み込みニューラルネットワーク(CNN)と Vision Transformer(ViT)の画像分類性能を,CIFAR-10 を対象に公平に比較する実験プロトコルを提示する。学習率スケジューリング,データ拡張,Test-Time Augmentation (TTA) を統一的に適用し,再現性を重視した評価を実施した。実験の結果,ViT-Ti は TTA により最大で +1.8pt の精度向上を得た一方,ResNet-18 では RandAugment を中心としたデータ拡張で +2.1pt の改善が確認された。これらの知見を通じて,限られた計算資源でも堅牢なベースラインを構築するための指針を示す。
\end{abstract}

\begin{IEEEkeywords}
画像分類, 深層学習, Vision Transformer, Test-Time Augmentation, CIFAR-10
\end{IEEEkeywords}

\input{01_introduction/main}
\input{02_related_work/main}
\input{03_method/main}
\input{04_results/main}
\input{05_discussion/main}
\input{06_conclusion/main}
\input{appendix/main}

\bibliographystyle{ieeetr}
\bibliography{references}

\end{document}

% !TEX program = xelatex
\documentclass[conference]{IEEEtran}

% 日本語出力と XeLaTeX 対応設定
\usepackage{xeCJK}
\usepackage{fontspec}
\setCJKmainfont{HaranoAjiMincho}

% 数式・図表・参考文献などの標準パッケージ
\usepackage{amsmath, amssymb}
\usepackage{graphicx}
\usepackage{url}
\usepackage{hyperref}

\hypersetup{
  colorlinks=true,
  linkcolor=blue,
  citecolor=blue,
  urlcolor=blue
}

\begin{document}

\title{軽量 CNN と Vision Transformer の公平比較に向けた実験的評価}

\author{\IEEEauthorblockN{著者 太郎\\ }
\IEEEauthorblockA{所属機関\\ 連絡先: author@example.com}}

\maketitle

\begin{abstract}
本稿では,軽量な畳み込みニューラルネットワーク(CNN)と Vision Transformer(ViT)の画像分類性能を,CIFAR-10 を対象に公平に比較する実験プロトコルを提示する。学習率スケジューリング,データ拡張,Test-Time Augmentation (TTA) を統一的に適用し,再現性を重視した評価を実施した。実験の結果,ViT-Ti は TTA により最大で +1.8pt の精度向上を得た一方,ResNet-18 では RandAugment を中心としたデータ拡張で +2.1pt の改善が確認された。これらの知見を通じて,限られた計算資源でも堅牢なベースラインを構築するための指針を示す。
\end{abstract}

\begin{IEEEkeywords}
画像分類, 深層学習, Vision Transformer, Test-Time Augmentation, CIFAR-10
\end{IEEEkeywords}

\input{01_introduction/main}
\input{02_related_work/main}
\input{03_method/main}
\input{04_results/main}
\input{05_discussion/main}
\input{06_conclusion/main}
\input{appendix/main}

\bibliographystyle{ieeetr}
\bibliography{references}

\end{document}

% !TEX program = xelatex
\documentclass[conference]{IEEEtran}

% 日本語出力と XeLaTeX 対応設定
\usepackage{xeCJK}
\usepackage{fontspec}
\setCJKmainfont{HaranoAjiMincho}

% 数式・図表・参考文献などの標準パッケージ
\usepackage{amsmath, amssymb}
\usepackage{graphicx}
\usepackage{url}
\usepackage{hyperref}

\hypersetup{
  colorlinks=true,
  linkcolor=blue,
  citecolor=blue,
  urlcolor=blue
}

\begin{document}

\title{軽量 CNN と Vision Transformer の公平比較に向けた実験的評価}

\author{\IEEEauthorblockN{著者 太郎\\ }
\IEEEauthorblockA{所属機関\\ 連絡先: author@example.com}}

\maketitle

\begin{abstract}
本稿では,軽量な畳み込みニューラルネットワーク(CNN)と Vision Transformer(ViT)の画像分類性能を,CIFAR-10 を対象に公平に比較する実験プロトコルを提示する。学習率スケジューリング,データ拡張,Test-Time Augmentation (TTA) を統一的に適用し,再現性を重視した評価を実施した。実験の結果,ViT-Ti は TTA により最大で +1.8pt の精度向上を得た一方,ResNet-18 では RandAugment を中心としたデータ拡張で +2.1pt の改善が確認された。これらの知見を通じて,限られた計算資源でも堅牢なベースラインを構築するための指針を示す。
\end{abstract}

\begin{IEEEkeywords}
画像分類, 深層学習, Vision Transformer, Test-Time Augmentation, CIFAR-10
\end{IEEEkeywords}

\input{01_introduction/main}
\input{02_related_work/main}
\input{03_method/main}
\input{04_results/main}
\input{05_discussion/main}
\input{06_conclusion/main}
\input{appendix/main}

\bibliographystyle{ieeetr}
\bibliography{references}

\end{document}

% !TEX program = xelatex
\documentclass[conference]{IEEEtran}

% 日本語出力と XeLaTeX 対応設定
\usepackage{xeCJK}
\usepackage{fontspec}
\setCJKmainfont{HaranoAjiMincho}

% 数式・図表・参考文献などの標準パッケージ
\usepackage{amsmath, amssymb}
\usepackage{graphicx}
\usepackage{url}
\usepackage{hyperref}

\hypersetup{
  colorlinks=true,
  linkcolor=blue,
  citecolor=blue,
  urlcolor=blue
}

\begin{document}

\title{軽量 CNN と Vision Transformer の公平比較に向けた実験的評価}

\author{\IEEEauthorblockN{著者 太郎\\ }
\IEEEauthorblockA{所属機関\\ 連絡先: author@example.com}}

\maketitle

\begin{abstract}
本稿では,軽量な畳み込みニューラルネットワーク(CNN)と Vision Transformer(ViT)の画像分類性能を,CIFAR-10 を対象に公平に比較する実験プロトコルを提示する。学習率スケジューリング,データ拡張,Test-Time Augmentation (TTA) を統一的に適用し,再現性を重視した評価を実施した。実験の結果,ViT-Ti は TTA により最大で +1.8pt の精度向上を得た一方,ResNet-18 では RandAugment を中心としたデータ拡張で +2.1pt の改善が確認された。これらの知見を通じて,限られた計算資源でも堅牢なベースラインを構築するための指針を示す。
\end{abstract}

\begin{IEEEkeywords}
画像分類, 深層学習, Vision Transformer, Test-Time Augmentation, CIFAR-10
\end{IEEEkeywords}

\input{01_introduction/main}
\input{02_related_work/main}
\input{03_method/main}
\input{04_results/main}
\input{05_discussion/main}
\input{06_conclusion/main}
\input{appendix/main}

\bibliographystyle{ieeetr}
\bibliography{references}

\end{document}

% !TEX program = xelatex
\documentclass[conference]{IEEEtran}

% 日本語出力と XeLaTeX 対応設定
\usepackage{xeCJK}
\usepackage{fontspec}
\setCJKmainfont{HaranoAjiMincho}

% 数式・図表・参考文献などの標準パッケージ
\usepackage{amsmath, amssymb}
\usepackage{graphicx}
\usepackage{url}
\usepackage{hyperref}

\hypersetup{
  colorlinks=true,
  linkcolor=blue,
  citecolor=blue,
  urlcolor=blue
}

\begin{document}

\title{軽量 CNN と Vision Transformer の公平比較に向けた実験的評価}

\author{\IEEEauthorblockN{著者 太郎\\ }
\IEEEauthorblockA{所属機関\\ 連絡先: author@example.com}}

\maketitle

\begin{abstract}
本稿では,軽量な畳み込みニューラルネットワーク(CNN)と Vision Transformer(ViT)の画像分類性能を,CIFAR-10 を対象に公平に比較する実験プロトコルを提示する。学習率スケジューリング,データ拡張,Test-Time Augmentation (TTA) を統一的に適用し,再現性を重視した評価を実施した。実験の結果,ViT-Ti は TTA により最大で +1.8pt の精度向上を得た一方,ResNet-18 では RandAugment を中心としたデータ拡張で +2.1pt の改善が確認された。これらの知見を通じて,限られた計算資源でも堅牢なベースラインを構築するための指針を示す。
\end{abstract}

\begin{IEEEkeywords}
画像分類, 深層学習, Vision Transformer, Test-Time Augmentation, CIFAR-10
\end{IEEEkeywords}

\input{01_introduction/main}
\input{02_related_work/main}
\input{03_method/main}
\input{04_results/main}
\input{05_discussion/main}
\input{06_conclusion/main}
\input{appendix/main}

\bibliographystyle{ieeetr}
\bibliography{references}

\end{document}


\bibliographystyle{ieeetr}
\bibliography{references}

\end{document}

% !TEX program = xelatex
\documentclass[conference]{IEEEtran}

% 日本語出力と XeLaTeX 対応設定
\usepackage{xeCJK}
\usepackage{fontspec}
\setCJKmainfont{HaranoAjiMincho}

% 数式・図表・参考文献などの標準パッケージ
\usepackage{amsmath, amssymb}
\usepackage{graphicx}
\usepackage{url}
\usepackage{hyperref}

\hypersetup{
  colorlinks=true,
  linkcolor=blue,
  citecolor=blue,
  urlcolor=blue
}

\begin{document}

\title{軽量 CNN と Vision Transformer の公平比較に向けた実験的評価}

\author{\IEEEauthorblockN{著者 太郎\\ }
\IEEEauthorblockA{所属機関\\ 連絡先: author@example.com}}

\maketitle

\begin{abstract}
本稿では,軽量な畳み込みニューラルネットワーク(CNN)と Vision Transformer(ViT)の画像分類性能を,CIFAR-10 を対象に公平に比較する実験プロトコルを提示する。学習率スケジューリング,データ拡張,Test-Time Augmentation (TTA) を統一的に適用し,再現性を重視した評価を実施した。実験の結果,ViT-Ti は TTA により最大で +1.8pt の精度向上を得た一方,ResNet-18 では RandAugment を中心としたデータ拡張で +2.1pt の改善が確認された。これらの知見を通じて,限られた計算資源でも堅牢なベースラインを構築するための指針を示す。
\end{abstract}

\begin{IEEEkeywords}
画像分類, 深層学習, Vision Transformer, Test-Time Augmentation, CIFAR-10
\end{IEEEkeywords}

% !TEX program = xelatex
\documentclass[conference]{IEEEtran}

% 日本語出力と XeLaTeX 対応設定
\usepackage{xeCJK}
\usepackage{fontspec}
\setCJKmainfont{HaranoAjiMincho}

% 数式・図表・参考文献などの標準パッケージ
\usepackage{amsmath, amssymb}
\usepackage{graphicx}
\usepackage{url}
\usepackage{hyperref}

\hypersetup{
  colorlinks=true,
  linkcolor=blue,
  citecolor=blue,
  urlcolor=blue
}

\begin{document}

\title{軽量 CNN と Vision Transformer の公平比較に向けた実験的評価}

\author{\IEEEauthorblockN{著者 太郎\\ }
\IEEEauthorblockA{所属機関\\ 連絡先: author@example.com}}

\maketitle

\begin{abstract}
本稿では,軽量な畳み込みニューラルネットワーク(CNN)と Vision Transformer(ViT)の画像分類性能を,CIFAR-10 を対象に公平に比較する実験プロトコルを提示する。学習率スケジューリング,データ拡張,Test-Time Augmentation (TTA) を統一的に適用し,再現性を重視した評価を実施した。実験の結果,ViT-Ti は TTA により最大で +1.8pt の精度向上を得た一方,ResNet-18 では RandAugment を中心としたデータ拡張で +2.1pt の改善が確認された。これらの知見を通じて,限られた計算資源でも堅牢なベースラインを構築するための指針を示す。
\end{abstract}

\begin{IEEEkeywords}
画像分類, 深層学習, Vision Transformer, Test-Time Augmentation, CIFAR-10
\end{IEEEkeywords}

\input{01_introduction/main}
\input{02_related_work/main}
\input{03_method/main}
\input{04_results/main}
\input{05_discussion/main}
\input{06_conclusion/main}
\input{appendix/main}

\bibliographystyle{ieeetr}
\bibliography{references}

\end{document}

% !TEX program = xelatex
\documentclass[conference]{IEEEtran}

% 日本語出力と XeLaTeX 対応設定
\usepackage{xeCJK}
\usepackage{fontspec}
\setCJKmainfont{HaranoAjiMincho}

% 数式・図表・参考文献などの標準パッケージ
\usepackage{amsmath, amssymb}
\usepackage{graphicx}
\usepackage{url}
\usepackage{hyperref}

\hypersetup{
  colorlinks=true,
  linkcolor=blue,
  citecolor=blue,
  urlcolor=blue
}

\begin{document}

\title{軽量 CNN と Vision Transformer の公平比較に向けた実験的評価}

\author{\IEEEauthorblockN{著者 太郎\\ }
\IEEEauthorblockA{所属機関\\ 連絡先: author@example.com}}

\maketitle

\begin{abstract}
本稿では,軽量な畳み込みニューラルネットワーク(CNN)と Vision Transformer(ViT)の画像分類性能を,CIFAR-10 を対象に公平に比較する実験プロトコルを提示する。学習率スケジューリング,データ拡張,Test-Time Augmentation (TTA) を統一的に適用し,再現性を重視した評価を実施した。実験の結果,ViT-Ti は TTA により最大で +1.8pt の精度向上を得た一方,ResNet-18 では RandAugment を中心としたデータ拡張で +2.1pt の改善が確認された。これらの知見を通じて,限られた計算資源でも堅牢なベースラインを構築するための指針を示す。
\end{abstract}

\begin{IEEEkeywords}
画像分類, 深層学習, Vision Transformer, Test-Time Augmentation, CIFAR-10
\end{IEEEkeywords}

\input{01_introduction/main}
\input{02_related_work/main}
\input{03_method/main}
\input{04_results/main}
\input{05_discussion/main}
\input{06_conclusion/main}
\input{appendix/main}

\bibliographystyle{ieeetr}
\bibliography{references}

\end{document}

% !TEX program = xelatex
\documentclass[conference]{IEEEtran}

% 日本語出力と XeLaTeX 対応設定
\usepackage{xeCJK}
\usepackage{fontspec}
\setCJKmainfont{HaranoAjiMincho}

% 数式・図表・参考文献などの標準パッケージ
\usepackage{amsmath, amssymb}
\usepackage{graphicx}
\usepackage{url}
\usepackage{hyperref}

\hypersetup{
  colorlinks=true,
  linkcolor=blue,
  citecolor=blue,
  urlcolor=blue
}

\begin{document}

\title{軽量 CNN と Vision Transformer の公平比較に向けた実験的評価}

\author{\IEEEauthorblockN{著者 太郎\\ }
\IEEEauthorblockA{所属機関\\ 連絡先: author@example.com}}

\maketitle

\begin{abstract}
本稿では,軽量な畳み込みニューラルネットワーク(CNN)と Vision Transformer(ViT)の画像分類性能を,CIFAR-10 を対象に公平に比較する実験プロトコルを提示する。学習率スケジューリング,データ拡張,Test-Time Augmentation (TTA) を統一的に適用し,再現性を重視した評価を実施した。実験の結果,ViT-Ti は TTA により最大で +1.8pt の精度向上を得た一方,ResNet-18 では RandAugment を中心としたデータ拡張で +2.1pt の改善が確認された。これらの知見を通じて,限られた計算資源でも堅牢なベースラインを構築するための指針を示す。
\end{abstract}

\begin{IEEEkeywords}
画像分類, 深層学習, Vision Transformer, Test-Time Augmentation, CIFAR-10
\end{IEEEkeywords}

\input{01_introduction/main}
\input{02_related_work/main}
\input{03_method/main}
\input{04_results/main}
\input{05_discussion/main}
\input{06_conclusion/main}
\input{appendix/main}

\bibliographystyle{ieeetr}
\bibliography{references}

\end{document}

% !TEX program = xelatex
\documentclass[conference]{IEEEtran}

% 日本語出力と XeLaTeX 対応設定
\usepackage{xeCJK}
\usepackage{fontspec}
\setCJKmainfont{HaranoAjiMincho}

% 数式・図表・参考文献などの標準パッケージ
\usepackage{amsmath, amssymb}
\usepackage{graphicx}
\usepackage{url}
\usepackage{hyperref}

\hypersetup{
  colorlinks=true,
  linkcolor=blue,
  citecolor=blue,
  urlcolor=blue
}

\begin{document}

\title{軽量 CNN と Vision Transformer の公平比較に向けた実験的評価}

\author{\IEEEauthorblockN{著者 太郎\\ }
\IEEEauthorblockA{所属機関\\ 連絡先: author@example.com}}

\maketitle

\begin{abstract}
本稿では,軽量な畳み込みニューラルネットワーク(CNN)と Vision Transformer(ViT)の画像分類性能を,CIFAR-10 を対象に公平に比較する実験プロトコルを提示する。学習率スケジューリング,データ拡張,Test-Time Augmentation (TTA) を統一的に適用し,再現性を重視した評価を実施した。実験の結果,ViT-Ti は TTA により最大で +1.8pt の精度向上を得た一方,ResNet-18 では RandAugment を中心としたデータ拡張で +2.1pt の改善が確認された。これらの知見を通じて,限られた計算資源でも堅牢なベースラインを構築するための指針を示す。
\end{abstract}

\begin{IEEEkeywords}
画像分類, 深層学習, Vision Transformer, Test-Time Augmentation, CIFAR-10
\end{IEEEkeywords}

\input{01_introduction/main}
\input{02_related_work/main}
\input{03_method/main}
\input{04_results/main}
\input{05_discussion/main}
\input{06_conclusion/main}
\input{appendix/main}

\bibliographystyle{ieeetr}
\bibliography{references}

\end{document}

% !TEX program = xelatex
\documentclass[conference]{IEEEtran}

% 日本語出力と XeLaTeX 対応設定
\usepackage{xeCJK}
\usepackage{fontspec}
\setCJKmainfont{HaranoAjiMincho}

% 数式・図表・参考文献などの標準パッケージ
\usepackage{amsmath, amssymb}
\usepackage{graphicx}
\usepackage{url}
\usepackage{hyperref}

\hypersetup{
  colorlinks=true,
  linkcolor=blue,
  citecolor=blue,
  urlcolor=blue
}

\begin{document}

\title{軽量 CNN と Vision Transformer の公平比較に向けた実験的評価}

\author{\IEEEauthorblockN{著者 太郎\\ }
\IEEEauthorblockA{所属機関\\ 連絡先: author@example.com}}

\maketitle

\begin{abstract}
本稿では,軽量な畳み込みニューラルネットワーク(CNN)と Vision Transformer(ViT)の画像分類性能を,CIFAR-10 を対象に公平に比較する実験プロトコルを提示する。学習率スケジューリング,データ拡張,Test-Time Augmentation (TTA) を統一的に適用し,再現性を重視した評価を実施した。実験の結果,ViT-Ti は TTA により最大で +1.8pt の精度向上を得た一方,ResNet-18 では RandAugment を中心としたデータ拡張で +2.1pt の改善が確認された。これらの知見を通じて,限られた計算資源でも堅牢なベースラインを構築するための指針を示す。
\end{abstract}

\begin{IEEEkeywords}
画像分類, 深層学習, Vision Transformer, Test-Time Augmentation, CIFAR-10
\end{IEEEkeywords}

\input{01_introduction/main}
\input{02_related_work/main}
\input{03_method/main}
\input{04_results/main}
\input{05_discussion/main}
\input{06_conclusion/main}
\input{appendix/main}

\bibliographystyle{ieeetr}
\bibliography{references}

\end{document}

% !TEX program = xelatex
\documentclass[conference]{IEEEtran}

% 日本語出力と XeLaTeX 対応設定
\usepackage{xeCJK}
\usepackage{fontspec}
\setCJKmainfont{HaranoAjiMincho}

% 数式・図表・参考文献などの標準パッケージ
\usepackage{amsmath, amssymb}
\usepackage{graphicx}
\usepackage{url}
\usepackage{hyperref}

\hypersetup{
  colorlinks=true,
  linkcolor=blue,
  citecolor=blue,
  urlcolor=blue
}

\begin{document}

\title{軽量 CNN と Vision Transformer の公平比較に向けた実験的評価}

\author{\IEEEauthorblockN{著者 太郎\\ }
\IEEEauthorblockA{所属機関\\ 連絡先: author@example.com}}

\maketitle

\begin{abstract}
本稿では,軽量な畳み込みニューラルネットワーク(CNN)と Vision Transformer(ViT)の画像分類性能を,CIFAR-10 を対象に公平に比較する実験プロトコルを提示する。学習率スケジューリング,データ拡張,Test-Time Augmentation (TTA) を統一的に適用し,再現性を重視した評価を実施した。実験の結果,ViT-Ti は TTA により最大で +1.8pt の精度向上を得た一方,ResNet-18 では RandAugment を中心としたデータ拡張で +2.1pt の改善が確認された。これらの知見を通じて,限られた計算資源でも堅牢なベースラインを構築するための指針を示す。
\end{abstract}

\begin{IEEEkeywords}
画像分類, 深層学習, Vision Transformer, Test-Time Augmentation, CIFAR-10
\end{IEEEkeywords}

\input{01_introduction/main}
\input{02_related_work/main}
\input{03_method/main}
\input{04_results/main}
\input{05_discussion/main}
\input{06_conclusion/main}
\input{appendix/main}

\bibliographystyle{ieeetr}
\bibliography{references}

\end{document}

% !TEX program = xelatex
\documentclass[conference]{IEEEtran}

% 日本語出力と XeLaTeX 対応設定
\usepackage{xeCJK}
\usepackage{fontspec}
\setCJKmainfont{HaranoAjiMincho}

% 数式・図表・参考文献などの標準パッケージ
\usepackage{amsmath, amssymb}
\usepackage{graphicx}
\usepackage{url}
\usepackage{hyperref}

\hypersetup{
  colorlinks=true,
  linkcolor=blue,
  citecolor=blue,
  urlcolor=blue
}

\begin{document}

\title{軽量 CNN と Vision Transformer の公平比較に向けた実験的評価}

\author{\IEEEauthorblockN{著者 太郎\\ }
\IEEEauthorblockA{所属機関\\ 連絡先: author@example.com}}

\maketitle

\begin{abstract}
本稿では,軽量な畳み込みニューラルネットワーク(CNN)と Vision Transformer(ViT)の画像分類性能を,CIFAR-10 を対象に公平に比較する実験プロトコルを提示する。学習率スケジューリング,データ拡張,Test-Time Augmentation (TTA) を統一的に適用し,再現性を重視した評価を実施した。実験の結果,ViT-Ti は TTA により最大で +1.8pt の精度向上を得た一方,ResNet-18 では RandAugment を中心としたデータ拡張で +2.1pt の改善が確認された。これらの知見を通じて,限られた計算資源でも堅牢なベースラインを構築するための指針を示す。
\end{abstract}

\begin{IEEEkeywords}
画像分類, 深層学習, Vision Transformer, Test-Time Augmentation, CIFAR-10
\end{IEEEkeywords}

\input{01_introduction/main}
\input{02_related_work/main}
\input{03_method/main}
\input{04_results/main}
\input{05_discussion/main}
\input{06_conclusion/main}
\input{appendix/main}

\bibliographystyle{ieeetr}
\bibliography{references}

\end{document}


\bibliographystyle{ieeetr}
\bibliography{references}

\end{document}

% !TEX program = xelatex
\documentclass[conference]{IEEEtran}

% 日本語出力と XeLaTeX 対応設定
\usepackage{xeCJK}
\usepackage{fontspec}
\setCJKmainfont{HaranoAjiMincho}

% 数式・図表・参考文献などの標準パッケージ
\usepackage{amsmath, amssymb}
\usepackage{graphicx}
\usepackage{url}
\usepackage{hyperref}

\hypersetup{
  colorlinks=true,
  linkcolor=blue,
  citecolor=blue,
  urlcolor=blue
}

\begin{document}

\title{軽量 CNN と Vision Transformer の公平比較に向けた実験的評価}

\author{\IEEEauthorblockN{著者 太郎\\ }
\IEEEauthorblockA{所属機関\\ 連絡先: author@example.com}}

\maketitle

\begin{abstract}
本稿では,軽量な畳み込みニューラルネットワーク(CNN)と Vision Transformer(ViT)の画像分類性能を,CIFAR-10 を対象に公平に比較する実験プロトコルを提示する。学習率スケジューリング,データ拡張,Test-Time Augmentation (TTA) を統一的に適用し,再現性を重視した評価を実施した。実験の結果,ViT-Ti は TTA により最大で +1.8pt の精度向上を得た一方,ResNet-18 では RandAugment を中心としたデータ拡張で +2.1pt の改善が確認された。これらの知見を通じて,限られた計算資源でも堅牢なベースラインを構築するための指針を示す。
\end{abstract}

\begin{IEEEkeywords}
画像分類, 深層学習, Vision Transformer, Test-Time Augmentation, CIFAR-10
\end{IEEEkeywords}

% !TEX program = xelatex
\documentclass[conference]{IEEEtran}

% 日本語出力と XeLaTeX 対応設定
\usepackage{xeCJK}
\usepackage{fontspec}
\setCJKmainfont{HaranoAjiMincho}

% 数式・図表・参考文献などの標準パッケージ
\usepackage{amsmath, amssymb}
\usepackage{graphicx}
\usepackage{url}
\usepackage{hyperref}

\hypersetup{
  colorlinks=true,
  linkcolor=blue,
  citecolor=blue,
  urlcolor=blue
}

\begin{document}

\title{軽量 CNN と Vision Transformer の公平比較に向けた実験的評価}

\author{\IEEEauthorblockN{著者 太郎\\ }
\IEEEauthorblockA{所属機関\\ 連絡先: author@example.com}}

\maketitle

\begin{abstract}
本稿では,軽量な畳み込みニューラルネットワーク(CNN)と Vision Transformer(ViT)の画像分類性能を,CIFAR-10 を対象に公平に比較する実験プロトコルを提示する。学習率スケジューリング,データ拡張,Test-Time Augmentation (TTA) を統一的に適用し,再現性を重視した評価を実施した。実験の結果,ViT-Ti は TTA により最大で +1.8pt の精度向上を得た一方,ResNet-18 では RandAugment を中心としたデータ拡張で +2.1pt の改善が確認された。これらの知見を通じて,限られた計算資源でも堅牢なベースラインを構築するための指針を示す。
\end{abstract}

\begin{IEEEkeywords}
画像分類, 深層学習, Vision Transformer, Test-Time Augmentation, CIFAR-10
\end{IEEEkeywords}

\input{01_introduction/main}
\input{02_related_work/main}
\input{03_method/main}
\input{04_results/main}
\input{05_discussion/main}
\input{06_conclusion/main}
\input{appendix/main}

\bibliographystyle{ieeetr}
\bibliography{references}

\end{document}

% !TEX program = xelatex
\documentclass[conference]{IEEEtran}

% 日本語出力と XeLaTeX 対応設定
\usepackage{xeCJK}
\usepackage{fontspec}
\setCJKmainfont{HaranoAjiMincho}

% 数式・図表・参考文献などの標準パッケージ
\usepackage{amsmath, amssymb}
\usepackage{graphicx}
\usepackage{url}
\usepackage{hyperref}

\hypersetup{
  colorlinks=true,
  linkcolor=blue,
  citecolor=blue,
  urlcolor=blue
}

\begin{document}

\title{軽量 CNN と Vision Transformer の公平比較に向けた実験的評価}

\author{\IEEEauthorblockN{著者 太郎\\ }
\IEEEauthorblockA{所属機関\\ 連絡先: author@example.com}}

\maketitle

\begin{abstract}
本稿では,軽量な畳み込みニューラルネットワーク(CNN)と Vision Transformer(ViT)の画像分類性能を,CIFAR-10 を対象に公平に比較する実験プロトコルを提示する。学習率スケジューリング,データ拡張,Test-Time Augmentation (TTA) を統一的に適用し,再現性を重視した評価を実施した。実験の結果,ViT-Ti は TTA により最大で +1.8pt の精度向上を得た一方,ResNet-18 では RandAugment を中心としたデータ拡張で +2.1pt の改善が確認された。これらの知見を通じて,限られた計算資源でも堅牢なベースラインを構築するための指針を示す。
\end{abstract}

\begin{IEEEkeywords}
画像分類, 深層学習, Vision Transformer, Test-Time Augmentation, CIFAR-10
\end{IEEEkeywords}

\input{01_introduction/main}
\input{02_related_work/main}
\input{03_method/main}
\input{04_results/main}
\input{05_discussion/main}
\input{06_conclusion/main}
\input{appendix/main}

\bibliographystyle{ieeetr}
\bibliography{references}

\end{document}

% !TEX program = xelatex
\documentclass[conference]{IEEEtran}

% 日本語出力と XeLaTeX 対応設定
\usepackage{xeCJK}
\usepackage{fontspec}
\setCJKmainfont{HaranoAjiMincho}

% 数式・図表・参考文献などの標準パッケージ
\usepackage{amsmath, amssymb}
\usepackage{graphicx}
\usepackage{url}
\usepackage{hyperref}

\hypersetup{
  colorlinks=true,
  linkcolor=blue,
  citecolor=blue,
  urlcolor=blue
}

\begin{document}

\title{軽量 CNN と Vision Transformer の公平比較に向けた実験的評価}

\author{\IEEEauthorblockN{著者 太郎\\ }
\IEEEauthorblockA{所属機関\\ 連絡先: author@example.com}}

\maketitle

\begin{abstract}
本稿では,軽量な畳み込みニューラルネットワーク(CNN)と Vision Transformer(ViT)の画像分類性能を,CIFAR-10 を対象に公平に比較する実験プロトコルを提示する。学習率スケジューリング,データ拡張,Test-Time Augmentation (TTA) を統一的に適用し,再現性を重視した評価を実施した。実験の結果,ViT-Ti は TTA により最大で +1.8pt の精度向上を得た一方,ResNet-18 では RandAugment を中心としたデータ拡張で +2.1pt の改善が確認された。これらの知見を通じて,限られた計算資源でも堅牢なベースラインを構築するための指針を示す。
\end{abstract}

\begin{IEEEkeywords}
画像分類, 深層学習, Vision Transformer, Test-Time Augmentation, CIFAR-10
\end{IEEEkeywords}

\input{01_introduction/main}
\input{02_related_work/main}
\input{03_method/main}
\input{04_results/main}
\input{05_discussion/main}
\input{06_conclusion/main}
\input{appendix/main}

\bibliographystyle{ieeetr}
\bibliography{references}

\end{document}

% !TEX program = xelatex
\documentclass[conference]{IEEEtran}

% 日本語出力と XeLaTeX 対応設定
\usepackage{xeCJK}
\usepackage{fontspec}
\setCJKmainfont{HaranoAjiMincho}

% 数式・図表・参考文献などの標準パッケージ
\usepackage{amsmath, amssymb}
\usepackage{graphicx}
\usepackage{url}
\usepackage{hyperref}

\hypersetup{
  colorlinks=true,
  linkcolor=blue,
  citecolor=blue,
  urlcolor=blue
}

\begin{document}

\title{軽量 CNN と Vision Transformer の公平比較に向けた実験的評価}

\author{\IEEEauthorblockN{著者 太郎\\ }
\IEEEauthorblockA{所属機関\\ 連絡先: author@example.com}}

\maketitle

\begin{abstract}
本稿では,軽量な畳み込みニューラルネットワーク(CNN)と Vision Transformer(ViT)の画像分類性能を,CIFAR-10 を対象に公平に比較する実験プロトコルを提示する。学習率スケジューリング,データ拡張,Test-Time Augmentation (TTA) を統一的に適用し,再現性を重視した評価を実施した。実験の結果,ViT-Ti は TTA により最大で +1.8pt の精度向上を得た一方,ResNet-18 では RandAugment を中心としたデータ拡張で +2.1pt の改善が確認された。これらの知見を通じて,限られた計算資源でも堅牢なベースラインを構築するための指針を示す。
\end{abstract}

\begin{IEEEkeywords}
画像分類, 深層学習, Vision Transformer, Test-Time Augmentation, CIFAR-10
\end{IEEEkeywords}

\input{01_introduction/main}
\input{02_related_work/main}
\input{03_method/main}
\input{04_results/main}
\input{05_discussion/main}
\input{06_conclusion/main}
\input{appendix/main}

\bibliographystyle{ieeetr}
\bibliography{references}

\end{document}

% !TEX program = xelatex
\documentclass[conference]{IEEEtran}

% 日本語出力と XeLaTeX 対応設定
\usepackage{xeCJK}
\usepackage{fontspec}
\setCJKmainfont{HaranoAjiMincho}

% 数式・図表・参考文献などの標準パッケージ
\usepackage{amsmath, amssymb}
\usepackage{graphicx}
\usepackage{url}
\usepackage{hyperref}

\hypersetup{
  colorlinks=true,
  linkcolor=blue,
  citecolor=blue,
  urlcolor=blue
}

\begin{document}

\title{軽量 CNN と Vision Transformer の公平比較に向けた実験的評価}

\author{\IEEEauthorblockN{著者 太郎\\ }
\IEEEauthorblockA{所属機関\\ 連絡先: author@example.com}}

\maketitle

\begin{abstract}
本稿では,軽量な畳み込みニューラルネットワーク(CNN)と Vision Transformer(ViT)の画像分類性能を,CIFAR-10 を対象に公平に比較する実験プロトコルを提示する。学習率スケジューリング,データ拡張,Test-Time Augmentation (TTA) を統一的に適用し,再現性を重視した評価を実施した。実験の結果,ViT-Ti は TTA により最大で +1.8pt の精度向上を得た一方,ResNet-18 では RandAugment を中心としたデータ拡張で +2.1pt の改善が確認された。これらの知見を通じて,限られた計算資源でも堅牢なベースラインを構築するための指針を示す。
\end{abstract}

\begin{IEEEkeywords}
画像分類, 深層学習, Vision Transformer, Test-Time Augmentation, CIFAR-10
\end{IEEEkeywords}

\input{01_introduction/main}
\input{02_related_work/main}
\input{03_method/main}
\input{04_results/main}
\input{05_discussion/main}
\input{06_conclusion/main}
\input{appendix/main}

\bibliographystyle{ieeetr}
\bibliography{references}

\end{document}

% !TEX program = xelatex
\documentclass[conference]{IEEEtran}

% 日本語出力と XeLaTeX 対応設定
\usepackage{xeCJK}
\usepackage{fontspec}
\setCJKmainfont{HaranoAjiMincho}

% 数式・図表・参考文献などの標準パッケージ
\usepackage{amsmath, amssymb}
\usepackage{graphicx}
\usepackage{url}
\usepackage{hyperref}

\hypersetup{
  colorlinks=true,
  linkcolor=blue,
  citecolor=blue,
  urlcolor=blue
}

\begin{document}

\title{軽量 CNN と Vision Transformer の公平比較に向けた実験的評価}

\author{\IEEEauthorblockN{著者 太郎\\ }
\IEEEauthorblockA{所属機関\\ 連絡先: author@example.com}}

\maketitle

\begin{abstract}
本稿では,軽量な畳み込みニューラルネットワーク(CNN)と Vision Transformer(ViT)の画像分類性能を,CIFAR-10 を対象に公平に比較する実験プロトコルを提示する。学習率スケジューリング,データ拡張,Test-Time Augmentation (TTA) を統一的に適用し,再現性を重視した評価を実施した。実験の結果,ViT-Ti は TTA により最大で +1.8pt の精度向上を得た一方,ResNet-18 では RandAugment を中心としたデータ拡張で +2.1pt の改善が確認された。これらの知見を通じて,限られた計算資源でも堅牢なベースラインを構築するための指針を示す。
\end{abstract}

\begin{IEEEkeywords}
画像分類, 深層学習, Vision Transformer, Test-Time Augmentation, CIFAR-10
\end{IEEEkeywords}

\input{01_introduction/main}
\input{02_related_work/main}
\input{03_method/main}
\input{04_results/main}
\input{05_discussion/main}
\input{06_conclusion/main}
\input{appendix/main}

\bibliographystyle{ieeetr}
\bibliography{references}

\end{document}

% !TEX program = xelatex
\documentclass[conference]{IEEEtran}

% 日本語出力と XeLaTeX 対応設定
\usepackage{xeCJK}
\usepackage{fontspec}
\setCJKmainfont{HaranoAjiMincho}

% 数式・図表・参考文献などの標準パッケージ
\usepackage{amsmath, amssymb}
\usepackage{graphicx}
\usepackage{url}
\usepackage{hyperref}

\hypersetup{
  colorlinks=true,
  linkcolor=blue,
  citecolor=blue,
  urlcolor=blue
}

\begin{document}

\title{軽量 CNN と Vision Transformer の公平比較に向けた実験的評価}

\author{\IEEEauthorblockN{著者 太郎\\ }
\IEEEauthorblockA{所属機関\\ 連絡先: author@example.com}}

\maketitle

\begin{abstract}
本稿では,軽量な畳み込みニューラルネットワーク(CNN)と Vision Transformer(ViT)の画像分類性能を,CIFAR-10 を対象に公平に比較する実験プロトコルを提示する。学習率スケジューリング,データ拡張,Test-Time Augmentation (TTA) を統一的に適用し,再現性を重視した評価を実施した。実験の結果,ViT-Ti は TTA により最大で +1.8pt の精度向上を得た一方,ResNet-18 では RandAugment を中心としたデータ拡張で +2.1pt の改善が確認された。これらの知見を通じて,限られた計算資源でも堅牢なベースラインを構築するための指針を示す。
\end{abstract}

\begin{IEEEkeywords}
画像分類, 深層学習, Vision Transformer, Test-Time Augmentation, CIFAR-10
\end{IEEEkeywords}

\input{01_introduction/main}
\input{02_related_work/main}
\input{03_method/main}
\input{04_results/main}
\input{05_discussion/main}
\input{06_conclusion/main}
\input{appendix/main}

\bibliographystyle{ieeetr}
\bibliography{references}

\end{document}


\bibliographystyle{ieeetr}
\bibliography{references}

\end{document}

% !TEX program = xelatex
\documentclass[conference]{IEEEtran}

% 日本語出力と XeLaTeX 対応設定
\usepackage{xeCJK}
\usepackage{fontspec}
\setCJKmainfont{HaranoAjiMincho}

% 数式・図表・参考文献などの標準パッケージ
\usepackage{amsmath, amssymb}
\usepackage{graphicx}
\usepackage{url}
\usepackage{hyperref}

\hypersetup{
  colorlinks=true,
  linkcolor=blue,
  citecolor=blue,
  urlcolor=blue
}

\begin{document}

\title{軽量 CNN と Vision Transformer の公平比較に向けた実験的評価}

\author{\IEEEauthorblockN{著者 太郎\\ }
\IEEEauthorblockA{所属機関\\ 連絡先: author@example.com}}

\maketitle

\begin{abstract}
本稿では,軽量な畳み込みニューラルネットワーク(CNN)と Vision Transformer(ViT)の画像分類性能を,CIFAR-10 を対象に公平に比較する実験プロトコルを提示する。学習率スケジューリング,データ拡張,Test-Time Augmentation (TTA) を統一的に適用し,再現性を重視した評価を実施した。実験の結果,ViT-Ti は TTA により最大で +1.8pt の精度向上を得た一方,ResNet-18 では RandAugment を中心としたデータ拡張で +2.1pt の改善が確認された。これらの知見を通じて,限られた計算資源でも堅牢なベースラインを構築するための指針を示す。
\end{abstract}

\begin{IEEEkeywords}
画像分類, 深層学習, Vision Transformer, Test-Time Augmentation, CIFAR-10
\end{IEEEkeywords}

% !TEX program = xelatex
\documentclass[conference]{IEEEtran}

% 日本語出力と XeLaTeX 対応設定
\usepackage{xeCJK}
\usepackage{fontspec}
\setCJKmainfont{HaranoAjiMincho}

% 数式・図表・参考文献などの標準パッケージ
\usepackage{amsmath, amssymb}
\usepackage{graphicx}
\usepackage{url}
\usepackage{hyperref}

\hypersetup{
  colorlinks=true,
  linkcolor=blue,
  citecolor=blue,
  urlcolor=blue
}

\begin{document}

\title{軽量 CNN と Vision Transformer の公平比較に向けた実験的評価}

\author{\IEEEauthorblockN{著者 太郎\\ }
\IEEEauthorblockA{所属機関\\ 連絡先: author@example.com}}

\maketitle

\begin{abstract}
本稿では,軽量な畳み込みニューラルネットワーク(CNN)と Vision Transformer(ViT)の画像分類性能を,CIFAR-10 を対象に公平に比較する実験プロトコルを提示する。学習率スケジューリング,データ拡張,Test-Time Augmentation (TTA) を統一的に適用し,再現性を重視した評価を実施した。実験の結果,ViT-Ti は TTA により最大で +1.8pt の精度向上を得た一方,ResNet-18 では RandAugment を中心としたデータ拡張で +2.1pt の改善が確認された。これらの知見を通じて,限られた計算資源でも堅牢なベースラインを構築するための指針を示す。
\end{abstract}

\begin{IEEEkeywords}
画像分類, 深層学習, Vision Transformer, Test-Time Augmentation, CIFAR-10
\end{IEEEkeywords}

\input{01_introduction/main}
\input{02_related_work/main}
\input{03_method/main}
\input{04_results/main}
\input{05_discussion/main}
\input{06_conclusion/main}
\input{appendix/main}

\bibliographystyle{ieeetr}
\bibliography{references}

\end{document}

% !TEX program = xelatex
\documentclass[conference]{IEEEtran}

% 日本語出力と XeLaTeX 対応設定
\usepackage{xeCJK}
\usepackage{fontspec}
\setCJKmainfont{HaranoAjiMincho}

% 数式・図表・参考文献などの標準パッケージ
\usepackage{amsmath, amssymb}
\usepackage{graphicx}
\usepackage{url}
\usepackage{hyperref}

\hypersetup{
  colorlinks=true,
  linkcolor=blue,
  citecolor=blue,
  urlcolor=blue
}

\begin{document}

\title{軽量 CNN と Vision Transformer の公平比較に向けた実験的評価}

\author{\IEEEauthorblockN{著者 太郎\\ }
\IEEEauthorblockA{所属機関\\ 連絡先: author@example.com}}

\maketitle

\begin{abstract}
本稿では,軽量な畳み込みニューラルネットワーク(CNN)と Vision Transformer(ViT)の画像分類性能を,CIFAR-10 を対象に公平に比較する実験プロトコルを提示する。学習率スケジューリング,データ拡張,Test-Time Augmentation (TTA) を統一的に適用し,再現性を重視した評価を実施した。実験の結果,ViT-Ti は TTA により最大で +1.8pt の精度向上を得た一方,ResNet-18 では RandAugment を中心としたデータ拡張で +2.1pt の改善が確認された。これらの知見を通じて,限られた計算資源でも堅牢なベースラインを構築するための指針を示す。
\end{abstract}

\begin{IEEEkeywords}
画像分類, 深層学習, Vision Transformer, Test-Time Augmentation, CIFAR-10
\end{IEEEkeywords}

\input{01_introduction/main}
\input{02_related_work/main}
\input{03_method/main}
\input{04_results/main}
\input{05_discussion/main}
\input{06_conclusion/main}
\input{appendix/main}

\bibliographystyle{ieeetr}
\bibliography{references}

\end{document}

% !TEX program = xelatex
\documentclass[conference]{IEEEtran}

% 日本語出力と XeLaTeX 対応設定
\usepackage{xeCJK}
\usepackage{fontspec}
\setCJKmainfont{HaranoAjiMincho}

% 数式・図表・参考文献などの標準パッケージ
\usepackage{amsmath, amssymb}
\usepackage{graphicx}
\usepackage{url}
\usepackage{hyperref}

\hypersetup{
  colorlinks=true,
  linkcolor=blue,
  citecolor=blue,
  urlcolor=blue
}

\begin{document}

\title{軽量 CNN と Vision Transformer の公平比較に向けた実験的評価}

\author{\IEEEauthorblockN{著者 太郎\\ }
\IEEEauthorblockA{所属機関\\ 連絡先: author@example.com}}

\maketitle

\begin{abstract}
本稿では,軽量な畳み込みニューラルネットワーク(CNN)と Vision Transformer(ViT)の画像分類性能を,CIFAR-10 を対象に公平に比較する実験プロトコルを提示する。学習率スケジューリング,データ拡張,Test-Time Augmentation (TTA) を統一的に適用し,再現性を重視した評価を実施した。実験の結果,ViT-Ti は TTA により最大で +1.8pt の精度向上を得た一方,ResNet-18 では RandAugment を中心としたデータ拡張で +2.1pt の改善が確認された。これらの知見を通じて,限られた計算資源でも堅牢なベースラインを構築するための指針を示す。
\end{abstract}

\begin{IEEEkeywords}
画像分類, 深層学習, Vision Transformer, Test-Time Augmentation, CIFAR-10
\end{IEEEkeywords}

\input{01_introduction/main}
\input{02_related_work/main}
\input{03_method/main}
\input{04_results/main}
\input{05_discussion/main}
\input{06_conclusion/main}
\input{appendix/main}

\bibliographystyle{ieeetr}
\bibliography{references}

\end{document}

% !TEX program = xelatex
\documentclass[conference]{IEEEtran}

% 日本語出力と XeLaTeX 対応設定
\usepackage{xeCJK}
\usepackage{fontspec}
\setCJKmainfont{HaranoAjiMincho}

% 数式・図表・参考文献などの標準パッケージ
\usepackage{amsmath, amssymb}
\usepackage{graphicx}
\usepackage{url}
\usepackage{hyperref}

\hypersetup{
  colorlinks=true,
  linkcolor=blue,
  citecolor=blue,
  urlcolor=blue
}

\begin{document}

\title{軽量 CNN と Vision Transformer の公平比較に向けた実験的評価}

\author{\IEEEauthorblockN{著者 太郎\\ }
\IEEEauthorblockA{所属機関\\ 連絡先: author@example.com}}

\maketitle

\begin{abstract}
本稿では,軽量な畳み込みニューラルネットワーク(CNN)と Vision Transformer(ViT)の画像分類性能を,CIFAR-10 を対象に公平に比較する実験プロトコルを提示する。学習率スケジューリング,データ拡張,Test-Time Augmentation (TTA) を統一的に適用し,再現性を重視した評価を実施した。実験の結果,ViT-Ti は TTA により最大で +1.8pt の精度向上を得た一方,ResNet-18 では RandAugment を中心としたデータ拡張で +2.1pt の改善が確認された。これらの知見を通じて,限られた計算資源でも堅牢なベースラインを構築するための指針を示す。
\end{abstract}

\begin{IEEEkeywords}
画像分類, 深層学習, Vision Transformer, Test-Time Augmentation, CIFAR-10
\end{IEEEkeywords}

\input{01_introduction/main}
\input{02_related_work/main}
\input{03_method/main}
\input{04_results/main}
\input{05_discussion/main}
\input{06_conclusion/main}
\input{appendix/main}

\bibliographystyle{ieeetr}
\bibliography{references}

\end{document}

% !TEX program = xelatex
\documentclass[conference]{IEEEtran}

% 日本語出力と XeLaTeX 対応設定
\usepackage{xeCJK}
\usepackage{fontspec}
\setCJKmainfont{HaranoAjiMincho}

% 数式・図表・参考文献などの標準パッケージ
\usepackage{amsmath, amssymb}
\usepackage{graphicx}
\usepackage{url}
\usepackage{hyperref}

\hypersetup{
  colorlinks=true,
  linkcolor=blue,
  citecolor=blue,
  urlcolor=blue
}

\begin{document}

\title{軽量 CNN と Vision Transformer の公平比較に向けた実験的評価}

\author{\IEEEauthorblockN{著者 太郎\\ }
\IEEEauthorblockA{所属機関\\ 連絡先: author@example.com}}

\maketitle

\begin{abstract}
本稿では,軽量な畳み込みニューラルネットワーク(CNN)と Vision Transformer(ViT)の画像分類性能を,CIFAR-10 を対象に公平に比較する実験プロトコルを提示する。学習率スケジューリング,データ拡張,Test-Time Augmentation (TTA) を統一的に適用し,再現性を重視した評価を実施した。実験の結果,ViT-Ti は TTA により最大で +1.8pt の精度向上を得た一方,ResNet-18 では RandAugment を中心としたデータ拡張で +2.1pt の改善が確認された。これらの知見を通じて,限られた計算資源でも堅牢なベースラインを構築するための指針を示す。
\end{abstract}

\begin{IEEEkeywords}
画像分類, 深層学習, Vision Transformer, Test-Time Augmentation, CIFAR-10
\end{IEEEkeywords}

\input{01_introduction/main}
\input{02_related_work/main}
\input{03_method/main}
\input{04_results/main}
\input{05_discussion/main}
\input{06_conclusion/main}
\input{appendix/main}

\bibliographystyle{ieeetr}
\bibliography{references}

\end{document}

% !TEX program = xelatex
\documentclass[conference]{IEEEtran}

% 日本語出力と XeLaTeX 対応設定
\usepackage{xeCJK}
\usepackage{fontspec}
\setCJKmainfont{HaranoAjiMincho}

% 数式・図表・参考文献などの標準パッケージ
\usepackage{amsmath, amssymb}
\usepackage{graphicx}
\usepackage{url}
\usepackage{hyperref}

\hypersetup{
  colorlinks=true,
  linkcolor=blue,
  citecolor=blue,
  urlcolor=blue
}

\begin{document}

\title{軽量 CNN と Vision Transformer の公平比較に向けた実験的評価}

\author{\IEEEauthorblockN{著者 太郎\\ }
\IEEEauthorblockA{所属機関\\ 連絡先: author@example.com}}

\maketitle

\begin{abstract}
本稿では,軽量な畳み込みニューラルネットワーク(CNN)と Vision Transformer(ViT)の画像分類性能を,CIFAR-10 を対象に公平に比較する実験プロトコルを提示する。学習率スケジューリング,データ拡張,Test-Time Augmentation (TTA) を統一的に適用し,再現性を重視した評価を実施した。実験の結果,ViT-Ti は TTA により最大で +1.8pt の精度向上を得た一方,ResNet-18 では RandAugment を中心としたデータ拡張で +2.1pt の改善が確認された。これらの知見を通じて,限られた計算資源でも堅牢なベースラインを構築するための指針を示す。
\end{abstract}

\begin{IEEEkeywords}
画像分類, 深層学習, Vision Transformer, Test-Time Augmentation, CIFAR-10
\end{IEEEkeywords}

\input{01_introduction/main}
\input{02_related_work/main}
\input{03_method/main}
\input{04_results/main}
\input{05_discussion/main}
\input{06_conclusion/main}
\input{appendix/main}

\bibliographystyle{ieeetr}
\bibliography{references}

\end{document}

% !TEX program = xelatex
\documentclass[conference]{IEEEtran}

% 日本語出力と XeLaTeX 対応設定
\usepackage{xeCJK}
\usepackage{fontspec}
\setCJKmainfont{HaranoAjiMincho}

% 数式・図表・参考文献などの標準パッケージ
\usepackage{amsmath, amssymb}
\usepackage{graphicx}
\usepackage{url}
\usepackage{hyperref}

\hypersetup{
  colorlinks=true,
  linkcolor=blue,
  citecolor=blue,
  urlcolor=blue
}

\begin{document}

\title{軽量 CNN と Vision Transformer の公平比較に向けた実験的評価}

\author{\IEEEauthorblockN{著者 太郎\\ }
\IEEEauthorblockA{所属機関\\ 連絡先: author@example.com}}

\maketitle

\begin{abstract}
本稿では,軽量な畳み込みニューラルネットワーク(CNN)と Vision Transformer(ViT)の画像分類性能を,CIFAR-10 を対象に公平に比較する実験プロトコルを提示する。学習率スケジューリング,データ拡張,Test-Time Augmentation (TTA) を統一的に適用し,再現性を重視した評価を実施した。実験の結果,ViT-Ti は TTA により最大で +1.8pt の精度向上を得た一方,ResNet-18 では RandAugment を中心としたデータ拡張で +2.1pt の改善が確認された。これらの知見を通じて,限られた計算資源でも堅牢なベースラインを構築するための指針を示す。
\end{abstract}

\begin{IEEEkeywords}
画像分類, 深層学習, Vision Transformer, Test-Time Augmentation, CIFAR-10
\end{IEEEkeywords}

\input{01_introduction/main}
\input{02_related_work/main}
\input{03_method/main}
\input{04_results/main}
\input{05_discussion/main}
\input{06_conclusion/main}
\input{appendix/main}

\bibliographystyle{ieeetr}
\bibliography{references}

\end{document}


\bibliographystyle{ieeetr}
\bibliography{references}

\end{document}

% !TEX program = xelatex
\documentclass[conference]{IEEEtran}

% 日本語出力と XeLaTeX 対応設定
\usepackage{xeCJK}
\usepackage{fontspec}
\setCJKmainfont{HaranoAjiMincho}

% 数式・図表・参考文献などの標準パッケージ
\usepackage{amsmath, amssymb}
\usepackage{graphicx}
\usepackage{url}
\usepackage{hyperref}

\hypersetup{
  colorlinks=true,
  linkcolor=blue,
  citecolor=blue,
  urlcolor=blue
}

\begin{document}

\title{軽量 CNN と Vision Transformer の公平比較に向けた実験的評価}

\author{\IEEEauthorblockN{著者 太郎\\ }
\IEEEauthorblockA{所属機関\\ 連絡先: author@example.com}}

\maketitle

\begin{abstract}
本稿では,軽量な畳み込みニューラルネットワーク(CNN)と Vision Transformer(ViT)の画像分類性能を,CIFAR-10 を対象に公平に比較する実験プロトコルを提示する。学習率スケジューリング,データ拡張,Test-Time Augmentation (TTA) を統一的に適用し,再現性を重視した評価を実施した。実験の結果,ViT-Ti は TTA により最大で +1.8pt の精度向上を得た一方,ResNet-18 では RandAugment を中心としたデータ拡張で +2.1pt の改善が確認された。これらの知見を通じて,限られた計算資源でも堅牢なベースラインを構築するための指針を示す。
\end{abstract}

\begin{IEEEkeywords}
画像分類, 深層学習, Vision Transformer, Test-Time Augmentation, CIFAR-10
\end{IEEEkeywords}

% !TEX program = xelatex
\documentclass[conference]{IEEEtran}

% 日本語出力と XeLaTeX 対応設定
\usepackage{xeCJK}
\usepackage{fontspec}
\setCJKmainfont{HaranoAjiMincho}

% 数式・図表・参考文献などの標準パッケージ
\usepackage{amsmath, amssymb}
\usepackage{graphicx}
\usepackage{url}
\usepackage{hyperref}

\hypersetup{
  colorlinks=true,
  linkcolor=blue,
  citecolor=blue,
  urlcolor=blue
}

\begin{document}

\title{軽量 CNN と Vision Transformer の公平比較に向けた実験的評価}

\author{\IEEEauthorblockN{著者 太郎\\ }
\IEEEauthorblockA{所属機関\\ 連絡先: author@example.com}}

\maketitle

\begin{abstract}
本稿では,軽量な畳み込みニューラルネットワーク(CNN)と Vision Transformer(ViT)の画像分類性能を,CIFAR-10 を対象に公平に比較する実験プロトコルを提示する。学習率スケジューリング,データ拡張,Test-Time Augmentation (TTA) を統一的に適用し,再現性を重視した評価を実施した。実験の結果,ViT-Ti は TTA により最大で +1.8pt の精度向上を得た一方,ResNet-18 では RandAugment を中心としたデータ拡張で +2.1pt の改善が確認された。これらの知見を通じて,限られた計算資源でも堅牢なベースラインを構築するための指針を示す。
\end{abstract}

\begin{IEEEkeywords}
画像分類, 深層学習, Vision Transformer, Test-Time Augmentation, CIFAR-10
\end{IEEEkeywords}

\input{01_introduction/main}
\input{02_related_work/main}
\input{03_method/main}
\input{04_results/main}
\input{05_discussion/main}
\input{06_conclusion/main}
\input{appendix/main}

\bibliographystyle{ieeetr}
\bibliography{references}

\end{document}

% !TEX program = xelatex
\documentclass[conference]{IEEEtran}

% 日本語出力と XeLaTeX 対応設定
\usepackage{xeCJK}
\usepackage{fontspec}
\setCJKmainfont{HaranoAjiMincho}

% 数式・図表・参考文献などの標準パッケージ
\usepackage{amsmath, amssymb}
\usepackage{graphicx}
\usepackage{url}
\usepackage{hyperref}

\hypersetup{
  colorlinks=true,
  linkcolor=blue,
  citecolor=blue,
  urlcolor=blue
}

\begin{document}

\title{軽量 CNN と Vision Transformer の公平比較に向けた実験的評価}

\author{\IEEEauthorblockN{著者 太郎\\ }
\IEEEauthorblockA{所属機関\\ 連絡先: author@example.com}}

\maketitle

\begin{abstract}
本稿では,軽量な畳み込みニューラルネットワーク(CNN)と Vision Transformer(ViT)の画像分類性能を,CIFAR-10 を対象に公平に比較する実験プロトコルを提示する。学習率スケジューリング,データ拡張,Test-Time Augmentation (TTA) を統一的に適用し,再現性を重視した評価を実施した。実験の結果,ViT-Ti は TTA により最大で +1.8pt の精度向上を得た一方,ResNet-18 では RandAugment を中心としたデータ拡張で +2.1pt の改善が確認された。これらの知見を通じて,限られた計算資源でも堅牢なベースラインを構築するための指針を示す。
\end{abstract}

\begin{IEEEkeywords}
画像分類, 深層学習, Vision Transformer, Test-Time Augmentation, CIFAR-10
\end{IEEEkeywords}

\input{01_introduction/main}
\input{02_related_work/main}
\input{03_method/main}
\input{04_results/main}
\input{05_discussion/main}
\input{06_conclusion/main}
\input{appendix/main}

\bibliographystyle{ieeetr}
\bibliography{references}

\end{document}

% !TEX program = xelatex
\documentclass[conference]{IEEEtran}

% 日本語出力と XeLaTeX 対応設定
\usepackage{xeCJK}
\usepackage{fontspec}
\setCJKmainfont{HaranoAjiMincho}

% 数式・図表・参考文献などの標準パッケージ
\usepackage{amsmath, amssymb}
\usepackage{graphicx}
\usepackage{url}
\usepackage{hyperref}

\hypersetup{
  colorlinks=true,
  linkcolor=blue,
  citecolor=blue,
  urlcolor=blue
}

\begin{document}

\title{軽量 CNN と Vision Transformer の公平比較に向けた実験的評価}

\author{\IEEEauthorblockN{著者 太郎\\ }
\IEEEauthorblockA{所属機関\\ 連絡先: author@example.com}}

\maketitle

\begin{abstract}
本稿では,軽量な畳み込みニューラルネットワーク(CNN)と Vision Transformer(ViT)の画像分類性能を,CIFAR-10 を対象に公平に比較する実験プロトコルを提示する。学習率スケジューリング,データ拡張,Test-Time Augmentation (TTA) を統一的に適用し,再現性を重視した評価を実施した。実験の結果,ViT-Ti は TTA により最大で +1.8pt の精度向上を得た一方,ResNet-18 では RandAugment を中心としたデータ拡張で +2.1pt の改善が確認された。これらの知見を通じて,限られた計算資源でも堅牢なベースラインを構築するための指針を示す。
\end{abstract}

\begin{IEEEkeywords}
画像分類, 深層学習, Vision Transformer, Test-Time Augmentation, CIFAR-10
\end{IEEEkeywords}

\input{01_introduction/main}
\input{02_related_work/main}
\input{03_method/main}
\input{04_results/main}
\input{05_discussion/main}
\input{06_conclusion/main}
\input{appendix/main}

\bibliographystyle{ieeetr}
\bibliography{references}

\end{document}

% !TEX program = xelatex
\documentclass[conference]{IEEEtran}

% 日本語出力と XeLaTeX 対応設定
\usepackage{xeCJK}
\usepackage{fontspec}
\setCJKmainfont{HaranoAjiMincho}

% 数式・図表・参考文献などの標準パッケージ
\usepackage{amsmath, amssymb}
\usepackage{graphicx}
\usepackage{url}
\usepackage{hyperref}

\hypersetup{
  colorlinks=true,
  linkcolor=blue,
  citecolor=blue,
  urlcolor=blue
}

\begin{document}

\title{軽量 CNN と Vision Transformer の公平比較に向けた実験的評価}

\author{\IEEEauthorblockN{著者 太郎\\ }
\IEEEauthorblockA{所属機関\\ 連絡先: author@example.com}}

\maketitle

\begin{abstract}
本稿では,軽量な畳み込みニューラルネットワーク(CNN)と Vision Transformer(ViT)の画像分類性能を,CIFAR-10 を対象に公平に比較する実験プロトコルを提示する。学習率スケジューリング,データ拡張,Test-Time Augmentation (TTA) を統一的に適用し,再現性を重視した評価を実施した。実験の結果,ViT-Ti は TTA により最大で +1.8pt の精度向上を得た一方,ResNet-18 では RandAugment を中心としたデータ拡張で +2.1pt の改善が確認された。これらの知見を通じて,限られた計算資源でも堅牢なベースラインを構築するための指針を示す。
\end{abstract}

\begin{IEEEkeywords}
画像分類, 深層学習, Vision Transformer, Test-Time Augmentation, CIFAR-10
\end{IEEEkeywords}

\input{01_introduction/main}
\input{02_related_work/main}
\input{03_method/main}
\input{04_results/main}
\input{05_discussion/main}
\input{06_conclusion/main}
\input{appendix/main}

\bibliographystyle{ieeetr}
\bibliography{references}

\end{document}

% !TEX program = xelatex
\documentclass[conference]{IEEEtran}

% 日本語出力と XeLaTeX 対応設定
\usepackage{xeCJK}
\usepackage{fontspec}
\setCJKmainfont{HaranoAjiMincho}

% 数式・図表・参考文献などの標準パッケージ
\usepackage{amsmath, amssymb}
\usepackage{graphicx}
\usepackage{url}
\usepackage{hyperref}

\hypersetup{
  colorlinks=true,
  linkcolor=blue,
  citecolor=blue,
  urlcolor=blue
}

\begin{document}

\title{軽量 CNN と Vision Transformer の公平比較に向けた実験的評価}

\author{\IEEEauthorblockN{著者 太郎\\ }
\IEEEauthorblockA{所属機関\\ 連絡先: author@example.com}}

\maketitle

\begin{abstract}
本稿では,軽量な畳み込みニューラルネットワーク(CNN)と Vision Transformer(ViT)の画像分類性能を,CIFAR-10 を対象に公平に比較する実験プロトコルを提示する。学習率スケジューリング,データ拡張,Test-Time Augmentation (TTA) を統一的に適用し,再現性を重視した評価を実施した。実験の結果,ViT-Ti は TTA により最大で +1.8pt の精度向上を得た一方,ResNet-18 では RandAugment を中心としたデータ拡張で +2.1pt の改善が確認された。これらの知見を通じて,限られた計算資源でも堅牢なベースラインを構築するための指針を示す。
\end{abstract}

\begin{IEEEkeywords}
画像分類, 深層学習, Vision Transformer, Test-Time Augmentation, CIFAR-10
\end{IEEEkeywords}

\input{01_introduction/main}
\input{02_related_work/main}
\input{03_method/main}
\input{04_results/main}
\input{05_discussion/main}
\input{06_conclusion/main}
\input{appendix/main}

\bibliographystyle{ieeetr}
\bibliography{references}

\end{document}

% !TEX program = xelatex
\documentclass[conference]{IEEEtran}

% 日本語出力と XeLaTeX 対応設定
\usepackage{xeCJK}
\usepackage{fontspec}
\setCJKmainfont{HaranoAjiMincho}

% 数式・図表・参考文献などの標準パッケージ
\usepackage{amsmath, amssymb}
\usepackage{graphicx}
\usepackage{url}
\usepackage{hyperref}

\hypersetup{
  colorlinks=true,
  linkcolor=blue,
  citecolor=blue,
  urlcolor=blue
}

\begin{document}

\title{軽量 CNN と Vision Transformer の公平比較に向けた実験的評価}

\author{\IEEEauthorblockN{著者 太郎\\ }
\IEEEauthorblockA{所属機関\\ 連絡先: author@example.com}}

\maketitle

\begin{abstract}
本稿では,軽量な畳み込みニューラルネットワーク(CNN)と Vision Transformer(ViT)の画像分類性能を,CIFAR-10 を対象に公平に比較する実験プロトコルを提示する。学習率スケジューリング,データ拡張,Test-Time Augmentation (TTA) を統一的に適用し,再現性を重視した評価を実施した。実験の結果,ViT-Ti は TTA により最大で +1.8pt の精度向上を得た一方,ResNet-18 では RandAugment を中心としたデータ拡張で +2.1pt の改善が確認された。これらの知見を通じて,限られた計算資源でも堅牢なベースラインを構築するための指針を示す。
\end{abstract}

\begin{IEEEkeywords}
画像分類, 深層学習, Vision Transformer, Test-Time Augmentation, CIFAR-10
\end{IEEEkeywords}

\input{01_introduction/main}
\input{02_related_work/main}
\input{03_method/main}
\input{04_results/main}
\input{05_discussion/main}
\input{06_conclusion/main}
\input{appendix/main}

\bibliographystyle{ieeetr}
\bibliography{references}

\end{document}

% !TEX program = xelatex
\documentclass[conference]{IEEEtran}

% 日本語出力と XeLaTeX 対応設定
\usepackage{xeCJK}
\usepackage{fontspec}
\setCJKmainfont{HaranoAjiMincho}

% 数式・図表・参考文献などの標準パッケージ
\usepackage{amsmath, amssymb}
\usepackage{graphicx}
\usepackage{url}
\usepackage{hyperref}

\hypersetup{
  colorlinks=true,
  linkcolor=blue,
  citecolor=blue,
  urlcolor=blue
}

\begin{document}

\title{軽量 CNN と Vision Transformer の公平比較に向けた実験的評価}

\author{\IEEEauthorblockN{著者 太郎\\ }
\IEEEauthorblockA{所属機関\\ 連絡先: author@example.com}}

\maketitle

\begin{abstract}
本稿では,軽量な畳み込みニューラルネットワーク(CNN)と Vision Transformer(ViT)の画像分類性能を,CIFAR-10 を対象に公平に比較する実験プロトコルを提示する。学習率スケジューリング,データ拡張,Test-Time Augmentation (TTA) を統一的に適用し,再現性を重視した評価を実施した。実験の結果,ViT-Ti は TTA により最大で +1.8pt の精度向上を得た一方,ResNet-18 では RandAugment を中心としたデータ拡張で +2.1pt の改善が確認された。これらの知見を通じて,限られた計算資源でも堅牢なベースラインを構築するための指針を示す。
\end{abstract}

\begin{IEEEkeywords}
画像分類, 深層学習, Vision Transformer, Test-Time Augmentation, CIFAR-10
\end{IEEEkeywords}

\input{01_introduction/main}
\input{02_related_work/main}
\input{03_method/main}
\input{04_results/main}
\input{05_discussion/main}
\input{06_conclusion/main}
\input{appendix/main}

\bibliographystyle{ieeetr}
\bibliography{references}

\end{document}


\bibliographystyle{ieeetr}
\bibliography{references}

\end{document}


\bibliographystyle{ieeetr}
\bibliography{references}

\end{document}

% !TEX program = xelatex
\documentclass[conference]{IEEEtran}

% 日本語出力と XeLaTeX 対応設定
\usepackage{xeCJK}
\usepackage{fontspec}
\setCJKmainfont{HaranoAjiMincho}

% 数式・図表・参考文献などの標準パッケージ
\usepackage{amsmath, amssymb}
\usepackage{graphicx}
\usepackage{url}
\usepackage{hyperref}

\hypersetup{
  colorlinks=true,
  linkcolor=blue,
  citecolor=blue,
  urlcolor=blue
}

\begin{document}

\title{軽量 CNN と Vision Transformer の公平比較に向けた実験的評価}

\author{\IEEEauthorblockN{著者 太郎\\ }
\IEEEauthorblockA{所属機関\\ 連絡先: author@example.com}}

\maketitle

\begin{abstract}
本稿では,軽量な畳み込みニューラルネットワーク(CNN)と Vision Transformer(ViT)の画像分類性能を,CIFAR-10 を対象に公平に比較する実験プロトコルを提示する。学習率スケジューリング,データ拡張,Test-Time Augmentation (TTA) を統一的に適用し,再現性を重視した評価を実施した。実験の結果,ViT-Ti は TTA により最大で +1.8pt の精度向上を得た一方,ResNet-18 では RandAugment を中心としたデータ拡張で +2.1pt の改善が確認された。これらの知見を通じて,限られた計算資源でも堅牢なベースラインを構築するための指針を示す。
\end{abstract}

\begin{IEEEkeywords}
画像分類, 深層学習, Vision Transformer, Test-Time Augmentation, CIFAR-10
\end{IEEEkeywords}

% !TEX program = xelatex
\documentclass[conference]{IEEEtran}

% 日本語出力と XeLaTeX 対応設定
\usepackage{xeCJK}
\usepackage{fontspec}
\setCJKmainfont{HaranoAjiMincho}

% 数式・図表・参考文献などの標準パッケージ
\usepackage{amsmath, amssymb}
\usepackage{graphicx}
\usepackage{url}
\usepackage{hyperref}

\hypersetup{
  colorlinks=true,
  linkcolor=blue,
  citecolor=blue,
  urlcolor=blue
}

\begin{document}

\title{軽量 CNN と Vision Transformer の公平比較に向けた実験的評価}

\author{\IEEEauthorblockN{著者 太郎\\ }
\IEEEauthorblockA{所属機関\\ 連絡先: author@example.com}}

\maketitle

\begin{abstract}
本稿では,軽量な畳み込みニューラルネットワーク(CNN)と Vision Transformer(ViT)の画像分類性能を,CIFAR-10 を対象に公平に比較する実験プロトコルを提示する。学習率スケジューリング,データ拡張,Test-Time Augmentation (TTA) を統一的に適用し,再現性を重視した評価を実施した。実験の結果,ViT-Ti は TTA により最大で +1.8pt の精度向上を得た一方,ResNet-18 では RandAugment を中心としたデータ拡張で +2.1pt の改善が確認された。これらの知見を通じて,限られた計算資源でも堅牢なベースラインを構築するための指針を示す。
\end{abstract}

\begin{IEEEkeywords}
画像分類, 深層学習, Vision Transformer, Test-Time Augmentation, CIFAR-10
\end{IEEEkeywords}

% !TEX program = xelatex
\documentclass[conference]{IEEEtran}

% 日本語出力と XeLaTeX 対応設定
\usepackage{xeCJK}
\usepackage{fontspec}
\setCJKmainfont{HaranoAjiMincho}

% 数式・図表・参考文献などの標準パッケージ
\usepackage{amsmath, amssymb}
\usepackage{graphicx}
\usepackage{url}
\usepackage{hyperref}

\hypersetup{
  colorlinks=true,
  linkcolor=blue,
  citecolor=blue,
  urlcolor=blue
}

\begin{document}

\title{軽量 CNN と Vision Transformer の公平比較に向けた実験的評価}

\author{\IEEEauthorblockN{著者 太郎\\ }
\IEEEauthorblockA{所属機関\\ 連絡先: author@example.com}}

\maketitle

\begin{abstract}
本稿では,軽量な畳み込みニューラルネットワーク(CNN)と Vision Transformer(ViT)の画像分類性能を,CIFAR-10 を対象に公平に比較する実験プロトコルを提示する。学習率スケジューリング,データ拡張,Test-Time Augmentation (TTA) を統一的に適用し,再現性を重視した評価を実施した。実験の結果,ViT-Ti は TTA により最大で +1.8pt の精度向上を得た一方,ResNet-18 では RandAugment を中心としたデータ拡張で +2.1pt の改善が確認された。これらの知見を通じて,限られた計算資源でも堅牢なベースラインを構築するための指針を示す。
\end{abstract}

\begin{IEEEkeywords}
画像分類, 深層学習, Vision Transformer, Test-Time Augmentation, CIFAR-10
\end{IEEEkeywords}

\input{01_introduction/main}
\input{02_related_work/main}
\input{03_method/main}
\input{04_results/main}
\input{05_discussion/main}
\input{06_conclusion/main}
\input{appendix/main}

\bibliographystyle{ieeetr}
\bibliography{references}

\end{document}

% !TEX program = xelatex
\documentclass[conference]{IEEEtran}

% 日本語出力と XeLaTeX 対応設定
\usepackage{xeCJK}
\usepackage{fontspec}
\setCJKmainfont{HaranoAjiMincho}

% 数式・図表・参考文献などの標準パッケージ
\usepackage{amsmath, amssymb}
\usepackage{graphicx}
\usepackage{url}
\usepackage{hyperref}

\hypersetup{
  colorlinks=true,
  linkcolor=blue,
  citecolor=blue,
  urlcolor=blue
}

\begin{document}

\title{軽量 CNN と Vision Transformer の公平比較に向けた実験的評価}

\author{\IEEEauthorblockN{著者 太郎\\ }
\IEEEauthorblockA{所属機関\\ 連絡先: author@example.com}}

\maketitle

\begin{abstract}
本稿では,軽量な畳み込みニューラルネットワーク(CNN)と Vision Transformer(ViT)の画像分類性能を,CIFAR-10 を対象に公平に比較する実験プロトコルを提示する。学習率スケジューリング,データ拡張,Test-Time Augmentation (TTA) を統一的に適用し,再現性を重視した評価を実施した。実験の結果,ViT-Ti は TTA により最大で +1.8pt の精度向上を得た一方,ResNet-18 では RandAugment を中心としたデータ拡張で +2.1pt の改善が確認された。これらの知見を通じて,限られた計算資源でも堅牢なベースラインを構築するための指針を示す。
\end{abstract}

\begin{IEEEkeywords}
画像分類, 深層学習, Vision Transformer, Test-Time Augmentation, CIFAR-10
\end{IEEEkeywords}

\input{01_introduction/main}
\input{02_related_work/main}
\input{03_method/main}
\input{04_results/main}
\input{05_discussion/main}
\input{06_conclusion/main}
\input{appendix/main}

\bibliographystyle{ieeetr}
\bibliography{references}

\end{document}

% !TEX program = xelatex
\documentclass[conference]{IEEEtran}

% 日本語出力と XeLaTeX 対応設定
\usepackage{xeCJK}
\usepackage{fontspec}
\setCJKmainfont{HaranoAjiMincho}

% 数式・図表・参考文献などの標準パッケージ
\usepackage{amsmath, amssymb}
\usepackage{graphicx}
\usepackage{url}
\usepackage{hyperref}

\hypersetup{
  colorlinks=true,
  linkcolor=blue,
  citecolor=blue,
  urlcolor=blue
}

\begin{document}

\title{軽量 CNN と Vision Transformer の公平比較に向けた実験的評価}

\author{\IEEEauthorblockN{著者 太郎\\ }
\IEEEauthorblockA{所属機関\\ 連絡先: author@example.com}}

\maketitle

\begin{abstract}
本稿では,軽量な畳み込みニューラルネットワーク(CNN)と Vision Transformer(ViT)の画像分類性能を,CIFAR-10 を対象に公平に比較する実験プロトコルを提示する。学習率スケジューリング,データ拡張,Test-Time Augmentation (TTA) を統一的に適用し,再現性を重視した評価を実施した。実験の結果,ViT-Ti は TTA により最大で +1.8pt の精度向上を得た一方,ResNet-18 では RandAugment を中心としたデータ拡張で +2.1pt の改善が確認された。これらの知見を通じて,限られた計算資源でも堅牢なベースラインを構築するための指針を示す。
\end{abstract}

\begin{IEEEkeywords}
画像分類, 深層学習, Vision Transformer, Test-Time Augmentation, CIFAR-10
\end{IEEEkeywords}

\input{01_introduction/main}
\input{02_related_work/main}
\input{03_method/main}
\input{04_results/main}
\input{05_discussion/main}
\input{06_conclusion/main}
\input{appendix/main}

\bibliographystyle{ieeetr}
\bibliography{references}

\end{document}

% !TEX program = xelatex
\documentclass[conference]{IEEEtran}

% 日本語出力と XeLaTeX 対応設定
\usepackage{xeCJK}
\usepackage{fontspec}
\setCJKmainfont{HaranoAjiMincho}

% 数式・図表・参考文献などの標準パッケージ
\usepackage{amsmath, amssymb}
\usepackage{graphicx}
\usepackage{url}
\usepackage{hyperref}

\hypersetup{
  colorlinks=true,
  linkcolor=blue,
  citecolor=blue,
  urlcolor=blue
}

\begin{document}

\title{軽量 CNN と Vision Transformer の公平比較に向けた実験的評価}

\author{\IEEEauthorblockN{著者 太郎\\ }
\IEEEauthorblockA{所属機関\\ 連絡先: author@example.com}}

\maketitle

\begin{abstract}
本稿では,軽量な畳み込みニューラルネットワーク(CNN)と Vision Transformer(ViT)の画像分類性能を,CIFAR-10 を対象に公平に比較する実験プロトコルを提示する。学習率スケジューリング,データ拡張,Test-Time Augmentation (TTA) を統一的に適用し,再現性を重視した評価を実施した。実験の結果,ViT-Ti は TTA により最大で +1.8pt の精度向上を得た一方,ResNet-18 では RandAugment を中心としたデータ拡張で +2.1pt の改善が確認された。これらの知見を通じて,限られた計算資源でも堅牢なベースラインを構築するための指針を示す。
\end{abstract}

\begin{IEEEkeywords}
画像分類, 深層学習, Vision Transformer, Test-Time Augmentation, CIFAR-10
\end{IEEEkeywords}

\input{01_introduction/main}
\input{02_related_work/main}
\input{03_method/main}
\input{04_results/main}
\input{05_discussion/main}
\input{06_conclusion/main}
\input{appendix/main}

\bibliographystyle{ieeetr}
\bibliography{references}

\end{document}

% !TEX program = xelatex
\documentclass[conference]{IEEEtran}

% 日本語出力と XeLaTeX 対応設定
\usepackage{xeCJK}
\usepackage{fontspec}
\setCJKmainfont{HaranoAjiMincho}

% 数式・図表・参考文献などの標準パッケージ
\usepackage{amsmath, amssymb}
\usepackage{graphicx}
\usepackage{url}
\usepackage{hyperref}

\hypersetup{
  colorlinks=true,
  linkcolor=blue,
  citecolor=blue,
  urlcolor=blue
}

\begin{document}

\title{軽量 CNN と Vision Transformer の公平比較に向けた実験的評価}

\author{\IEEEauthorblockN{著者 太郎\\ }
\IEEEauthorblockA{所属機関\\ 連絡先: author@example.com}}

\maketitle

\begin{abstract}
本稿では,軽量な畳み込みニューラルネットワーク(CNN)と Vision Transformer(ViT)の画像分類性能を,CIFAR-10 を対象に公平に比較する実験プロトコルを提示する。学習率スケジューリング,データ拡張,Test-Time Augmentation (TTA) を統一的に適用し,再現性を重視した評価を実施した。実験の結果,ViT-Ti は TTA により最大で +1.8pt の精度向上を得た一方,ResNet-18 では RandAugment を中心としたデータ拡張で +2.1pt の改善が確認された。これらの知見を通じて,限られた計算資源でも堅牢なベースラインを構築するための指針を示す。
\end{abstract}

\begin{IEEEkeywords}
画像分類, 深層学習, Vision Transformer, Test-Time Augmentation, CIFAR-10
\end{IEEEkeywords}

\input{01_introduction/main}
\input{02_related_work/main}
\input{03_method/main}
\input{04_results/main}
\input{05_discussion/main}
\input{06_conclusion/main}
\input{appendix/main}

\bibliographystyle{ieeetr}
\bibliography{references}

\end{document}

% !TEX program = xelatex
\documentclass[conference]{IEEEtran}

% 日本語出力と XeLaTeX 対応設定
\usepackage{xeCJK}
\usepackage{fontspec}
\setCJKmainfont{HaranoAjiMincho}

% 数式・図表・参考文献などの標準パッケージ
\usepackage{amsmath, amssymb}
\usepackage{graphicx}
\usepackage{url}
\usepackage{hyperref}

\hypersetup{
  colorlinks=true,
  linkcolor=blue,
  citecolor=blue,
  urlcolor=blue
}

\begin{document}

\title{軽量 CNN と Vision Transformer の公平比較に向けた実験的評価}

\author{\IEEEauthorblockN{著者 太郎\\ }
\IEEEauthorblockA{所属機関\\ 連絡先: author@example.com}}

\maketitle

\begin{abstract}
本稿では,軽量な畳み込みニューラルネットワーク(CNN)と Vision Transformer(ViT)の画像分類性能を,CIFAR-10 を対象に公平に比較する実験プロトコルを提示する。学習率スケジューリング,データ拡張,Test-Time Augmentation (TTA) を統一的に適用し,再現性を重視した評価を実施した。実験の結果,ViT-Ti は TTA により最大で +1.8pt の精度向上を得た一方,ResNet-18 では RandAugment を中心としたデータ拡張で +2.1pt の改善が確認された。これらの知見を通じて,限られた計算資源でも堅牢なベースラインを構築するための指針を示す。
\end{abstract}

\begin{IEEEkeywords}
画像分類, 深層学習, Vision Transformer, Test-Time Augmentation, CIFAR-10
\end{IEEEkeywords}

\input{01_introduction/main}
\input{02_related_work/main}
\input{03_method/main}
\input{04_results/main}
\input{05_discussion/main}
\input{06_conclusion/main}
\input{appendix/main}

\bibliographystyle{ieeetr}
\bibliography{references}

\end{document}

% !TEX program = xelatex
\documentclass[conference]{IEEEtran}

% 日本語出力と XeLaTeX 対応設定
\usepackage{xeCJK}
\usepackage{fontspec}
\setCJKmainfont{HaranoAjiMincho}

% 数式・図表・参考文献などの標準パッケージ
\usepackage{amsmath, amssymb}
\usepackage{graphicx}
\usepackage{url}
\usepackage{hyperref}

\hypersetup{
  colorlinks=true,
  linkcolor=blue,
  citecolor=blue,
  urlcolor=blue
}

\begin{document}

\title{軽量 CNN と Vision Transformer の公平比較に向けた実験的評価}

\author{\IEEEauthorblockN{著者 太郎\\ }
\IEEEauthorblockA{所属機関\\ 連絡先: author@example.com}}

\maketitle

\begin{abstract}
本稿では,軽量な畳み込みニューラルネットワーク(CNN)と Vision Transformer(ViT)の画像分類性能を,CIFAR-10 を対象に公平に比較する実験プロトコルを提示する。学習率スケジューリング,データ拡張,Test-Time Augmentation (TTA) を統一的に適用し,再現性を重視した評価を実施した。実験の結果,ViT-Ti は TTA により最大で +1.8pt の精度向上を得た一方,ResNet-18 では RandAugment を中心としたデータ拡張で +2.1pt の改善が確認された。これらの知見を通じて,限られた計算資源でも堅牢なベースラインを構築するための指針を示す。
\end{abstract}

\begin{IEEEkeywords}
画像分類, 深層学習, Vision Transformer, Test-Time Augmentation, CIFAR-10
\end{IEEEkeywords}

\input{01_introduction/main}
\input{02_related_work/main}
\input{03_method/main}
\input{04_results/main}
\input{05_discussion/main}
\input{06_conclusion/main}
\input{appendix/main}

\bibliographystyle{ieeetr}
\bibliography{references}

\end{document}


\bibliographystyle{ieeetr}
\bibliography{references}

\end{document}

% !TEX program = xelatex
\documentclass[conference]{IEEEtran}

% 日本語出力と XeLaTeX 対応設定
\usepackage{xeCJK}
\usepackage{fontspec}
\setCJKmainfont{HaranoAjiMincho}

% 数式・図表・参考文献などの標準パッケージ
\usepackage{amsmath, amssymb}
\usepackage{graphicx}
\usepackage{url}
\usepackage{hyperref}

\hypersetup{
  colorlinks=true,
  linkcolor=blue,
  citecolor=blue,
  urlcolor=blue
}

\begin{document}

\title{軽量 CNN と Vision Transformer の公平比較に向けた実験的評価}

\author{\IEEEauthorblockN{著者 太郎\\ }
\IEEEauthorblockA{所属機関\\ 連絡先: author@example.com}}

\maketitle

\begin{abstract}
本稿では,軽量な畳み込みニューラルネットワーク(CNN)と Vision Transformer(ViT)の画像分類性能を,CIFAR-10 を対象に公平に比較する実験プロトコルを提示する。学習率スケジューリング,データ拡張,Test-Time Augmentation (TTA) を統一的に適用し,再現性を重視した評価を実施した。実験の結果,ViT-Ti は TTA により最大で +1.8pt の精度向上を得た一方,ResNet-18 では RandAugment を中心としたデータ拡張で +2.1pt の改善が確認された。これらの知見を通じて,限られた計算資源でも堅牢なベースラインを構築するための指針を示す。
\end{abstract}

\begin{IEEEkeywords}
画像分類, 深層学習, Vision Transformer, Test-Time Augmentation, CIFAR-10
\end{IEEEkeywords}

% !TEX program = xelatex
\documentclass[conference]{IEEEtran}

% 日本語出力と XeLaTeX 対応設定
\usepackage{xeCJK}
\usepackage{fontspec}
\setCJKmainfont{HaranoAjiMincho}

% 数式・図表・参考文献などの標準パッケージ
\usepackage{amsmath, amssymb}
\usepackage{graphicx}
\usepackage{url}
\usepackage{hyperref}

\hypersetup{
  colorlinks=true,
  linkcolor=blue,
  citecolor=blue,
  urlcolor=blue
}

\begin{document}

\title{軽量 CNN と Vision Transformer の公平比較に向けた実験的評価}

\author{\IEEEauthorblockN{著者 太郎\\ }
\IEEEauthorblockA{所属機関\\ 連絡先: author@example.com}}

\maketitle

\begin{abstract}
本稿では,軽量な畳み込みニューラルネットワーク(CNN)と Vision Transformer(ViT)の画像分類性能を,CIFAR-10 を対象に公平に比較する実験プロトコルを提示する。学習率スケジューリング,データ拡張,Test-Time Augmentation (TTA) を統一的に適用し,再現性を重視した評価を実施した。実験の結果,ViT-Ti は TTA により最大で +1.8pt の精度向上を得た一方,ResNet-18 では RandAugment を中心としたデータ拡張で +2.1pt の改善が確認された。これらの知見を通じて,限られた計算資源でも堅牢なベースラインを構築するための指針を示す。
\end{abstract}

\begin{IEEEkeywords}
画像分類, 深層学習, Vision Transformer, Test-Time Augmentation, CIFAR-10
\end{IEEEkeywords}

\input{01_introduction/main}
\input{02_related_work/main}
\input{03_method/main}
\input{04_results/main}
\input{05_discussion/main}
\input{06_conclusion/main}
\input{appendix/main}

\bibliographystyle{ieeetr}
\bibliography{references}

\end{document}

% !TEX program = xelatex
\documentclass[conference]{IEEEtran}

% 日本語出力と XeLaTeX 対応設定
\usepackage{xeCJK}
\usepackage{fontspec}
\setCJKmainfont{HaranoAjiMincho}

% 数式・図表・参考文献などの標準パッケージ
\usepackage{amsmath, amssymb}
\usepackage{graphicx}
\usepackage{url}
\usepackage{hyperref}

\hypersetup{
  colorlinks=true,
  linkcolor=blue,
  citecolor=blue,
  urlcolor=blue
}

\begin{document}

\title{軽量 CNN と Vision Transformer の公平比較に向けた実験的評価}

\author{\IEEEauthorblockN{著者 太郎\\ }
\IEEEauthorblockA{所属機関\\ 連絡先: author@example.com}}

\maketitle

\begin{abstract}
本稿では,軽量な畳み込みニューラルネットワーク(CNN)と Vision Transformer(ViT)の画像分類性能を,CIFAR-10 を対象に公平に比較する実験プロトコルを提示する。学習率スケジューリング,データ拡張,Test-Time Augmentation (TTA) を統一的に適用し,再現性を重視した評価を実施した。実験の結果,ViT-Ti は TTA により最大で +1.8pt の精度向上を得た一方,ResNet-18 では RandAugment を中心としたデータ拡張で +2.1pt の改善が確認された。これらの知見を通じて,限られた計算資源でも堅牢なベースラインを構築するための指針を示す。
\end{abstract}

\begin{IEEEkeywords}
画像分類, 深層学習, Vision Transformer, Test-Time Augmentation, CIFAR-10
\end{IEEEkeywords}

\input{01_introduction/main}
\input{02_related_work/main}
\input{03_method/main}
\input{04_results/main}
\input{05_discussion/main}
\input{06_conclusion/main}
\input{appendix/main}

\bibliographystyle{ieeetr}
\bibliography{references}

\end{document}

% !TEX program = xelatex
\documentclass[conference]{IEEEtran}

% 日本語出力と XeLaTeX 対応設定
\usepackage{xeCJK}
\usepackage{fontspec}
\setCJKmainfont{HaranoAjiMincho}

% 数式・図表・参考文献などの標準パッケージ
\usepackage{amsmath, amssymb}
\usepackage{graphicx}
\usepackage{url}
\usepackage{hyperref}

\hypersetup{
  colorlinks=true,
  linkcolor=blue,
  citecolor=blue,
  urlcolor=blue
}

\begin{document}

\title{軽量 CNN と Vision Transformer の公平比較に向けた実験的評価}

\author{\IEEEauthorblockN{著者 太郎\\ }
\IEEEauthorblockA{所属機関\\ 連絡先: author@example.com}}

\maketitle

\begin{abstract}
本稿では,軽量な畳み込みニューラルネットワーク(CNN)と Vision Transformer(ViT)の画像分類性能を,CIFAR-10 を対象に公平に比較する実験プロトコルを提示する。学習率スケジューリング,データ拡張,Test-Time Augmentation (TTA) を統一的に適用し,再現性を重視した評価を実施した。実験の結果,ViT-Ti は TTA により最大で +1.8pt の精度向上を得た一方,ResNet-18 では RandAugment を中心としたデータ拡張で +2.1pt の改善が確認された。これらの知見を通じて,限られた計算資源でも堅牢なベースラインを構築するための指針を示す。
\end{abstract}

\begin{IEEEkeywords}
画像分類, 深層学習, Vision Transformer, Test-Time Augmentation, CIFAR-10
\end{IEEEkeywords}

\input{01_introduction/main}
\input{02_related_work/main}
\input{03_method/main}
\input{04_results/main}
\input{05_discussion/main}
\input{06_conclusion/main}
\input{appendix/main}

\bibliographystyle{ieeetr}
\bibliography{references}

\end{document}

% !TEX program = xelatex
\documentclass[conference]{IEEEtran}

% 日本語出力と XeLaTeX 対応設定
\usepackage{xeCJK}
\usepackage{fontspec}
\setCJKmainfont{HaranoAjiMincho}

% 数式・図表・参考文献などの標準パッケージ
\usepackage{amsmath, amssymb}
\usepackage{graphicx}
\usepackage{url}
\usepackage{hyperref}

\hypersetup{
  colorlinks=true,
  linkcolor=blue,
  citecolor=blue,
  urlcolor=blue
}

\begin{document}

\title{軽量 CNN と Vision Transformer の公平比較に向けた実験的評価}

\author{\IEEEauthorblockN{著者 太郎\\ }
\IEEEauthorblockA{所属機関\\ 連絡先: author@example.com}}

\maketitle

\begin{abstract}
本稿では,軽量な畳み込みニューラルネットワーク(CNN)と Vision Transformer(ViT)の画像分類性能を,CIFAR-10 を対象に公平に比較する実験プロトコルを提示する。学習率スケジューリング,データ拡張,Test-Time Augmentation (TTA) を統一的に適用し,再現性を重視した評価を実施した。実験の結果,ViT-Ti は TTA により最大で +1.8pt の精度向上を得た一方,ResNet-18 では RandAugment を中心としたデータ拡張で +2.1pt の改善が確認された。これらの知見を通じて,限られた計算資源でも堅牢なベースラインを構築するための指針を示す。
\end{abstract}

\begin{IEEEkeywords}
画像分類, 深層学習, Vision Transformer, Test-Time Augmentation, CIFAR-10
\end{IEEEkeywords}

\input{01_introduction/main}
\input{02_related_work/main}
\input{03_method/main}
\input{04_results/main}
\input{05_discussion/main}
\input{06_conclusion/main}
\input{appendix/main}

\bibliographystyle{ieeetr}
\bibliography{references}

\end{document}

% !TEX program = xelatex
\documentclass[conference]{IEEEtran}

% 日本語出力と XeLaTeX 対応設定
\usepackage{xeCJK}
\usepackage{fontspec}
\setCJKmainfont{HaranoAjiMincho}

% 数式・図表・参考文献などの標準パッケージ
\usepackage{amsmath, amssymb}
\usepackage{graphicx}
\usepackage{url}
\usepackage{hyperref}

\hypersetup{
  colorlinks=true,
  linkcolor=blue,
  citecolor=blue,
  urlcolor=blue
}

\begin{document}

\title{軽量 CNN と Vision Transformer の公平比較に向けた実験的評価}

\author{\IEEEauthorblockN{著者 太郎\\ }
\IEEEauthorblockA{所属機関\\ 連絡先: author@example.com}}

\maketitle

\begin{abstract}
本稿では,軽量な畳み込みニューラルネットワーク(CNN)と Vision Transformer(ViT)の画像分類性能を,CIFAR-10 を対象に公平に比較する実験プロトコルを提示する。学習率スケジューリング,データ拡張,Test-Time Augmentation (TTA) を統一的に適用し,再現性を重視した評価を実施した。実験の結果,ViT-Ti は TTA により最大で +1.8pt の精度向上を得た一方,ResNet-18 では RandAugment を中心としたデータ拡張で +2.1pt の改善が確認された。これらの知見を通じて,限られた計算資源でも堅牢なベースラインを構築するための指針を示す。
\end{abstract}

\begin{IEEEkeywords}
画像分類, 深層学習, Vision Transformer, Test-Time Augmentation, CIFAR-10
\end{IEEEkeywords}

\input{01_introduction/main}
\input{02_related_work/main}
\input{03_method/main}
\input{04_results/main}
\input{05_discussion/main}
\input{06_conclusion/main}
\input{appendix/main}

\bibliographystyle{ieeetr}
\bibliography{references}

\end{document}

% !TEX program = xelatex
\documentclass[conference]{IEEEtran}

% 日本語出力と XeLaTeX 対応設定
\usepackage{xeCJK}
\usepackage{fontspec}
\setCJKmainfont{HaranoAjiMincho}

% 数式・図表・参考文献などの標準パッケージ
\usepackage{amsmath, amssymb}
\usepackage{graphicx}
\usepackage{url}
\usepackage{hyperref}

\hypersetup{
  colorlinks=true,
  linkcolor=blue,
  citecolor=blue,
  urlcolor=blue
}

\begin{document}

\title{軽量 CNN と Vision Transformer の公平比較に向けた実験的評価}

\author{\IEEEauthorblockN{著者 太郎\\ }
\IEEEauthorblockA{所属機関\\ 連絡先: author@example.com}}

\maketitle

\begin{abstract}
本稿では,軽量な畳み込みニューラルネットワーク(CNN)と Vision Transformer(ViT)の画像分類性能を,CIFAR-10 を対象に公平に比較する実験プロトコルを提示する。学習率スケジューリング,データ拡張,Test-Time Augmentation (TTA) を統一的に適用し,再現性を重視した評価を実施した。実験の結果,ViT-Ti は TTA により最大で +1.8pt の精度向上を得た一方,ResNet-18 では RandAugment を中心としたデータ拡張で +2.1pt の改善が確認された。これらの知見を通じて,限られた計算資源でも堅牢なベースラインを構築するための指針を示す。
\end{abstract}

\begin{IEEEkeywords}
画像分類, 深層学習, Vision Transformer, Test-Time Augmentation, CIFAR-10
\end{IEEEkeywords}

\input{01_introduction/main}
\input{02_related_work/main}
\input{03_method/main}
\input{04_results/main}
\input{05_discussion/main}
\input{06_conclusion/main}
\input{appendix/main}

\bibliographystyle{ieeetr}
\bibliography{references}

\end{document}

% !TEX program = xelatex
\documentclass[conference]{IEEEtran}

% 日本語出力と XeLaTeX 対応設定
\usepackage{xeCJK}
\usepackage{fontspec}
\setCJKmainfont{HaranoAjiMincho}

% 数式・図表・参考文献などの標準パッケージ
\usepackage{amsmath, amssymb}
\usepackage{graphicx}
\usepackage{url}
\usepackage{hyperref}

\hypersetup{
  colorlinks=true,
  linkcolor=blue,
  citecolor=blue,
  urlcolor=blue
}

\begin{document}

\title{軽量 CNN と Vision Transformer の公平比較に向けた実験的評価}

\author{\IEEEauthorblockN{著者 太郎\\ }
\IEEEauthorblockA{所属機関\\ 連絡先: author@example.com}}

\maketitle

\begin{abstract}
本稿では,軽量な畳み込みニューラルネットワーク(CNN)と Vision Transformer(ViT)の画像分類性能を,CIFAR-10 を対象に公平に比較する実験プロトコルを提示する。学習率スケジューリング,データ拡張,Test-Time Augmentation (TTA) を統一的に適用し,再現性を重視した評価を実施した。実験の結果,ViT-Ti は TTA により最大で +1.8pt の精度向上を得た一方,ResNet-18 では RandAugment を中心としたデータ拡張で +2.1pt の改善が確認された。これらの知見を通じて,限られた計算資源でも堅牢なベースラインを構築するための指針を示す。
\end{abstract}

\begin{IEEEkeywords}
画像分類, 深層学習, Vision Transformer, Test-Time Augmentation, CIFAR-10
\end{IEEEkeywords}

\input{01_introduction/main}
\input{02_related_work/main}
\input{03_method/main}
\input{04_results/main}
\input{05_discussion/main}
\input{06_conclusion/main}
\input{appendix/main}

\bibliographystyle{ieeetr}
\bibliography{references}

\end{document}


\bibliographystyle{ieeetr}
\bibliography{references}

\end{document}

% !TEX program = xelatex
\documentclass[conference]{IEEEtran}

% 日本語出力と XeLaTeX 対応設定
\usepackage{xeCJK}
\usepackage{fontspec}
\setCJKmainfont{HaranoAjiMincho}

% 数式・図表・参考文献などの標準パッケージ
\usepackage{amsmath, amssymb}
\usepackage{graphicx}
\usepackage{url}
\usepackage{hyperref}

\hypersetup{
  colorlinks=true,
  linkcolor=blue,
  citecolor=blue,
  urlcolor=blue
}

\begin{document}

\title{軽量 CNN と Vision Transformer の公平比較に向けた実験的評価}

\author{\IEEEauthorblockN{著者 太郎\\ }
\IEEEauthorblockA{所属機関\\ 連絡先: author@example.com}}

\maketitle

\begin{abstract}
本稿では,軽量な畳み込みニューラルネットワーク(CNN)と Vision Transformer(ViT)の画像分類性能を,CIFAR-10 を対象に公平に比較する実験プロトコルを提示する。学習率スケジューリング,データ拡張,Test-Time Augmentation (TTA) を統一的に適用し,再現性を重視した評価を実施した。実験の結果,ViT-Ti は TTA により最大で +1.8pt の精度向上を得た一方,ResNet-18 では RandAugment を中心としたデータ拡張で +2.1pt の改善が確認された。これらの知見を通じて,限られた計算資源でも堅牢なベースラインを構築するための指針を示す。
\end{abstract}

\begin{IEEEkeywords}
画像分類, 深層学習, Vision Transformer, Test-Time Augmentation, CIFAR-10
\end{IEEEkeywords}

% !TEX program = xelatex
\documentclass[conference]{IEEEtran}

% 日本語出力と XeLaTeX 対応設定
\usepackage{xeCJK}
\usepackage{fontspec}
\setCJKmainfont{HaranoAjiMincho}

% 数式・図表・参考文献などの標準パッケージ
\usepackage{amsmath, amssymb}
\usepackage{graphicx}
\usepackage{url}
\usepackage{hyperref}

\hypersetup{
  colorlinks=true,
  linkcolor=blue,
  citecolor=blue,
  urlcolor=blue
}

\begin{document}

\title{軽量 CNN と Vision Transformer の公平比較に向けた実験的評価}

\author{\IEEEauthorblockN{著者 太郎\\ }
\IEEEauthorblockA{所属機関\\ 連絡先: author@example.com}}

\maketitle

\begin{abstract}
本稿では,軽量な畳み込みニューラルネットワーク(CNN)と Vision Transformer(ViT)の画像分類性能を,CIFAR-10 を対象に公平に比較する実験プロトコルを提示する。学習率スケジューリング,データ拡張,Test-Time Augmentation (TTA) を統一的に適用し,再現性を重視した評価を実施した。実験の結果,ViT-Ti は TTA により最大で +1.8pt の精度向上を得た一方,ResNet-18 では RandAugment を中心としたデータ拡張で +2.1pt の改善が確認された。これらの知見を通じて,限られた計算資源でも堅牢なベースラインを構築するための指針を示す。
\end{abstract}

\begin{IEEEkeywords}
画像分類, 深層学習, Vision Transformer, Test-Time Augmentation, CIFAR-10
\end{IEEEkeywords}

\input{01_introduction/main}
\input{02_related_work/main}
\input{03_method/main}
\input{04_results/main}
\input{05_discussion/main}
\input{06_conclusion/main}
\input{appendix/main}

\bibliographystyle{ieeetr}
\bibliography{references}

\end{document}

% !TEX program = xelatex
\documentclass[conference]{IEEEtran}

% 日本語出力と XeLaTeX 対応設定
\usepackage{xeCJK}
\usepackage{fontspec}
\setCJKmainfont{HaranoAjiMincho}

% 数式・図表・参考文献などの標準パッケージ
\usepackage{amsmath, amssymb}
\usepackage{graphicx}
\usepackage{url}
\usepackage{hyperref}

\hypersetup{
  colorlinks=true,
  linkcolor=blue,
  citecolor=blue,
  urlcolor=blue
}

\begin{document}

\title{軽量 CNN と Vision Transformer の公平比較に向けた実験的評価}

\author{\IEEEauthorblockN{著者 太郎\\ }
\IEEEauthorblockA{所属機関\\ 連絡先: author@example.com}}

\maketitle

\begin{abstract}
本稿では,軽量な畳み込みニューラルネットワーク(CNN)と Vision Transformer(ViT)の画像分類性能を,CIFAR-10 を対象に公平に比較する実験プロトコルを提示する。学習率スケジューリング,データ拡張,Test-Time Augmentation (TTA) を統一的に適用し,再現性を重視した評価を実施した。実験の結果,ViT-Ti は TTA により最大で +1.8pt の精度向上を得た一方,ResNet-18 では RandAugment を中心としたデータ拡張で +2.1pt の改善が確認された。これらの知見を通じて,限られた計算資源でも堅牢なベースラインを構築するための指針を示す。
\end{abstract}

\begin{IEEEkeywords}
画像分類, 深層学習, Vision Transformer, Test-Time Augmentation, CIFAR-10
\end{IEEEkeywords}

\input{01_introduction/main}
\input{02_related_work/main}
\input{03_method/main}
\input{04_results/main}
\input{05_discussion/main}
\input{06_conclusion/main}
\input{appendix/main}

\bibliographystyle{ieeetr}
\bibliography{references}

\end{document}

% !TEX program = xelatex
\documentclass[conference]{IEEEtran}

% 日本語出力と XeLaTeX 対応設定
\usepackage{xeCJK}
\usepackage{fontspec}
\setCJKmainfont{HaranoAjiMincho}

% 数式・図表・参考文献などの標準パッケージ
\usepackage{amsmath, amssymb}
\usepackage{graphicx}
\usepackage{url}
\usepackage{hyperref}

\hypersetup{
  colorlinks=true,
  linkcolor=blue,
  citecolor=blue,
  urlcolor=blue
}

\begin{document}

\title{軽量 CNN と Vision Transformer の公平比較に向けた実験的評価}

\author{\IEEEauthorblockN{著者 太郎\\ }
\IEEEauthorblockA{所属機関\\ 連絡先: author@example.com}}

\maketitle

\begin{abstract}
本稿では,軽量な畳み込みニューラルネットワーク(CNN)と Vision Transformer(ViT)の画像分類性能を,CIFAR-10 を対象に公平に比較する実験プロトコルを提示する。学習率スケジューリング,データ拡張,Test-Time Augmentation (TTA) を統一的に適用し,再現性を重視した評価を実施した。実験の結果,ViT-Ti は TTA により最大で +1.8pt の精度向上を得た一方,ResNet-18 では RandAugment を中心としたデータ拡張で +2.1pt の改善が確認された。これらの知見を通じて,限られた計算資源でも堅牢なベースラインを構築するための指針を示す。
\end{abstract}

\begin{IEEEkeywords}
画像分類, 深層学習, Vision Transformer, Test-Time Augmentation, CIFAR-10
\end{IEEEkeywords}

\input{01_introduction/main}
\input{02_related_work/main}
\input{03_method/main}
\input{04_results/main}
\input{05_discussion/main}
\input{06_conclusion/main}
\input{appendix/main}

\bibliographystyle{ieeetr}
\bibliography{references}

\end{document}

% !TEX program = xelatex
\documentclass[conference]{IEEEtran}

% 日本語出力と XeLaTeX 対応設定
\usepackage{xeCJK}
\usepackage{fontspec}
\setCJKmainfont{HaranoAjiMincho}

% 数式・図表・参考文献などの標準パッケージ
\usepackage{amsmath, amssymb}
\usepackage{graphicx}
\usepackage{url}
\usepackage{hyperref}

\hypersetup{
  colorlinks=true,
  linkcolor=blue,
  citecolor=blue,
  urlcolor=blue
}

\begin{document}

\title{軽量 CNN と Vision Transformer の公平比較に向けた実験的評価}

\author{\IEEEauthorblockN{著者 太郎\\ }
\IEEEauthorblockA{所属機関\\ 連絡先: author@example.com}}

\maketitle

\begin{abstract}
本稿では,軽量な畳み込みニューラルネットワーク(CNN)と Vision Transformer(ViT)の画像分類性能を,CIFAR-10 を対象に公平に比較する実験プロトコルを提示する。学習率スケジューリング,データ拡張,Test-Time Augmentation (TTA) を統一的に適用し,再現性を重視した評価を実施した。実験の結果,ViT-Ti は TTA により最大で +1.8pt の精度向上を得た一方,ResNet-18 では RandAugment を中心としたデータ拡張で +2.1pt の改善が確認された。これらの知見を通じて,限られた計算資源でも堅牢なベースラインを構築するための指針を示す。
\end{abstract}

\begin{IEEEkeywords}
画像分類, 深層学習, Vision Transformer, Test-Time Augmentation, CIFAR-10
\end{IEEEkeywords}

\input{01_introduction/main}
\input{02_related_work/main}
\input{03_method/main}
\input{04_results/main}
\input{05_discussion/main}
\input{06_conclusion/main}
\input{appendix/main}

\bibliographystyle{ieeetr}
\bibliography{references}

\end{document}

% !TEX program = xelatex
\documentclass[conference]{IEEEtran}

% 日本語出力と XeLaTeX 対応設定
\usepackage{xeCJK}
\usepackage{fontspec}
\setCJKmainfont{HaranoAjiMincho}

% 数式・図表・参考文献などの標準パッケージ
\usepackage{amsmath, amssymb}
\usepackage{graphicx}
\usepackage{url}
\usepackage{hyperref}

\hypersetup{
  colorlinks=true,
  linkcolor=blue,
  citecolor=blue,
  urlcolor=blue
}

\begin{document}

\title{軽量 CNN と Vision Transformer の公平比較に向けた実験的評価}

\author{\IEEEauthorblockN{著者 太郎\\ }
\IEEEauthorblockA{所属機関\\ 連絡先: author@example.com}}

\maketitle

\begin{abstract}
本稿では,軽量な畳み込みニューラルネットワーク(CNN)と Vision Transformer(ViT)の画像分類性能を,CIFAR-10 を対象に公平に比較する実験プロトコルを提示する。学習率スケジューリング,データ拡張,Test-Time Augmentation (TTA) を統一的に適用し,再現性を重視した評価を実施した。実験の結果,ViT-Ti は TTA により最大で +1.8pt の精度向上を得た一方,ResNet-18 では RandAugment を中心としたデータ拡張で +2.1pt の改善が確認された。これらの知見を通じて,限られた計算資源でも堅牢なベースラインを構築するための指針を示す。
\end{abstract}

\begin{IEEEkeywords}
画像分類, 深層学習, Vision Transformer, Test-Time Augmentation, CIFAR-10
\end{IEEEkeywords}

\input{01_introduction/main}
\input{02_related_work/main}
\input{03_method/main}
\input{04_results/main}
\input{05_discussion/main}
\input{06_conclusion/main}
\input{appendix/main}

\bibliographystyle{ieeetr}
\bibliography{references}

\end{document}

% !TEX program = xelatex
\documentclass[conference]{IEEEtran}

% 日本語出力と XeLaTeX 対応設定
\usepackage{xeCJK}
\usepackage{fontspec}
\setCJKmainfont{HaranoAjiMincho}

% 数式・図表・参考文献などの標準パッケージ
\usepackage{amsmath, amssymb}
\usepackage{graphicx}
\usepackage{url}
\usepackage{hyperref}

\hypersetup{
  colorlinks=true,
  linkcolor=blue,
  citecolor=blue,
  urlcolor=blue
}

\begin{document}

\title{軽量 CNN と Vision Transformer の公平比較に向けた実験的評価}

\author{\IEEEauthorblockN{著者 太郎\\ }
\IEEEauthorblockA{所属機関\\ 連絡先: author@example.com}}

\maketitle

\begin{abstract}
本稿では,軽量な畳み込みニューラルネットワーク(CNN)と Vision Transformer(ViT)の画像分類性能を,CIFAR-10 を対象に公平に比較する実験プロトコルを提示する。学習率スケジューリング,データ拡張,Test-Time Augmentation (TTA) を統一的に適用し,再現性を重視した評価を実施した。実験の結果,ViT-Ti は TTA により最大で +1.8pt の精度向上を得た一方,ResNet-18 では RandAugment を中心としたデータ拡張で +2.1pt の改善が確認された。これらの知見を通じて,限られた計算資源でも堅牢なベースラインを構築するための指針を示す。
\end{abstract}

\begin{IEEEkeywords}
画像分類, 深層学習, Vision Transformer, Test-Time Augmentation, CIFAR-10
\end{IEEEkeywords}

\input{01_introduction/main}
\input{02_related_work/main}
\input{03_method/main}
\input{04_results/main}
\input{05_discussion/main}
\input{06_conclusion/main}
\input{appendix/main}

\bibliographystyle{ieeetr}
\bibliography{references}

\end{document}

% !TEX program = xelatex
\documentclass[conference]{IEEEtran}

% 日本語出力と XeLaTeX 対応設定
\usepackage{xeCJK}
\usepackage{fontspec}
\setCJKmainfont{HaranoAjiMincho}

% 数式・図表・参考文献などの標準パッケージ
\usepackage{amsmath, amssymb}
\usepackage{graphicx}
\usepackage{url}
\usepackage{hyperref}

\hypersetup{
  colorlinks=true,
  linkcolor=blue,
  citecolor=blue,
  urlcolor=blue
}

\begin{document}

\title{軽量 CNN と Vision Transformer の公平比較に向けた実験的評価}

\author{\IEEEauthorblockN{著者 太郎\\ }
\IEEEauthorblockA{所属機関\\ 連絡先: author@example.com}}

\maketitle

\begin{abstract}
本稿では,軽量な畳み込みニューラルネットワーク(CNN)と Vision Transformer(ViT)の画像分類性能を,CIFAR-10 を対象に公平に比較する実験プロトコルを提示する。学習率スケジューリング,データ拡張,Test-Time Augmentation (TTA) を統一的に適用し,再現性を重視した評価を実施した。実験の結果,ViT-Ti は TTA により最大で +1.8pt の精度向上を得た一方,ResNet-18 では RandAugment を中心としたデータ拡張で +2.1pt の改善が確認された。これらの知見を通じて,限られた計算資源でも堅牢なベースラインを構築するための指針を示す。
\end{abstract}

\begin{IEEEkeywords}
画像分類, 深層学習, Vision Transformer, Test-Time Augmentation, CIFAR-10
\end{IEEEkeywords}

\input{01_introduction/main}
\input{02_related_work/main}
\input{03_method/main}
\input{04_results/main}
\input{05_discussion/main}
\input{06_conclusion/main}
\input{appendix/main}

\bibliographystyle{ieeetr}
\bibliography{references}

\end{document}


\bibliographystyle{ieeetr}
\bibliography{references}

\end{document}

% !TEX program = xelatex
\documentclass[conference]{IEEEtran}

% 日本語出力と XeLaTeX 対応設定
\usepackage{xeCJK}
\usepackage{fontspec}
\setCJKmainfont{HaranoAjiMincho}

% 数式・図表・参考文献などの標準パッケージ
\usepackage{amsmath, amssymb}
\usepackage{graphicx}
\usepackage{url}
\usepackage{hyperref}

\hypersetup{
  colorlinks=true,
  linkcolor=blue,
  citecolor=blue,
  urlcolor=blue
}

\begin{document}

\title{軽量 CNN と Vision Transformer の公平比較に向けた実験的評価}

\author{\IEEEauthorblockN{著者 太郎\\ }
\IEEEauthorblockA{所属機関\\ 連絡先: author@example.com}}

\maketitle

\begin{abstract}
本稿では,軽量な畳み込みニューラルネットワーク(CNN)と Vision Transformer(ViT)の画像分類性能を,CIFAR-10 を対象に公平に比較する実験プロトコルを提示する。学習率スケジューリング,データ拡張,Test-Time Augmentation (TTA) を統一的に適用し,再現性を重視した評価を実施した。実験の結果,ViT-Ti は TTA により最大で +1.8pt の精度向上を得た一方,ResNet-18 では RandAugment を中心としたデータ拡張で +2.1pt の改善が確認された。これらの知見を通じて,限られた計算資源でも堅牢なベースラインを構築するための指針を示す。
\end{abstract}

\begin{IEEEkeywords}
画像分類, 深層学習, Vision Transformer, Test-Time Augmentation, CIFAR-10
\end{IEEEkeywords}

% !TEX program = xelatex
\documentclass[conference]{IEEEtran}

% 日本語出力と XeLaTeX 対応設定
\usepackage{xeCJK}
\usepackage{fontspec}
\setCJKmainfont{HaranoAjiMincho}

% 数式・図表・参考文献などの標準パッケージ
\usepackage{amsmath, amssymb}
\usepackage{graphicx}
\usepackage{url}
\usepackage{hyperref}

\hypersetup{
  colorlinks=true,
  linkcolor=blue,
  citecolor=blue,
  urlcolor=blue
}

\begin{document}

\title{軽量 CNN と Vision Transformer の公平比較に向けた実験的評価}

\author{\IEEEauthorblockN{著者 太郎\\ }
\IEEEauthorblockA{所属機関\\ 連絡先: author@example.com}}

\maketitle

\begin{abstract}
本稿では,軽量な畳み込みニューラルネットワーク(CNN)と Vision Transformer(ViT)の画像分類性能を,CIFAR-10 を対象に公平に比較する実験プロトコルを提示する。学習率スケジューリング,データ拡張,Test-Time Augmentation (TTA) を統一的に適用し,再現性を重視した評価を実施した。実験の結果,ViT-Ti は TTA により最大で +1.8pt の精度向上を得た一方,ResNet-18 では RandAugment を中心としたデータ拡張で +2.1pt の改善が確認された。これらの知見を通じて,限られた計算資源でも堅牢なベースラインを構築するための指針を示す。
\end{abstract}

\begin{IEEEkeywords}
画像分類, 深層学習, Vision Transformer, Test-Time Augmentation, CIFAR-10
\end{IEEEkeywords}

\input{01_introduction/main}
\input{02_related_work/main}
\input{03_method/main}
\input{04_results/main}
\input{05_discussion/main}
\input{06_conclusion/main}
\input{appendix/main}

\bibliographystyle{ieeetr}
\bibliography{references}

\end{document}

% !TEX program = xelatex
\documentclass[conference]{IEEEtran}

% 日本語出力と XeLaTeX 対応設定
\usepackage{xeCJK}
\usepackage{fontspec}
\setCJKmainfont{HaranoAjiMincho}

% 数式・図表・参考文献などの標準パッケージ
\usepackage{amsmath, amssymb}
\usepackage{graphicx}
\usepackage{url}
\usepackage{hyperref}

\hypersetup{
  colorlinks=true,
  linkcolor=blue,
  citecolor=blue,
  urlcolor=blue
}

\begin{document}

\title{軽量 CNN と Vision Transformer の公平比較に向けた実験的評価}

\author{\IEEEauthorblockN{著者 太郎\\ }
\IEEEauthorblockA{所属機関\\ 連絡先: author@example.com}}

\maketitle

\begin{abstract}
本稿では,軽量な畳み込みニューラルネットワーク(CNN)と Vision Transformer(ViT)の画像分類性能を,CIFAR-10 を対象に公平に比較する実験プロトコルを提示する。学習率スケジューリング,データ拡張,Test-Time Augmentation (TTA) を統一的に適用し,再現性を重視した評価を実施した。実験の結果,ViT-Ti は TTA により最大で +1.8pt の精度向上を得た一方,ResNet-18 では RandAugment を中心としたデータ拡張で +2.1pt の改善が確認された。これらの知見を通じて,限られた計算資源でも堅牢なベースラインを構築するための指針を示す。
\end{abstract}

\begin{IEEEkeywords}
画像分類, 深層学習, Vision Transformer, Test-Time Augmentation, CIFAR-10
\end{IEEEkeywords}

\input{01_introduction/main}
\input{02_related_work/main}
\input{03_method/main}
\input{04_results/main}
\input{05_discussion/main}
\input{06_conclusion/main}
\input{appendix/main}

\bibliographystyle{ieeetr}
\bibliography{references}

\end{document}

% !TEX program = xelatex
\documentclass[conference]{IEEEtran}

% 日本語出力と XeLaTeX 対応設定
\usepackage{xeCJK}
\usepackage{fontspec}
\setCJKmainfont{HaranoAjiMincho}

% 数式・図表・参考文献などの標準パッケージ
\usepackage{amsmath, amssymb}
\usepackage{graphicx}
\usepackage{url}
\usepackage{hyperref}

\hypersetup{
  colorlinks=true,
  linkcolor=blue,
  citecolor=blue,
  urlcolor=blue
}

\begin{document}

\title{軽量 CNN と Vision Transformer の公平比較に向けた実験的評価}

\author{\IEEEauthorblockN{著者 太郎\\ }
\IEEEauthorblockA{所属機関\\ 連絡先: author@example.com}}

\maketitle

\begin{abstract}
本稿では,軽量な畳み込みニューラルネットワーク(CNN)と Vision Transformer(ViT)の画像分類性能を,CIFAR-10 を対象に公平に比較する実験プロトコルを提示する。学習率スケジューリング,データ拡張,Test-Time Augmentation (TTA) を統一的に適用し,再現性を重視した評価を実施した。実験の結果,ViT-Ti は TTA により最大で +1.8pt の精度向上を得た一方,ResNet-18 では RandAugment を中心としたデータ拡張で +2.1pt の改善が確認された。これらの知見を通じて,限られた計算資源でも堅牢なベースラインを構築するための指針を示す。
\end{abstract}

\begin{IEEEkeywords}
画像分類, 深層学習, Vision Transformer, Test-Time Augmentation, CIFAR-10
\end{IEEEkeywords}

\input{01_introduction/main}
\input{02_related_work/main}
\input{03_method/main}
\input{04_results/main}
\input{05_discussion/main}
\input{06_conclusion/main}
\input{appendix/main}

\bibliographystyle{ieeetr}
\bibliography{references}

\end{document}

% !TEX program = xelatex
\documentclass[conference]{IEEEtran}

% 日本語出力と XeLaTeX 対応設定
\usepackage{xeCJK}
\usepackage{fontspec}
\setCJKmainfont{HaranoAjiMincho}

% 数式・図表・参考文献などの標準パッケージ
\usepackage{amsmath, amssymb}
\usepackage{graphicx}
\usepackage{url}
\usepackage{hyperref}

\hypersetup{
  colorlinks=true,
  linkcolor=blue,
  citecolor=blue,
  urlcolor=blue
}

\begin{document}

\title{軽量 CNN と Vision Transformer の公平比較に向けた実験的評価}

\author{\IEEEauthorblockN{著者 太郎\\ }
\IEEEauthorblockA{所属機関\\ 連絡先: author@example.com}}

\maketitle

\begin{abstract}
本稿では,軽量な畳み込みニューラルネットワーク(CNN)と Vision Transformer(ViT)の画像分類性能を,CIFAR-10 を対象に公平に比較する実験プロトコルを提示する。学習率スケジューリング,データ拡張,Test-Time Augmentation (TTA) を統一的に適用し,再現性を重視した評価を実施した。実験の結果,ViT-Ti は TTA により最大で +1.8pt の精度向上を得た一方,ResNet-18 では RandAugment を中心としたデータ拡張で +2.1pt の改善が確認された。これらの知見を通じて,限られた計算資源でも堅牢なベースラインを構築するための指針を示す。
\end{abstract}

\begin{IEEEkeywords}
画像分類, 深層学習, Vision Transformer, Test-Time Augmentation, CIFAR-10
\end{IEEEkeywords}

\input{01_introduction/main}
\input{02_related_work/main}
\input{03_method/main}
\input{04_results/main}
\input{05_discussion/main}
\input{06_conclusion/main}
\input{appendix/main}

\bibliographystyle{ieeetr}
\bibliography{references}

\end{document}

% !TEX program = xelatex
\documentclass[conference]{IEEEtran}

% 日本語出力と XeLaTeX 対応設定
\usepackage{xeCJK}
\usepackage{fontspec}
\setCJKmainfont{HaranoAjiMincho}

% 数式・図表・参考文献などの標準パッケージ
\usepackage{amsmath, amssymb}
\usepackage{graphicx}
\usepackage{url}
\usepackage{hyperref}

\hypersetup{
  colorlinks=true,
  linkcolor=blue,
  citecolor=blue,
  urlcolor=blue
}

\begin{document}

\title{軽量 CNN と Vision Transformer の公平比較に向けた実験的評価}

\author{\IEEEauthorblockN{著者 太郎\\ }
\IEEEauthorblockA{所属機関\\ 連絡先: author@example.com}}

\maketitle

\begin{abstract}
本稿では,軽量な畳み込みニューラルネットワーク(CNN)と Vision Transformer(ViT)の画像分類性能を,CIFAR-10 を対象に公平に比較する実験プロトコルを提示する。学習率スケジューリング,データ拡張,Test-Time Augmentation (TTA) を統一的に適用し,再現性を重視した評価を実施した。実験の結果,ViT-Ti は TTA により最大で +1.8pt の精度向上を得た一方,ResNet-18 では RandAugment を中心としたデータ拡張で +2.1pt の改善が確認された。これらの知見を通じて,限られた計算資源でも堅牢なベースラインを構築するための指針を示す。
\end{abstract}

\begin{IEEEkeywords}
画像分類, 深層学習, Vision Transformer, Test-Time Augmentation, CIFAR-10
\end{IEEEkeywords}

\input{01_introduction/main}
\input{02_related_work/main}
\input{03_method/main}
\input{04_results/main}
\input{05_discussion/main}
\input{06_conclusion/main}
\input{appendix/main}

\bibliographystyle{ieeetr}
\bibliography{references}

\end{document}

% !TEX program = xelatex
\documentclass[conference]{IEEEtran}

% 日本語出力と XeLaTeX 対応設定
\usepackage{xeCJK}
\usepackage{fontspec}
\setCJKmainfont{HaranoAjiMincho}

% 数式・図表・参考文献などの標準パッケージ
\usepackage{amsmath, amssymb}
\usepackage{graphicx}
\usepackage{url}
\usepackage{hyperref}

\hypersetup{
  colorlinks=true,
  linkcolor=blue,
  citecolor=blue,
  urlcolor=blue
}

\begin{document}

\title{軽量 CNN と Vision Transformer の公平比較に向けた実験的評価}

\author{\IEEEauthorblockN{著者 太郎\\ }
\IEEEauthorblockA{所属機関\\ 連絡先: author@example.com}}

\maketitle

\begin{abstract}
本稿では,軽量な畳み込みニューラルネットワーク(CNN)と Vision Transformer(ViT)の画像分類性能を,CIFAR-10 を対象に公平に比較する実験プロトコルを提示する。学習率スケジューリング,データ拡張,Test-Time Augmentation (TTA) を統一的に適用し,再現性を重視した評価を実施した。実験の結果,ViT-Ti は TTA により最大で +1.8pt の精度向上を得た一方,ResNet-18 では RandAugment を中心としたデータ拡張で +2.1pt の改善が確認された。これらの知見を通じて,限られた計算資源でも堅牢なベースラインを構築するための指針を示す。
\end{abstract}

\begin{IEEEkeywords}
画像分類, 深層学習, Vision Transformer, Test-Time Augmentation, CIFAR-10
\end{IEEEkeywords}

\input{01_introduction/main}
\input{02_related_work/main}
\input{03_method/main}
\input{04_results/main}
\input{05_discussion/main}
\input{06_conclusion/main}
\input{appendix/main}

\bibliographystyle{ieeetr}
\bibliography{references}

\end{document}

% !TEX program = xelatex
\documentclass[conference]{IEEEtran}

% 日本語出力と XeLaTeX 対応設定
\usepackage{xeCJK}
\usepackage{fontspec}
\setCJKmainfont{HaranoAjiMincho}

% 数式・図表・参考文献などの標準パッケージ
\usepackage{amsmath, amssymb}
\usepackage{graphicx}
\usepackage{url}
\usepackage{hyperref}

\hypersetup{
  colorlinks=true,
  linkcolor=blue,
  citecolor=blue,
  urlcolor=blue
}

\begin{document}

\title{軽量 CNN と Vision Transformer の公平比較に向けた実験的評価}

\author{\IEEEauthorblockN{著者 太郎\\ }
\IEEEauthorblockA{所属機関\\ 連絡先: author@example.com}}

\maketitle

\begin{abstract}
本稿では,軽量な畳み込みニューラルネットワーク(CNN)と Vision Transformer(ViT)の画像分類性能を,CIFAR-10 を対象に公平に比較する実験プロトコルを提示する。学習率スケジューリング,データ拡張,Test-Time Augmentation (TTA) を統一的に適用し,再現性を重視した評価を実施した。実験の結果,ViT-Ti は TTA により最大で +1.8pt の精度向上を得た一方,ResNet-18 では RandAugment を中心としたデータ拡張で +2.1pt の改善が確認された。これらの知見を通じて,限られた計算資源でも堅牢なベースラインを構築するための指針を示す。
\end{abstract}

\begin{IEEEkeywords}
画像分類, 深層学習, Vision Transformer, Test-Time Augmentation, CIFAR-10
\end{IEEEkeywords}

\input{01_introduction/main}
\input{02_related_work/main}
\input{03_method/main}
\input{04_results/main}
\input{05_discussion/main}
\input{06_conclusion/main}
\input{appendix/main}

\bibliographystyle{ieeetr}
\bibliography{references}

\end{document}


\bibliographystyle{ieeetr}
\bibliography{references}

\end{document}

% !TEX program = xelatex
\documentclass[conference]{IEEEtran}

% 日本語出力と XeLaTeX 対応設定
\usepackage{xeCJK}
\usepackage{fontspec}
\setCJKmainfont{HaranoAjiMincho}

% 数式・図表・参考文献などの標準パッケージ
\usepackage{amsmath, amssymb}
\usepackage{graphicx}
\usepackage{url}
\usepackage{hyperref}

\hypersetup{
  colorlinks=true,
  linkcolor=blue,
  citecolor=blue,
  urlcolor=blue
}

\begin{document}

\title{軽量 CNN と Vision Transformer の公平比較に向けた実験的評価}

\author{\IEEEauthorblockN{著者 太郎\\ }
\IEEEauthorblockA{所属機関\\ 連絡先: author@example.com}}

\maketitle

\begin{abstract}
本稿では,軽量な畳み込みニューラルネットワーク(CNN)と Vision Transformer(ViT)の画像分類性能を,CIFAR-10 を対象に公平に比較する実験プロトコルを提示する。学習率スケジューリング,データ拡張,Test-Time Augmentation (TTA) を統一的に適用し,再現性を重視した評価を実施した。実験の結果,ViT-Ti は TTA により最大で +1.8pt の精度向上を得た一方,ResNet-18 では RandAugment を中心としたデータ拡張で +2.1pt の改善が確認された。これらの知見を通じて,限られた計算資源でも堅牢なベースラインを構築するための指針を示す。
\end{abstract}

\begin{IEEEkeywords}
画像分類, 深層学習, Vision Transformer, Test-Time Augmentation, CIFAR-10
\end{IEEEkeywords}

% !TEX program = xelatex
\documentclass[conference]{IEEEtran}

% 日本語出力と XeLaTeX 対応設定
\usepackage{xeCJK}
\usepackage{fontspec}
\setCJKmainfont{HaranoAjiMincho}

% 数式・図表・参考文献などの標準パッケージ
\usepackage{amsmath, amssymb}
\usepackage{graphicx}
\usepackage{url}
\usepackage{hyperref}

\hypersetup{
  colorlinks=true,
  linkcolor=blue,
  citecolor=blue,
  urlcolor=blue
}

\begin{document}

\title{軽量 CNN と Vision Transformer の公平比較に向けた実験的評価}

\author{\IEEEauthorblockN{著者 太郎\\ }
\IEEEauthorblockA{所属機関\\ 連絡先: author@example.com}}

\maketitle

\begin{abstract}
本稿では,軽量な畳み込みニューラルネットワーク(CNN)と Vision Transformer(ViT)の画像分類性能を,CIFAR-10 を対象に公平に比較する実験プロトコルを提示する。学習率スケジューリング,データ拡張,Test-Time Augmentation (TTA) を統一的に適用し,再現性を重視した評価を実施した。実験の結果,ViT-Ti は TTA により最大で +1.8pt の精度向上を得た一方,ResNet-18 では RandAugment を中心としたデータ拡張で +2.1pt の改善が確認された。これらの知見を通じて,限られた計算資源でも堅牢なベースラインを構築するための指針を示す。
\end{abstract}

\begin{IEEEkeywords}
画像分類, 深層学習, Vision Transformer, Test-Time Augmentation, CIFAR-10
\end{IEEEkeywords}

\input{01_introduction/main}
\input{02_related_work/main}
\input{03_method/main}
\input{04_results/main}
\input{05_discussion/main}
\input{06_conclusion/main}
\input{appendix/main}

\bibliographystyle{ieeetr}
\bibliography{references}

\end{document}

% !TEX program = xelatex
\documentclass[conference]{IEEEtran}

% 日本語出力と XeLaTeX 対応設定
\usepackage{xeCJK}
\usepackage{fontspec}
\setCJKmainfont{HaranoAjiMincho}

% 数式・図表・参考文献などの標準パッケージ
\usepackage{amsmath, amssymb}
\usepackage{graphicx}
\usepackage{url}
\usepackage{hyperref}

\hypersetup{
  colorlinks=true,
  linkcolor=blue,
  citecolor=blue,
  urlcolor=blue
}

\begin{document}

\title{軽量 CNN と Vision Transformer の公平比較に向けた実験的評価}

\author{\IEEEauthorblockN{著者 太郎\\ }
\IEEEauthorblockA{所属機関\\ 連絡先: author@example.com}}

\maketitle

\begin{abstract}
本稿では,軽量な畳み込みニューラルネットワーク(CNN)と Vision Transformer(ViT)の画像分類性能を,CIFAR-10 を対象に公平に比較する実験プロトコルを提示する。学習率スケジューリング,データ拡張,Test-Time Augmentation (TTA) を統一的に適用し,再現性を重視した評価を実施した。実験の結果,ViT-Ti は TTA により最大で +1.8pt の精度向上を得た一方,ResNet-18 では RandAugment を中心としたデータ拡張で +2.1pt の改善が確認された。これらの知見を通じて,限られた計算資源でも堅牢なベースラインを構築するための指針を示す。
\end{abstract}

\begin{IEEEkeywords}
画像分類, 深層学習, Vision Transformer, Test-Time Augmentation, CIFAR-10
\end{IEEEkeywords}

\input{01_introduction/main}
\input{02_related_work/main}
\input{03_method/main}
\input{04_results/main}
\input{05_discussion/main}
\input{06_conclusion/main}
\input{appendix/main}

\bibliographystyle{ieeetr}
\bibliography{references}

\end{document}

% !TEX program = xelatex
\documentclass[conference]{IEEEtran}

% 日本語出力と XeLaTeX 対応設定
\usepackage{xeCJK}
\usepackage{fontspec}
\setCJKmainfont{HaranoAjiMincho}

% 数式・図表・参考文献などの標準パッケージ
\usepackage{amsmath, amssymb}
\usepackage{graphicx}
\usepackage{url}
\usepackage{hyperref}

\hypersetup{
  colorlinks=true,
  linkcolor=blue,
  citecolor=blue,
  urlcolor=blue
}

\begin{document}

\title{軽量 CNN と Vision Transformer の公平比較に向けた実験的評価}

\author{\IEEEauthorblockN{著者 太郎\\ }
\IEEEauthorblockA{所属機関\\ 連絡先: author@example.com}}

\maketitle

\begin{abstract}
本稿では,軽量な畳み込みニューラルネットワーク(CNN)と Vision Transformer(ViT)の画像分類性能を,CIFAR-10 を対象に公平に比較する実験プロトコルを提示する。学習率スケジューリング,データ拡張,Test-Time Augmentation (TTA) を統一的に適用し,再現性を重視した評価を実施した。実験の結果,ViT-Ti は TTA により最大で +1.8pt の精度向上を得た一方,ResNet-18 では RandAugment を中心としたデータ拡張で +2.1pt の改善が確認された。これらの知見を通じて,限られた計算資源でも堅牢なベースラインを構築するための指針を示す。
\end{abstract}

\begin{IEEEkeywords}
画像分類, 深層学習, Vision Transformer, Test-Time Augmentation, CIFAR-10
\end{IEEEkeywords}

\input{01_introduction/main}
\input{02_related_work/main}
\input{03_method/main}
\input{04_results/main}
\input{05_discussion/main}
\input{06_conclusion/main}
\input{appendix/main}

\bibliographystyle{ieeetr}
\bibliography{references}

\end{document}

% !TEX program = xelatex
\documentclass[conference]{IEEEtran}

% 日本語出力と XeLaTeX 対応設定
\usepackage{xeCJK}
\usepackage{fontspec}
\setCJKmainfont{HaranoAjiMincho}

% 数式・図表・参考文献などの標準パッケージ
\usepackage{amsmath, amssymb}
\usepackage{graphicx}
\usepackage{url}
\usepackage{hyperref}

\hypersetup{
  colorlinks=true,
  linkcolor=blue,
  citecolor=blue,
  urlcolor=blue
}

\begin{document}

\title{軽量 CNN と Vision Transformer の公平比較に向けた実験的評価}

\author{\IEEEauthorblockN{著者 太郎\\ }
\IEEEauthorblockA{所属機関\\ 連絡先: author@example.com}}

\maketitle

\begin{abstract}
本稿では,軽量な畳み込みニューラルネットワーク(CNN)と Vision Transformer(ViT)の画像分類性能を,CIFAR-10 を対象に公平に比較する実験プロトコルを提示する。学習率スケジューリング,データ拡張,Test-Time Augmentation (TTA) を統一的に適用し,再現性を重視した評価を実施した。実験の結果,ViT-Ti は TTA により最大で +1.8pt の精度向上を得た一方,ResNet-18 では RandAugment を中心としたデータ拡張で +2.1pt の改善が確認された。これらの知見を通じて,限られた計算資源でも堅牢なベースラインを構築するための指針を示す。
\end{abstract}

\begin{IEEEkeywords}
画像分類, 深層学習, Vision Transformer, Test-Time Augmentation, CIFAR-10
\end{IEEEkeywords}

\input{01_introduction/main}
\input{02_related_work/main}
\input{03_method/main}
\input{04_results/main}
\input{05_discussion/main}
\input{06_conclusion/main}
\input{appendix/main}

\bibliographystyle{ieeetr}
\bibliography{references}

\end{document}

% !TEX program = xelatex
\documentclass[conference]{IEEEtran}

% 日本語出力と XeLaTeX 対応設定
\usepackage{xeCJK}
\usepackage{fontspec}
\setCJKmainfont{HaranoAjiMincho}

% 数式・図表・参考文献などの標準パッケージ
\usepackage{amsmath, amssymb}
\usepackage{graphicx}
\usepackage{url}
\usepackage{hyperref}

\hypersetup{
  colorlinks=true,
  linkcolor=blue,
  citecolor=blue,
  urlcolor=blue
}

\begin{document}

\title{軽量 CNN と Vision Transformer の公平比較に向けた実験的評価}

\author{\IEEEauthorblockN{著者 太郎\\ }
\IEEEauthorblockA{所属機関\\ 連絡先: author@example.com}}

\maketitle

\begin{abstract}
本稿では,軽量な畳み込みニューラルネットワーク(CNN)と Vision Transformer(ViT)の画像分類性能を,CIFAR-10 を対象に公平に比較する実験プロトコルを提示する。学習率スケジューリング,データ拡張,Test-Time Augmentation (TTA) を統一的に適用し,再現性を重視した評価を実施した。実験の結果,ViT-Ti は TTA により最大で +1.8pt の精度向上を得た一方,ResNet-18 では RandAugment を中心としたデータ拡張で +2.1pt の改善が確認された。これらの知見を通じて,限られた計算資源でも堅牢なベースラインを構築するための指針を示す。
\end{abstract}

\begin{IEEEkeywords}
画像分類, 深層学習, Vision Transformer, Test-Time Augmentation, CIFAR-10
\end{IEEEkeywords}

\input{01_introduction/main}
\input{02_related_work/main}
\input{03_method/main}
\input{04_results/main}
\input{05_discussion/main}
\input{06_conclusion/main}
\input{appendix/main}

\bibliographystyle{ieeetr}
\bibliography{references}

\end{document}

% !TEX program = xelatex
\documentclass[conference]{IEEEtran}

% 日本語出力と XeLaTeX 対応設定
\usepackage{xeCJK}
\usepackage{fontspec}
\setCJKmainfont{HaranoAjiMincho}

% 数式・図表・参考文献などの標準パッケージ
\usepackage{amsmath, amssymb}
\usepackage{graphicx}
\usepackage{url}
\usepackage{hyperref}

\hypersetup{
  colorlinks=true,
  linkcolor=blue,
  citecolor=blue,
  urlcolor=blue
}

\begin{document}

\title{軽量 CNN と Vision Transformer の公平比較に向けた実験的評価}

\author{\IEEEauthorblockN{著者 太郎\\ }
\IEEEauthorblockA{所属機関\\ 連絡先: author@example.com}}

\maketitle

\begin{abstract}
本稿では,軽量な畳み込みニューラルネットワーク(CNN)と Vision Transformer(ViT)の画像分類性能を,CIFAR-10 を対象に公平に比較する実験プロトコルを提示する。学習率スケジューリング,データ拡張,Test-Time Augmentation (TTA) を統一的に適用し,再現性を重視した評価を実施した。実験の結果,ViT-Ti は TTA により最大で +1.8pt の精度向上を得た一方,ResNet-18 では RandAugment を中心としたデータ拡張で +2.1pt の改善が確認された。これらの知見を通じて,限られた計算資源でも堅牢なベースラインを構築するための指針を示す。
\end{abstract}

\begin{IEEEkeywords}
画像分類, 深層学習, Vision Transformer, Test-Time Augmentation, CIFAR-10
\end{IEEEkeywords}

\input{01_introduction/main}
\input{02_related_work/main}
\input{03_method/main}
\input{04_results/main}
\input{05_discussion/main}
\input{06_conclusion/main}
\input{appendix/main}

\bibliographystyle{ieeetr}
\bibliography{references}

\end{document}

% !TEX program = xelatex
\documentclass[conference]{IEEEtran}

% 日本語出力と XeLaTeX 対応設定
\usepackage{xeCJK}
\usepackage{fontspec}
\setCJKmainfont{HaranoAjiMincho}

% 数式・図表・参考文献などの標準パッケージ
\usepackage{amsmath, amssymb}
\usepackage{graphicx}
\usepackage{url}
\usepackage{hyperref}

\hypersetup{
  colorlinks=true,
  linkcolor=blue,
  citecolor=blue,
  urlcolor=blue
}

\begin{document}

\title{軽量 CNN と Vision Transformer の公平比較に向けた実験的評価}

\author{\IEEEauthorblockN{著者 太郎\\ }
\IEEEauthorblockA{所属機関\\ 連絡先: author@example.com}}

\maketitle

\begin{abstract}
本稿では,軽量な畳み込みニューラルネットワーク(CNN)と Vision Transformer(ViT)の画像分類性能を,CIFAR-10 を対象に公平に比較する実験プロトコルを提示する。学習率スケジューリング,データ拡張,Test-Time Augmentation (TTA) を統一的に適用し,再現性を重視した評価を実施した。実験の結果,ViT-Ti は TTA により最大で +1.8pt の精度向上を得た一方,ResNet-18 では RandAugment を中心としたデータ拡張で +2.1pt の改善が確認された。これらの知見を通じて,限られた計算資源でも堅牢なベースラインを構築するための指針を示す。
\end{abstract}

\begin{IEEEkeywords}
画像分類, 深層学習, Vision Transformer, Test-Time Augmentation, CIFAR-10
\end{IEEEkeywords}

\input{01_introduction/main}
\input{02_related_work/main}
\input{03_method/main}
\input{04_results/main}
\input{05_discussion/main}
\input{06_conclusion/main}
\input{appendix/main}

\bibliographystyle{ieeetr}
\bibliography{references}

\end{document}


\bibliographystyle{ieeetr}
\bibliography{references}

\end{document}

% !TEX program = xelatex
\documentclass[conference]{IEEEtran}

% 日本語出力と XeLaTeX 対応設定
\usepackage{xeCJK}
\usepackage{fontspec}
\setCJKmainfont{HaranoAjiMincho}

% 数式・図表・参考文献などの標準パッケージ
\usepackage{amsmath, amssymb}
\usepackage{graphicx}
\usepackage{url}
\usepackage{hyperref}

\hypersetup{
  colorlinks=true,
  linkcolor=blue,
  citecolor=blue,
  urlcolor=blue
}

\begin{document}

\title{軽量 CNN と Vision Transformer の公平比較に向けた実験的評価}

\author{\IEEEauthorblockN{著者 太郎\\ }
\IEEEauthorblockA{所属機関\\ 連絡先: author@example.com}}

\maketitle

\begin{abstract}
本稿では,軽量な畳み込みニューラルネットワーク(CNN)と Vision Transformer(ViT)の画像分類性能を,CIFAR-10 を対象に公平に比較する実験プロトコルを提示する。学習率スケジューリング,データ拡張,Test-Time Augmentation (TTA) を統一的に適用し,再現性を重視した評価を実施した。実験の結果,ViT-Ti は TTA により最大で +1.8pt の精度向上を得た一方,ResNet-18 では RandAugment を中心としたデータ拡張で +2.1pt の改善が確認された。これらの知見を通じて,限られた計算資源でも堅牢なベースラインを構築するための指針を示す。
\end{abstract}

\begin{IEEEkeywords}
画像分類, 深層学習, Vision Transformer, Test-Time Augmentation, CIFAR-10
\end{IEEEkeywords}

% !TEX program = xelatex
\documentclass[conference]{IEEEtran}

% 日本語出力と XeLaTeX 対応設定
\usepackage{xeCJK}
\usepackage{fontspec}
\setCJKmainfont{HaranoAjiMincho}

% 数式・図表・参考文献などの標準パッケージ
\usepackage{amsmath, amssymb}
\usepackage{graphicx}
\usepackage{url}
\usepackage{hyperref}

\hypersetup{
  colorlinks=true,
  linkcolor=blue,
  citecolor=blue,
  urlcolor=blue
}

\begin{document}

\title{軽量 CNN と Vision Transformer の公平比較に向けた実験的評価}

\author{\IEEEauthorblockN{著者 太郎\\ }
\IEEEauthorblockA{所属機関\\ 連絡先: author@example.com}}

\maketitle

\begin{abstract}
本稿では,軽量な畳み込みニューラルネットワーク(CNN)と Vision Transformer(ViT)の画像分類性能を,CIFAR-10 を対象に公平に比較する実験プロトコルを提示する。学習率スケジューリング,データ拡張,Test-Time Augmentation (TTA) を統一的に適用し,再現性を重視した評価を実施した。実験の結果,ViT-Ti は TTA により最大で +1.8pt の精度向上を得た一方,ResNet-18 では RandAugment を中心としたデータ拡張で +2.1pt の改善が確認された。これらの知見を通じて,限られた計算資源でも堅牢なベースラインを構築するための指針を示す。
\end{abstract}

\begin{IEEEkeywords}
画像分類, 深層学習, Vision Transformer, Test-Time Augmentation, CIFAR-10
\end{IEEEkeywords}

\input{01_introduction/main}
\input{02_related_work/main}
\input{03_method/main}
\input{04_results/main}
\input{05_discussion/main}
\input{06_conclusion/main}
\input{appendix/main}

\bibliographystyle{ieeetr}
\bibliography{references}

\end{document}

% !TEX program = xelatex
\documentclass[conference]{IEEEtran}

% 日本語出力と XeLaTeX 対応設定
\usepackage{xeCJK}
\usepackage{fontspec}
\setCJKmainfont{HaranoAjiMincho}

% 数式・図表・参考文献などの標準パッケージ
\usepackage{amsmath, amssymb}
\usepackage{graphicx}
\usepackage{url}
\usepackage{hyperref}

\hypersetup{
  colorlinks=true,
  linkcolor=blue,
  citecolor=blue,
  urlcolor=blue
}

\begin{document}

\title{軽量 CNN と Vision Transformer の公平比較に向けた実験的評価}

\author{\IEEEauthorblockN{著者 太郎\\ }
\IEEEauthorblockA{所属機関\\ 連絡先: author@example.com}}

\maketitle

\begin{abstract}
本稿では,軽量な畳み込みニューラルネットワーク(CNN)と Vision Transformer(ViT)の画像分類性能を,CIFAR-10 を対象に公平に比較する実験プロトコルを提示する。学習率スケジューリング,データ拡張,Test-Time Augmentation (TTA) を統一的に適用し,再現性を重視した評価を実施した。実験の結果,ViT-Ti は TTA により最大で +1.8pt の精度向上を得た一方,ResNet-18 では RandAugment を中心としたデータ拡張で +2.1pt の改善が確認された。これらの知見を通じて,限られた計算資源でも堅牢なベースラインを構築するための指針を示す。
\end{abstract}

\begin{IEEEkeywords}
画像分類, 深層学習, Vision Transformer, Test-Time Augmentation, CIFAR-10
\end{IEEEkeywords}

\input{01_introduction/main}
\input{02_related_work/main}
\input{03_method/main}
\input{04_results/main}
\input{05_discussion/main}
\input{06_conclusion/main}
\input{appendix/main}

\bibliographystyle{ieeetr}
\bibliography{references}

\end{document}

% !TEX program = xelatex
\documentclass[conference]{IEEEtran}

% 日本語出力と XeLaTeX 対応設定
\usepackage{xeCJK}
\usepackage{fontspec}
\setCJKmainfont{HaranoAjiMincho}

% 数式・図表・参考文献などの標準パッケージ
\usepackage{amsmath, amssymb}
\usepackage{graphicx}
\usepackage{url}
\usepackage{hyperref}

\hypersetup{
  colorlinks=true,
  linkcolor=blue,
  citecolor=blue,
  urlcolor=blue
}

\begin{document}

\title{軽量 CNN と Vision Transformer の公平比較に向けた実験的評価}

\author{\IEEEauthorblockN{著者 太郎\\ }
\IEEEauthorblockA{所属機関\\ 連絡先: author@example.com}}

\maketitle

\begin{abstract}
本稿では,軽量な畳み込みニューラルネットワーク(CNN)と Vision Transformer(ViT)の画像分類性能を,CIFAR-10 を対象に公平に比較する実験プロトコルを提示する。学習率スケジューリング,データ拡張,Test-Time Augmentation (TTA) を統一的に適用し,再現性を重視した評価を実施した。実験の結果,ViT-Ti は TTA により最大で +1.8pt の精度向上を得た一方,ResNet-18 では RandAugment を中心としたデータ拡張で +2.1pt の改善が確認された。これらの知見を通じて,限られた計算資源でも堅牢なベースラインを構築するための指針を示す。
\end{abstract}

\begin{IEEEkeywords}
画像分類, 深層学習, Vision Transformer, Test-Time Augmentation, CIFAR-10
\end{IEEEkeywords}

\input{01_introduction/main}
\input{02_related_work/main}
\input{03_method/main}
\input{04_results/main}
\input{05_discussion/main}
\input{06_conclusion/main}
\input{appendix/main}

\bibliographystyle{ieeetr}
\bibliography{references}

\end{document}

% !TEX program = xelatex
\documentclass[conference]{IEEEtran}

% 日本語出力と XeLaTeX 対応設定
\usepackage{xeCJK}
\usepackage{fontspec}
\setCJKmainfont{HaranoAjiMincho}

% 数式・図表・参考文献などの標準パッケージ
\usepackage{amsmath, amssymb}
\usepackage{graphicx}
\usepackage{url}
\usepackage{hyperref}

\hypersetup{
  colorlinks=true,
  linkcolor=blue,
  citecolor=blue,
  urlcolor=blue
}

\begin{document}

\title{軽量 CNN と Vision Transformer の公平比較に向けた実験的評価}

\author{\IEEEauthorblockN{著者 太郎\\ }
\IEEEauthorblockA{所属機関\\ 連絡先: author@example.com}}

\maketitle

\begin{abstract}
本稿では,軽量な畳み込みニューラルネットワーク(CNN)と Vision Transformer(ViT)の画像分類性能を,CIFAR-10 を対象に公平に比較する実験プロトコルを提示する。学習率スケジューリング,データ拡張,Test-Time Augmentation (TTA) を統一的に適用し,再現性を重視した評価を実施した。実験の結果,ViT-Ti は TTA により最大で +1.8pt の精度向上を得た一方,ResNet-18 では RandAugment を中心としたデータ拡張で +2.1pt の改善が確認された。これらの知見を通じて,限られた計算資源でも堅牢なベースラインを構築するための指針を示す。
\end{abstract}

\begin{IEEEkeywords}
画像分類, 深層学習, Vision Transformer, Test-Time Augmentation, CIFAR-10
\end{IEEEkeywords}

\input{01_introduction/main}
\input{02_related_work/main}
\input{03_method/main}
\input{04_results/main}
\input{05_discussion/main}
\input{06_conclusion/main}
\input{appendix/main}

\bibliographystyle{ieeetr}
\bibliography{references}

\end{document}

% !TEX program = xelatex
\documentclass[conference]{IEEEtran}

% 日本語出力と XeLaTeX 対応設定
\usepackage{xeCJK}
\usepackage{fontspec}
\setCJKmainfont{HaranoAjiMincho}

% 数式・図表・参考文献などの標準パッケージ
\usepackage{amsmath, amssymb}
\usepackage{graphicx}
\usepackage{url}
\usepackage{hyperref}

\hypersetup{
  colorlinks=true,
  linkcolor=blue,
  citecolor=blue,
  urlcolor=blue
}

\begin{document}

\title{軽量 CNN と Vision Transformer の公平比較に向けた実験的評価}

\author{\IEEEauthorblockN{著者 太郎\\ }
\IEEEauthorblockA{所属機関\\ 連絡先: author@example.com}}

\maketitle

\begin{abstract}
本稿では,軽量な畳み込みニューラルネットワーク(CNN)と Vision Transformer(ViT)の画像分類性能を,CIFAR-10 を対象に公平に比較する実験プロトコルを提示する。学習率スケジューリング,データ拡張,Test-Time Augmentation (TTA) を統一的に適用し,再現性を重視した評価を実施した。実験の結果,ViT-Ti は TTA により最大で +1.8pt の精度向上を得た一方,ResNet-18 では RandAugment を中心としたデータ拡張で +2.1pt の改善が確認された。これらの知見を通じて,限られた計算資源でも堅牢なベースラインを構築するための指針を示す。
\end{abstract}

\begin{IEEEkeywords}
画像分類, 深層学習, Vision Transformer, Test-Time Augmentation, CIFAR-10
\end{IEEEkeywords}

\input{01_introduction/main}
\input{02_related_work/main}
\input{03_method/main}
\input{04_results/main}
\input{05_discussion/main}
\input{06_conclusion/main}
\input{appendix/main}

\bibliographystyle{ieeetr}
\bibliography{references}

\end{document}

% !TEX program = xelatex
\documentclass[conference]{IEEEtran}

% 日本語出力と XeLaTeX 対応設定
\usepackage{xeCJK}
\usepackage{fontspec}
\setCJKmainfont{HaranoAjiMincho}

% 数式・図表・参考文献などの標準パッケージ
\usepackage{amsmath, amssymb}
\usepackage{graphicx}
\usepackage{url}
\usepackage{hyperref}

\hypersetup{
  colorlinks=true,
  linkcolor=blue,
  citecolor=blue,
  urlcolor=blue
}

\begin{document}

\title{軽量 CNN と Vision Transformer の公平比較に向けた実験的評価}

\author{\IEEEauthorblockN{著者 太郎\\ }
\IEEEauthorblockA{所属機関\\ 連絡先: author@example.com}}

\maketitle

\begin{abstract}
本稿では,軽量な畳み込みニューラルネットワーク(CNN)と Vision Transformer(ViT)の画像分類性能を,CIFAR-10 を対象に公平に比較する実験プロトコルを提示する。学習率スケジューリング,データ拡張,Test-Time Augmentation (TTA) を統一的に適用し,再現性を重視した評価を実施した。実験の結果,ViT-Ti は TTA により最大で +1.8pt の精度向上を得た一方,ResNet-18 では RandAugment を中心としたデータ拡張で +2.1pt の改善が確認された。これらの知見を通じて,限られた計算資源でも堅牢なベースラインを構築するための指針を示す。
\end{abstract}

\begin{IEEEkeywords}
画像分類, 深層学習, Vision Transformer, Test-Time Augmentation, CIFAR-10
\end{IEEEkeywords}

\input{01_introduction/main}
\input{02_related_work/main}
\input{03_method/main}
\input{04_results/main}
\input{05_discussion/main}
\input{06_conclusion/main}
\input{appendix/main}

\bibliographystyle{ieeetr}
\bibliography{references}

\end{document}

% !TEX program = xelatex
\documentclass[conference]{IEEEtran}

% 日本語出力と XeLaTeX 対応設定
\usepackage{xeCJK}
\usepackage{fontspec}
\setCJKmainfont{HaranoAjiMincho}

% 数式・図表・参考文献などの標準パッケージ
\usepackage{amsmath, amssymb}
\usepackage{graphicx}
\usepackage{url}
\usepackage{hyperref}

\hypersetup{
  colorlinks=true,
  linkcolor=blue,
  citecolor=blue,
  urlcolor=blue
}

\begin{document}

\title{軽量 CNN と Vision Transformer の公平比較に向けた実験的評価}

\author{\IEEEauthorblockN{著者 太郎\\ }
\IEEEauthorblockA{所属機関\\ 連絡先: author@example.com}}

\maketitle

\begin{abstract}
本稿では,軽量な畳み込みニューラルネットワーク(CNN)と Vision Transformer(ViT)の画像分類性能を,CIFAR-10 を対象に公平に比較する実験プロトコルを提示する。学習率スケジューリング,データ拡張,Test-Time Augmentation (TTA) を統一的に適用し,再現性を重視した評価を実施した。実験の結果,ViT-Ti は TTA により最大で +1.8pt の精度向上を得た一方,ResNet-18 では RandAugment を中心としたデータ拡張で +2.1pt の改善が確認された。これらの知見を通じて,限られた計算資源でも堅牢なベースラインを構築するための指針を示す。
\end{abstract}

\begin{IEEEkeywords}
画像分類, 深層学習, Vision Transformer, Test-Time Augmentation, CIFAR-10
\end{IEEEkeywords}

\input{01_introduction/main}
\input{02_related_work/main}
\input{03_method/main}
\input{04_results/main}
\input{05_discussion/main}
\input{06_conclusion/main}
\input{appendix/main}

\bibliographystyle{ieeetr}
\bibliography{references}

\end{document}


\bibliographystyle{ieeetr}
\bibliography{references}

\end{document}

% !TEX program = xelatex
\documentclass[conference]{IEEEtran}

% 日本語出力と XeLaTeX 対応設定
\usepackage{xeCJK}
\usepackage{fontspec}
\setCJKmainfont{HaranoAjiMincho}

% 数式・図表・参考文献などの標準パッケージ
\usepackage{amsmath, amssymb}
\usepackage{graphicx}
\usepackage{url}
\usepackage{hyperref}

\hypersetup{
  colorlinks=true,
  linkcolor=blue,
  citecolor=blue,
  urlcolor=blue
}

\begin{document}

\title{軽量 CNN と Vision Transformer の公平比較に向けた実験的評価}

\author{\IEEEauthorblockN{著者 太郎\\ }
\IEEEauthorblockA{所属機関\\ 連絡先: author@example.com}}

\maketitle

\begin{abstract}
本稿では,軽量な畳み込みニューラルネットワーク(CNN)と Vision Transformer(ViT)の画像分類性能を,CIFAR-10 を対象に公平に比較する実験プロトコルを提示する。学習率スケジューリング,データ拡張,Test-Time Augmentation (TTA) を統一的に適用し,再現性を重視した評価を実施した。実験の結果,ViT-Ti は TTA により最大で +1.8pt の精度向上を得た一方,ResNet-18 では RandAugment を中心としたデータ拡張で +2.1pt の改善が確認された。これらの知見を通じて,限られた計算資源でも堅牢なベースラインを構築するための指針を示す。
\end{abstract}

\begin{IEEEkeywords}
画像分類, 深層学習, Vision Transformer, Test-Time Augmentation, CIFAR-10
\end{IEEEkeywords}

% !TEX program = xelatex
\documentclass[conference]{IEEEtran}

% 日本語出力と XeLaTeX 対応設定
\usepackage{xeCJK}
\usepackage{fontspec}
\setCJKmainfont{HaranoAjiMincho}

% 数式・図表・参考文献などの標準パッケージ
\usepackage{amsmath, amssymb}
\usepackage{graphicx}
\usepackage{url}
\usepackage{hyperref}

\hypersetup{
  colorlinks=true,
  linkcolor=blue,
  citecolor=blue,
  urlcolor=blue
}

\begin{document}

\title{軽量 CNN と Vision Transformer の公平比較に向けた実験的評価}

\author{\IEEEauthorblockN{著者 太郎\\ }
\IEEEauthorblockA{所属機関\\ 連絡先: author@example.com}}

\maketitle

\begin{abstract}
本稿では,軽量な畳み込みニューラルネットワーク(CNN)と Vision Transformer(ViT)の画像分類性能を,CIFAR-10 を対象に公平に比較する実験プロトコルを提示する。学習率スケジューリング,データ拡張,Test-Time Augmentation (TTA) を統一的に適用し,再現性を重視した評価を実施した。実験の結果,ViT-Ti は TTA により最大で +1.8pt の精度向上を得た一方,ResNet-18 では RandAugment を中心としたデータ拡張で +2.1pt の改善が確認された。これらの知見を通じて,限られた計算資源でも堅牢なベースラインを構築するための指針を示す。
\end{abstract}

\begin{IEEEkeywords}
画像分類, 深層学習, Vision Transformer, Test-Time Augmentation, CIFAR-10
\end{IEEEkeywords}

\input{01_introduction/main}
\input{02_related_work/main}
\input{03_method/main}
\input{04_results/main}
\input{05_discussion/main}
\input{06_conclusion/main}
\input{appendix/main}

\bibliographystyle{ieeetr}
\bibliography{references}

\end{document}

% !TEX program = xelatex
\documentclass[conference]{IEEEtran}

% 日本語出力と XeLaTeX 対応設定
\usepackage{xeCJK}
\usepackage{fontspec}
\setCJKmainfont{HaranoAjiMincho}

% 数式・図表・参考文献などの標準パッケージ
\usepackage{amsmath, amssymb}
\usepackage{graphicx}
\usepackage{url}
\usepackage{hyperref}

\hypersetup{
  colorlinks=true,
  linkcolor=blue,
  citecolor=blue,
  urlcolor=blue
}

\begin{document}

\title{軽量 CNN と Vision Transformer の公平比較に向けた実験的評価}

\author{\IEEEauthorblockN{著者 太郎\\ }
\IEEEauthorblockA{所属機関\\ 連絡先: author@example.com}}

\maketitle

\begin{abstract}
本稿では,軽量な畳み込みニューラルネットワーク(CNN)と Vision Transformer(ViT)の画像分類性能を,CIFAR-10 を対象に公平に比較する実験プロトコルを提示する。学習率スケジューリング,データ拡張,Test-Time Augmentation (TTA) を統一的に適用し,再現性を重視した評価を実施した。実験の結果,ViT-Ti は TTA により最大で +1.8pt の精度向上を得た一方,ResNet-18 では RandAugment を中心としたデータ拡張で +2.1pt の改善が確認された。これらの知見を通じて,限られた計算資源でも堅牢なベースラインを構築するための指針を示す。
\end{abstract}

\begin{IEEEkeywords}
画像分類, 深層学習, Vision Transformer, Test-Time Augmentation, CIFAR-10
\end{IEEEkeywords}

\input{01_introduction/main}
\input{02_related_work/main}
\input{03_method/main}
\input{04_results/main}
\input{05_discussion/main}
\input{06_conclusion/main}
\input{appendix/main}

\bibliographystyle{ieeetr}
\bibliography{references}

\end{document}

% !TEX program = xelatex
\documentclass[conference]{IEEEtran}

% 日本語出力と XeLaTeX 対応設定
\usepackage{xeCJK}
\usepackage{fontspec}
\setCJKmainfont{HaranoAjiMincho}

% 数式・図表・参考文献などの標準パッケージ
\usepackage{amsmath, amssymb}
\usepackage{graphicx}
\usepackage{url}
\usepackage{hyperref}

\hypersetup{
  colorlinks=true,
  linkcolor=blue,
  citecolor=blue,
  urlcolor=blue
}

\begin{document}

\title{軽量 CNN と Vision Transformer の公平比較に向けた実験的評価}

\author{\IEEEauthorblockN{著者 太郎\\ }
\IEEEauthorblockA{所属機関\\ 連絡先: author@example.com}}

\maketitle

\begin{abstract}
本稿では,軽量な畳み込みニューラルネットワーク(CNN)と Vision Transformer(ViT)の画像分類性能を,CIFAR-10 を対象に公平に比較する実験プロトコルを提示する。学習率スケジューリング,データ拡張,Test-Time Augmentation (TTA) を統一的に適用し,再現性を重視した評価を実施した。実験の結果,ViT-Ti は TTA により最大で +1.8pt の精度向上を得た一方,ResNet-18 では RandAugment を中心としたデータ拡張で +2.1pt の改善が確認された。これらの知見を通じて,限られた計算資源でも堅牢なベースラインを構築するための指針を示す。
\end{abstract}

\begin{IEEEkeywords}
画像分類, 深層学習, Vision Transformer, Test-Time Augmentation, CIFAR-10
\end{IEEEkeywords}

\input{01_introduction/main}
\input{02_related_work/main}
\input{03_method/main}
\input{04_results/main}
\input{05_discussion/main}
\input{06_conclusion/main}
\input{appendix/main}

\bibliographystyle{ieeetr}
\bibliography{references}

\end{document}

% !TEX program = xelatex
\documentclass[conference]{IEEEtran}

% 日本語出力と XeLaTeX 対応設定
\usepackage{xeCJK}
\usepackage{fontspec}
\setCJKmainfont{HaranoAjiMincho}

% 数式・図表・参考文献などの標準パッケージ
\usepackage{amsmath, amssymb}
\usepackage{graphicx}
\usepackage{url}
\usepackage{hyperref}

\hypersetup{
  colorlinks=true,
  linkcolor=blue,
  citecolor=blue,
  urlcolor=blue
}

\begin{document}

\title{軽量 CNN と Vision Transformer の公平比較に向けた実験的評価}

\author{\IEEEauthorblockN{著者 太郎\\ }
\IEEEauthorblockA{所属機関\\ 連絡先: author@example.com}}

\maketitle

\begin{abstract}
本稿では,軽量な畳み込みニューラルネットワーク(CNN)と Vision Transformer(ViT)の画像分類性能を,CIFAR-10 を対象に公平に比較する実験プロトコルを提示する。学習率スケジューリング,データ拡張,Test-Time Augmentation (TTA) を統一的に適用し,再現性を重視した評価を実施した。実験の結果,ViT-Ti は TTA により最大で +1.8pt の精度向上を得た一方,ResNet-18 では RandAugment を中心としたデータ拡張で +2.1pt の改善が確認された。これらの知見を通じて,限られた計算資源でも堅牢なベースラインを構築するための指針を示す。
\end{abstract}

\begin{IEEEkeywords}
画像分類, 深層学習, Vision Transformer, Test-Time Augmentation, CIFAR-10
\end{IEEEkeywords}

\input{01_introduction/main}
\input{02_related_work/main}
\input{03_method/main}
\input{04_results/main}
\input{05_discussion/main}
\input{06_conclusion/main}
\input{appendix/main}

\bibliographystyle{ieeetr}
\bibliography{references}

\end{document}

% !TEX program = xelatex
\documentclass[conference]{IEEEtran}

% 日本語出力と XeLaTeX 対応設定
\usepackage{xeCJK}
\usepackage{fontspec}
\setCJKmainfont{HaranoAjiMincho}

% 数式・図表・参考文献などの標準パッケージ
\usepackage{amsmath, amssymb}
\usepackage{graphicx}
\usepackage{url}
\usepackage{hyperref}

\hypersetup{
  colorlinks=true,
  linkcolor=blue,
  citecolor=blue,
  urlcolor=blue
}

\begin{document}

\title{軽量 CNN と Vision Transformer の公平比較に向けた実験的評価}

\author{\IEEEauthorblockN{著者 太郎\\ }
\IEEEauthorblockA{所属機関\\ 連絡先: author@example.com}}

\maketitle

\begin{abstract}
本稿では,軽量な畳み込みニューラルネットワーク(CNN)と Vision Transformer(ViT)の画像分類性能を,CIFAR-10 を対象に公平に比較する実験プロトコルを提示する。学習率スケジューリング,データ拡張,Test-Time Augmentation (TTA) を統一的に適用し,再現性を重視した評価を実施した。実験の結果,ViT-Ti は TTA により最大で +1.8pt の精度向上を得た一方,ResNet-18 では RandAugment を中心としたデータ拡張で +2.1pt の改善が確認された。これらの知見を通じて,限られた計算資源でも堅牢なベースラインを構築するための指針を示す。
\end{abstract}

\begin{IEEEkeywords}
画像分類, 深層学習, Vision Transformer, Test-Time Augmentation, CIFAR-10
\end{IEEEkeywords}

\input{01_introduction/main}
\input{02_related_work/main}
\input{03_method/main}
\input{04_results/main}
\input{05_discussion/main}
\input{06_conclusion/main}
\input{appendix/main}

\bibliographystyle{ieeetr}
\bibliography{references}

\end{document}

% !TEX program = xelatex
\documentclass[conference]{IEEEtran}

% 日本語出力と XeLaTeX 対応設定
\usepackage{xeCJK}
\usepackage{fontspec}
\setCJKmainfont{HaranoAjiMincho}

% 数式・図表・参考文献などの標準パッケージ
\usepackage{amsmath, amssymb}
\usepackage{graphicx}
\usepackage{url}
\usepackage{hyperref}

\hypersetup{
  colorlinks=true,
  linkcolor=blue,
  citecolor=blue,
  urlcolor=blue
}

\begin{document}

\title{軽量 CNN と Vision Transformer の公平比較に向けた実験的評価}

\author{\IEEEauthorblockN{著者 太郎\\ }
\IEEEauthorblockA{所属機関\\ 連絡先: author@example.com}}

\maketitle

\begin{abstract}
本稿では,軽量な畳み込みニューラルネットワーク(CNN)と Vision Transformer(ViT)の画像分類性能を,CIFAR-10 を対象に公平に比較する実験プロトコルを提示する。学習率スケジューリング,データ拡張,Test-Time Augmentation (TTA) を統一的に適用し,再現性を重視した評価を実施した。実験の結果,ViT-Ti は TTA により最大で +1.8pt の精度向上を得た一方,ResNet-18 では RandAugment を中心としたデータ拡張で +2.1pt の改善が確認された。これらの知見を通じて,限られた計算資源でも堅牢なベースラインを構築するための指針を示す。
\end{abstract}

\begin{IEEEkeywords}
画像分類, 深層学習, Vision Transformer, Test-Time Augmentation, CIFAR-10
\end{IEEEkeywords}

\input{01_introduction/main}
\input{02_related_work/main}
\input{03_method/main}
\input{04_results/main}
\input{05_discussion/main}
\input{06_conclusion/main}
\input{appendix/main}

\bibliographystyle{ieeetr}
\bibliography{references}

\end{document}

% !TEX program = xelatex
\documentclass[conference]{IEEEtran}

% 日本語出力と XeLaTeX 対応設定
\usepackage{xeCJK}
\usepackage{fontspec}
\setCJKmainfont{HaranoAjiMincho}

% 数式・図表・参考文献などの標準パッケージ
\usepackage{amsmath, amssymb}
\usepackage{graphicx}
\usepackage{url}
\usepackage{hyperref}

\hypersetup{
  colorlinks=true,
  linkcolor=blue,
  citecolor=blue,
  urlcolor=blue
}

\begin{document}

\title{軽量 CNN と Vision Transformer の公平比較に向けた実験的評価}

\author{\IEEEauthorblockN{著者 太郎\\ }
\IEEEauthorblockA{所属機関\\ 連絡先: author@example.com}}

\maketitle

\begin{abstract}
本稿では,軽量な畳み込みニューラルネットワーク(CNN)と Vision Transformer(ViT)の画像分類性能を,CIFAR-10 を対象に公平に比較する実験プロトコルを提示する。学習率スケジューリング,データ拡張,Test-Time Augmentation (TTA) を統一的に適用し,再現性を重視した評価を実施した。実験の結果,ViT-Ti は TTA により最大で +1.8pt の精度向上を得た一方,ResNet-18 では RandAugment を中心としたデータ拡張で +2.1pt の改善が確認された。これらの知見を通じて,限られた計算資源でも堅牢なベースラインを構築するための指針を示す。
\end{abstract}

\begin{IEEEkeywords}
画像分類, 深層学習, Vision Transformer, Test-Time Augmentation, CIFAR-10
\end{IEEEkeywords}

\input{01_introduction/main}
\input{02_related_work/main}
\input{03_method/main}
\input{04_results/main}
\input{05_discussion/main}
\input{06_conclusion/main}
\input{appendix/main}

\bibliographystyle{ieeetr}
\bibliography{references}

\end{document}


\bibliographystyle{ieeetr}
\bibliography{references}

\end{document}


\bibliographystyle{ieeetr}
\bibliography{references}

\end{document}

% !TEX program = xelatex
\documentclass[conference]{IEEEtran}

% 日本語出力と XeLaTeX 対応設定
\usepackage{xeCJK}
\usepackage{fontspec}
\setCJKmainfont{HaranoAjiMincho}

% 数式・図表・参考文献などの標準パッケージ
\usepackage{amsmath, amssymb}
\usepackage{graphicx}
\usepackage{url}
\usepackage{hyperref}

\hypersetup{
  colorlinks=true,
  linkcolor=blue,
  citecolor=blue,
  urlcolor=blue
}

\begin{document}

\title{軽量 CNN と Vision Transformer の公平比較に向けた実験的評価}

\author{\IEEEauthorblockN{著者 太郎\\ }
\IEEEauthorblockA{所属機関\\ 連絡先: author@example.com}}

\maketitle

\begin{abstract}
本稿では,軽量な畳み込みニューラルネットワーク(CNN)と Vision Transformer(ViT)の画像分類性能を,CIFAR-10 を対象に公平に比較する実験プロトコルを提示する。学習率スケジューリング,データ拡張,Test-Time Augmentation (TTA) を統一的に適用し,再現性を重視した評価を実施した。実験の結果,ViT-Ti は TTA により最大で +1.8pt の精度向上を得た一方,ResNet-18 では RandAugment を中心としたデータ拡張で +2.1pt の改善が確認された。これらの知見を通じて,限られた計算資源でも堅牢なベースラインを構築するための指針を示す。
\end{abstract}

\begin{IEEEkeywords}
画像分類, 深層学習, Vision Transformer, Test-Time Augmentation, CIFAR-10
\end{IEEEkeywords}

% !TEX program = xelatex
\documentclass[conference]{IEEEtran}

% 日本語出力と XeLaTeX 対応設定
\usepackage{xeCJK}
\usepackage{fontspec}
\setCJKmainfont{HaranoAjiMincho}

% 数式・図表・参考文献などの標準パッケージ
\usepackage{amsmath, amssymb}
\usepackage{graphicx}
\usepackage{url}
\usepackage{hyperref}

\hypersetup{
  colorlinks=true,
  linkcolor=blue,
  citecolor=blue,
  urlcolor=blue
}

\begin{document}

\title{軽量 CNN と Vision Transformer の公平比較に向けた実験的評価}

\author{\IEEEauthorblockN{著者 太郎\\ }
\IEEEauthorblockA{所属機関\\ 連絡先: author@example.com}}

\maketitle

\begin{abstract}
本稿では,軽量な畳み込みニューラルネットワーク(CNN)と Vision Transformer(ViT)の画像分類性能を,CIFAR-10 を対象に公平に比較する実験プロトコルを提示する。学習率スケジューリング,データ拡張,Test-Time Augmentation (TTA) を統一的に適用し,再現性を重視した評価を実施した。実験の結果,ViT-Ti は TTA により最大で +1.8pt の精度向上を得た一方,ResNet-18 では RandAugment を中心としたデータ拡張で +2.1pt の改善が確認された。これらの知見を通じて,限られた計算資源でも堅牢なベースラインを構築するための指針を示す。
\end{abstract}

\begin{IEEEkeywords}
画像分類, 深層学習, Vision Transformer, Test-Time Augmentation, CIFAR-10
\end{IEEEkeywords}

% !TEX program = xelatex
\documentclass[conference]{IEEEtran}

% 日本語出力と XeLaTeX 対応設定
\usepackage{xeCJK}
\usepackage{fontspec}
\setCJKmainfont{HaranoAjiMincho}

% 数式・図表・参考文献などの標準パッケージ
\usepackage{amsmath, amssymb}
\usepackage{graphicx}
\usepackage{url}
\usepackage{hyperref}

\hypersetup{
  colorlinks=true,
  linkcolor=blue,
  citecolor=blue,
  urlcolor=blue
}

\begin{document}

\title{軽量 CNN と Vision Transformer の公平比較に向けた実験的評価}

\author{\IEEEauthorblockN{著者 太郎\\ }
\IEEEauthorblockA{所属機関\\ 連絡先: author@example.com}}

\maketitle

\begin{abstract}
本稿では,軽量な畳み込みニューラルネットワーク(CNN)と Vision Transformer(ViT)の画像分類性能を,CIFAR-10 を対象に公平に比較する実験プロトコルを提示する。学習率スケジューリング,データ拡張,Test-Time Augmentation (TTA) を統一的に適用し,再現性を重視した評価を実施した。実験の結果,ViT-Ti は TTA により最大で +1.8pt の精度向上を得た一方,ResNet-18 では RandAugment を中心としたデータ拡張で +2.1pt の改善が確認された。これらの知見を通じて,限られた計算資源でも堅牢なベースラインを構築するための指針を示す。
\end{abstract}

\begin{IEEEkeywords}
画像分類, 深層学習, Vision Transformer, Test-Time Augmentation, CIFAR-10
\end{IEEEkeywords}

\input{01_introduction/main}
\input{02_related_work/main}
\input{03_method/main}
\input{04_results/main}
\input{05_discussion/main}
\input{06_conclusion/main}
\input{appendix/main}

\bibliographystyle{ieeetr}
\bibliography{references}

\end{document}

% !TEX program = xelatex
\documentclass[conference]{IEEEtran}

% 日本語出力と XeLaTeX 対応設定
\usepackage{xeCJK}
\usepackage{fontspec}
\setCJKmainfont{HaranoAjiMincho}

% 数式・図表・参考文献などの標準パッケージ
\usepackage{amsmath, amssymb}
\usepackage{graphicx}
\usepackage{url}
\usepackage{hyperref}

\hypersetup{
  colorlinks=true,
  linkcolor=blue,
  citecolor=blue,
  urlcolor=blue
}

\begin{document}

\title{軽量 CNN と Vision Transformer の公平比較に向けた実験的評価}

\author{\IEEEauthorblockN{著者 太郎\\ }
\IEEEauthorblockA{所属機関\\ 連絡先: author@example.com}}

\maketitle

\begin{abstract}
本稿では,軽量な畳み込みニューラルネットワーク(CNN)と Vision Transformer(ViT)の画像分類性能を,CIFAR-10 を対象に公平に比較する実験プロトコルを提示する。学習率スケジューリング,データ拡張,Test-Time Augmentation (TTA) を統一的に適用し,再現性を重視した評価を実施した。実験の結果,ViT-Ti は TTA により最大で +1.8pt の精度向上を得た一方,ResNet-18 では RandAugment を中心としたデータ拡張で +2.1pt の改善が確認された。これらの知見を通じて,限られた計算資源でも堅牢なベースラインを構築するための指針を示す。
\end{abstract}

\begin{IEEEkeywords}
画像分類, 深層学習, Vision Transformer, Test-Time Augmentation, CIFAR-10
\end{IEEEkeywords}

\input{01_introduction/main}
\input{02_related_work/main}
\input{03_method/main}
\input{04_results/main}
\input{05_discussion/main}
\input{06_conclusion/main}
\input{appendix/main}

\bibliographystyle{ieeetr}
\bibliography{references}

\end{document}

% !TEX program = xelatex
\documentclass[conference]{IEEEtran}

% 日本語出力と XeLaTeX 対応設定
\usepackage{xeCJK}
\usepackage{fontspec}
\setCJKmainfont{HaranoAjiMincho}

% 数式・図表・参考文献などの標準パッケージ
\usepackage{amsmath, amssymb}
\usepackage{graphicx}
\usepackage{url}
\usepackage{hyperref}

\hypersetup{
  colorlinks=true,
  linkcolor=blue,
  citecolor=blue,
  urlcolor=blue
}

\begin{document}

\title{軽量 CNN と Vision Transformer の公平比較に向けた実験的評価}

\author{\IEEEauthorblockN{著者 太郎\\ }
\IEEEauthorblockA{所属機関\\ 連絡先: author@example.com}}

\maketitle

\begin{abstract}
本稿では,軽量な畳み込みニューラルネットワーク(CNN)と Vision Transformer(ViT)の画像分類性能を,CIFAR-10 を対象に公平に比較する実験プロトコルを提示する。学習率スケジューリング,データ拡張,Test-Time Augmentation (TTA) を統一的に適用し,再現性を重視した評価を実施した。実験の結果,ViT-Ti は TTA により最大で +1.8pt の精度向上を得た一方,ResNet-18 では RandAugment を中心としたデータ拡張で +2.1pt の改善が確認された。これらの知見を通じて,限られた計算資源でも堅牢なベースラインを構築するための指針を示す。
\end{abstract}

\begin{IEEEkeywords}
画像分類, 深層学習, Vision Transformer, Test-Time Augmentation, CIFAR-10
\end{IEEEkeywords}

\input{01_introduction/main}
\input{02_related_work/main}
\input{03_method/main}
\input{04_results/main}
\input{05_discussion/main}
\input{06_conclusion/main}
\input{appendix/main}

\bibliographystyle{ieeetr}
\bibliography{references}

\end{document}

% !TEX program = xelatex
\documentclass[conference]{IEEEtran}

% 日本語出力と XeLaTeX 対応設定
\usepackage{xeCJK}
\usepackage{fontspec}
\setCJKmainfont{HaranoAjiMincho}

% 数式・図表・参考文献などの標準パッケージ
\usepackage{amsmath, amssymb}
\usepackage{graphicx}
\usepackage{url}
\usepackage{hyperref}

\hypersetup{
  colorlinks=true,
  linkcolor=blue,
  citecolor=blue,
  urlcolor=blue
}

\begin{document}

\title{軽量 CNN と Vision Transformer の公平比較に向けた実験的評価}

\author{\IEEEauthorblockN{著者 太郎\\ }
\IEEEauthorblockA{所属機関\\ 連絡先: author@example.com}}

\maketitle

\begin{abstract}
本稿では,軽量な畳み込みニューラルネットワーク(CNN)と Vision Transformer(ViT)の画像分類性能を,CIFAR-10 を対象に公平に比較する実験プロトコルを提示する。学習率スケジューリング,データ拡張,Test-Time Augmentation (TTA) を統一的に適用し,再現性を重視した評価を実施した。実験の結果,ViT-Ti は TTA により最大で +1.8pt の精度向上を得た一方,ResNet-18 では RandAugment を中心としたデータ拡張で +2.1pt の改善が確認された。これらの知見を通じて,限られた計算資源でも堅牢なベースラインを構築するための指針を示す。
\end{abstract}

\begin{IEEEkeywords}
画像分類, 深層学習, Vision Transformer, Test-Time Augmentation, CIFAR-10
\end{IEEEkeywords}

\input{01_introduction/main}
\input{02_related_work/main}
\input{03_method/main}
\input{04_results/main}
\input{05_discussion/main}
\input{06_conclusion/main}
\input{appendix/main}

\bibliographystyle{ieeetr}
\bibliography{references}

\end{document}

% !TEX program = xelatex
\documentclass[conference]{IEEEtran}

% 日本語出力と XeLaTeX 対応設定
\usepackage{xeCJK}
\usepackage{fontspec}
\setCJKmainfont{HaranoAjiMincho}

% 数式・図表・参考文献などの標準パッケージ
\usepackage{amsmath, amssymb}
\usepackage{graphicx}
\usepackage{url}
\usepackage{hyperref}

\hypersetup{
  colorlinks=true,
  linkcolor=blue,
  citecolor=blue,
  urlcolor=blue
}

\begin{document}

\title{軽量 CNN と Vision Transformer の公平比較に向けた実験的評価}

\author{\IEEEauthorblockN{著者 太郎\\ }
\IEEEauthorblockA{所属機関\\ 連絡先: author@example.com}}

\maketitle

\begin{abstract}
本稿では,軽量な畳み込みニューラルネットワーク(CNN)と Vision Transformer(ViT)の画像分類性能を,CIFAR-10 を対象に公平に比較する実験プロトコルを提示する。学習率スケジューリング,データ拡張,Test-Time Augmentation (TTA) を統一的に適用し,再現性を重視した評価を実施した。実験の結果,ViT-Ti は TTA により最大で +1.8pt の精度向上を得た一方,ResNet-18 では RandAugment を中心としたデータ拡張で +2.1pt の改善が確認された。これらの知見を通じて,限られた計算資源でも堅牢なベースラインを構築するための指針を示す。
\end{abstract}

\begin{IEEEkeywords}
画像分類, 深層学習, Vision Transformer, Test-Time Augmentation, CIFAR-10
\end{IEEEkeywords}

\input{01_introduction/main}
\input{02_related_work/main}
\input{03_method/main}
\input{04_results/main}
\input{05_discussion/main}
\input{06_conclusion/main}
\input{appendix/main}

\bibliographystyle{ieeetr}
\bibliography{references}

\end{document}

% !TEX program = xelatex
\documentclass[conference]{IEEEtran}

% 日本語出力と XeLaTeX 対応設定
\usepackage{xeCJK}
\usepackage{fontspec}
\setCJKmainfont{HaranoAjiMincho}

% 数式・図表・参考文献などの標準パッケージ
\usepackage{amsmath, amssymb}
\usepackage{graphicx}
\usepackage{url}
\usepackage{hyperref}

\hypersetup{
  colorlinks=true,
  linkcolor=blue,
  citecolor=blue,
  urlcolor=blue
}

\begin{document}

\title{軽量 CNN と Vision Transformer の公平比較に向けた実験的評価}

\author{\IEEEauthorblockN{著者 太郎\\ }
\IEEEauthorblockA{所属機関\\ 連絡先: author@example.com}}

\maketitle

\begin{abstract}
本稿では,軽量な畳み込みニューラルネットワーク(CNN)と Vision Transformer(ViT)の画像分類性能を,CIFAR-10 を対象に公平に比較する実験プロトコルを提示する。学習率スケジューリング,データ拡張,Test-Time Augmentation (TTA) を統一的に適用し,再現性を重視した評価を実施した。実験の結果,ViT-Ti は TTA により最大で +1.8pt の精度向上を得た一方,ResNet-18 では RandAugment を中心としたデータ拡張で +2.1pt の改善が確認された。これらの知見を通じて,限られた計算資源でも堅牢なベースラインを構築するための指針を示す。
\end{abstract}

\begin{IEEEkeywords}
画像分類, 深層学習, Vision Transformer, Test-Time Augmentation, CIFAR-10
\end{IEEEkeywords}

\input{01_introduction/main}
\input{02_related_work/main}
\input{03_method/main}
\input{04_results/main}
\input{05_discussion/main}
\input{06_conclusion/main}
\input{appendix/main}

\bibliographystyle{ieeetr}
\bibliography{references}

\end{document}

% !TEX program = xelatex
\documentclass[conference]{IEEEtran}

% 日本語出力と XeLaTeX 対応設定
\usepackage{xeCJK}
\usepackage{fontspec}
\setCJKmainfont{HaranoAjiMincho}

% 数式・図表・参考文献などの標準パッケージ
\usepackage{amsmath, amssymb}
\usepackage{graphicx}
\usepackage{url}
\usepackage{hyperref}

\hypersetup{
  colorlinks=true,
  linkcolor=blue,
  citecolor=blue,
  urlcolor=blue
}

\begin{document}

\title{軽量 CNN と Vision Transformer の公平比較に向けた実験的評価}

\author{\IEEEauthorblockN{著者 太郎\\ }
\IEEEauthorblockA{所属機関\\ 連絡先: author@example.com}}

\maketitle

\begin{abstract}
本稿では,軽量な畳み込みニューラルネットワーク(CNN)と Vision Transformer(ViT)の画像分類性能を,CIFAR-10 を対象に公平に比較する実験プロトコルを提示する。学習率スケジューリング,データ拡張,Test-Time Augmentation (TTA) を統一的に適用し,再現性を重視した評価を実施した。実験の結果,ViT-Ti は TTA により最大で +1.8pt の精度向上を得た一方,ResNet-18 では RandAugment を中心としたデータ拡張で +2.1pt の改善が確認された。これらの知見を通じて,限られた計算資源でも堅牢なベースラインを構築するための指針を示す。
\end{abstract}

\begin{IEEEkeywords}
画像分類, 深層学習, Vision Transformer, Test-Time Augmentation, CIFAR-10
\end{IEEEkeywords}

\input{01_introduction/main}
\input{02_related_work/main}
\input{03_method/main}
\input{04_results/main}
\input{05_discussion/main}
\input{06_conclusion/main}
\input{appendix/main}

\bibliographystyle{ieeetr}
\bibliography{references}

\end{document}


\bibliographystyle{ieeetr}
\bibliography{references}

\end{document}

% !TEX program = xelatex
\documentclass[conference]{IEEEtran}

% 日本語出力と XeLaTeX 対応設定
\usepackage{xeCJK}
\usepackage{fontspec}
\setCJKmainfont{HaranoAjiMincho}

% 数式・図表・参考文献などの標準パッケージ
\usepackage{amsmath, amssymb}
\usepackage{graphicx}
\usepackage{url}
\usepackage{hyperref}

\hypersetup{
  colorlinks=true,
  linkcolor=blue,
  citecolor=blue,
  urlcolor=blue
}

\begin{document}

\title{軽量 CNN と Vision Transformer の公平比較に向けた実験的評価}

\author{\IEEEauthorblockN{著者 太郎\\ }
\IEEEauthorblockA{所属機関\\ 連絡先: author@example.com}}

\maketitle

\begin{abstract}
本稿では,軽量な畳み込みニューラルネットワーク(CNN)と Vision Transformer(ViT)の画像分類性能を,CIFAR-10 を対象に公平に比較する実験プロトコルを提示する。学習率スケジューリング,データ拡張,Test-Time Augmentation (TTA) を統一的に適用し,再現性を重視した評価を実施した。実験の結果,ViT-Ti は TTA により最大で +1.8pt の精度向上を得た一方,ResNet-18 では RandAugment を中心としたデータ拡張で +2.1pt の改善が確認された。これらの知見を通じて,限られた計算資源でも堅牢なベースラインを構築するための指針を示す。
\end{abstract}

\begin{IEEEkeywords}
画像分類, 深層学習, Vision Transformer, Test-Time Augmentation, CIFAR-10
\end{IEEEkeywords}

% !TEX program = xelatex
\documentclass[conference]{IEEEtran}

% 日本語出力と XeLaTeX 対応設定
\usepackage{xeCJK}
\usepackage{fontspec}
\setCJKmainfont{HaranoAjiMincho}

% 数式・図表・参考文献などの標準パッケージ
\usepackage{amsmath, amssymb}
\usepackage{graphicx}
\usepackage{url}
\usepackage{hyperref}

\hypersetup{
  colorlinks=true,
  linkcolor=blue,
  citecolor=blue,
  urlcolor=blue
}

\begin{document}

\title{軽量 CNN と Vision Transformer の公平比較に向けた実験的評価}

\author{\IEEEauthorblockN{著者 太郎\\ }
\IEEEauthorblockA{所属機関\\ 連絡先: author@example.com}}

\maketitle

\begin{abstract}
本稿では,軽量な畳み込みニューラルネットワーク(CNN)と Vision Transformer(ViT)の画像分類性能を,CIFAR-10 を対象に公平に比較する実験プロトコルを提示する。学習率スケジューリング,データ拡張,Test-Time Augmentation (TTA) を統一的に適用し,再現性を重視した評価を実施した。実験の結果,ViT-Ti は TTA により最大で +1.8pt の精度向上を得た一方,ResNet-18 では RandAugment を中心としたデータ拡張で +2.1pt の改善が確認された。これらの知見を通じて,限られた計算資源でも堅牢なベースラインを構築するための指針を示す。
\end{abstract}

\begin{IEEEkeywords}
画像分類, 深層学習, Vision Transformer, Test-Time Augmentation, CIFAR-10
\end{IEEEkeywords}

\input{01_introduction/main}
\input{02_related_work/main}
\input{03_method/main}
\input{04_results/main}
\input{05_discussion/main}
\input{06_conclusion/main}
\input{appendix/main}

\bibliographystyle{ieeetr}
\bibliography{references}

\end{document}

% !TEX program = xelatex
\documentclass[conference]{IEEEtran}

% 日本語出力と XeLaTeX 対応設定
\usepackage{xeCJK}
\usepackage{fontspec}
\setCJKmainfont{HaranoAjiMincho}

% 数式・図表・参考文献などの標準パッケージ
\usepackage{amsmath, amssymb}
\usepackage{graphicx}
\usepackage{url}
\usepackage{hyperref}

\hypersetup{
  colorlinks=true,
  linkcolor=blue,
  citecolor=blue,
  urlcolor=blue
}

\begin{document}

\title{軽量 CNN と Vision Transformer の公平比較に向けた実験的評価}

\author{\IEEEauthorblockN{著者 太郎\\ }
\IEEEauthorblockA{所属機関\\ 連絡先: author@example.com}}

\maketitle

\begin{abstract}
本稿では,軽量な畳み込みニューラルネットワーク(CNN)と Vision Transformer(ViT)の画像分類性能を,CIFAR-10 を対象に公平に比較する実験プロトコルを提示する。学習率スケジューリング,データ拡張,Test-Time Augmentation (TTA) を統一的に適用し,再現性を重視した評価を実施した。実験の結果,ViT-Ti は TTA により最大で +1.8pt の精度向上を得た一方,ResNet-18 では RandAugment を中心としたデータ拡張で +2.1pt の改善が確認された。これらの知見を通じて,限られた計算資源でも堅牢なベースラインを構築するための指針を示す。
\end{abstract}

\begin{IEEEkeywords}
画像分類, 深層学習, Vision Transformer, Test-Time Augmentation, CIFAR-10
\end{IEEEkeywords}

\input{01_introduction/main}
\input{02_related_work/main}
\input{03_method/main}
\input{04_results/main}
\input{05_discussion/main}
\input{06_conclusion/main}
\input{appendix/main}

\bibliographystyle{ieeetr}
\bibliography{references}

\end{document}

% !TEX program = xelatex
\documentclass[conference]{IEEEtran}

% 日本語出力と XeLaTeX 対応設定
\usepackage{xeCJK}
\usepackage{fontspec}
\setCJKmainfont{HaranoAjiMincho}

% 数式・図表・参考文献などの標準パッケージ
\usepackage{amsmath, amssymb}
\usepackage{graphicx}
\usepackage{url}
\usepackage{hyperref}

\hypersetup{
  colorlinks=true,
  linkcolor=blue,
  citecolor=blue,
  urlcolor=blue
}

\begin{document}

\title{軽量 CNN と Vision Transformer の公平比較に向けた実験的評価}

\author{\IEEEauthorblockN{著者 太郎\\ }
\IEEEauthorblockA{所属機関\\ 連絡先: author@example.com}}

\maketitle

\begin{abstract}
本稿では,軽量な畳み込みニューラルネットワーク(CNN)と Vision Transformer(ViT)の画像分類性能を,CIFAR-10 を対象に公平に比較する実験プロトコルを提示する。学習率スケジューリング,データ拡張,Test-Time Augmentation (TTA) を統一的に適用し,再現性を重視した評価を実施した。実験の結果,ViT-Ti は TTA により最大で +1.8pt の精度向上を得た一方,ResNet-18 では RandAugment を中心としたデータ拡張で +2.1pt の改善が確認された。これらの知見を通じて,限られた計算資源でも堅牢なベースラインを構築するための指針を示す。
\end{abstract}

\begin{IEEEkeywords}
画像分類, 深層学習, Vision Transformer, Test-Time Augmentation, CIFAR-10
\end{IEEEkeywords}

\input{01_introduction/main}
\input{02_related_work/main}
\input{03_method/main}
\input{04_results/main}
\input{05_discussion/main}
\input{06_conclusion/main}
\input{appendix/main}

\bibliographystyle{ieeetr}
\bibliography{references}

\end{document}

% !TEX program = xelatex
\documentclass[conference]{IEEEtran}

% 日本語出力と XeLaTeX 対応設定
\usepackage{xeCJK}
\usepackage{fontspec}
\setCJKmainfont{HaranoAjiMincho}

% 数式・図表・参考文献などの標準パッケージ
\usepackage{amsmath, amssymb}
\usepackage{graphicx}
\usepackage{url}
\usepackage{hyperref}

\hypersetup{
  colorlinks=true,
  linkcolor=blue,
  citecolor=blue,
  urlcolor=blue
}

\begin{document}

\title{軽量 CNN と Vision Transformer の公平比較に向けた実験的評価}

\author{\IEEEauthorblockN{著者 太郎\\ }
\IEEEauthorblockA{所属機関\\ 連絡先: author@example.com}}

\maketitle

\begin{abstract}
本稿では,軽量な畳み込みニューラルネットワーク(CNN)と Vision Transformer(ViT)の画像分類性能を,CIFAR-10 を対象に公平に比較する実験プロトコルを提示する。学習率スケジューリング,データ拡張,Test-Time Augmentation (TTA) を統一的に適用し,再現性を重視した評価を実施した。実験の結果,ViT-Ti は TTA により最大で +1.8pt の精度向上を得た一方,ResNet-18 では RandAugment を中心としたデータ拡張で +2.1pt の改善が確認された。これらの知見を通じて,限られた計算資源でも堅牢なベースラインを構築するための指針を示す。
\end{abstract}

\begin{IEEEkeywords}
画像分類, 深層学習, Vision Transformer, Test-Time Augmentation, CIFAR-10
\end{IEEEkeywords}

\input{01_introduction/main}
\input{02_related_work/main}
\input{03_method/main}
\input{04_results/main}
\input{05_discussion/main}
\input{06_conclusion/main}
\input{appendix/main}

\bibliographystyle{ieeetr}
\bibliography{references}

\end{document}

% !TEX program = xelatex
\documentclass[conference]{IEEEtran}

% 日本語出力と XeLaTeX 対応設定
\usepackage{xeCJK}
\usepackage{fontspec}
\setCJKmainfont{HaranoAjiMincho}

% 数式・図表・参考文献などの標準パッケージ
\usepackage{amsmath, amssymb}
\usepackage{graphicx}
\usepackage{url}
\usepackage{hyperref}

\hypersetup{
  colorlinks=true,
  linkcolor=blue,
  citecolor=blue,
  urlcolor=blue
}

\begin{document}

\title{軽量 CNN と Vision Transformer の公平比較に向けた実験的評価}

\author{\IEEEauthorblockN{著者 太郎\\ }
\IEEEauthorblockA{所属機関\\ 連絡先: author@example.com}}

\maketitle

\begin{abstract}
本稿では,軽量な畳み込みニューラルネットワーク(CNN)と Vision Transformer(ViT)の画像分類性能を,CIFAR-10 を対象に公平に比較する実験プロトコルを提示する。学習率スケジューリング,データ拡張,Test-Time Augmentation (TTA) を統一的に適用し,再現性を重視した評価を実施した。実験の結果,ViT-Ti は TTA により最大で +1.8pt の精度向上を得た一方,ResNet-18 では RandAugment を中心としたデータ拡張で +2.1pt の改善が確認された。これらの知見を通じて,限られた計算資源でも堅牢なベースラインを構築するための指針を示す。
\end{abstract}

\begin{IEEEkeywords}
画像分類, 深層学習, Vision Transformer, Test-Time Augmentation, CIFAR-10
\end{IEEEkeywords}

\input{01_introduction/main}
\input{02_related_work/main}
\input{03_method/main}
\input{04_results/main}
\input{05_discussion/main}
\input{06_conclusion/main}
\input{appendix/main}

\bibliographystyle{ieeetr}
\bibliography{references}

\end{document}

% !TEX program = xelatex
\documentclass[conference]{IEEEtran}

% 日本語出力と XeLaTeX 対応設定
\usepackage{xeCJK}
\usepackage{fontspec}
\setCJKmainfont{HaranoAjiMincho}

% 数式・図表・参考文献などの標準パッケージ
\usepackage{amsmath, amssymb}
\usepackage{graphicx}
\usepackage{url}
\usepackage{hyperref}

\hypersetup{
  colorlinks=true,
  linkcolor=blue,
  citecolor=blue,
  urlcolor=blue
}

\begin{document}

\title{軽量 CNN と Vision Transformer の公平比較に向けた実験的評価}

\author{\IEEEauthorblockN{著者 太郎\\ }
\IEEEauthorblockA{所属機関\\ 連絡先: author@example.com}}

\maketitle

\begin{abstract}
本稿では,軽量な畳み込みニューラルネットワーク(CNN)と Vision Transformer(ViT)の画像分類性能を,CIFAR-10 を対象に公平に比較する実験プロトコルを提示する。学習率スケジューリング,データ拡張,Test-Time Augmentation (TTA) を統一的に適用し,再現性を重視した評価を実施した。実験の結果,ViT-Ti は TTA により最大で +1.8pt の精度向上を得た一方,ResNet-18 では RandAugment を中心としたデータ拡張で +2.1pt の改善が確認された。これらの知見を通じて,限られた計算資源でも堅牢なベースラインを構築するための指針を示す。
\end{abstract}

\begin{IEEEkeywords}
画像分類, 深層学習, Vision Transformer, Test-Time Augmentation, CIFAR-10
\end{IEEEkeywords}

\input{01_introduction/main}
\input{02_related_work/main}
\input{03_method/main}
\input{04_results/main}
\input{05_discussion/main}
\input{06_conclusion/main}
\input{appendix/main}

\bibliographystyle{ieeetr}
\bibliography{references}

\end{document}

% !TEX program = xelatex
\documentclass[conference]{IEEEtran}

% 日本語出力と XeLaTeX 対応設定
\usepackage{xeCJK}
\usepackage{fontspec}
\setCJKmainfont{HaranoAjiMincho}

% 数式・図表・参考文献などの標準パッケージ
\usepackage{amsmath, amssymb}
\usepackage{graphicx}
\usepackage{url}
\usepackage{hyperref}

\hypersetup{
  colorlinks=true,
  linkcolor=blue,
  citecolor=blue,
  urlcolor=blue
}

\begin{document}

\title{軽量 CNN と Vision Transformer の公平比較に向けた実験的評価}

\author{\IEEEauthorblockN{著者 太郎\\ }
\IEEEauthorblockA{所属機関\\ 連絡先: author@example.com}}

\maketitle

\begin{abstract}
本稿では,軽量な畳み込みニューラルネットワーク(CNN)と Vision Transformer(ViT)の画像分類性能を,CIFAR-10 を対象に公平に比較する実験プロトコルを提示する。学習率スケジューリング,データ拡張,Test-Time Augmentation (TTA) を統一的に適用し,再現性を重視した評価を実施した。実験の結果,ViT-Ti は TTA により最大で +1.8pt の精度向上を得た一方,ResNet-18 では RandAugment を中心としたデータ拡張で +2.1pt の改善が確認された。これらの知見を通じて,限られた計算資源でも堅牢なベースラインを構築するための指針を示す。
\end{abstract}

\begin{IEEEkeywords}
画像分類, 深層学習, Vision Transformer, Test-Time Augmentation, CIFAR-10
\end{IEEEkeywords}

\input{01_introduction/main}
\input{02_related_work/main}
\input{03_method/main}
\input{04_results/main}
\input{05_discussion/main}
\input{06_conclusion/main}
\input{appendix/main}

\bibliographystyle{ieeetr}
\bibliography{references}

\end{document}


\bibliographystyle{ieeetr}
\bibliography{references}

\end{document}

% !TEX program = xelatex
\documentclass[conference]{IEEEtran}

% 日本語出力と XeLaTeX 対応設定
\usepackage{xeCJK}
\usepackage{fontspec}
\setCJKmainfont{HaranoAjiMincho}

% 数式・図表・参考文献などの標準パッケージ
\usepackage{amsmath, amssymb}
\usepackage{graphicx}
\usepackage{url}
\usepackage{hyperref}

\hypersetup{
  colorlinks=true,
  linkcolor=blue,
  citecolor=blue,
  urlcolor=blue
}

\begin{document}

\title{軽量 CNN と Vision Transformer の公平比較に向けた実験的評価}

\author{\IEEEauthorblockN{著者 太郎\\ }
\IEEEauthorblockA{所属機関\\ 連絡先: author@example.com}}

\maketitle

\begin{abstract}
本稿では,軽量な畳み込みニューラルネットワーク(CNN)と Vision Transformer(ViT)の画像分類性能を,CIFAR-10 を対象に公平に比較する実験プロトコルを提示する。学習率スケジューリング,データ拡張,Test-Time Augmentation (TTA) を統一的に適用し,再現性を重視した評価を実施した。実験の結果,ViT-Ti は TTA により最大で +1.8pt の精度向上を得た一方,ResNet-18 では RandAugment を中心としたデータ拡張で +2.1pt の改善が確認された。これらの知見を通じて,限られた計算資源でも堅牢なベースラインを構築するための指針を示す。
\end{abstract}

\begin{IEEEkeywords}
画像分類, 深層学習, Vision Transformer, Test-Time Augmentation, CIFAR-10
\end{IEEEkeywords}

% !TEX program = xelatex
\documentclass[conference]{IEEEtran}

% 日本語出力と XeLaTeX 対応設定
\usepackage{xeCJK}
\usepackage{fontspec}
\setCJKmainfont{HaranoAjiMincho}

% 数式・図表・参考文献などの標準パッケージ
\usepackage{amsmath, amssymb}
\usepackage{graphicx}
\usepackage{url}
\usepackage{hyperref}

\hypersetup{
  colorlinks=true,
  linkcolor=blue,
  citecolor=blue,
  urlcolor=blue
}

\begin{document}

\title{軽量 CNN と Vision Transformer の公平比較に向けた実験的評価}

\author{\IEEEauthorblockN{著者 太郎\\ }
\IEEEauthorblockA{所属機関\\ 連絡先: author@example.com}}

\maketitle

\begin{abstract}
本稿では,軽量な畳み込みニューラルネットワーク(CNN)と Vision Transformer(ViT)の画像分類性能を,CIFAR-10 を対象に公平に比較する実験プロトコルを提示する。学習率スケジューリング,データ拡張,Test-Time Augmentation (TTA) を統一的に適用し,再現性を重視した評価を実施した。実験の結果,ViT-Ti は TTA により最大で +1.8pt の精度向上を得た一方,ResNet-18 では RandAugment を中心としたデータ拡張で +2.1pt の改善が確認された。これらの知見を通じて,限られた計算資源でも堅牢なベースラインを構築するための指針を示す。
\end{abstract}

\begin{IEEEkeywords}
画像分類, 深層学習, Vision Transformer, Test-Time Augmentation, CIFAR-10
\end{IEEEkeywords}

\input{01_introduction/main}
\input{02_related_work/main}
\input{03_method/main}
\input{04_results/main}
\input{05_discussion/main}
\input{06_conclusion/main}
\input{appendix/main}

\bibliographystyle{ieeetr}
\bibliography{references}

\end{document}

% !TEX program = xelatex
\documentclass[conference]{IEEEtran}

% 日本語出力と XeLaTeX 対応設定
\usepackage{xeCJK}
\usepackage{fontspec}
\setCJKmainfont{HaranoAjiMincho}

% 数式・図表・参考文献などの標準パッケージ
\usepackage{amsmath, amssymb}
\usepackage{graphicx}
\usepackage{url}
\usepackage{hyperref}

\hypersetup{
  colorlinks=true,
  linkcolor=blue,
  citecolor=blue,
  urlcolor=blue
}

\begin{document}

\title{軽量 CNN と Vision Transformer の公平比較に向けた実験的評価}

\author{\IEEEauthorblockN{著者 太郎\\ }
\IEEEauthorblockA{所属機関\\ 連絡先: author@example.com}}

\maketitle

\begin{abstract}
本稿では,軽量な畳み込みニューラルネットワーク(CNN)と Vision Transformer(ViT)の画像分類性能を,CIFAR-10 を対象に公平に比較する実験プロトコルを提示する。学習率スケジューリング,データ拡張,Test-Time Augmentation (TTA) を統一的に適用し,再現性を重視した評価を実施した。実験の結果,ViT-Ti は TTA により最大で +1.8pt の精度向上を得た一方,ResNet-18 では RandAugment を中心としたデータ拡張で +2.1pt の改善が確認された。これらの知見を通じて,限られた計算資源でも堅牢なベースラインを構築するための指針を示す。
\end{abstract}

\begin{IEEEkeywords}
画像分類, 深層学習, Vision Transformer, Test-Time Augmentation, CIFAR-10
\end{IEEEkeywords}

\input{01_introduction/main}
\input{02_related_work/main}
\input{03_method/main}
\input{04_results/main}
\input{05_discussion/main}
\input{06_conclusion/main}
\input{appendix/main}

\bibliographystyle{ieeetr}
\bibliography{references}

\end{document}

% !TEX program = xelatex
\documentclass[conference]{IEEEtran}

% 日本語出力と XeLaTeX 対応設定
\usepackage{xeCJK}
\usepackage{fontspec}
\setCJKmainfont{HaranoAjiMincho}

% 数式・図表・参考文献などの標準パッケージ
\usepackage{amsmath, amssymb}
\usepackage{graphicx}
\usepackage{url}
\usepackage{hyperref}

\hypersetup{
  colorlinks=true,
  linkcolor=blue,
  citecolor=blue,
  urlcolor=blue
}

\begin{document}

\title{軽量 CNN と Vision Transformer の公平比較に向けた実験的評価}

\author{\IEEEauthorblockN{著者 太郎\\ }
\IEEEauthorblockA{所属機関\\ 連絡先: author@example.com}}

\maketitle

\begin{abstract}
本稿では,軽量な畳み込みニューラルネットワーク(CNN)と Vision Transformer(ViT)の画像分類性能を,CIFAR-10 を対象に公平に比較する実験プロトコルを提示する。学習率スケジューリング,データ拡張,Test-Time Augmentation (TTA) を統一的に適用し,再現性を重視した評価を実施した。実験の結果,ViT-Ti は TTA により最大で +1.8pt の精度向上を得た一方,ResNet-18 では RandAugment を中心としたデータ拡張で +2.1pt の改善が確認された。これらの知見を通じて,限られた計算資源でも堅牢なベースラインを構築するための指針を示す。
\end{abstract}

\begin{IEEEkeywords}
画像分類, 深層学習, Vision Transformer, Test-Time Augmentation, CIFAR-10
\end{IEEEkeywords}

\input{01_introduction/main}
\input{02_related_work/main}
\input{03_method/main}
\input{04_results/main}
\input{05_discussion/main}
\input{06_conclusion/main}
\input{appendix/main}

\bibliographystyle{ieeetr}
\bibliography{references}

\end{document}

% !TEX program = xelatex
\documentclass[conference]{IEEEtran}

% 日本語出力と XeLaTeX 対応設定
\usepackage{xeCJK}
\usepackage{fontspec}
\setCJKmainfont{HaranoAjiMincho}

% 数式・図表・参考文献などの標準パッケージ
\usepackage{amsmath, amssymb}
\usepackage{graphicx}
\usepackage{url}
\usepackage{hyperref}

\hypersetup{
  colorlinks=true,
  linkcolor=blue,
  citecolor=blue,
  urlcolor=blue
}

\begin{document}

\title{軽量 CNN と Vision Transformer の公平比較に向けた実験的評価}

\author{\IEEEauthorblockN{著者 太郎\\ }
\IEEEauthorblockA{所属機関\\ 連絡先: author@example.com}}

\maketitle

\begin{abstract}
本稿では,軽量な畳み込みニューラルネットワーク(CNN)と Vision Transformer(ViT)の画像分類性能を,CIFAR-10 を対象に公平に比較する実験プロトコルを提示する。学習率スケジューリング,データ拡張,Test-Time Augmentation (TTA) を統一的に適用し,再現性を重視した評価を実施した。実験の結果,ViT-Ti は TTA により最大で +1.8pt の精度向上を得た一方,ResNet-18 では RandAugment を中心としたデータ拡張で +2.1pt の改善が確認された。これらの知見を通じて,限られた計算資源でも堅牢なベースラインを構築するための指針を示す。
\end{abstract}

\begin{IEEEkeywords}
画像分類, 深層学習, Vision Transformer, Test-Time Augmentation, CIFAR-10
\end{IEEEkeywords}

\input{01_introduction/main}
\input{02_related_work/main}
\input{03_method/main}
\input{04_results/main}
\input{05_discussion/main}
\input{06_conclusion/main}
\input{appendix/main}

\bibliographystyle{ieeetr}
\bibliography{references}

\end{document}

% !TEX program = xelatex
\documentclass[conference]{IEEEtran}

% 日本語出力と XeLaTeX 対応設定
\usepackage{xeCJK}
\usepackage{fontspec}
\setCJKmainfont{HaranoAjiMincho}

% 数式・図表・参考文献などの標準パッケージ
\usepackage{amsmath, amssymb}
\usepackage{graphicx}
\usepackage{url}
\usepackage{hyperref}

\hypersetup{
  colorlinks=true,
  linkcolor=blue,
  citecolor=blue,
  urlcolor=blue
}

\begin{document}

\title{軽量 CNN と Vision Transformer の公平比較に向けた実験的評価}

\author{\IEEEauthorblockN{著者 太郎\\ }
\IEEEauthorblockA{所属機関\\ 連絡先: author@example.com}}

\maketitle

\begin{abstract}
本稿では,軽量な畳み込みニューラルネットワーク(CNN)と Vision Transformer(ViT)の画像分類性能を,CIFAR-10 を対象に公平に比較する実験プロトコルを提示する。学習率スケジューリング,データ拡張,Test-Time Augmentation (TTA) を統一的に適用し,再現性を重視した評価を実施した。実験の結果,ViT-Ti は TTA により最大で +1.8pt の精度向上を得た一方,ResNet-18 では RandAugment を中心としたデータ拡張で +2.1pt の改善が確認された。これらの知見を通じて,限られた計算資源でも堅牢なベースラインを構築するための指針を示す。
\end{abstract}

\begin{IEEEkeywords}
画像分類, 深層学習, Vision Transformer, Test-Time Augmentation, CIFAR-10
\end{IEEEkeywords}

\input{01_introduction/main}
\input{02_related_work/main}
\input{03_method/main}
\input{04_results/main}
\input{05_discussion/main}
\input{06_conclusion/main}
\input{appendix/main}

\bibliographystyle{ieeetr}
\bibliography{references}

\end{document}

% !TEX program = xelatex
\documentclass[conference]{IEEEtran}

% 日本語出力と XeLaTeX 対応設定
\usepackage{xeCJK}
\usepackage{fontspec}
\setCJKmainfont{HaranoAjiMincho}

% 数式・図表・参考文献などの標準パッケージ
\usepackage{amsmath, amssymb}
\usepackage{graphicx}
\usepackage{url}
\usepackage{hyperref}

\hypersetup{
  colorlinks=true,
  linkcolor=blue,
  citecolor=blue,
  urlcolor=blue
}

\begin{document}

\title{軽量 CNN と Vision Transformer の公平比較に向けた実験的評価}

\author{\IEEEauthorblockN{著者 太郎\\ }
\IEEEauthorblockA{所属機関\\ 連絡先: author@example.com}}

\maketitle

\begin{abstract}
本稿では,軽量な畳み込みニューラルネットワーク(CNN)と Vision Transformer(ViT)の画像分類性能を,CIFAR-10 を対象に公平に比較する実験プロトコルを提示する。学習率スケジューリング,データ拡張,Test-Time Augmentation (TTA) を統一的に適用し,再現性を重視した評価を実施した。実験の結果,ViT-Ti は TTA により最大で +1.8pt の精度向上を得た一方,ResNet-18 では RandAugment を中心としたデータ拡張で +2.1pt の改善が確認された。これらの知見を通じて,限られた計算資源でも堅牢なベースラインを構築するための指針を示す。
\end{abstract}

\begin{IEEEkeywords}
画像分類, 深層学習, Vision Transformer, Test-Time Augmentation, CIFAR-10
\end{IEEEkeywords}

\input{01_introduction/main}
\input{02_related_work/main}
\input{03_method/main}
\input{04_results/main}
\input{05_discussion/main}
\input{06_conclusion/main}
\input{appendix/main}

\bibliographystyle{ieeetr}
\bibliography{references}

\end{document}

% !TEX program = xelatex
\documentclass[conference]{IEEEtran}

% 日本語出力と XeLaTeX 対応設定
\usepackage{xeCJK}
\usepackage{fontspec}
\setCJKmainfont{HaranoAjiMincho}

% 数式・図表・参考文献などの標準パッケージ
\usepackage{amsmath, amssymb}
\usepackage{graphicx}
\usepackage{url}
\usepackage{hyperref}

\hypersetup{
  colorlinks=true,
  linkcolor=blue,
  citecolor=blue,
  urlcolor=blue
}

\begin{document}

\title{軽量 CNN と Vision Transformer の公平比較に向けた実験的評価}

\author{\IEEEauthorblockN{著者 太郎\\ }
\IEEEauthorblockA{所属機関\\ 連絡先: author@example.com}}

\maketitle

\begin{abstract}
本稿では,軽量な畳み込みニューラルネットワーク(CNN)と Vision Transformer(ViT)の画像分類性能を,CIFAR-10 を対象に公平に比較する実験プロトコルを提示する。学習率スケジューリング,データ拡張,Test-Time Augmentation (TTA) を統一的に適用し,再現性を重視した評価を実施した。実験の結果,ViT-Ti は TTA により最大で +1.8pt の精度向上を得た一方,ResNet-18 では RandAugment を中心としたデータ拡張で +2.1pt の改善が確認された。これらの知見を通じて,限られた計算資源でも堅牢なベースラインを構築するための指針を示す。
\end{abstract}

\begin{IEEEkeywords}
画像分類, 深層学習, Vision Transformer, Test-Time Augmentation, CIFAR-10
\end{IEEEkeywords}

\input{01_introduction/main}
\input{02_related_work/main}
\input{03_method/main}
\input{04_results/main}
\input{05_discussion/main}
\input{06_conclusion/main}
\input{appendix/main}

\bibliographystyle{ieeetr}
\bibliography{references}

\end{document}


\bibliographystyle{ieeetr}
\bibliography{references}

\end{document}

% !TEX program = xelatex
\documentclass[conference]{IEEEtran}

% 日本語出力と XeLaTeX 対応設定
\usepackage{xeCJK}
\usepackage{fontspec}
\setCJKmainfont{HaranoAjiMincho}

% 数式・図表・参考文献などの標準パッケージ
\usepackage{amsmath, amssymb}
\usepackage{graphicx}
\usepackage{url}
\usepackage{hyperref}

\hypersetup{
  colorlinks=true,
  linkcolor=blue,
  citecolor=blue,
  urlcolor=blue
}

\begin{document}

\title{軽量 CNN と Vision Transformer の公平比較に向けた実験的評価}

\author{\IEEEauthorblockN{著者 太郎\\ }
\IEEEauthorblockA{所属機関\\ 連絡先: author@example.com}}

\maketitle

\begin{abstract}
本稿では,軽量な畳み込みニューラルネットワーク(CNN)と Vision Transformer(ViT)の画像分類性能を,CIFAR-10 を対象に公平に比較する実験プロトコルを提示する。学習率スケジューリング,データ拡張,Test-Time Augmentation (TTA) を統一的に適用し,再現性を重視した評価を実施した。実験の結果,ViT-Ti は TTA により最大で +1.8pt の精度向上を得た一方,ResNet-18 では RandAugment を中心としたデータ拡張で +2.1pt の改善が確認された。これらの知見を通じて,限られた計算資源でも堅牢なベースラインを構築するための指針を示す。
\end{abstract}

\begin{IEEEkeywords}
画像分類, 深層学習, Vision Transformer, Test-Time Augmentation, CIFAR-10
\end{IEEEkeywords}

% !TEX program = xelatex
\documentclass[conference]{IEEEtran}

% 日本語出力と XeLaTeX 対応設定
\usepackage{xeCJK}
\usepackage{fontspec}
\setCJKmainfont{HaranoAjiMincho}

% 数式・図表・参考文献などの標準パッケージ
\usepackage{amsmath, amssymb}
\usepackage{graphicx}
\usepackage{url}
\usepackage{hyperref}

\hypersetup{
  colorlinks=true,
  linkcolor=blue,
  citecolor=blue,
  urlcolor=blue
}

\begin{document}

\title{軽量 CNN と Vision Transformer の公平比較に向けた実験的評価}

\author{\IEEEauthorblockN{著者 太郎\\ }
\IEEEauthorblockA{所属機関\\ 連絡先: author@example.com}}

\maketitle

\begin{abstract}
本稿では,軽量な畳み込みニューラルネットワーク(CNN)と Vision Transformer(ViT)の画像分類性能を,CIFAR-10 を対象に公平に比較する実験プロトコルを提示する。学習率スケジューリング,データ拡張,Test-Time Augmentation (TTA) を統一的に適用し,再現性を重視した評価を実施した。実験の結果,ViT-Ti は TTA により最大で +1.8pt の精度向上を得た一方,ResNet-18 では RandAugment を中心としたデータ拡張で +2.1pt の改善が確認された。これらの知見を通じて,限られた計算資源でも堅牢なベースラインを構築するための指針を示す。
\end{abstract}

\begin{IEEEkeywords}
画像分類, 深層学習, Vision Transformer, Test-Time Augmentation, CIFAR-10
\end{IEEEkeywords}

\input{01_introduction/main}
\input{02_related_work/main}
\input{03_method/main}
\input{04_results/main}
\input{05_discussion/main}
\input{06_conclusion/main}
\input{appendix/main}

\bibliographystyle{ieeetr}
\bibliography{references}

\end{document}

% !TEX program = xelatex
\documentclass[conference]{IEEEtran}

% 日本語出力と XeLaTeX 対応設定
\usepackage{xeCJK}
\usepackage{fontspec}
\setCJKmainfont{HaranoAjiMincho}

% 数式・図表・参考文献などの標準パッケージ
\usepackage{amsmath, amssymb}
\usepackage{graphicx}
\usepackage{url}
\usepackage{hyperref}

\hypersetup{
  colorlinks=true,
  linkcolor=blue,
  citecolor=blue,
  urlcolor=blue
}

\begin{document}

\title{軽量 CNN と Vision Transformer の公平比較に向けた実験的評価}

\author{\IEEEauthorblockN{著者 太郎\\ }
\IEEEauthorblockA{所属機関\\ 連絡先: author@example.com}}

\maketitle

\begin{abstract}
本稿では,軽量な畳み込みニューラルネットワーク(CNN)と Vision Transformer(ViT)の画像分類性能を,CIFAR-10 を対象に公平に比較する実験プロトコルを提示する。学習率スケジューリング,データ拡張,Test-Time Augmentation (TTA) を統一的に適用し,再現性を重視した評価を実施した。実験の結果,ViT-Ti は TTA により最大で +1.8pt の精度向上を得た一方,ResNet-18 では RandAugment を中心としたデータ拡張で +2.1pt の改善が確認された。これらの知見を通じて,限られた計算資源でも堅牢なベースラインを構築するための指針を示す。
\end{abstract}

\begin{IEEEkeywords}
画像分類, 深層学習, Vision Transformer, Test-Time Augmentation, CIFAR-10
\end{IEEEkeywords}

\input{01_introduction/main}
\input{02_related_work/main}
\input{03_method/main}
\input{04_results/main}
\input{05_discussion/main}
\input{06_conclusion/main}
\input{appendix/main}

\bibliographystyle{ieeetr}
\bibliography{references}

\end{document}

% !TEX program = xelatex
\documentclass[conference]{IEEEtran}

% 日本語出力と XeLaTeX 対応設定
\usepackage{xeCJK}
\usepackage{fontspec}
\setCJKmainfont{HaranoAjiMincho}

% 数式・図表・参考文献などの標準パッケージ
\usepackage{amsmath, amssymb}
\usepackage{graphicx}
\usepackage{url}
\usepackage{hyperref}

\hypersetup{
  colorlinks=true,
  linkcolor=blue,
  citecolor=blue,
  urlcolor=blue
}

\begin{document}

\title{軽量 CNN と Vision Transformer の公平比較に向けた実験的評価}

\author{\IEEEauthorblockN{著者 太郎\\ }
\IEEEauthorblockA{所属機関\\ 連絡先: author@example.com}}

\maketitle

\begin{abstract}
本稿では,軽量な畳み込みニューラルネットワーク(CNN)と Vision Transformer(ViT)の画像分類性能を,CIFAR-10 を対象に公平に比較する実験プロトコルを提示する。学習率スケジューリング,データ拡張,Test-Time Augmentation (TTA) を統一的に適用し,再現性を重視した評価を実施した。実験の結果,ViT-Ti は TTA により最大で +1.8pt の精度向上を得た一方,ResNet-18 では RandAugment を中心としたデータ拡張で +2.1pt の改善が確認された。これらの知見を通じて,限られた計算資源でも堅牢なベースラインを構築するための指針を示す。
\end{abstract}

\begin{IEEEkeywords}
画像分類, 深層学習, Vision Transformer, Test-Time Augmentation, CIFAR-10
\end{IEEEkeywords}

\input{01_introduction/main}
\input{02_related_work/main}
\input{03_method/main}
\input{04_results/main}
\input{05_discussion/main}
\input{06_conclusion/main}
\input{appendix/main}

\bibliographystyle{ieeetr}
\bibliography{references}

\end{document}

% !TEX program = xelatex
\documentclass[conference]{IEEEtran}

% 日本語出力と XeLaTeX 対応設定
\usepackage{xeCJK}
\usepackage{fontspec}
\setCJKmainfont{HaranoAjiMincho}

% 数式・図表・参考文献などの標準パッケージ
\usepackage{amsmath, amssymb}
\usepackage{graphicx}
\usepackage{url}
\usepackage{hyperref}

\hypersetup{
  colorlinks=true,
  linkcolor=blue,
  citecolor=blue,
  urlcolor=blue
}

\begin{document}

\title{軽量 CNN と Vision Transformer の公平比較に向けた実験的評価}

\author{\IEEEauthorblockN{著者 太郎\\ }
\IEEEauthorblockA{所属機関\\ 連絡先: author@example.com}}

\maketitle

\begin{abstract}
本稿では,軽量な畳み込みニューラルネットワーク(CNN)と Vision Transformer(ViT)の画像分類性能を,CIFAR-10 を対象に公平に比較する実験プロトコルを提示する。学習率スケジューリング,データ拡張,Test-Time Augmentation (TTA) を統一的に適用し,再現性を重視した評価を実施した。実験の結果,ViT-Ti は TTA により最大で +1.8pt の精度向上を得た一方,ResNet-18 では RandAugment を中心としたデータ拡張で +2.1pt の改善が確認された。これらの知見を通じて,限られた計算資源でも堅牢なベースラインを構築するための指針を示す。
\end{abstract}

\begin{IEEEkeywords}
画像分類, 深層学習, Vision Transformer, Test-Time Augmentation, CIFAR-10
\end{IEEEkeywords}

\input{01_introduction/main}
\input{02_related_work/main}
\input{03_method/main}
\input{04_results/main}
\input{05_discussion/main}
\input{06_conclusion/main}
\input{appendix/main}

\bibliographystyle{ieeetr}
\bibliography{references}

\end{document}

% !TEX program = xelatex
\documentclass[conference]{IEEEtran}

% 日本語出力と XeLaTeX 対応設定
\usepackage{xeCJK}
\usepackage{fontspec}
\setCJKmainfont{HaranoAjiMincho}

% 数式・図表・参考文献などの標準パッケージ
\usepackage{amsmath, amssymb}
\usepackage{graphicx}
\usepackage{url}
\usepackage{hyperref}

\hypersetup{
  colorlinks=true,
  linkcolor=blue,
  citecolor=blue,
  urlcolor=blue
}

\begin{document}

\title{軽量 CNN と Vision Transformer の公平比較に向けた実験的評価}

\author{\IEEEauthorblockN{著者 太郎\\ }
\IEEEauthorblockA{所属機関\\ 連絡先: author@example.com}}

\maketitle

\begin{abstract}
本稿では,軽量な畳み込みニューラルネットワーク(CNN)と Vision Transformer(ViT)の画像分類性能を,CIFAR-10 を対象に公平に比較する実験プロトコルを提示する。学習率スケジューリング,データ拡張,Test-Time Augmentation (TTA) を統一的に適用し,再現性を重視した評価を実施した。実験の結果,ViT-Ti は TTA により最大で +1.8pt の精度向上を得た一方,ResNet-18 では RandAugment を中心としたデータ拡張で +2.1pt の改善が確認された。これらの知見を通じて,限られた計算資源でも堅牢なベースラインを構築するための指針を示す。
\end{abstract}

\begin{IEEEkeywords}
画像分類, 深層学習, Vision Transformer, Test-Time Augmentation, CIFAR-10
\end{IEEEkeywords}

\input{01_introduction/main}
\input{02_related_work/main}
\input{03_method/main}
\input{04_results/main}
\input{05_discussion/main}
\input{06_conclusion/main}
\input{appendix/main}

\bibliographystyle{ieeetr}
\bibliography{references}

\end{document}

% !TEX program = xelatex
\documentclass[conference]{IEEEtran}

% 日本語出力と XeLaTeX 対応設定
\usepackage{xeCJK}
\usepackage{fontspec}
\setCJKmainfont{HaranoAjiMincho}

% 数式・図表・参考文献などの標準パッケージ
\usepackage{amsmath, amssymb}
\usepackage{graphicx}
\usepackage{url}
\usepackage{hyperref}

\hypersetup{
  colorlinks=true,
  linkcolor=blue,
  citecolor=blue,
  urlcolor=blue
}

\begin{document}

\title{軽量 CNN と Vision Transformer の公平比較に向けた実験的評価}

\author{\IEEEauthorblockN{著者 太郎\\ }
\IEEEauthorblockA{所属機関\\ 連絡先: author@example.com}}

\maketitle

\begin{abstract}
本稿では,軽量な畳み込みニューラルネットワーク(CNN)と Vision Transformer(ViT)の画像分類性能を,CIFAR-10 を対象に公平に比較する実験プロトコルを提示する。学習率スケジューリング,データ拡張,Test-Time Augmentation (TTA) を統一的に適用し,再現性を重視した評価を実施した。実験の結果,ViT-Ti は TTA により最大で +1.8pt の精度向上を得た一方,ResNet-18 では RandAugment を中心としたデータ拡張で +2.1pt の改善が確認された。これらの知見を通じて,限られた計算資源でも堅牢なベースラインを構築するための指針を示す。
\end{abstract}

\begin{IEEEkeywords}
画像分類, 深層学習, Vision Transformer, Test-Time Augmentation, CIFAR-10
\end{IEEEkeywords}

\input{01_introduction/main}
\input{02_related_work/main}
\input{03_method/main}
\input{04_results/main}
\input{05_discussion/main}
\input{06_conclusion/main}
\input{appendix/main}

\bibliographystyle{ieeetr}
\bibliography{references}

\end{document}

% !TEX program = xelatex
\documentclass[conference]{IEEEtran}

% 日本語出力と XeLaTeX 対応設定
\usepackage{xeCJK}
\usepackage{fontspec}
\setCJKmainfont{HaranoAjiMincho}

% 数式・図表・参考文献などの標準パッケージ
\usepackage{amsmath, amssymb}
\usepackage{graphicx}
\usepackage{url}
\usepackage{hyperref}

\hypersetup{
  colorlinks=true,
  linkcolor=blue,
  citecolor=blue,
  urlcolor=blue
}

\begin{document}

\title{軽量 CNN と Vision Transformer の公平比較に向けた実験的評価}

\author{\IEEEauthorblockN{著者 太郎\\ }
\IEEEauthorblockA{所属機関\\ 連絡先: author@example.com}}

\maketitle

\begin{abstract}
本稿では,軽量な畳み込みニューラルネットワーク(CNN)と Vision Transformer(ViT)の画像分類性能を,CIFAR-10 を対象に公平に比較する実験プロトコルを提示する。学習率スケジューリング,データ拡張,Test-Time Augmentation (TTA) を統一的に適用し,再現性を重視した評価を実施した。実験の結果,ViT-Ti は TTA により最大で +1.8pt の精度向上を得た一方,ResNet-18 では RandAugment を中心としたデータ拡張で +2.1pt の改善が確認された。これらの知見を通じて,限られた計算資源でも堅牢なベースラインを構築するための指針を示す。
\end{abstract}

\begin{IEEEkeywords}
画像分類, 深層学習, Vision Transformer, Test-Time Augmentation, CIFAR-10
\end{IEEEkeywords}

\input{01_introduction/main}
\input{02_related_work/main}
\input{03_method/main}
\input{04_results/main}
\input{05_discussion/main}
\input{06_conclusion/main}
\input{appendix/main}

\bibliographystyle{ieeetr}
\bibliography{references}

\end{document}


\bibliographystyle{ieeetr}
\bibliography{references}

\end{document}

% !TEX program = xelatex
\documentclass[conference]{IEEEtran}

% 日本語出力と XeLaTeX 対応設定
\usepackage{xeCJK}
\usepackage{fontspec}
\setCJKmainfont{HaranoAjiMincho}

% 数式・図表・参考文献などの標準パッケージ
\usepackage{amsmath, amssymb}
\usepackage{graphicx}
\usepackage{url}
\usepackage{hyperref}

\hypersetup{
  colorlinks=true,
  linkcolor=blue,
  citecolor=blue,
  urlcolor=blue
}

\begin{document}

\title{軽量 CNN と Vision Transformer の公平比較に向けた実験的評価}

\author{\IEEEauthorblockN{著者 太郎\\ }
\IEEEauthorblockA{所属機関\\ 連絡先: author@example.com}}

\maketitle

\begin{abstract}
本稿では,軽量な畳み込みニューラルネットワーク(CNN)と Vision Transformer(ViT)の画像分類性能を,CIFAR-10 を対象に公平に比較する実験プロトコルを提示する。学習率スケジューリング,データ拡張,Test-Time Augmentation (TTA) を統一的に適用し,再現性を重視した評価を実施した。実験の結果,ViT-Ti は TTA により最大で +1.8pt の精度向上を得た一方,ResNet-18 では RandAugment を中心としたデータ拡張で +2.1pt の改善が確認された。これらの知見を通じて,限られた計算資源でも堅牢なベースラインを構築するための指針を示す。
\end{abstract}

\begin{IEEEkeywords}
画像分類, 深層学習, Vision Transformer, Test-Time Augmentation, CIFAR-10
\end{IEEEkeywords}

% !TEX program = xelatex
\documentclass[conference]{IEEEtran}

% 日本語出力と XeLaTeX 対応設定
\usepackage{xeCJK}
\usepackage{fontspec}
\setCJKmainfont{HaranoAjiMincho}

% 数式・図表・参考文献などの標準パッケージ
\usepackage{amsmath, amssymb}
\usepackage{graphicx}
\usepackage{url}
\usepackage{hyperref}

\hypersetup{
  colorlinks=true,
  linkcolor=blue,
  citecolor=blue,
  urlcolor=blue
}

\begin{document}

\title{軽量 CNN と Vision Transformer の公平比較に向けた実験的評価}

\author{\IEEEauthorblockN{著者 太郎\\ }
\IEEEauthorblockA{所属機関\\ 連絡先: author@example.com}}

\maketitle

\begin{abstract}
本稿では,軽量な畳み込みニューラルネットワーク(CNN)と Vision Transformer(ViT)の画像分類性能を,CIFAR-10 を対象に公平に比較する実験プロトコルを提示する。学習率スケジューリング,データ拡張,Test-Time Augmentation (TTA) を統一的に適用し,再現性を重視した評価を実施した。実験の結果,ViT-Ti は TTA により最大で +1.8pt の精度向上を得た一方,ResNet-18 では RandAugment を中心としたデータ拡張で +2.1pt の改善が確認された。これらの知見を通じて,限られた計算資源でも堅牢なベースラインを構築するための指針を示す。
\end{abstract}

\begin{IEEEkeywords}
画像分類, 深層学習, Vision Transformer, Test-Time Augmentation, CIFAR-10
\end{IEEEkeywords}

\input{01_introduction/main}
\input{02_related_work/main}
\input{03_method/main}
\input{04_results/main}
\input{05_discussion/main}
\input{06_conclusion/main}
\input{appendix/main}

\bibliographystyle{ieeetr}
\bibliography{references}

\end{document}

% !TEX program = xelatex
\documentclass[conference]{IEEEtran}

% 日本語出力と XeLaTeX 対応設定
\usepackage{xeCJK}
\usepackage{fontspec}
\setCJKmainfont{HaranoAjiMincho}

% 数式・図表・参考文献などの標準パッケージ
\usepackage{amsmath, amssymb}
\usepackage{graphicx}
\usepackage{url}
\usepackage{hyperref}

\hypersetup{
  colorlinks=true,
  linkcolor=blue,
  citecolor=blue,
  urlcolor=blue
}

\begin{document}

\title{軽量 CNN と Vision Transformer の公平比較に向けた実験的評価}

\author{\IEEEauthorblockN{著者 太郎\\ }
\IEEEauthorblockA{所属機関\\ 連絡先: author@example.com}}

\maketitle

\begin{abstract}
本稿では,軽量な畳み込みニューラルネットワーク(CNN)と Vision Transformer(ViT)の画像分類性能を,CIFAR-10 を対象に公平に比較する実験プロトコルを提示する。学習率スケジューリング,データ拡張,Test-Time Augmentation (TTA) を統一的に適用し,再現性を重視した評価を実施した。実験の結果,ViT-Ti は TTA により最大で +1.8pt の精度向上を得た一方,ResNet-18 では RandAugment を中心としたデータ拡張で +2.1pt の改善が確認された。これらの知見を通じて,限られた計算資源でも堅牢なベースラインを構築するための指針を示す。
\end{abstract}

\begin{IEEEkeywords}
画像分類, 深層学習, Vision Transformer, Test-Time Augmentation, CIFAR-10
\end{IEEEkeywords}

\input{01_introduction/main}
\input{02_related_work/main}
\input{03_method/main}
\input{04_results/main}
\input{05_discussion/main}
\input{06_conclusion/main}
\input{appendix/main}

\bibliographystyle{ieeetr}
\bibliography{references}

\end{document}

% !TEX program = xelatex
\documentclass[conference]{IEEEtran}

% 日本語出力と XeLaTeX 対応設定
\usepackage{xeCJK}
\usepackage{fontspec}
\setCJKmainfont{HaranoAjiMincho}

% 数式・図表・参考文献などの標準パッケージ
\usepackage{amsmath, amssymb}
\usepackage{graphicx}
\usepackage{url}
\usepackage{hyperref}

\hypersetup{
  colorlinks=true,
  linkcolor=blue,
  citecolor=blue,
  urlcolor=blue
}

\begin{document}

\title{軽量 CNN と Vision Transformer の公平比較に向けた実験的評価}

\author{\IEEEauthorblockN{著者 太郎\\ }
\IEEEauthorblockA{所属機関\\ 連絡先: author@example.com}}

\maketitle

\begin{abstract}
本稿では,軽量な畳み込みニューラルネットワーク(CNN)と Vision Transformer(ViT)の画像分類性能を,CIFAR-10 を対象に公平に比較する実験プロトコルを提示する。学習率スケジューリング,データ拡張,Test-Time Augmentation (TTA) を統一的に適用し,再現性を重視した評価を実施した。実験の結果,ViT-Ti は TTA により最大で +1.8pt の精度向上を得た一方,ResNet-18 では RandAugment を中心としたデータ拡張で +2.1pt の改善が確認された。これらの知見を通じて,限られた計算資源でも堅牢なベースラインを構築するための指針を示す。
\end{abstract}

\begin{IEEEkeywords}
画像分類, 深層学習, Vision Transformer, Test-Time Augmentation, CIFAR-10
\end{IEEEkeywords}

\input{01_introduction/main}
\input{02_related_work/main}
\input{03_method/main}
\input{04_results/main}
\input{05_discussion/main}
\input{06_conclusion/main}
\input{appendix/main}

\bibliographystyle{ieeetr}
\bibliography{references}

\end{document}

% !TEX program = xelatex
\documentclass[conference]{IEEEtran}

% 日本語出力と XeLaTeX 対応設定
\usepackage{xeCJK}
\usepackage{fontspec}
\setCJKmainfont{HaranoAjiMincho}

% 数式・図表・参考文献などの標準パッケージ
\usepackage{amsmath, amssymb}
\usepackage{graphicx}
\usepackage{url}
\usepackage{hyperref}

\hypersetup{
  colorlinks=true,
  linkcolor=blue,
  citecolor=blue,
  urlcolor=blue
}

\begin{document}

\title{軽量 CNN と Vision Transformer の公平比較に向けた実験的評価}

\author{\IEEEauthorblockN{著者 太郎\\ }
\IEEEauthorblockA{所属機関\\ 連絡先: author@example.com}}

\maketitle

\begin{abstract}
本稿では,軽量な畳み込みニューラルネットワーク(CNN)と Vision Transformer(ViT)の画像分類性能を,CIFAR-10 を対象に公平に比較する実験プロトコルを提示する。学習率スケジューリング,データ拡張,Test-Time Augmentation (TTA) を統一的に適用し,再現性を重視した評価を実施した。実験の結果,ViT-Ti は TTA により最大で +1.8pt の精度向上を得た一方,ResNet-18 では RandAugment を中心としたデータ拡張で +2.1pt の改善が確認された。これらの知見を通じて,限られた計算資源でも堅牢なベースラインを構築するための指針を示す。
\end{abstract}

\begin{IEEEkeywords}
画像分類, 深層学習, Vision Transformer, Test-Time Augmentation, CIFAR-10
\end{IEEEkeywords}

\input{01_introduction/main}
\input{02_related_work/main}
\input{03_method/main}
\input{04_results/main}
\input{05_discussion/main}
\input{06_conclusion/main}
\input{appendix/main}

\bibliographystyle{ieeetr}
\bibliography{references}

\end{document}

% !TEX program = xelatex
\documentclass[conference]{IEEEtran}

% 日本語出力と XeLaTeX 対応設定
\usepackage{xeCJK}
\usepackage{fontspec}
\setCJKmainfont{HaranoAjiMincho}

% 数式・図表・参考文献などの標準パッケージ
\usepackage{amsmath, amssymb}
\usepackage{graphicx}
\usepackage{url}
\usepackage{hyperref}

\hypersetup{
  colorlinks=true,
  linkcolor=blue,
  citecolor=blue,
  urlcolor=blue
}

\begin{document}

\title{軽量 CNN と Vision Transformer の公平比較に向けた実験的評価}

\author{\IEEEauthorblockN{著者 太郎\\ }
\IEEEauthorblockA{所属機関\\ 連絡先: author@example.com}}

\maketitle

\begin{abstract}
本稿では,軽量な畳み込みニューラルネットワーク(CNN)と Vision Transformer(ViT)の画像分類性能を,CIFAR-10 を対象に公平に比較する実験プロトコルを提示する。学習率スケジューリング,データ拡張,Test-Time Augmentation (TTA) を統一的に適用し,再現性を重視した評価を実施した。実験の結果,ViT-Ti は TTA により最大で +1.8pt の精度向上を得た一方,ResNet-18 では RandAugment を中心としたデータ拡張で +2.1pt の改善が確認された。これらの知見を通じて,限られた計算資源でも堅牢なベースラインを構築するための指針を示す。
\end{abstract}

\begin{IEEEkeywords}
画像分類, 深層学習, Vision Transformer, Test-Time Augmentation, CIFAR-10
\end{IEEEkeywords}

\input{01_introduction/main}
\input{02_related_work/main}
\input{03_method/main}
\input{04_results/main}
\input{05_discussion/main}
\input{06_conclusion/main}
\input{appendix/main}

\bibliographystyle{ieeetr}
\bibliography{references}

\end{document}

% !TEX program = xelatex
\documentclass[conference]{IEEEtran}

% 日本語出力と XeLaTeX 対応設定
\usepackage{xeCJK}
\usepackage{fontspec}
\setCJKmainfont{HaranoAjiMincho}

% 数式・図表・参考文献などの標準パッケージ
\usepackage{amsmath, amssymb}
\usepackage{graphicx}
\usepackage{url}
\usepackage{hyperref}

\hypersetup{
  colorlinks=true,
  linkcolor=blue,
  citecolor=blue,
  urlcolor=blue
}

\begin{document}

\title{軽量 CNN と Vision Transformer の公平比較に向けた実験的評価}

\author{\IEEEauthorblockN{著者 太郎\\ }
\IEEEauthorblockA{所属機関\\ 連絡先: author@example.com}}

\maketitle

\begin{abstract}
本稿では,軽量な畳み込みニューラルネットワーク(CNN)と Vision Transformer(ViT)の画像分類性能を,CIFAR-10 を対象に公平に比較する実験プロトコルを提示する。学習率スケジューリング,データ拡張,Test-Time Augmentation (TTA) を統一的に適用し,再現性を重視した評価を実施した。実験の結果,ViT-Ti は TTA により最大で +1.8pt の精度向上を得た一方,ResNet-18 では RandAugment を中心としたデータ拡張で +2.1pt の改善が確認された。これらの知見を通じて,限られた計算資源でも堅牢なベースラインを構築するための指針を示す。
\end{abstract}

\begin{IEEEkeywords}
画像分類, 深層学習, Vision Transformer, Test-Time Augmentation, CIFAR-10
\end{IEEEkeywords}

\input{01_introduction/main}
\input{02_related_work/main}
\input{03_method/main}
\input{04_results/main}
\input{05_discussion/main}
\input{06_conclusion/main}
\input{appendix/main}

\bibliographystyle{ieeetr}
\bibliography{references}

\end{document}

% !TEX program = xelatex
\documentclass[conference]{IEEEtran}

% 日本語出力と XeLaTeX 対応設定
\usepackage{xeCJK}
\usepackage{fontspec}
\setCJKmainfont{HaranoAjiMincho}

% 数式・図表・参考文献などの標準パッケージ
\usepackage{amsmath, amssymb}
\usepackage{graphicx}
\usepackage{url}
\usepackage{hyperref}

\hypersetup{
  colorlinks=true,
  linkcolor=blue,
  citecolor=blue,
  urlcolor=blue
}

\begin{document}

\title{軽量 CNN と Vision Transformer の公平比較に向けた実験的評価}

\author{\IEEEauthorblockN{著者 太郎\\ }
\IEEEauthorblockA{所属機関\\ 連絡先: author@example.com}}

\maketitle

\begin{abstract}
本稿では,軽量な畳み込みニューラルネットワーク(CNN)と Vision Transformer(ViT)の画像分類性能を,CIFAR-10 を対象に公平に比較する実験プロトコルを提示する。学習率スケジューリング,データ拡張,Test-Time Augmentation (TTA) を統一的に適用し,再現性を重視した評価を実施した。実験の結果,ViT-Ti は TTA により最大で +1.8pt の精度向上を得た一方,ResNet-18 では RandAugment を中心としたデータ拡張で +2.1pt の改善が確認された。これらの知見を通じて,限られた計算資源でも堅牢なベースラインを構築するための指針を示す。
\end{abstract}

\begin{IEEEkeywords}
画像分類, 深層学習, Vision Transformer, Test-Time Augmentation, CIFAR-10
\end{IEEEkeywords}

\input{01_introduction/main}
\input{02_related_work/main}
\input{03_method/main}
\input{04_results/main}
\input{05_discussion/main}
\input{06_conclusion/main}
\input{appendix/main}

\bibliographystyle{ieeetr}
\bibliography{references}

\end{document}


\bibliographystyle{ieeetr}
\bibliography{references}

\end{document}

% !TEX program = xelatex
\documentclass[conference]{IEEEtran}

% 日本語出力と XeLaTeX 対応設定
\usepackage{xeCJK}
\usepackage{fontspec}
\setCJKmainfont{HaranoAjiMincho}

% 数式・図表・参考文献などの標準パッケージ
\usepackage{amsmath, amssymb}
\usepackage{graphicx}
\usepackage{url}
\usepackage{hyperref}

\hypersetup{
  colorlinks=true,
  linkcolor=blue,
  citecolor=blue,
  urlcolor=blue
}

\begin{document}

\title{軽量 CNN と Vision Transformer の公平比較に向けた実験的評価}

\author{\IEEEauthorblockN{著者 太郎\\ }
\IEEEauthorblockA{所属機関\\ 連絡先: author@example.com}}

\maketitle

\begin{abstract}
本稿では,軽量な畳み込みニューラルネットワーク(CNN)と Vision Transformer(ViT)の画像分類性能を,CIFAR-10 を対象に公平に比較する実験プロトコルを提示する。学習率スケジューリング,データ拡張,Test-Time Augmentation (TTA) を統一的に適用し,再現性を重視した評価を実施した。実験の結果,ViT-Ti は TTA により最大で +1.8pt の精度向上を得た一方,ResNet-18 では RandAugment を中心としたデータ拡張で +2.1pt の改善が確認された。これらの知見を通じて,限られた計算資源でも堅牢なベースラインを構築するための指針を示す。
\end{abstract}

\begin{IEEEkeywords}
画像分類, 深層学習, Vision Transformer, Test-Time Augmentation, CIFAR-10
\end{IEEEkeywords}

% !TEX program = xelatex
\documentclass[conference]{IEEEtran}

% 日本語出力と XeLaTeX 対応設定
\usepackage{xeCJK}
\usepackage{fontspec}
\setCJKmainfont{HaranoAjiMincho}

% 数式・図表・参考文献などの標準パッケージ
\usepackage{amsmath, amssymb}
\usepackage{graphicx}
\usepackage{url}
\usepackage{hyperref}

\hypersetup{
  colorlinks=true,
  linkcolor=blue,
  citecolor=blue,
  urlcolor=blue
}

\begin{document}

\title{軽量 CNN と Vision Transformer の公平比較に向けた実験的評価}

\author{\IEEEauthorblockN{著者 太郎\\ }
\IEEEauthorblockA{所属機関\\ 連絡先: author@example.com}}

\maketitle

\begin{abstract}
本稿では,軽量な畳み込みニューラルネットワーク(CNN)と Vision Transformer(ViT)の画像分類性能を,CIFAR-10 を対象に公平に比較する実験プロトコルを提示する。学習率スケジューリング,データ拡張,Test-Time Augmentation (TTA) を統一的に適用し,再現性を重視した評価を実施した。実験の結果,ViT-Ti は TTA により最大で +1.8pt の精度向上を得た一方,ResNet-18 では RandAugment を中心としたデータ拡張で +2.1pt の改善が確認された。これらの知見を通じて,限られた計算資源でも堅牢なベースラインを構築するための指針を示す。
\end{abstract}

\begin{IEEEkeywords}
画像分類, 深層学習, Vision Transformer, Test-Time Augmentation, CIFAR-10
\end{IEEEkeywords}

\input{01_introduction/main}
\input{02_related_work/main}
\input{03_method/main}
\input{04_results/main}
\input{05_discussion/main}
\input{06_conclusion/main}
\input{appendix/main}

\bibliographystyle{ieeetr}
\bibliography{references}

\end{document}

% !TEX program = xelatex
\documentclass[conference]{IEEEtran}

% 日本語出力と XeLaTeX 対応設定
\usepackage{xeCJK}
\usepackage{fontspec}
\setCJKmainfont{HaranoAjiMincho}

% 数式・図表・参考文献などの標準パッケージ
\usepackage{amsmath, amssymb}
\usepackage{graphicx}
\usepackage{url}
\usepackage{hyperref}

\hypersetup{
  colorlinks=true,
  linkcolor=blue,
  citecolor=blue,
  urlcolor=blue
}

\begin{document}

\title{軽量 CNN と Vision Transformer の公平比較に向けた実験的評価}

\author{\IEEEauthorblockN{著者 太郎\\ }
\IEEEauthorblockA{所属機関\\ 連絡先: author@example.com}}

\maketitle

\begin{abstract}
本稿では,軽量な畳み込みニューラルネットワーク(CNN)と Vision Transformer(ViT)の画像分類性能を,CIFAR-10 を対象に公平に比較する実験プロトコルを提示する。学習率スケジューリング,データ拡張,Test-Time Augmentation (TTA) を統一的に適用し,再現性を重視した評価を実施した。実験の結果,ViT-Ti は TTA により最大で +1.8pt の精度向上を得た一方,ResNet-18 では RandAugment を中心としたデータ拡張で +2.1pt の改善が確認された。これらの知見を通じて,限られた計算資源でも堅牢なベースラインを構築するための指針を示す。
\end{abstract}

\begin{IEEEkeywords}
画像分類, 深層学習, Vision Transformer, Test-Time Augmentation, CIFAR-10
\end{IEEEkeywords}

\input{01_introduction/main}
\input{02_related_work/main}
\input{03_method/main}
\input{04_results/main}
\input{05_discussion/main}
\input{06_conclusion/main}
\input{appendix/main}

\bibliographystyle{ieeetr}
\bibliography{references}

\end{document}

% !TEX program = xelatex
\documentclass[conference]{IEEEtran}

% 日本語出力と XeLaTeX 対応設定
\usepackage{xeCJK}
\usepackage{fontspec}
\setCJKmainfont{HaranoAjiMincho}

% 数式・図表・参考文献などの標準パッケージ
\usepackage{amsmath, amssymb}
\usepackage{graphicx}
\usepackage{url}
\usepackage{hyperref}

\hypersetup{
  colorlinks=true,
  linkcolor=blue,
  citecolor=blue,
  urlcolor=blue
}

\begin{document}

\title{軽量 CNN と Vision Transformer の公平比較に向けた実験的評価}

\author{\IEEEauthorblockN{著者 太郎\\ }
\IEEEauthorblockA{所属機関\\ 連絡先: author@example.com}}

\maketitle

\begin{abstract}
本稿では,軽量な畳み込みニューラルネットワーク(CNN)と Vision Transformer(ViT)の画像分類性能を,CIFAR-10 を対象に公平に比較する実験プロトコルを提示する。学習率スケジューリング,データ拡張,Test-Time Augmentation (TTA) を統一的に適用し,再現性を重視した評価を実施した。実験の結果,ViT-Ti は TTA により最大で +1.8pt の精度向上を得た一方,ResNet-18 では RandAugment を中心としたデータ拡張で +2.1pt の改善が確認された。これらの知見を通じて,限られた計算資源でも堅牢なベースラインを構築するための指針を示す。
\end{abstract}

\begin{IEEEkeywords}
画像分類, 深層学習, Vision Transformer, Test-Time Augmentation, CIFAR-10
\end{IEEEkeywords}

\input{01_introduction/main}
\input{02_related_work/main}
\input{03_method/main}
\input{04_results/main}
\input{05_discussion/main}
\input{06_conclusion/main}
\input{appendix/main}

\bibliographystyle{ieeetr}
\bibliography{references}

\end{document}

% !TEX program = xelatex
\documentclass[conference]{IEEEtran}

% 日本語出力と XeLaTeX 対応設定
\usepackage{xeCJK}
\usepackage{fontspec}
\setCJKmainfont{HaranoAjiMincho}

% 数式・図表・参考文献などの標準パッケージ
\usepackage{amsmath, amssymb}
\usepackage{graphicx}
\usepackage{url}
\usepackage{hyperref}

\hypersetup{
  colorlinks=true,
  linkcolor=blue,
  citecolor=blue,
  urlcolor=blue
}

\begin{document}

\title{軽量 CNN と Vision Transformer の公平比較に向けた実験的評価}

\author{\IEEEauthorblockN{著者 太郎\\ }
\IEEEauthorblockA{所属機関\\ 連絡先: author@example.com}}

\maketitle

\begin{abstract}
本稿では,軽量な畳み込みニューラルネットワーク(CNN)と Vision Transformer(ViT)の画像分類性能を,CIFAR-10 を対象に公平に比較する実験プロトコルを提示する。学習率スケジューリング,データ拡張,Test-Time Augmentation (TTA) を統一的に適用し,再現性を重視した評価を実施した。実験の結果,ViT-Ti は TTA により最大で +1.8pt の精度向上を得た一方,ResNet-18 では RandAugment を中心としたデータ拡張で +2.1pt の改善が確認された。これらの知見を通じて,限られた計算資源でも堅牢なベースラインを構築するための指針を示す。
\end{abstract}

\begin{IEEEkeywords}
画像分類, 深層学習, Vision Transformer, Test-Time Augmentation, CIFAR-10
\end{IEEEkeywords}

\input{01_introduction/main}
\input{02_related_work/main}
\input{03_method/main}
\input{04_results/main}
\input{05_discussion/main}
\input{06_conclusion/main}
\input{appendix/main}

\bibliographystyle{ieeetr}
\bibliography{references}

\end{document}

% !TEX program = xelatex
\documentclass[conference]{IEEEtran}

% 日本語出力と XeLaTeX 対応設定
\usepackage{xeCJK}
\usepackage{fontspec}
\setCJKmainfont{HaranoAjiMincho}

% 数式・図表・参考文献などの標準パッケージ
\usepackage{amsmath, amssymb}
\usepackage{graphicx}
\usepackage{url}
\usepackage{hyperref}

\hypersetup{
  colorlinks=true,
  linkcolor=blue,
  citecolor=blue,
  urlcolor=blue
}

\begin{document}

\title{軽量 CNN と Vision Transformer の公平比較に向けた実験的評価}

\author{\IEEEauthorblockN{著者 太郎\\ }
\IEEEauthorblockA{所属機関\\ 連絡先: author@example.com}}

\maketitle

\begin{abstract}
本稿では,軽量な畳み込みニューラルネットワーク(CNN)と Vision Transformer(ViT)の画像分類性能を,CIFAR-10 を対象に公平に比較する実験プロトコルを提示する。学習率スケジューリング,データ拡張,Test-Time Augmentation (TTA) を統一的に適用し,再現性を重視した評価を実施した。実験の結果,ViT-Ti は TTA により最大で +1.8pt の精度向上を得た一方,ResNet-18 では RandAugment を中心としたデータ拡張で +2.1pt の改善が確認された。これらの知見を通じて,限られた計算資源でも堅牢なベースラインを構築するための指針を示す。
\end{abstract}

\begin{IEEEkeywords}
画像分類, 深層学習, Vision Transformer, Test-Time Augmentation, CIFAR-10
\end{IEEEkeywords}

\input{01_introduction/main}
\input{02_related_work/main}
\input{03_method/main}
\input{04_results/main}
\input{05_discussion/main}
\input{06_conclusion/main}
\input{appendix/main}

\bibliographystyle{ieeetr}
\bibliography{references}

\end{document}

% !TEX program = xelatex
\documentclass[conference]{IEEEtran}

% 日本語出力と XeLaTeX 対応設定
\usepackage{xeCJK}
\usepackage{fontspec}
\setCJKmainfont{HaranoAjiMincho}

% 数式・図表・参考文献などの標準パッケージ
\usepackage{amsmath, amssymb}
\usepackage{graphicx}
\usepackage{url}
\usepackage{hyperref}

\hypersetup{
  colorlinks=true,
  linkcolor=blue,
  citecolor=blue,
  urlcolor=blue
}

\begin{document}

\title{軽量 CNN と Vision Transformer の公平比較に向けた実験的評価}

\author{\IEEEauthorblockN{著者 太郎\\ }
\IEEEauthorblockA{所属機関\\ 連絡先: author@example.com}}

\maketitle

\begin{abstract}
本稿では,軽量な畳み込みニューラルネットワーク(CNN)と Vision Transformer(ViT)の画像分類性能を,CIFAR-10 を対象に公平に比較する実験プロトコルを提示する。学習率スケジューリング,データ拡張,Test-Time Augmentation (TTA) を統一的に適用し,再現性を重視した評価を実施した。実験の結果,ViT-Ti は TTA により最大で +1.8pt の精度向上を得た一方,ResNet-18 では RandAugment を中心としたデータ拡張で +2.1pt の改善が確認された。これらの知見を通じて,限られた計算資源でも堅牢なベースラインを構築するための指針を示す。
\end{abstract}

\begin{IEEEkeywords}
画像分類, 深層学習, Vision Transformer, Test-Time Augmentation, CIFAR-10
\end{IEEEkeywords}

\input{01_introduction/main}
\input{02_related_work/main}
\input{03_method/main}
\input{04_results/main}
\input{05_discussion/main}
\input{06_conclusion/main}
\input{appendix/main}

\bibliographystyle{ieeetr}
\bibliography{references}

\end{document}

% !TEX program = xelatex
\documentclass[conference]{IEEEtran}

% 日本語出力と XeLaTeX 対応設定
\usepackage{xeCJK}
\usepackage{fontspec}
\setCJKmainfont{HaranoAjiMincho}

% 数式・図表・参考文献などの標準パッケージ
\usepackage{amsmath, amssymb}
\usepackage{graphicx}
\usepackage{url}
\usepackage{hyperref}

\hypersetup{
  colorlinks=true,
  linkcolor=blue,
  citecolor=blue,
  urlcolor=blue
}

\begin{document}

\title{軽量 CNN と Vision Transformer の公平比較に向けた実験的評価}

\author{\IEEEauthorblockN{著者 太郎\\ }
\IEEEauthorblockA{所属機関\\ 連絡先: author@example.com}}

\maketitle

\begin{abstract}
本稿では,軽量な畳み込みニューラルネットワーク(CNN)と Vision Transformer(ViT)の画像分類性能を,CIFAR-10 を対象に公平に比較する実験プロトコルを提示する。学習率スケジューリング,データ拡張,Test-Time Augmentation (TTA) を統一的に適用し,再現性を重視した評価を実施した。実験の結果,ViT-Ti は TTA により最大で +1.8pt の精度向上を得た一方,ResNet-18 では RandAugment を中心としたデータ拡張で +2.1pt の改善が確認された。これらの知見を通じて,限られた計算資源でも堅牢なベースラインを構築するための指針を示す。
\end{abstract}

\begin{IEEEkeywords}
画像分類, 深層学習, Vision Transformer, Test-Time Augmentation, CIFAR-10
\end{IEEEkeywords}

\input{01_introduction/main}
\input{02_related_work/main}
\input{03_method/main}
\input{04_results/main}
\input{05_discussion/main}
\input{06_conclusion/main}
\input{appendix/main}

\bibliographystyle{ieeetr}
\bibliography{references}

\end{document}


\bibliographystyle{ieeetr}
\bibliography{references}

\end{document}

% !TEX program = xelatex
\documentclass[conference]{IEEEtran}

% 日本語出力と XeLaTeX 対応設定
\usepackage{xeCJK}
\usepackage{fontspec}
\setCJKmainfont{HaranoAjiMincho}

% 数式・図表・参考文献などの標準パッケージ
\usepackage{amsmath, amssymb}
\usepackage{graphicx}
\usepackage{url}
\usepackage{hyperref}

\hypersetup{
  colorlinks=true,
  linkcolor=blue,
  citecolor=blue,
  urlcolor=blue
}

\begin{document}

\title{軽量 CNN と Vision Transformer の公平比較に向けた実験的評価}

\author{\IEEEauthorblockN{著者 太郎\\ }
\IEEEauthorblockA{所属機関\\ 連絡先: author@example.com}}

\maketitle

\begin{abstract}
本稿では,軽量な畳み込みニューラルネットワーク(CNN)と Vision Transformer(ViT)の画像分類性能を,CIFAR-10 を対象に公平に比較する実験プロトコルを提示する。学習率スケジューリング,データ拡張,Test-Time Augmentation (TTA) を統一的に適用し,再現性を重視した評価を実施した。実験の結果,ViT-Ti は TTA により最大で +1.8pt の精度向上を得た一方,ResNet-18 では RandAugment を中心としたデータ拡張で +2.1pt の改善が確認された。これらの知見を通じて,限られた計算資源でも堅牢なベースラインを構築するための指針を示す。
\end{abstract}

\begin{IEEEkeywords}
画像分類, 深層学習, Vision Transformer, Test-Time Augmentation, CIFAR-10
\end{IEEEkeywords}

% !TEX program = xelatex
\documentclass[conference]{IEEEtran}

% 日本語出力と XeLaTeX 対応設定
\usepackage{xeCJK}
\usepackage{fontspec}
\setCJKmainfont{HaranoAjiMincho}

% 数式・図表・参考文献などの標準パッケージ
\usepackage{amsmath, amssymb}
\usepackage{graphicx}
\usepackage{url}
\usepackage{hyperref}

\hypersetup{
  colorlinks=true,
  linkcolor=blue,
  citecolor=blue,
  urlcolor=blue
}

\begin{document}

\title{軽量 CNN と Vision Transformer の公平比較に向けた実験的評価}

\author{\IEEEauthorblockN{著者 太郎\\ }
\IEEEauthorblockA{所属機関\\ 連絡先: author@example.com}}

\maketitle

\begin{abstract}
本稿では,軽量な畳み込みニューラルネットワーク(CNN)と Vision Transformer(ViT)の画像分類性能を,CIFAR-10 を対象に公平に比較する実験プロトコルを提示する。学習率スケジューリング,データ拡張,Test-Time Augmentation (TTA) を統一的に適用し,再現性を重視した評価を実施した。実験の結果,ViT-Ti は TTA により最大で +1.8pt の精度向上を得た一方,ResNet-18 では RandAugment を中心としたデータ拡張で +2.1pt の改善が確認された。これらの知見を通じて,限られた計算資源でも堅牢なベースラインを構築するための指針を示す。
\end{abstract}

\begin{IEEEkeywords}
画像分類, 深層学習, Vision Transformer, Test-Time Augmentation, CIFAR-10
\end{IEEEkeywords}

\input{01_introduction/main}
\input{02_related_work/main}
\input{03_method/main}
\input{04_results/main}
\input{05_discussion/main}
\input{06_conclusion/main}
\input{appendix/main}

\bibliographystyle{ieeetr}
\bibliography{references}

\end{document}

% !TEX program = xelatex
\documentclass[conference]{IEEEtran}

% 日本語出力と XeLaTeX 対応設定
\usepackage{xeCJK}
\usepackage{fontspec}
\setCJKmainfont{HaranoAjiMincho}

% 数式・図表・参考文献などの標準パッケージ
\usepackage{amsmath, amssymb}
\usepackage{graphicx}
\usepackage{url}
\usepackage{hyperref}

\hypersetup{
  colorlinks=true,
  linkcolor=blue,
  citecolor=blue,
  urlcolor=blue
}

\begin{document}

\title{軽量 CNN と Vision Transformer の公平比較に向けた実験的評価}

\author{\IEEEauthorblockN{著者 太郎\\ }
\IEEEauthorblockA{所属機関\\ 連絡先: author@example.com}}

\maketitle

\begin{abstract}
本稿では,軽量な畳み込みニューラルネットワーク(CNN)と Vision Transformer(ViT)の画像分類性能を,CIFAR-10 を対象に公平に比較する実験プロトコルを提示する。学習率スケジューリング,データ拡張,Test-Time Augmentation (TTA) を統一的に適用し,再現性を重視した評価を実施した。実験の結果,ViT-Ti は TTA により最大で +1.8pt の精度向上を得た一方,ResNet-18 では RandAugment を中心としたデータ拡張で +2.1pt の改善が確認された。これらの知見を通じて,限られた計算資源でも堅牢なベースラインを構築するための指針を示す。
\end{abstract}

\begin{IEEEkeywords}
画像分類, 深層学習, Vision Transformer, Test-Time Augmentation, CIFAR-10
\end{IEEEkeywords}

\input{01_introduction/main}
\input{02_related_work/main}
\input{03_method/main}
\input{04_results/main}
\input{05_discussion/main}
\input{06_conclusion/main}
\input{appendix/main}

\bibliographystyle{ieeetr}
\bibliography{references}

\end{document}

% !TEX program = xelatex
\documentclass[conference]{IEEEtran}

% 日本語出力と XeLaTeX 対応設定
\usepackage{xeCJK}
\usepackage{fontspec}
\setCJKmainfont{HaranoAjiMincho}

% 数式・図表・参考文献などの標準パッケージ
\usepackage{amsmath, amssymb}
\usepackage{graphicx}
\usepackage{url}
\usepackage{hyperref}

\hypersetup{
  colorlinks=true,
  linkcolor=blue,
  citecolor=blue,
  urlcolor=blue
}

\begin{document}

\title{軽量 CNN と Vision Transformer の公平比較に向けた実験的評価}

\author{\IEEEauthorblockN{著者 太郎\\ }
\IEEEauthorblockA{所属機関\\ 連絡先: author@example.com}}

\maketitle

\begin{abstract}
本稿では,軽量な畳み込みニューラルネットワーク(CNN)と Vision Transformer(ViT)の画像分類性能を,CIFAR-10 を対象に公平に比較する実験プロトコルを提示する。学習率スケジューリング,データ拡張,Test-Time Augmentation (TTA) を統一的に適用し,再現性を重視した評価を実施した。実験の結果,ViT-Ti は TTA により最大で +1.8pt の精度向上を得た一方,ResNet-18 では RandAugment を中心としたデータ拡張で +2.1pt の改善が確認された。これらの知見を通じて,限られた計算資源でも堅牢なベースラインを構築するための指針を示す。
\end{abstract}

\begin{IEEEkeywords}
画像分類, 深層学習, Vision Transformer, Test-Time Augmentation, CIFAR-10
\end{IEEEkeywords}

\input{01_introduction/main}
\input{02_related_work/main}
\input{03_method/main}
\input{04_results/main}
\input{05_discussion/main}
\input{06_conclusion/main}
\input{appendix/main}

\bibliographystyle{ieeetr}
\bibliography{references}

\end{document}

% !TEX program = xelatex
\documentclass[conference]{IEEEtran}

% 日本語出力と XeLaTeX 対応設定
\usepackage{xeCJK}
\usepackage{fontspec}
\setCJKmainfont{HaranoAjiMincho}

% 数式・図表・参考文献などの標準パッケージ
\usepackage{amsmath, amssymb}
\usepackage{graphicx}
\usepackage{url}
\usepackage{hyperref}

\hypersetup{
  colorlinks=true,
  linkcolor=blue,
  citecolor=blue,
  urlcolor=blue
}

\begin{document}

\title{軽量 CNN と Vision Transformer の公平比較に向けた実験的評価}

\author{\IEEEauthorblockN{著者 太郎\\ }
\IEEEauthorblockA{所属機関\\ 連絡先: author@example.com}}

\maketitle

\begin{abstract}
本稿では,軽量な畳み込みニューラルネットワーク(CNN)と Vision Transformer(ViT)の画像分類性能を,CIFAR-10 を対象に公平に比較する実験プロトコルを提示する。学習率スケジューリング,データ拡張,Test-Time Augmentation (TTA) を統一的に適用し,再現性を重視した評価を実施した。実験の結果,ViT-Ti は TTA により最大で +1.8pt の精度向上を得た一方,ResNet-18 では RandAugment を中心としたデータ拡張で +2.1pt の改善が確認された。これらの知見を通じて,限られた計算資源でも堅牢なベースラインを構築するための指針を示す。
\end{abstract}

\begin{IEEEkeywords}
画像分類, 深層学習, Vision Transformer, Test-Time Augmentation, CIFAR-10
\end{IEEEkeywords}

\input{01_introduction/main}
\input{02_related_work/main}
\input{03_method/main}
\input{04_results/main}
\input{05_discussion/main}
\input{06_conclusion/main}
\input{appendix/main}

\bibliographystyle{ieeetr}
\bibliography{references}

\end{document}

% !TEX program = xelatex
\documentclass[conference]{IEEEtran}

% 日本語出力と XeLaTeX 対応設定
\usepackage{xeCJK}
\usepackage{fontspec}
\setCJKmainfont{HaranoAjiMincho}

% 数式・図表・参考文献などの標準パッケージ
\usepackage{amsmath, amssymb}
\usepackage{graphicx}
\usepackage{url}
\usepackage{hyperref}

\hypersetup{
  colorlinks=true,
  linkcolor=blue,
  citecolor=blue,
  urlcolor=blue
}

\begin{document}

\title{軽量 CNN と Vision Transformer の公平比較に向けた実験的評価}

\author{\IEEEauthorblockN{著者 太郎\\ }
\IEEEauthorblockA{所属機関\\ 連絡先: author@example.com}}

\maketitle

\begin{abstract}
本稿では,軽量な畳み込みニューラルネットワーク(CNN)と Vision Transformer(ViT)の画像分類性能を,CIFAR-10 を対象に公平に比較する実験プロトコルを提示する。学習率スケジューリング,データ拡張,Test-Time Augmentation (TTA) を統一的に適用し,再現性を重視した評価を実施した。実験の結果,ViT-Ti は TTA により最大で +1.8pt の精度向上を得た一方,ResNet-18 では RandAugment を中心としたデータ拡張で +2.1pt の改善が確認された。これらの知見を通じて,限られた計算資源でも堅牢なベースラインを構築するための指針を示す。
\end{abstract}

\begin{IEEEkeywords}
画像分類, 深層学習, Vision Transformer, Test-Time Augmentation, CIFAR-10
\end{IEEEkeywords}

\input{01_introduction/main}
\input{02_related_work/main}
\input{03_method/main}
\input{04_results/main}
\input{05_discussion/main}
\input{06_conclusion/main}
\input{appendix/main}

\bibliographystyle{ieeetr}
\bibliography{references}

\end{document}

% !TEX program = xelatex
\documentclass[conference]{IEEEtran}

% 日本語出力と XeLaTeX 対応設定
\usepackage{xeCJK}
\usepackage{fontspec}
\setCJKmainfont{HaranoAjiMincho}

% 数式・図表・参考文献などの標準パッケージ
\usepackage{amsmath, amssymb}
\usepackage{graphicx}
\usepackage{url}
\usepackage{hyperref}

\hypersetup{
  colorlinks=true,
  linkcolor=blue,
  citecolor=blue,
  urlcolor=blue
}

\begin{document}

\title{軽量 CNN と Vision Transformer の公平比較に向けた実験的評価}

\author{\IEEEauthorblockN{著者 太郎\\ }
\IEEEauthorblockA{所属機関\\ 連絡先: author@example.com}}

\maketitle

\begin{abstract}
本稿では,軽量な畳み込みニューラルネットワーク(CNN)と Vision Transformer(ViT)の画像分類性能を,CIFAR-10 を対象に公平に比較する実験プロトコルを提示する。学習率スケジューリング,データ拡張,Test-Time Augmentation (TTA) を統一的に適用し,再現性を重視した評価を実施した。実験の結果,ViT-Ti は TTA により最大で +1.8pt の精度向上を得た一方,ResNet-18 では RandAugment を中心としたデータ拡張で +2.1pt の改善が確認された。これらの知見を通じて,限られた計算資源でも堅牢なベースラインを構築するための指針を示す。
\end{abstract}

\begin{IEEEkeywords}
画像分類, 深層学習, Vision Transformer, Test-Time Augmentation, CIFAR-10
\end{IEEEkeywords}

\input{01_introduction/main}
\input{02_related_work/main}
\input{03_method/main}
\input{04_results/main}
\input{05_discussion/main}
\input{06_conclusion/main}
\input{appendix/main}

\bibliographystyle{ieeetr}
\bibliography{references}

\end{document}

% !TEX program = xelatex
\documentclass[conference]{IEEEtran}

% 日本語出力と XeLaTeX 対応設定
\usepackage{xeCJK}
\usepackage{fontspec}
\setCJKmainfont{HaranoAjiMincho}

% 数式・図表・参考文献などの標準パッケージ
\usepackage{amsmath, amssymb}
\usepackage{graphicx}
\usepackage{url}
\usepackage{hyperref}

\hypersetup{
  colorlinks=true,
  linkcolor=blue,
  citecolor=blue,
  urlcolor=blue
}

\begin{document}

\title{軽量 CNN と Vision Transformer の公平比較に向けた実験的評価}

\author{\IEEEauthorblockN{著者 太郎\\ }
\IEEEauthorblockA{所属機関\\ 連絡先: author@example.com}}

\maketitle

\begin{abstract}
本稿では,軽量な畳み込みニューラルネットワーク(CNN)と Vision Transformer(ViT)の画像分類性能を,CIFAR-10 を対象に公平に比較する実験プロトコルを提示する。学習率スケジューリング,データ拡張,Test-Time Augmentation (TTA) を統一的に適用し,再現性を重視した評価を実施した。実験の結果,ViT-Ti は TTA により最大で +1.8pt の精度向上を得た一方,ResNet-18 では RandAugment を中心としたデータ拡張で +2.1pt の改善が確認された。これらの知見を通じて,限られた計算資源でも堅牢なベースラインを構築するための指針を示す。
\end{abstract}

\begin{IEEEkeywords}
画像分類, 深層学習, Vision Transformer, Test-Time Augmentation, CIFAR-10
\end{IEEEkeywords}

\input{01_introduction/main}
\input{02_related_work/main}
\input{03_method/main}
\input{04_results/main}
\input{05_discussion/main}
\input{06_conclusion/main}
\input{appendix/main}

\bibliographystyle{ieeetr}
\bibliography{references}

\end{document}


\bibliographystyle{ieeetr}
\bibliography{references}

\end{document}


\bibliographystyle{ieeetr}
\bibliography{references}

\end{document}

% !TEX program = xelatex
\documentclass[conference]{IEEEtran}

% 日本語出力と XeLaTeX 対応設定
\usepackage{xeCJK}
\usepackage{fontspec}
\setCJKmainfont{HaranoAjiMincho}

% 数式・図表・参考文献などの標準パッケージ
\usepackage{amsmath, amssymb}
\usepackage{graphicx}
\usepackage{url}
\usepackage{hyperref}

\hypersetup{
  colorlinks=true,
  linkcolor=blue,
  citecolor=blue,
  urlcolor=blue
}

\begin{document}

\title{軽量 CNN と Vision Transformer の公平比較に向けた実験的評価}

\author{\IEEEauthorblockN{著者 太郎\\ }
\IEEEauthorblockA{所属機関\\ 連絡先: author@example.com}}

\maketitle

\begin{abstract}
本稿では,軽量な畳み込みニューラルネットワーク(CNN)と Vision Transformer(ViT)の画像分類性能を,CIFAR-10 を対象に公平に比較する実験プロトコルを提示する。学習率スケジューリング,データ拡張,Test-Time Augmentation (TTA) を統一的に適用し,再現性を重視した評価を実施した。実験の結果,ViT-Ti は TTA により最大で +1.8pt の精度向上を得た一方,ResNet-18 では RandAugment を中心としたデータ拡張で +2.1pt の改善が確認された。これらの知見を通じて,限られた計算資源でも堅牢なベースラインを構築するための指針を示す。
\end{abstract}

\begin{IEEEkeywords}
画像分類, 深層学習, Vision Transformer, Test-Time Augmentation, CIFAR-10
\end{IEEEkeywords}

% !TEX program = xelatex
\documentclass[conference]{IEEEtran}

% 日本語出力と XeLaTeX 対応設定
\usepackage{xeCJK}
\usepackage{fontspec}
\setCJKmainfont{HaranoAjiMincho}

% 数式・図表・参考文献などの標準パッケージ
\usepackage{amsmath, amssymb}
\usepackage{graphicx}
\usepackage{url}
\usepackage{hyperref}

\hypersetup{
  colorlinks=true,
  linkcolor=blue,
  citecolor=blue,
  urlcolor=blue
}

\begin{document}

\title{軽量 CNN と Vision Transformer の公平比較に向けた実験的評価}

\author{\IEEEauthorblockN{著者 太郎\\ }
\IEEEauthorblockA{所属機関\\ 連絡先: author@example.com}}

\maketitle

\begin{abstract}
本稿では,軽量な畳み込みニューラルネットワーク(CNN)と Vision Transformer(ViT)の画像分類性能を,CIFAR-10 を対象に公平に比較する実験プロトコルを提示する。学習率スケジューリング,データ拡張,Test-Time Augmentation (TTA) を統一的に適用し,再現性を重視した評価を実施した。実験の結果,ViT-Ti は TTA により最大で +1.8pt の精度向上を得た一方,ResNet-18 では RandAugment を中心としたデータ拡張で +2.1pt の改善が確認された。これらの知見を通じて,限られた計算資源でも堅牢なベースラインを構築するための指針を示す。
\end{abstract}

\begin{IEEEkeywords}
画像分類, 深層学習, Vision Transformer, Test-Time Augmentation, CIFAR-10
\end{IEEEkeywords}

% !TEX program = xelatex
\documentclass[conference]{IEEEtran}

% 日本語出力と XeLaTeX 対応設定
\usepackage{xeCJK}
\usepackage{fontspec}
\setCJKmainfont{HaranoAjiMincho}

% 数式・図表・参考文献などの標準パッケージ
\usepackage{amsmath, amssymb}
\usepackage{graphicx}
\usepackage{url}
\usepackage{hyperref}

\hypersetup{
  colorlinks=true,
  linkcolor=blue,
  citecolor=blue,
  urlcolor=blue
}

\begin{document}

\title{軽量 CNN と Vision Transformer の公平比較に向けた実験的評価}

\author{\IEEEauthorblockN{著者 太郎\\ }
\IEEEauthorblockA{所属機関\\ 連絡先: author@example.com}}

\maketitle

\begin{abstract}
本稿では,軽量な畳み込みニューラルネットワーク(CNN)と Vision Transformer(ViT)の画像分類性能を,CIFAR-10 を対象に公平に比較する実験プロトコルを提示する。学習率スケジューリング,データ拡張,Test-Time Augmentation (TTA) を統一的に適用し,再現性を重視した評価を実施した。実験の結果,ViT-Ti は TTA により最大で +1.8pt の精度向上を得た一方,ResNet-18 では RandAugment を中心としたデータ拡張で +2.1pt の改善が確認された。これらの知見を通じて,限られた計算資源でも堅牢なベースラインを構築するための指針を示す。
\end{abstract}

\begin{IEEEkeywords}
画像分類, 深層学習, Vision Transformer, Test-Time Augmentation, CIFAR-10
\end{IEEEkeywords}

\input{01_introduction/main}
\input{02_related_work/main}
\input{03_method/main}
\input{04_results/main}
\input{05_discussion/main}
\input{06_conclusion/main}
\input{appendix/main}

\bibliographystyle{ieeetr}
\bibliography{references}

\end{document}

% !TEX program = xelatex
\documentclass[conference]{IEEEtran}

% 日本語出力と XeLaTeX 対応設定
\usepackage{xeCJK}
\usepackage{fontspec}
\setCJKmainfont{HaranoAjiMincho}

% 数式・図表・参考文献などの標準パッケージ
\usepackage{amsmath, amssymb}
\usepackage{graphicx}
\usepackage{url}
\usepackage{hyperref}

\hypersetup{
  colorlinks=true,
  linkcolor=blue,
  citecolor=blue,
  urlcolor=blue
}

\begin{document}

\title{軽量 CNN と Vision Transformer の公平比較に向けた実験的評価}

\author{\IEEEauthorblockN{著者 太郎\\ }
\IEEEauthorblockA{所属機関\\ 連絡先: author@example.com}}

\maketitle

\begin{abstract}
本稿では,軽量な畳み込みニューラルネットワーク(CNN)と Vision Transformer(ViT)の画像分類性能を,CIFAR-10 を対象に公平に比較する実験プロトコルを提示する。学習率スケジューリング,データ拡張,Test-Time Augmentation (TTA) を統一的に適用し,再現性を重視した評価を実施した。実験の結果,ViT-Ti は TTA により最大で +1.8pt の精度向上を得た一方,ResNet-18 では RandAugment を中心としたデータ拡張で +2.1pt の改善が確認された。これらの知見を通じて,限られた計算資源でも堅牢なベースラインを構築するための指針を示す。
\end{abstract}

\begin{IEEEkeywords}
画像分類, 深層学習, Vision Transformer, Test-Time Augmentation, CIFAR-10
\end{IEEEkeywords}

\input{01_introduction/main}
\input{02_related_work/main}
\input{03_method/main}
\input{04_results/main}
\input{05_discussion/main}
\input{06_conclusion/main}
\input{appendix/main}

\bibliographystyle{ieeetr}
\bibliography{references}

\end{document}

% !TEX program = xelatex
\documentclass[conference]{IEEEtran}

% 日本語出力と XeLaTeX 対応設定
\usepackage{xeCJK}
\usepackage{fontspec}
\setCJKmainfont{HaranoAjiMincho}

% 数式・図表・参考文献などの標準パッケージ
\usepackage{amsmath, amssymb}
\usepackage{graphicx}
\usepackage{url}
\usepackage{hyperref}

\hypersetup{
  colorlinks=true,
  linkcolor=blue,
  citecolor=blue,
  urlcolor=blue
}

\begin{document}

\title{軽量 CNN と Vision Transformer の公平比較に向けた実験的評価}

\author{\IEEEauthorblockN{著者 太郎\\ }
\IEEEauthorblockA{所属機関\\ 連絡先: author@example.com}}

\maketitle

\begin{abstract}
本稿では,軽量な畳み込みニューラルネットワーク(CNN)と Vision Transformer(ViT)の画像分類性能を,CIFAR-10 を対象に公平に比較する実験プロトコルを提示する。学習率スケジューリング,データ拡張,Test-Time Augmentation (TTA) を統一的に適用し,再現性を重視した評価を実施した。実験の結果,ViT-Ti は TTA により最大で +1.8pt の精度向上を得た一方,ResNet-18 では RandAugment を中心としたデータ拡張で +2.1pt の改善が確認された。これらの知見を通じて,限られた計算資源でも堅牢なベースラインを構築するための指針を示す。
\end{abstract}

\begin{IEEEkeywords}
画像分類, 深層学習, Vision Transformer, Test-Time Augmentation, CIFAR-10
\end{IEEEkeywords}

\input{01_introduction/main}
\input{02_related_work/main}
\input{03_method/main}
\input{04_results/main}
\input{05_discussion/main}
\input{06_conclusion/main}
\input{appendix/main}

\bibliographystyle{ieeetr}
\bibliography{references}

\end{document}

% !TEX program = xelatex
\documentclass[conference]{IEEEtran}

% 日本語出力と XeLaTeX 対応設定
\usepackage{xeCJK}
\usepackage{fontspec}
\setCJKmainfont{HaranoAjiMincho}

% 数式・図表・参考文献などの標準パッケージ
\usepackage{amsmath, amssymb}
\usepackage{graphicx}
\usepackage{url}
\usepackage{hyperref}

\hypersetup{
  colorlinks=true,
  linkcolor=blue,
  citecolor=blue,
  urlcolor=blue
}

\begin{document}

\title{軽量 CNN と Vision Transformer の公平比較に向けた実験的評価}

\author{\IEEEauthorblockN{著者 太郎\\ }
\IEEEauthorblockA{所属機関\\ 連絡先: author@example.com}}

\maketitle

\begin{abstract}
本稿では,軽量な畳み込みニューラルネットワーク(CNN)と Vision Transformer(ViT)の画像分類性能を,CIFAR-10 を対象に公平に比較する実験プロトコルを提示する。学習率スケジューリング,データ拡張,Test-Time Augmentation (TTA) を統一的に適用し,再現性を重視した評価を実施した。実験の結果,ViT-Ti は TTA により最大で +1.8pt の精度向上を得た一方,ResNet-18 では RandAugment を中心としたデータ拡張で +2.1pt の改善が確認された。これらの知見を通じて,限られた計算資源でも堅牢なベースラインを構築するための指針を示す。
\end{abstract}

\begin{IEEEkeywords}
画像分類, 深層学習, Vision Transformer, Test-Time Augmentation, CIFAR-10
\end{IEEEkeywords}

\input{01_introduction/main}
\input{02_related_work/main}
\input{03_method/main}
\input{04_results/main}
\input{05_discussion/main}
\input{06_conclusion/main}
\input{appendix/main}

\bibliographystyle{ieeetr}
\bibliography{references}

\end{document}

% !TEX program = xelatex
\documentclass[conference]{IEEEtran}

% 日本語出力と XeLaTeX 対応設定
\usepackage{xeCJK}
\usepackage{fontspec}
\setCJKmainfont{HaranoAjiMincho}

% 数式・図表・参考文献などの標準パッケージ
\usepackage{amsmath, amssymb}
\usepackage{graphicx}
\usepackage{url}
\usepackage{hyperref}

\hypersetup{
  colorlinks=true,
  linkcolor=blue,
  citecolor=blue,
  urlcolor=blue
}

\begin{document}

\title{軽量 CNN と Vision Transformer の公平比較に向けた実験的評価}

\author{\IEEEauthorblockN{著者 太郎\\ }
\IEEEauthorblockA{所属機関\\ 連絡先: author@example.com}}

\maketitle

\begin{abstract}
本稿では,軽量な畳み込みニューラルネットワーク(CNN)と Vision Transformer(ViT)の画像分類性能を,CIFAR-10 を対象に公平に比較する実験プロトコルを提示する。学習率スケジューリング,データ拡張,Test-Time Augmentation (TTA) を統一的に適用し,再現性を重視した評価を実施した。実験の結果,ViT-Ti は TTA により最大で +1.8pt の精度向上を得た一方,ResNet-18 では RandAugment を中心としたデータ拡張で +2.1pt の改善が確認された。これらの知見を通じて,限られた計算資源でも堅牢なベースラインを構築するための指針を示す。
\end{abstract}

\begin{IEEEkeywords}
画像分類, 深層学習, Vision Transformer, Test-Time Augmentation, CIFAR-10
\end{IEEEkeywords}

\input{01_introduction/main}
\input{02_related_work/main}
\input{03_method/main}
\input{04_results/main}
\input{05_discussion/main}
\input{06_conclusion/main}
\input{appendix/main}

\bibliographystyle{ieeetr}
\bibliography{references}

\end{document}

% !TEX program = xelatex
\documentclass[conference]{IEEEtran}

% 日本語出力と XeLaTeX 対応設定
\usepackage{xeCJK}
\usepackage{fontspec}
\setCJKmainfont{HaranoAjiMincho}

% 数式・図表・参考文献などの標準パッケージ
\usepackage{amsmath, amssymb}
\usepackage{graphicx}
\usepackage{url}
\usepackage{hyperref}

\hypersetup{
  colorlinks=true,
  linkcolor=blue,
  citecolor=blue,
  urlcolor=blue
}

\begin{document}

\title{軽量 CNN と Vision Transformer の公平比較に向けた実験的評価}

\author{\IEEEauthorblockN{著者 太郎\\ }
\IEEEauthorblockA{所属機関\\ 連絡先: author@example.com}}

\maketitle

\begin{abstract}
本稿では,軽量な畳み込みニューラルネットワーク(CNN)と Vision Transformer(ViT)の画像分類性能を,CIFAR-10 を対象に公平に比較する実験プロトコルを提示する。学習率スケジューリング,データ拡張,Test-Time Augmentation (TTA) を統一的に適用し,再現性を重視した評価を実施した。実験の結果,ViT-Ti は TTA により最大で +1.8pt の精度向上を得た一方,ResNet-18 では RandAugment を中心としたデータ拡張で +2.1pt の改善が確認された。これらの知見を通じて,限られた計算資源でも堅牢なベースラインを構築するための指針を示す。
\end{abstract}

\begin{IEEEkeywords}
画像分類, 深層学習, Vision Transformer, Test-Time Augmentation, CIFAR-10
\end{IEEEkeywords}

\input{01_introduction/main}
\input{02_related_work/main}
\input{03_method/main}
\input{04_results/main}
\input{05_discussion/main}
\input{06_conclusion/main}
\input{appendix/main}

\bibliographystyle{ieeetr}
\bibliography{references}

\end{document}

% !TEX program = xelatex
\documentclass[conference]{IEEEtran}

% 日本語出力と XeLaTeX 対応設定
\usepackage{xeCJK}
\usepackage{fontspec}
\setCJKmainfont{HaranoAjiMincho}

% 数式・図表・参考文献などの標準パッケージ
\usepackage{amsmath, amssymb}
\usepackage{graphicx}
\usepackage{url}
\usepackage{hyperref}

\hypersetup{
  colorlinks=true,
  linkcolor=blue,
  citecolor=blue,
  urlcolor=blue
}

\begin{document}

\title{軽量 CNN と Vision Transformer の公平比較に向けた実験的評価}

\author{\IEEEauthorblockN{著者 太郎\\ }
\IEEEauthorblockA{所属機関\\ 連絡先: author@example.com}}

\maketitle

\begin{abstract}
本稿では,軽量な畳み込みニューラルネットワーク(CNN)と Vision Transformer(ViT)の画像分類性能を,CIFAR-10 を対象に公平に比較する実験プロトコルを提示する。学習率スケジューリング,データ拡張,Test-Time Augmentation (TTA) を統一的に適用し,再現性を重視した評価を実施した。実験の結果,ViT-Ti は TTA により最大で +1.8pt の精度向上を得た一方,ResNet-18 では RandAugment を中心としたデータ拡張で +2.1pt の改善が確認された。これらの知見を通じて,限られた計算資源でも堅牢なベースラインを構築するための指針を示す。
\end{abstract}

\begin{IEEEkeywords}
画像分類, 深層学習, Vision Transformer, Test-Time Augmentation, CIFAR-10
\end{IEEEkeywords}

\input{01_introduction/main}
\input{02_related_work/main}
\input{03_method/main}
\input{04_results/main}
\input{05_discussion/main}
\input{06_conclusion/main}
\input{appendix/main}

\bibliographystyle{ieeetr}
\bibliography{references}

\end{document}


\bibliographystyle{ieeetr}
\bibliography{references}

\end{document}

% !TEX program = xelatex
\documentclass[conference]{IEEEtran}

% 日本語出力と XeLaTeX 対応設定
\usepackage{xeCJK}
\usepackage{fontspec}
\setCJKmainfont{HaranoAjiMincho}

% 数式・図表・参考文献などの標準パッケージ
\usepackage{amsmath, amssymb}
\usepackage{graphicx}
\usepackage{url}
\usepackage{hyperref}

\hypersetup{
  colorlinks=true,
  linkcolor=blue,
  citecolor=blue,
  urlcolor=blue
}

\begin{document}

\title{軽量 CNN と Vision Transformer の公平比較に向けた実験的評価}

\author{\IEEEauthorblockN{著者 太郎\\ }
\IEEEauthorblockA{所属機関\\ 連絡先: author@example.com}}

\maketitle

\begin{abstract}
本稿では,軽量な畳み込みニューラルネットワーク(CNN)と Vision Transformer(ViT)の画像分類性能を,CIFAR-10 を対象に公平に比較する実験プロトコルを提示する。学習率スケジューリング,データ拡張,Test-Time Augmentation (TTA) を統一的に適用し,再現性を重視した評価を実施した。実験の結果,ViT-Ti は TTA により最大で +1.8pt の精度向上を得た一方,ResNet-18 では RandAugment を中心としたデータ拡張で +2.1pt の改善が確認された。これらの知見を通じて,限られた計算資源でも堅牢なベースラインを構築するための指針を示す。
\end{abstract}

\begin{IEEEkeywords}
画像分類, 深層学習, Vision Transformer, Test-Time Augmentation, CIFAR-10
\end{IEEEkeywords}

% !TEX program = xelatex
\documentclass[conference]{IEEEtran}

% 日本語出力と XeLaTeX 対応設定
\usepackage{xeCJK}
\usepackage{fontspec}
\setCJKmainfont{HaranoAjiMincho}

% 数式・図表・参考文献などの標準パッケージ
\usepackage{amsmath, amssymb}
\usepackage{graphicx}
\usepackage{url}
\usepackage{hyperref}

\hypersetup{
  colorlinks=true,
  linkcolor=blue,
  citecolor=blue,
  urlcolor=blue
}

\begin{document}

\title{軽量 CNN と Vision Transformer の公平比較に向けた実験的評価}

\author{\IEEEauthorblockN{著者 太郎\\ }
\IEEEauthorblockA{所属機関\\ 連絡先: author@example.com}}

\maketitle

\begin{abstract}
本稿では,軽量な畳み込みニューラルネットワーク(CNN)と Vision Transformer(ViT)の画像分類性能を,CIFAR-10 を対象に公平に比較する実験プロトコルを提示する。学習率スケジューリング,データ拡張,Test-Time Augmentation (TTA) を統一的に適用し,再現性を重視した評価を実施した。実験の結果,ViT-Ti は TTA により最大で +1.8pt の精度向上を得た一方,ResNet-18 では RandAugment を中心としたデータ拡張で +2.1pt の改善が確認された。これらの知見を通じて,限られた計算資源でも堅牢なベースラインを構築するための指針を示す。
\end{abstract}

\begin{IEEEkeywords}
画像分類, 深層学習, Vision Transformer, Test-Time Augmentation, CIFAR-10
\end{IEEEkeywords}

\input{01_introduction/main}
\input{02_related_work/main}
\input{03_method/main}
\input{04_results/main}
\input{05_discussion/main}
\input{06_conclusion/main}
\input{appendix/main}

\bibliographystyle{ieeetr}
\bibliography{references}

\end{document}

% !TEX program = xelatex
\documentclass[conference]{IEEEtran}

% 日本語出力と XeLaTeX 対応設定
\usepackage{xeCJK}
\usepackage{fontspec}
\setCJKmainfont{HaranoAjiMincho}

% 数式・図表・参考文献などの標準パッケージ
\usepackage{amsmath, amssymb}
\usepackage{graphicx}
\usepackage{url}
\usepackage{hyperref}

\hypersetup{
  colorlinks=true,
  linkcolor=blue,
  citecolor=blue,
  urlcolor=blue
}

\begin{document}

\title{軽量 CNN と Vision Transformer の公平比較に向けた実験的評価}

\author{\IEEEauthorblockN{著者 太郎\\ }
\IEEEauthorblockA{所属機関\\ 連絡先: author@example.com}}

\maketitle

\begin{abstract}
本稿では,軽量な畳み込みニューラルネットワーク(CNN)と Vision Transformer(ViT)の画像分類性能を,CIFAR-10 を対象に公平に比較する実験プロトコルを提示する。学習率スケジューリング,データ拡張,Test-Time Augmentation (TTA) を統一的に適用し,再現性を重視した評価を実施した。実験の結果,ViT-Ti は TTA により最大で +1.8pt の精度向上を得た一方,ResNet-18 では RandAugment を中心としたデータ拡張で +2.1pt の改善が確認された。これらの知見を通じて,限られた計算資源でも堅牢なベースラインを構築するための指針を示す。
\end{abstract}

\begin{IEEEkeywords}
画像分類, 深層学習, Vision Transformer, Test-Time Augmentation, CIFAR-10
\end{IEEEkeywords}

\input{01_introduction/main}
\input{02_related_work/main}
\input{03_method/main}
\input{04_results/main}
\input{05_discussion/main}
\input{06_conclusion/main}
\input{appendix/main}

\bibliographystyle{ieeetr}
\bibliography{references}

\end{document}

% !TEX program = xelatex
\documentclass[conference]{IEEEtran}

% 日本語出力と XeLaTeX 対応設定
\usepackage{xeCJK}
\usepackage{fontspec}
\setCJKmainfont{HaranoAjiMincho}

% 数式・図表・参考文献などの標準パッケージ
\usepackage{amsmath, amssymb}
\usepackage{graphicx}
\usepackage{url}
\usepackage{hyperref}

\hypersetup{
  colorlinks=true,
  linkcolor=blue,
  citecolor=blue,
  urlcolor=blue
}

\begin{document}

\title{軽量 CNN と Vision Transformer の公平比較に向けた実験的評価}

\author{\IEEEauthorblockN{著者 太郎\\ }
\IEEEauthorblockA{所属機関\\ 連絡先: author@example.com}}

\maketitle

\begin{abstract}
本稿では,軽量な畳み込みニューラルネットワーク(CNN)と Vision Transformer(ViT)の画像分類性能を,CIFAR-10 を対象に公平に比較する実験プロトコルを提示する。学習率スケジューリング,データ拡張,Test-Time Augmentation (TTA) を統一的に適用し,再現性を重視した評価を実施した。実験の結果,ViT-Ti は TTA により最大で +1.8pt の精度向上を得た一方,ResNet-18 では RandAugment を中心としたデータ拡張で +2.1pt の改善が確認された。これらの知見を通じて,限られた計算資源でも堅牢なベースラインを構築するための指針を示す。
\end{abstract}

\begin{IEEEkeywords}
画像分類, 深層学習, Vision Transformer, Test-Time Augmentation, CIFAR-10
\end{IEEEkeywords}

\input{01_introduction/main}
\input{02_related_work/main}
\input{03_method/main}
\input{04_results/main}
\input{05_discussion/main}
\input{06_conclusion/main}
\input{appendix/main}

\bibliographystyle{ieeetr}
\bibliography{references}

\end{document}

% !TEX program = xelatex
\documentclass[conference]{IEEEtran}

% 日本語出力と XeLaTeX 対応設定
\usepackage{xeCJK}
\usepackage{fontspec}
\setCJKmainfont{HaranoAjiMincho}

% 数式・図表・参考文献などの標準パッケージ
\usepackage{amsmath, amssymb}
\usepackage{graphicx}
\usepackage{url}
\usepackage{hyperref}

\hypersetup{
  colorlinks=true,
  linkcolor=blue,
  citecolor=blue,
  urlcolor=blue
}

\begin{document}

\title{軽量 CNN と Vision Transformer の公平比較に向けた実験的評価}

\author{\IEEEauthorblockN{著者 太郎\\ }
\IEEEauthorblockA{所属機関\\ 連絡先: author@example.com}}

\maketitle

\begin{abstract}
本稿では,軽量な畳み込みニューラルネットワーク(CNN)と Vision Transformer(ViT)の画像分類性能を,CIFAR-10 を対象に公平に比較する実験プロトコルを提示する。学習率スケジューリング,データ拡張,Test-Time Augmentation (TTA) を統一的に適用し,再現性を重視した評価を実施した。実験の結果,ViT-Ti は TTA により最大で +1.8pt の精度向上を得た一方,ResNet-18 では RandAugment を中心としたデータ拡張で +2.1pt の改善が確認された。これらの知見を通じて,限られた計算資源でも堅牢なベースラインを構築するための指針を示す。
\end{abstract}

\begin{IEEEkeywords}
画像分類, 深層学習, Vision Transformer, Test-Time Augmentation, CIFAR-10
\end{IEEEkeywords}

\input{01_introduction/main}
\input{02_related_work/main}
\input{03_method/main}
\input{04_results/main}
\input{05_discussion/main}
\input{06_conclusion/main}
\input{appendix/main}

\bibliographystyle{ieeetr}
\bibliography{references}

\end{document}

% !TEX program = xelatex
\documentclass[conference]{IEEEtran}

% 日本語出力と XeLaTeX 対応設定
\usepackage{xeCJK}
\usepackage{fontspec}
\setCJKmainfont{HaranoAjiMincho}

% 数式・図表・参考文献などの標準パッケージ
\usepackage{amsmath, amssymb}
\usepackage{graphicx}
\usepackage{url}
\usepackage{hyperref}

\hypersetup{
  colorlinks=true,
  linkcolor=blue,
  citecolor=blue,
  urlcolor=blue
}

\begin{document}

\title{軽量 CNN と Vision Transformer の公平比較に向けた実験的評価}

\author{\IEEEauthorblockN{著者 太郎\\ }
\IEEEauthorblockA{所属機関\\ 連絡先: author@example.com}}

\maketitle

\begin{abstract}
本稿では,軽量な畳み込みニューラルネットワーク(CNN)と Vision Transformer(ViT)の画像分類性能を,CIFAR-10 を対象に公平に比較する実験プロトコルを提示する。学習率スケジューリング,データ拡張,Test-Time Augmentation (TTA) を統一的に適用し,再現性を重視した評価を実施した。実験の結果,ViT-Ti は TTA により最大で +1.8pt の精度向上を得た一方,ResNet-18 では RandAugment を中心としたデータ拡張で +2.1pt の改善が確認された。これらの知見を通じて,限られた計算資源でも堅牢なベースラインを構築するための指針を示す。
\end{abstract}

\begin{IEEEkeywords}
画像分類, 深層学習, Vision Transformer, Test-Time Augmentation, CIFAR-10
\end{IEEEkeywords}

\input{01_introduction/main}
\input{02_related_work/main}
\input{03_method/main}
\input{04_results/main}
\input{05_discussion/main}
\input{06_conclusion/main}
\input{appendix/main}

\bibliographystyle{ieeetr}
\bibliography{references}

\end{document}

% !TEX program = xelatex
\documentclass[conference]{IEEEtran}

% 日本語出力と XeLaTeX 対応設定
\usepackage{xeCJK}
\usepackage{fontspec}
\setCJKmainfont{HaranoAjiMincho}

% 数式・図表・参考文献などの標準パッケージ
\usepackage{amsmath, amssymb}
\usepackage{graphicx}
\usepackage{url}
\usepackage{hyperref}

\hypersetup{
  colorlinks=true,
  linkcolor=blue,
  citecolor=blue,
  urlcolor=blue
}

\begin{document}

\title{軽量 CNN と Vision Transformer の公平比較に向けた実験的評価}

\author{\IEEEauthorblockN{著者 太郎\\ }
\IEEEauthorblockA{所属機関\\ 連絡先: author@example.com}}

\maketitle

\begin{abstract}
本稿では,軽量な畳み込みニューラルネットワーク(CNN)と Vision Transformer(ViT)の画像分類性能を,CIFAR-10 を対象に公平に比較する実験プロトコルを提示する。学習率スケジューリング,データ拡張,Test-Time Augmentation (TTA) を統一的に適用し,再現性を重視した評価を実施した。実験の結果,ViT-Ti は TTA により最大で +1.8pt の精度向上を得た一方,ResNet-18 では RandAugment を中心としたデータ拡張で +2.1pt の改善が確認された。これらの知見を通じて,限られた計算資源でも堅牢なベースラインを構築するための指針を示す。
\end{abstract}

\begin{IEEEkeywords}
画像分類, 深層学習, Vision Transformer, Test-Time Augmentation, CIFAR-10
\end{IEEEkeywords}

\input{01_introduction/main}
\input{02_related_work/main}
\input{03_method/main}
\input{04_results/main}
\input{05_discussion/main}
\input{06_conclusion/main}
\input{appendix/main}

\bibliographystyle{ieeetr}
\bibliography{references}

\end{document}

% !TEX program = xelatex
\documentclass[conference]{IEEEtran}

% 日本語出力と XeLaTeX 対応設定
\usepackage{xeCJK}
\usepackage{fontspec}
\setCJKmainfont{HaranoAjiMincho}

% 数式・図表・参考文献などの標準パッケージ
\usepackage{amsmath, amssymb}
\usepackage{graphicx}
\usepackage{url}
\usepackage{hyperref}

\hypersetup{
  colorlinks=true,
  linkcolor=blue,
  citecolor=blue,
  urlcolor=blue
}

\begin{document}

\title{軽量 CNN と Vision Transformer の公平比較に向けた実験的評価}

\author{\IEEEauthorblockN{著者 太郎\\ }
\IEEEauthorblockA{所属機関\\ 連絡先: author@example.com}}

\maketitle

\begin{abstract}
本稿では,軽量な畳み込みニューラルネットワーク(CNN)と Vision Transformer(ViT)の画像分類性能を,CIFAR-10 を対象に公平に比較する実験プロトコルを提示する。学習率スケジューリング,データ拡張,Test-Time Augmentation (TTA) を統一的に適用し,再現性を重視した評価を実施した。実験の結果,ViT-Ti は TTA により最大で +1.8pt の精度向上を得た一方,ResNet-18 では RandAugment を中心としたデータ拡張で +2.1pt の改善が確認された。これらの知見を通じて,限られた計算資源でも堅牢なベースラインを構築するための指針を示す。
\end{abstract}

\begin{IEEEkeywords}
画像分類, 深層学習, Vision Transformer, Test-Time Augmentation, CIFAR-10
\end{IEEEkeywords}

\input{01_introduction/main}
\input{02_related_work/main}
\input{03_method/main}
\input{04_results/main}
\input{05_discussion/main}
\input{06_conclusion/main}
\input{appendix/main}

\bibliographystyle{ieeetr}
\bibliography{references}

\end{document}


\bibliographystyle{ieeetr}
\bibliography{references}

\end{document}

% !TEX program = xelatex
\documentclass[conference]{IEEEtran}

% 日本語出力と XeLaTeX 対応設定
\usepackage{xeCJK}
\usepackage{fontspec}
\setCJKmainfont{HaranoAjiMincho}

% 数式・図表・参考文献などの標準パッケージ
\usepackage{amsmath, amssymb}
\usepackage{graphicx}
\usepackage{url}
\usepackage{hyperref}

\hypersetup{
  colorlinks=true,
  linkcolor=blue,
  citecolor=blue,
  urlcolor=blue
}

\begin{document}

\title{軽量 CNN と Vision Transformer の公平比較に向けた実験的評価}

\author{\IEEEauthorblockN{著者 太郎\\ }
\IEEEauthorblockA{所属機関\\ 連絡先: author@example.com}}

\maketitle

\begin{abstract}
本稿では,軽量な畳み込みニューラルネットワーク(CNN)と Vision Transformer(ViT)の画像分類性能を,CIFAR-10 を対象に公平に比較する実験プロトコルを提示する。学習率スケジューリング,データ拡張,Test-Time Augmentation (TTA) を統一的に適用し,再現性を重視した評価を実施した。実験の結果,ViT-Ti は TTA により最大で +1.8pt の精度向上を得た一方,ResNet-18 では RandAugment を中心としたデータ拡張で +2.1pt の改善が確認された。これらの知見を通じて,限られた計算資源でも堅牢なベースラインを構築するための指針を示す。
\end{abstract}

\begin{IEEEkeywords}
画像分類, 深層学習, Vision Transformer, Test-Time Augmentation, CIFAR-10
\end{IEEEkeywords}

% !TEX program = xelatex
\documentclass[conference]{IEEEtran}

% 日本語出力と XeLaTeX 対応設定
\usepackage{xeCJK}
\usepackage{fontspec}
\setCJKmainfont{HaranoAjiMincho}

% 数式・図表・参考文献などの標準パッケージ
\usepackage{amsmath, amssymb}
\usepackage{graphicx}
\usepackage{url}
\usepackage{hyperref}

\hypersetup{
  colorlinks=true,
  linkcolor=blue,
  citecolor=blue,
  urlcolor=blue
}

\begin{document}

\title{軽量 CNN と Vision Transformer の公平比較に向けた実験的評価}

\author{\IEEEauthorblockN{著者 太郎\\ }
\IEEEauthorblockA{所属機関\\ 連絡先: author@example.com}}

\maketitle

\begin{abstract}
本稿では,軽量な畳み込みニューラルネットワーク(CNN)と Vision Transformer(ViT)の画像分類性能を,CIFAR-10 を対象に公平に比較する実験プロトコルを提示する。学習率スケジューリング,データ拡張,Test-Time Augmentation (TTA) を統一的に適用し,再現性を重視した評価を実施した。実験の結果,ViT-Ti は TTA により最大で +1.8pt の精度向上を得た一方,ResNet-18 では RandAugment を中心としたデータ拡張で +2.1pt の改善が確認された。これらの知見を通じて,限られた計算資源でも堅牢なベースラインを構築するための指針を示す。
\end{abstract}

\begin{IEEEkeywords}
画像分類, 深層学習, Vision Transformer, Test-Time Augmentation, CIFAR-10
\end{IEEEkeywords}

\input{01_introduction/main}
\input{02_related_work/main}
\input{03_method/main}
\input{04_results/main}
\input{05_discussion/main}
\input{06_conclusion/main}
\input{appendix/main}

\bibliographystyle{ieeetr}
\bibliography{references}

\end{document}

% !TEX program = xelatex
\documentclass[conference]{IEEEtran}

% 日本語出力と XeLaTeX 対応設定
\usepackage{xeCJK}
\usepackage{fontspec}
\setCJKmainfont{HaranoAjiMincho}

% 数式・図表・参考文献などの標準パッケージ
\usepackage{amsmath, amssymb}
\usepackage{graphicx}
\usepackage{url}
\usepackage{hyperref}

\hypersetup{
  colorlinks=true,
  linkcolor=blue,
  citecolor=blue,
  urlcolor=blue
}

\begin{document}

\title{軽量 CNN と Vision Transformer の公平比較に向けた実験的評価}

\author{\IEEEauthorblockN{著者 太郎\\ }
\IEEEauthorblockA{所属機関\\ 連絡先: author@example.com}}

\maketitle

\begin{abstract}
本稿では,軽量な畳み込みニューラルネットワーク(CNN)と Vision Transformer(ViT)の画像分類性能を,CIFAR-10 を対象に公平に比較する実験プロトコルを提示する。学習率スケジューリング,データ拡張,Test-Time Augmentation (TTA) を統一的に適用し,再現性を重視した評価を実施した。実験の結果,ViT-Ti は TTA により最大で +1.8pt の精度向上を得た一方,ResNet-18 では RandAugment を中心としたデータ拡張で +2.1pt の改善が確認された。これらの知見を通じて,限られた計算資源でも堅牢なベースラインを構築するための指針を示す。
\end{abstract}

\begin{IEEEkeywords}
画像分類, 深層学習, Vision Transformer, Test-Time Augmentation, CIFAR-10
\end{IEEEkeywords}

\input{01_introduction/main}
\input{02_related_work/main}
\input{03_method/main}
\input{04_results/main}
\input{05_discussion/main}
\input{06_conclusion/main}
\input{appendix/main}

\bibliographystyle{ieeetr}
\bibliography{references}

\end{document}

% !TEX program = xelatex
\documentclass[conference]{IEEEtran}

% 日本語出力と XeLaTeX 対応設定
\usepackage{xeCJK}
\usepackage{fontspec}
\setCJKmainfont{HaranoAjiMincho}

% 数式・図表・参考文献などの標準パッケージ
\usepackage{amsmath, amssymb}
\usepackage{graphicx}
\usepackage{url}
\usepackage{hyperref}

\hypersetup{
  colorlinks=true,
  linkcolor=blue,
  citecolor=blue,
  urlcolor=blue
}

\begin{document}

\title{軽量 CNN と Vision Transformer の公平比較に向けた実験的評価}

\author{\IEEEauthorblockN{著者 太郎\\ }
\IEEEauthorblockA{所属機関\\ 連絡先: author@example.com}}

\maketitle

\begin{abstract}
本稿では,軽量な畳み込みニューラルネットワーク(CNN)と Vision Transformer(ViT)の画像分類性能を,CIFAR-10 を対象に公平に比較する実験プロトコルを提示する。学習率スケジューリング,データ拡張,Test-Time Augmentation (TTA) を統一的に適用し,再現性を重視した評価を実施した。実験の結果,ViT-Ti は TTA により最大で +1.8pt の精度向上を得た一方,ResNet-18 では RandAugment を中心としたデータ拡張で +2.1pt の改善が確認された。これらの知見を通じて,限られた計算資源でも堅牢なベースラインを構築するための指針を示す。
\end{abstract}

\begin{IEEEkeywords}
画像分類, 深層学習, Vision Transformer, Test-Time Augmentation, CIFAR-10
\end{IEEEkeywords}

\input{01_introduction/main}
\input{02_related_work/main}
\input{03_method/main}
\input{04_results/main}
\input{05_discussion/main}
\input{06_conclusion/main}
\input{appendix/main}

\bibliographystyle{ieeetr}
\bibliography{references}

\end{document}

% !TEX program = xelatex
\documentclass[conference]{IEEEtran}

% 日本語出力と XeLaTeX 対応設定
\usepackage{xeCJK}
\usepackage{fontspec}
\setCJKmainfont{HaranoAjiMincho}

% 数式・図表・参考文献などの標準パッケージ
\usepackage{amsmath, amssymb}
\usepackage{graphicx}
\usepackage{url}
\usepackage{hyperref}

\hypersetup{
  colorlinks=true,
  linkcolor=blue,
  citecolor=blue,
  urlcolor=blue
}

\begin{document}

\title{軽量 CNN と Vision Transformer の公平比較に向けた実験的評価}

\author{\IEEEauthorblockN{著者 太郎\\ }
\IEEEauthorblockA{所属機関\\ 連絡先: author@example.com}}

\maketitle

\begin{abstract}
本稿では,軽量な畳み込みニューラルネットワーク(CNN)と Vision Transformer(ViT)の画像分類性能を,CIFAR-10 を対象に公平に比較する実験プロトコルを提示する。学習率スケジューリング,データ拡張,Test-Time Augmentation (TTA) を統一的に適用し,再現性を重視した評価を実施した。実験の結果,ViT-Ti は TTA により最大で +1.8pt の精度向上を得た一方,ResNet-18 では RandAugment を中心としたデータ拡張で +2.1pt の改善が確認された。これらの知見を通じて,限られた計算資源でも堅牢なベースラインを構築するための指針を示す。
\end{abstract}

\begin{IEEEkeywords}
画像分類, 深層学習, Vision Transformer, Test-Time Augmentation, CIFAR-10
\end{IEEEkeywords}

\input{01_introduction/main}
\input{02_related_work/main}
\input{03_method/main}
\input{04_results/main}
\input{05_discussion/main}
\input{06_conclusion/main}
\input{appendix/main}

\bibliographystyle{ieeetr}
\bibliography{references}

\end{document}

% !TEX program = xelatex
\documentclass[conference]{IEEEtran}

% 日本語出力と XeLaTeX 対応設定
\usepackage{xeCJK}
\usepackage{fontspec}
\setCJKmainfont{HaranoAjiMincho}

% 数式・図表・参考文献などの標準パッケージ
\usepackage{amsmath, amssymb}
\usepackage{graphicx}
\usepackage{url}
\usepackage{hyperref}

\hypersetup{
  colorlinks=true,
  linkcolor=blue,
  citecolor=blue,
  urlcolor=blue
}

\begin{document}

\title{軽量 CNN と Vision Transformer の公平比較に向けた実験的評価}

\author{\IEEEauthorblockN{著者 太郎\\ }
\IEEEauthorblockA{所属機関\\ 連絡先: author@example.com}}

\maketitle

\begin{abstract}
本稿では,軽量な畳み込みニューラルネットワーク(CNN)と Vision Transformer(ViT)の画像分類性能を,CIFAR-10 を対象に公平に比較する実験プロトコルを提示する。学習率スケジューリング,データ拡張,Test-Time Augmentation (TTA) を統一的に適用し,再現性を重視した評価を実施した。実験の結果,ViT-Ti は TTA により最大で +1.8pt の精度向上を得た一方,ResNet-18 では RandAugment を中心としたデータ拡張で +2.1pt の改善が確認された。これらの知見を通じて,限られた計算資源でも堅牢なベースラインを構築するための指針を示す。
\end{abstract}

\begin{IEEEkeywords}
画像分類, 深層学習, Vision Transformer, Test-Time Augmentation, CIFAR-10
\end{IEEEkeywords}

\input{01_introduction/main}
\input{02_related_work/main}
\input{03_method/main}
\input{04_results/main}
\input{05_discussion/main}
\input{06_conclusion/main}
\input{appendix/main}

\bibliographystyle{ieeetr}
\bibliography{references}

\end{document}

% !TEX program = xelatex
\documentclass[conference]{IEEEtran}

% 日本語出力と XeLaTeX 対応設定
\usepackage{xeCJK}
\usepackage{fontspec}
\setCJKmainfont{HaranoAjiMincho}

% 数式・図表・参考文献などの標準パッケージ
\usepackage{amsmath, amssymb}
\usepackage{graphicx}
\usepackage{url}
\usepackage{hyperref}

\hypersetup{
  colorlinks=true,
  linkcolor=blue,
  citecolor=blue,
  urlcolor=blue
}

\begin{document}

\title{軽量 CNN と Vision Transformer の公平比較に向けた実験的評価}

\author{\IEEEauthorblockN{著者 太郎\\ }
\IEEEauthorblockA{所属機関\\ 連絡先: author@example.com}}

\maketitle

\begin{abstract}
本稿では,軽量な畳み込みニューラルネットワーク(CNN)と Vision Transformer(ViT)の画像分類性能を,CIFAR-10 を対象に公平に比較する実験プロトコルを提示する。学習率スケジューリング,データ拡張,Test-Time Augmentation (TTA) を統一的に適用し,再現性を重視した評価を実施した。実験の結果,ViT-Ti は TTA により最大で +1.8pt の精度向上を得た一方,ResNet-18 では RandAugment を中心としたデータ拡張で +2.1pt の改善が確認された。これらの知見を通じて,限られた計算資源でも堅牢なベースラインを構築するための指針を示す。
\end{abstract}

\begin{IEEEkeywords}
画像分類, 深層学習, Vision Transformer, Test-Time Augmentation, CIFAR-10
\end{IEEEkeywords}

\input{01_introduction/main}
\input{02_related_work/main}
\input{03_method/main}
\input{04_results/main}
\input{05_discussion/main}
\input{06_conclusion/main}
\input{appendix/main}

\bibliographystyle{ieeetr}
\bibliography{references}

\end{document}

% !TEX program = xelatex
\documentclass[conference]{IEEEtran}

% 日本語出力と XeLaTeX 対応設定
\usepackage{xeCJK}
\usepackage{fontspec}
\setCJKmainfont{HaranoAjiMincho}

% 数式・図表・参考文献などの標準パッケージ
\usepackage{amsmath, amssymb}
\usepackage{graphicx}
\usepackage{url}
\usepackage{hyperref}

\hypersetup{
  colorlinks=true,
  linkcolor=blue,
  citecolor=blue,
  urlcolor=blue
}

\begin{document}

\title{軽量 CNN と Vision Transformer の公平比較に向けた実験的評価}

\author{\IEEEauthorblockN{著者 太郎\\ }
\IEEEauthorblockA{所属機関\\ 連絡先: author@example.com}}

\maketitle

\begin{abstract}
本稿では,軽量な畳み込みニューラルネットワーク(CNN)と Vision Transformer(ViT)の画像分類性能を,CIFAR-10 を対象に公平に比較する実験プロトコルを提示する。学習率スケジューリング,データ拡張,Test-Time Augmentation (TTA) を統一的に適用し,再現性を重視した評価を実施した。実験の結果,ViT-Ti は TTA により最大で +1.8pt の精度向上を得た一方,ResNet-18 では RandAugment を中心としたデータ拡張で +2.1pt の改善が確認された。これらの知見を通じて,限られた計算資源でも堅牢なベースラインを構築するための指針を示す。
\end{abstract}

\begin{IEEEkeywords}
画像分類, 深層学習, Vision Transformer, Test-Time Augmentation, CIFAR-10
\end{IEEEkeywords}

\input{01_introduction/main}
\input{02_related_work/main}
\input{03_method/main}
\input{04_results/main}
\input{05_discussion/main}
\input{06_conclusion/main}
\input{appendix/main}

\bibliographystyle{ieeetr}
\bibliography{references}

\end{document}


\bibliographystyle{ieeetr}
\bibliography{references}

\end{document}

% !TEX program = xelatex
\documentclass[conference]{IEEEtran}

% 日本語出力と XeLaTeX 対応設定
\usepackage{xeCJK}
\usepackage{fontspec}
\setCJKmainfont{HaranoAjiMincho}

% 数式・図表・参考文献などの標準パッケージ
\usepackage{amsmath, amssymb}
\usepackage{graphicx}
\usepackage{url}
\usepackage{hyperref}

\hypersetup{
  colorlinks=true,
  linkcolor=blue,
  citecolor=blue,
  urlcolor=blue
}

\begin{document}

\title{軽量 CNN と Vision Transformer の公平比較に向けた実験的評価}

\author{\IEEEauthorblockN{著者 太郎\\ }
\IEEEauthorblockA{所属機関\\ 連絡先: author@example.com}}

\maketitle

\begin{abstract}
本稿では,軽量な畳み込みニューラルネットワーク(CNN)と Vision Transformer(ViT)の画像分類性能を,CIFAR-10 を対象に公平に比較する実験プロトコルを提示する。学習率スケジューリング,データ拡張,Test-Time Augmentation (TTA) を統一的に適用し,再現性を重視した評価を実施した。実験の結果,ViT-Ti は TTA により最大で +1.8pt の精度向上を得た一方,ResNet-18 では RandAugment を中心としたデータ拡張で +2.1pt の改善が確認された。これらの知見を通じて,限られた計算資源でも堅牢なベースラインを構築するための指針を示す。
\end{abstract}

\begin{IEEEkeywords}
画像分類, 深層学習, Vision Transformer, Test-Time Augmentation, CIFAR-10
\end{IEEEkeywords}

% !TEX program = xelatex
\documentclass[conference]{IEEEtran}

% 日本語出力と XeLaTeX 対応設定
\usepackage{xeCJK}
\usepackage{fontspec}
\setCJKmainfont{HaranoAjiMincho}

% 数式・図表・参考文献などの標準パッケージ
\usepackage{amsmath, amssymb}
\usepackage{graphicx}
\usepackage{url}
\usepackage{hyperref}

\hypersetup{
  colorlinks=true,
  linkcolor=blue,
  citecolor=blue,
  urlcolor=blue
}

\begin{document}

\title{軽量 CNN と Vision Transformer の公平比較に向けた実験的評価}

\author{\IEEEauthorblockN{著者 太郎\\ }
\IEEEauthorblockA{所属機関\\ 連絡先: author@example.com}}

\maketitle

\begin{abstract}
本稿では,軽量な畳み込みニューラルネットワーク(CNN)と Vision Transformer(ViT)の画像分類性能を,CIFAR-10 を対象に公平に比較する実験プロトコルを提示する。学習率スケジューリング,データ拡張,Test-Time Augmentation (TTA) を統一的に適用し,再現性を重視した評価を実施した。実験の結果,ViT-Ti は TTA により最大で +1.8pt の精度向上を得た一方,ResNet-18 では RandAugment を中心としたデータ拡張で +2.1pt の改善が確認された。これらの知見を通じて,限られた計算資源でも堅牢なベースラインを構築するための指針を示す。
\end{abstract}

\begin{IEEEkeywords}
画像分類, 深層学習, Vision Transformer, Test-Time Augmentation, CIFAR-10
\end{IEEEkeywords}

\input{01_introduction/main}
\input{02_related_work/main}
\input{03_method/main}
\input{04_results/main}
\input{05_discussion/main}
\input{06_conclusion/main}
\input{appendix/main}

\bibliographystyle{ieeetr}
\bibliography{references}

\end{document}

% !TEX program = xelatex
\documentclass[conference]{IEEEtran}

% 日本語出力と XeLaTeX 対応設定
\usepackage{xeCJK}
\usepackage{fontspec}
\setCJKmainfont{HaranoAjiMincho}

% 数式・図表・参考文献などの標準パッケージ
\usepackage{amsmath, amssymb}
\usepackage{graphicx}
\usepackage{url}
\usepackage{hyperref}

\hypersetup{
  colorlinks=true,
  linkcolor=blue,
  citecolor=blue,
  urlcolor=blue
}

\begin{document}

\title{軽量 CNN と Vision Transformer の公平比較に向けた実験的評価}

\author{\IEEEauthorblockN{著者 太郎\\ }
\IEEEauthorblockA{所属機関\\ 連絡先: author@example.com}}

\maketitle

\begin{abstract}
本稿では,軽量な畳み込みニューラルネットワーク(CNN)と Vision Transformer(ViT)の画像分類性能を,CIFAR-10 を対象に公平に比較する実験プロトコルを提示する。学習率スケジューリング,データ拡張,Test-Time Augmentation (TTA) を統一的に適用し,再現性を重視した評価を実施した。実験の結果,ViT-Ti は TTA により最大で +1.8pt の精度向上を得た一方,ResNet-18 では RandAugment を中心としたデータ拡張で +2.1pt の改善が確認された。これらの知見を通じて,限られた計算資源でも堅牢なベースラインを構築するための指針を示す。
\end{abstract}

\begin{IEEEkeywords}
画像分類, 深層学習, Vision Transformer, Test-Time Augmentation, CIFAR-10
\end{IEEEkeywords}

\input{01_introduction/main}
\input{02_related_work/main}
\input{03_method/main}
\input{04_results/main}
\input{05_discussion/main}
\input{06_conclusion/main}
\input{appendix/main}

\bibliographystyle{ieeetr}
\bibliography{references}

\end{document}

% !TEX program = xelatex
\documentclass[conference]{IEEEtran}

% 日本語出力と XeLaTeX 対応設定
\usepackage{xeCJK}
\usepackage{fontspec}
\setCJKmainfont{HaranoAjiMincho}

% 数式・図表・参考文献などの標準パッケージ
\usepackage{amsmath, amssymb}
\usepackage{graphicx}
\usepackage{url}
\usepackage{hyperref}

\hypersetup{
  colorlinks=true,
  linkcolor=blue,
  citecolor=blue,
  urlcolor=blue
}

\begin{document}

\title{軽量 CNN と Vision Transformer の公平比較に向けた実験的評価}

\author{\IEEEauthorblockN{著者 太郎\\ }
\IEEEauthorblockA{所属機関\\ 連絡先: author@example.com}}

\maketitle

\begin{abstract}
本稿では,軽量な畳み込みニューラルネットワーク(CNN)と Vision Transformer(ViT)の画像分類性能を,CIFAR-10 を対象に公平に比較する実験プロトコルを提示する。学習率スケジューリング,データ拡張,Test-Time Augmentation (TTA) を統一的に適用し,再現性を重視した評価を実施した。実験の結果,ViT-Ti は TTA により最大で +1.8pt の精度向上を得た一方,ResNet-18 では RandAugment を中心としたデータ拡張で +2.1pt の改善が確認された。これらの知見を通じて,限られた計算資源でも堅牢なベースラインを構築するための指針を示す。
\end{abstract}

\begin{IEEEkeywords}
画像分類, 深層学習, Vision Transformer, Test-Time Augmentation, CIFAR-10
\end{IEEEkeywords}

\input{01_introduction/main}
\input{02_related_work/main}
\input{03_method/main}
\input{04_results/main}
\input{05_discussion/main}
\input{06_conclusion/main}
\input{appendix/main}

\bibliographystyle{ieeetr}
\bibliography{references}

\end{document}

% !TEX program = xelatex
\documentclass[conference]{IEEEtran}

% 日本語出力と XeLaTeX 対応設定
\usepackage{xeCJK}
\usepackage{fontspec}
\setCJKmainfont{HaranoAjiMincho}

% 数式・図表・参考文献などの標準パッケージ
\usepackage{amsmath, amssymb}
\usepackage{graphicx}
\usepackage{url}
\usepackage{hyperref}

\hypersetup{
  colorlinks=true,
  linkcolor=blue,
  citecolor=blue,
  urlcolor=blue
}

\begin{document}

\title{軽量 CNN と Vision Transformer の公平比較に向けた実験的評価}

\author{\IEEEauthorblockN{著者 太郎\\ }
\IEEEauthorblockA{所属機関\\ 連絡先: author@example.com}}

\maketitle

\begin{abstract}
本稿では,軽量な畳み込みニューラルネットワーク(CNN)と Vision Transformer(ViT)の画像分類性能を,CIFAR-10 を対象に公平に比較する実験プロトコルを提示する。学習率スケジューリング,データ拡張,Test-Time Augmentation (TTA) を統一的に適用し,再現性を重視した評価を実施した。実験の結果,ViT-Ti は TTA により最大で +1.8pt の精度向上を得た一方,ResNet-18 では RandAugment を中心としたデータ拡張で +2.1pt の改善が確認された。これらの知見を通じて,限られた計算資源でも堅牢なベースラインを構築するための指針を示す。
\end{abstract}

\begin{IEEEkeywords}
画像分類, 深層学習, Vision Transformer, Test-Time Augmentation, CIFAR-10
\end{IEEEkeywords}

\input{01_introduction/main}
\input{02_related_work/main}
\input{03_method/main}
\input{04_results/main}
\input{05_discussion/main}
\input{06_conclusion/main}
\input{appendix/main}

\bibliographystyle{ieeetr}
\bibliography{references}

\end{document}

% !TEX program = xelatex
\documentclass[conference]{IEEEtran}

% 日本語出力と XeLaTeX 対応設定
\usepackage{xeCJK}
\usepackage{fontspec}
\setCJKmainfont{HaranoAjiMincho}

% 数式・図表・参考文献などの標準パッケージ
\usepackage{amsmath, amssymb}
\usepackage{graphicx}
\usepackage{url}
\usepackage{hyperref}

\hypersetup{
  colorlinks=true,
  linkcolor=blue,
  citecolor=blue,
  urlcolor=blue
}

\begin{document}

\title{軽量 CNN と Vision Transformer の公平比較に向けた実験的評価}

\author{\IEEEauthorblockN{著者 太郎\\ }
\IEEEauthorblockA{所属機関\\ 連絡先: author@example.com}}

\maketitle

\begin{abstract}
本稿では,軽量な畳み込みニューラルネットワーク(CNN)と Vision Transformer(ViT)の画像分類性能を,CIFAR-10 を対象に公平に比較する実験プロトコルを提示する。学習率スケジューリング,データ拡張,Test-Time Augmentation (TTA) を統一的に適用し,再現性を重視した評価を実施した。実験の結果,ViT-Ti は TTA により最大で +1.8pt の精度向上を得た一方,ResNet-18 では RandAugment を中心としたデータ拡張で +2.1pt の改善が確認された。これらの知見を通じて,限られた計算資源でも堅牢なベースラインを構築するための指針を示す。
\end{abstract}

\begin{IEEEkeywords}
画像分類, 深層学習, Vision Transformer, Test-Time Augmentation, CIFAR-10
\end{IEEEkeywords}

\input{01_introduction/main}
\input{02_related_work/main}
\input{03_method/main}
\input{04_results/main}
\input{05_discussion/main}
\input{06_conclusion/main}
\input{appendix/main}

\bibliographystyle{ieeetr}
\bibliography{references}

\end{document}

% !TEX program = xelatex
\documentclass[conference]{IEEEtran}

% 日本語出力と XeLaTeX 対応設定
\usepackage{xeCJK}
\usepackage{fontspec}
\setCJKmainfont{HaranoAjiMincho}

% 数式・図表・参考文献などの標準パッケージ
\usepackage{amsmath, amssymb}
\usepackage{graphicx}
\usepackage{url}
\usepackage{hyperref}

\hypersetup{
  colorlinks=true,
  linkcolor=blue,
  citecolor=blue,
  urlcolor=blue
}

\begin{document}

\title{軽量 CNN と Vision Transformer の公平比較に向けた実験的評価}

\author{\IEEEauthorblockN{著者 太郎\\ }
\IEEEauthorblockA{所属機関\\ 連絡先: author@example.com}}

\maketitle

\begin{abstract}
本稿では,軽量な畳み込みニューラルネットワーク(CNN)と Vision Transformer(ViT)の画像分類性能を,CIFAR-10 を対象に公平に比較する実験プロトコルを提示する。学習率スケジューリング,データ拡張,Test-Time Augmentation (TTA) を統一的に適用し,再現性を重視した評価を実施した。実験の結果,ViT-Ti は TTA により最大で +1.8pt の精度向上を得た一方,ResNet-18 では RandAugment を中心としたデータ拡張で +2.1pt の改善が確認された。これらの知見を通じて,限られた計算資源でも堅牢なベースラインを構築するための指針を示す。
\end{abstract}

\begin{IEEEkeywords}
画像分類, 深層学習, Vision Transformer, Test-Time Augmentation, CIFAR-10
\end{IEEEkeywords}

\input{01_introduction/main}
\input{02_related_work/main}
\input{03_method/main}
\input{04_results/main}
\input{05_discussion/main}
\input{06_conclusion/main}
\input{appendix/main}

\bibliographystyle{ieeetr}
\bibliography{references}

\end{document}

% !TEX program = xelatex
\documentclass[conference]{IEEEtran}

% 日本語出力と XeLaTeX 対応設定
\usepackage{xeCJK}
\usepackage{fontspec}
\setCJKmainfont{HaranoAjiMincho}

% 数式・図表・参考文献などの標準パッケージ
\usepackage{amsmath, amssymb}
\usepackage{graphicx}
\usepackage{url}
\usepackage{hyperref}

\hypersetup{
  colorlinks=true,
  linkcolor=blue,
  citecolor=blue,
  urlcolor=blue
}

\begin{document}

\title{軽量 CNN と Vision Transformer の公平比較に向けた実験的評価}

\author{\IEEEauthorblockN{著者 太郎\\ }
\IEEEauthorblockA{所属機関\\ 連絡先: author@example.com}}

\maketitle

\begin{abstract}
本稿では,軽量な畳み込みニューラルネットワーク(CNN)と Vision Transformer(ViT)の画像分類性能を,CIFAR-10 を対象に公平に比較する実験プロトコルを提示する。学習率スケジューリング,データ拡張,Test-Time Augmentation (TTA) を統一的に適用し,再現性を重視した評価を実施した。実験の結果,ViT-Ti は TTA により最大で +1.8pt の精度向上を得た一方,ResNet-18 では RandAugment を中心としたデータ拡張で +2.1pt の改善が確認された。これらの知見を通じて,限られた計算資源でも堅牢なベースラインを構築するための指針を示す。
\end{abstract}

\begin{IEEEkeywords}
画像分類, 深層学習, Vision Transformer, Test-Time Augmentation, CIFAR-10
\end{IEEEkeywords}

\input{01_introduction/main}
\input{02_related_work/main}
\input{03_method/main}
\input{04_results/main}
\input{05_discussion/main}
\input{06_conclusion/main}
\input{appendix/main}

\bibliographystyle{ieeetr}
\bibliography{references}

\end{document}


\bibliographystyle{ieeetr}
\bibliography{references}

\end{document}

% !TEX program = xelatex
\documentclass[conference]{IEEEtran}

% 日本語出力と XeLaTeX 対応設定
\usepackage{xeCJK}
\usepackage{fontspec}
\setCJKmainfont{HaranoAjiMincho}

% 数式・図表・参考文献などの標準パッケージ
\usepackage{amsmath, amssymb}
\usepackage{graphicx}
\usepackage{url}
\usepackage{hyperref}

\hypersetup{
  colorlinks=true,
  linkcolor=blue,
  citecolor=blue,
  urlcolor=blue
}

\begin{document}

\title{軽量 CNN と Vision Transformer の公平比較に向けた実験的評価}

\author{\IEEEauthorblockN{著者 太郎\\ }
\IEEEauthorblockA{所属機関\\ 連絡先: author@example.com}}

\maketitle

\begin{abstract}
本稿では,軽量な畳み込みニューラルネットワーク(CNN)と Vision Transformer(ViT)の画像分類性能を,CIFAR-10 を対象に公平に比較する実験プロトコルを提示する。学習率スケジューリング,データ拡張,Test-Time Augmentation (TTA) を統一的に適用し,再現性を重視した評価を実施した。実験の結果,ViT-Ti は TTA により最大で +1.8pt の精度向上を得た一方,ResNet-18 では RandAugment を中心としたデータ拡張で +2.1pt の改善が確認された。これらの知見を通じて,限られた計算資源でも堅牢なベースラインを構築するための指針を示す。
\end{abstract}

\begin{IEEEkeywords}
画像分類, 深層学習, Vision Transformer, Test-Time Augmentation, CIFAR-10
\end{IEEEkeywords}

% !TEX program = xelatex
\documentclass[conference]{IEEEtran}

% 日本語出力と XeLaTeX 対応設定
\usepackage{xeCJK}
\usepackage{fontspec}
\setCJKmainfont{HaranoAjiMincho}

% 数式・図表・参考文献などの標準パッケージ
\usepackage{amsmath, amssymb}
\usepackage{graphicx}
\usepackage{url}
\usepackage{hyperref}

\hypersetup{
  colorlinks=true,
  linkcolor=blue,
  citecolor=blue,
  urlcolor=blue
}

\begin{document}

\title{軽量 CNN と Vision Transformer の公平比較に向けた実験的評価}

\author{\IEEEauthorblockN{著者 太郎\\ }
\IEEEauthorblockA{所属機関\\ 連絡先: author@example.com}}

\maketitle

\begin{abstract}
本稿では,軽量な畳み込みニューラルネットワーク(CNN)と Vision Transformer(ViT)の画像分類性能を,CIFAR-10 を対象に公平に比較する実験プロトコルを提示する。学習率スケジューリング,データ拡張,Test-Time Augmentation (TTA) を統一的に適用し,再現性を重視した評価を実施した。実験の結果,ViT-Ti は TTA により最大で +1.8pt の精度向上を得た一方,ResNet-18 では RandAugment を中心としたデータ拡張で +2.1pt の改善が確認された。これらの知見を通じて,限られた計算資源でも堅牢なベースラインを構築するための指針を示す。
\end{abstract}

\begin{IEEEkeywords}
画像分類, 深層学習, Vision Transformer, Test-Time Augmentation, CIFAR-10
\end{IEEEkeywords}

\input{01_introduction/main}
\input{02_related_work/main}
\input{03_method/main}
\input{04_results/main}
\input{05_discussion/main}
\input{06_conclusion/main}
\input{appendix/main}

\bibliographystyle{ieeetr}
\bibliography{references}

\end{document}

% !TEX program = xelatex
\documentclass[conference]{IEEEtran}

% 日本語出力と XeLaTeX 対応設定
\usepackage{xeCJK}
\usepackage{fontspec}
\setCJKmainfont{HaranoAjiMincho}

% 数式・図表・参考文献などの標準パッケージ
\usepackage{amsmath, amssymb}
\usepackage{graphicx}
\usepackage{url}
\usepackage{hyperref}

\hypersetup{
  colorlinks=true,
  linkcolor=blue,
  citecolor=blue,
  urlcolor=blue
}

\begin{document}

\title{軽量 CNN と Vision Transformer の公平比較に向けた実験的評価}

\author{\IEEEauthorblockN{著者 太郎\\ }
\IEEEauthorblockA{所属機関\\ 連絡先: author@example.com}}

\maketitle

\begin{abstract}
本稿では,軽量な畳み込みニューラルネットワーク(CNN)と Vision Transformer(ViT)の画像分類性能を,CIFAR-10 を対象に公平に比較する実験プロトコルを提示する。学習率スケジューリング,データ拡張,Test-Time Augmentation (TTA) を統一的に適用し,再現性を重視した評価を実施した。実験の結果,ViT-Ti は TTA により最大で +1.8pt の精度向上を得た一方,ResNet-18 では RandAugment を中心としたデータ拡張で +2.1pt の改善が確認された。これらの知見を通じて,限られた計算資源でも堅牢なベースラインを構築するための指針を示す。
\end{abstract}

\begin{IEEEkeywords}
画像分類, 深層学習, Vision Transformer, Test-Time Augmentation, CIFAR-10
\end{IEEEkeywords}

\input{01_introduction/main}
\input{02_related_work/main}
\input{03_method/main}
\input{04_results/main}
\input{05_discussion/main}
\input{06_conclusion/main}
\input{appendix/main}

\bibliographystyle{ieeetr}
\bibliography{references}

\end{document}

% !TEX program = xelatex
\documentclass[conference]{IEEEtran}

% 日本語出力と XeLaTeX 対応設定
\usepackage{xeCJK}
\usepackage{fontspec}
\setCJKmainfont{HaranoAjiMincho}

% 数式・図表・参考文献などの標準パッケージ
\usepackage{amsmath, amssymb}
\usepackage{graphicx}
\usepackage{url}
\usepackage{hyperref}

\hypersetup{
  colorlinks=true,
  linkcolor=blue,
  citecolor=blue,
  urlcolor=blue
}

\begin{document}

\title{軽量 CNN と Vision Transformer の公平比較に向けた実験的評価}

\author{\IEEEauthorblockN{著者 太郎\\ }
\IEEEauthorblockA{所属機関\\ 連絡先: author@example.com}}

\maketitle

\begin{abstract}
本稿では,軽量な畳み込みニューラルネットワーク(CNN)と Vision Transformer(ViT)の画像分類性能を,CIFAR-10 を対象に公平に比較する実験プロトコルを提示する。学習率スケジューリング,データ拡張,Test-Time Augmentation (TTA) を統一的に適用し,再現性を重視した評価を実施した。実験の結果,ViT-Ti は TTA により最大で +1.8pt の精度向上を得た一方,ResNet-18 では RandAugment を中心としたデータ拡張で +2.1pt の改善が確認された。これらの知見を通じて,限られた計算資源でも堅牢なベースラインを構築するための指針を示す。
\end{abstract}

\begin{IEEEkeywords}
画像分類, 深層学習, Vision Transformer, Test-Time Augmentation, CIFAR-10
\end{IEEEkeywords}

\input{01_introduction/main}
\input{02_related_work/main}
\input{03_method/main}
\input{04_results/main}
\input{05_discussion/main}
\input{06_conclusion/main}
\input{appendix/main}

\bibliographystyle{ieeetr}
\bibliography{references}

\end{document}

% !TEX program = xelatex
\documentclass[conference]{IEEEtran}

% 日本語出力と XeLaTeX 対応設定
\usepackage{xeCJK}
\usepackage{fontspec}
\setCJKmainfont{HaranoAjiMincho}

% 数式・図表・参考文献などの標準パッケージ
\usepackage{amsmath, amssymb}
\usepackage{graphicx}
\usepackage{url}
\usepackage{hyperref}

\hypersetup{
  colorlinks=true,
  linkcolor=blue,
  citecolor=blue,
  urlcolor=blue
}

\begin{document}

\title{軽量 CNN と Vision Transformer の公平比較に向けた実験的評価}

\author{\IEEEauthorblockN{著者 太郎\\ }
\IEEEauthorblockA{所属機関\\ 連絡先: author@example.com}}

\maketitle

\begin{abstract}
本稿では,軽量な畳み込みニューラルネットワーク(CNN)と Vision Transformer(ViT)の画像分類性能を,CIFAR-10 を対象に公平に比較する実験プロトコルを提示する。学習率スケジューリング,データ拡張,Test-Time Augmentation (TTA) を統一的に適用し,再現性を重視した評価を実施した。実験の結果,ViT-Ti は TTA により最大で +1.8pt の精度向上を得た一方,ResNet-18 では RandAugment を中心としたデータ拡張で +2.1pt の改善が確認された。これらの知見を通じて,限られた計算資源でも堅牢なベースラインを構築するための指針を示す。
\end{abstract}

\begin{IEEEkeywords}
画像分類, 深層学習, Vision Transformer, Test-Time Augmentation, CIFAR-10
\end{IEEEkeywords}

\input{01_introduction/main}
\input{02_related_work/main}
\input{03_method/main}
\input{04_results/main}
\input{05_discussion/main}
\input{06_conclusion/main}
\input{appendix/main}

\bibliographystyle{ieeetr}
\bibliography{references}

\end{document}

% !TEX program = xelatex
\documentclass[conference]{IEEEtran}

% 日本語出力と XeLaTeX 対応設定
\usepackage{xeCJK}
\usepackage{fontspec}
\setCJKmainfont{HaranoAjiMincho}

% 数式・図表・参考文献などの標準パッケージ
\usepackage{amsmath, amssymb}
\usepackage{graphicx}
\usepackage{url}
\usepackage{hyperref}

\hypersetup{
  colorlinks=true,
  linkcolor=blue,
  citecolor=blue,
  urlcolor=blue
}

\begin{document}

\title{軽量 CNN と Vision Transformer の公平比較に向けた実験的評価}

\author{\IEEEauthorblockN{著者 太郎\\ }
\IEEEauthorblockA{所属機関\\ 連絡先: author@example.com}}

\maketitle

\begin{abstract}
本稿では,軽量な畳み込みニューラルネットワーク(CNN)と Vision Transformer(ViT)の画像分類性能を,CIFAR-10 を対象に公平に比較する実験プロトコルを提示する。学習率スケジューリング,データ拡張,Test-Time Augmentation (TTA) を統一的に適用し,再現性を重視した評価を実施した。実験の結果,ViT-Ti は TTA により最大で +1.8pt の精度向上を得た一方,ResNet-18 では RandAugment を中心としたデータ拡張で +2.1pt の改善が確認された。これらの知見を通じて,限られた計算資源でも堅牢なベースラインを構築するための指針を示す。
\end{abstract}

\begin{IEEEkeywords}
画像分類, 深層学習, Vision Transformer, Test-Time Augmentation, CIFAR-10
\end{IEEEkeywords}

\input{01_introduction/main}
\input{02_related_work/main}
\input{03_method/main}
\input{04_results/main}
\input{05_discussion/main}
\input{06_conclusion/main}
\input{appendix/main}

\bibliographystyle{ieeetr}
\bibliography{references}

\end{document}

% !TEX program = xelatex
\documentclass[conference]{IEEEtran}

% 日本語出力と XeLaTeX 対応設定
\usepackage{xeCJK}
\usepackage{fontspec}
\setCJKmainfont{HaranoAjiMincho}

% 数式・図表・参考文献などの標準パッケージ
\usepackage{amsmath, amssymb}
\usepackage{graphicx}
\usepackage{url}
\usepackage{hyperref}

\hypersetup{
  colorlinks=true,
  linkcolor=blue,
  citecolor=blue,
  urlcolor=blue
}

\begin{document}

\title{軽量 CNN と Vision Transformer の公平比較に向けた実験的評価}

\author{\IEEEauthorblockN{著者 太郎\\ }
\IEEEauthorblockA{所属機関\\ 連絡先: author@example.com}}

\maketitle

\begin{abstract}
本稿では,軽量な畳み込みニューラルネットワーク(CNN)と Vision Transformer(ViT)の画像分類性能を,CIFAR-10 を対象に公平に比較する実験プロトコルを提示する。学習率スケジューリング,データ拡張,Test-Time Augmentation (TTA) を統一的に適用し,再現性を重視した評価を実施した。実験の結果,ViT-Ti は TTA により最大で +1.8pt の精度向上を得た一方,ResNet-18 では RandAugment を中心としたデータ拡張で +2.1pt の改善が確認された。これらの知見を通じて,限られた計算資源でも堅牢なベースラインを構築するための指針を示す。
\end{abstract}

\begin{IEEEkeywords}
画像分類, 深層学習, Vision Transformer, Test-Time Augmentation, CIFAR-10
\end{IEEEkeywords}

\input{01_introduction/main}
\input{02_related_work/main}
\input{03_method/main}
\input{04_results/main}
\input{05_discussion/main}
\input{06_conclusion/main}
\input{appendix/main}

\bibliographystyle{ieeetr}
\bibliography{references}

\end{document}

% !TEX program = xelatex
\documentclass[conference]{IEEEtran}

% 日本語出力と XeLaTeX 対応設定
\usepackage{xeCJK}
\usepackage{fontspec}
\setCJKmainfont{HaranoAjiMincho}

% 数式・図表・参考文献などの標準パッケージ
\usepackage{amsmath, amssymb}
\usepackage{graphicx}
\usepackage{url}
\usepackage{hyperref}

\hypersetup{
  colorlinks=true,
  linkcolor=blue,
  citecolor=blue,
  urlcolor=blue
}

\begin{document}

\title{軽量 CNN と Vision Transformer の公平比較に向けた実験的評価}

\author{\IEEEauthorblockN{著者 太郎\\ }
\IEEEauthorblockA{所属機関\\ 連絡先: author@example.com}}

\maketitle

\begin{abstract}
本稿では,軽量な畳み込みニューラルネットワーク(CNN)と Vision Transformer(ViT)の画像分類性能を,CIFAR-10 を対象に公平に比較する実験プロトコルを提示する。学習率スケジューリング,データ拡張,Test-Time Augmentation (TTA) を統一的に適用し,再現性を重視した評価を実施した。実験の結果,ViT-Ti は TTA により最大で +1.8pt の精度向上を得た一方,ResNet-18 では RandAugment を中心としたデータ拡張で +2.1pt の改善が確認された。これらの知見を通じて,限られた計算資源でも堅牢なベースラインを構築するための指針を示す。
\end{abstract}

\begin{IEEEkeywords}
画像分類, 深層学習, Vision Transformer, Test-Time Augmentation, CIFAR-10
\end{IEEEkeywords}

\input{01_introduction/main}
\input{02_related_work/main}
\input{03_method/main}
\input{04_results/main}
\input{05_discussion/main}
\input{06_conclusion/main}
\input{appendix/main}

\bibliographystyle{ieeetr}
\bibliography{references}

\end{document}


\bibliographystyle{ieeetr}
\bibliography{references}

\end{document}

% !TEX program = xelatex
\documentclass[conference]{IEEEtran}

% 日本語出力と XeLaTeX 対応設定
\usepackage{xeCJK}
\usepackage{fontspec}
\setCJKmainfont{HaranoAjiMincho}

% 数式・図表・参考文献などの標準パッケージ
\usepackage{amsmath, amssymb}
\usepackage{graphicx}
\usepackage{url}
\usepackage{hyperref}

\hypersetup{
  colorlinks=true,
  linkcolor=blue,
  citecolor=blue,
  urlcolor=blue
}

\begin{document}

\title{軽量 CNN と Vision Transformer の公平比較に向けた実験的評価}

\author{\IEEEauthorblockN{著者 太郎\\ }
\IEEEauthorblockA{所属機関\\ 連絡先: author@example.com}}

\maketitle

\begin{abstract}
本稿では,軽量な畳み込みニューラルネットワーク(CNN)と Vision Transformer(ViT)の画像分類性能を,CIFAR-10 を対象に公平に比較する実験プロトコルを提示する。学習率スケジューリング,データ拡張,Test-Time Augmentation (TTA) を統一的に適用し,再現性を重視した評価を実施した。実験の結果,ViT-Ti は TTA により最大で +1.8pt の精度向上を得た一方,ResNet-18 では RandAugment を中心としたデータ拡張で +2.1pt の改善が確認された。これらの知見を通じて,限られた計算資源でも堅牢なベースラインを構築するための指針を示す。
\end{abstract}

\begin{IEEEkeywords}
画像分類, 深層学習, Vision Transformer, Test-Time Augmentation, CIFAR-10
\end{IEEEkeywords}

% !TEX program = xelatex
\documentclass[conference]{IEEEtran}

% 日本語出力と XeLaTeX 対応設定
\usepackage{xeCJK}
\usepackage{fontspec}
\setCJKmainfont{HaranoAjiMincho}

% 数式・図表・参考文献などの標準パッケージ
\usepackage{amsmath, amssymb}
\usepackage{graphicx}
\usepackage{url}
\usepackage{hyperref}

\hypersetup{
  colorlinks=true,
  linkcolor=blue,
  citecolor=blue,
  urlcolor=blue
}

\begin{document}

\title{軽量 CNN と Vision Transformer の公平比較に向けた実験的評価}

\author{\IEEEauthorblockN{著者 太郎\\ }
\IEEEauthorblockA{所属機関\\ 連絡先: author@example.com}}

\maketitle

\begin{abstract}
本稿では,軽量な畳み込みニューラルネットワーク(CNN)と Vision Transformer(ViT)の画像分類性能を,CIFAR-10 を対象に公平に比較する実験プロトコルを提示する。学習率スケジューリング,データ拡張,Test-Time Augmentation (TTA) を統一的に適用し,再現性を重視した評価を実施した。実験の結果,ViT-Ti は TTA により最大で +1.8pt の精度向上を得た一方,ResNet-18 では RandAugment を中心としたデータ拡張で +2.1pt の改善が確認された。これらの知見を通じて,限られた計算資源でも堅牢なベースラインを構築するための指針を示す。
\end{abstract}

\begin{IEEEkeywords}
画像分類, 深層学習, Vision Transformer, Test-Time Augmentation, CIFAR-10
\end{IEEEkeywords}

\input{01_introduction/main}
\input{02_related_work/main}
\input{03_method/main}
\input{04_results/main}
\input{05_discussion/main}
\input{06_conclusion/main}
\input{appendix/main}

\bibliographystyle{ieeetr}
\bibliography{references}

\end{document}

% !TEX program = xelatex
\documentclass[conference]{IEEEtran}

% 日本語出力と XeLaTeX 対応設定
\usepackage{xeCJK}
\usepackage{fontspec}
\setCJKmainfont{HaranoAjiMincho}

% 数式・図表・参考文献などの標準パッケージ
\usepackage{amsmath, amssymb}
\usepackage{graphicx}
\usepackage{url}
\usepackage{hyperref}

\hypersetup{
  colorlinks=true,
  linkcolor=blue,
  citecolor=blue,
  urlcolor=blue
}

\begin{document}

\title{軽量 CNN と Vision Transformer の公平比較に向けた実験的評価}

\author{\IEEEauthorblockN{著者 太郎\\ }
\IEEEauthorblockA{所属機関\\ 連絡先: author@example.com}}

\maketitle

\begin{abstract}
本稿では,軽量な畳み込みニューラルネットワーク(CNN)と Vision Transformer(ViT)の画像分類性能を,CIFAR-10 を対象に公平に比較する実験プロトコルを提示する。学習率スケジューリング,データ拡張,Test-Time Augmentation (TTA) を統一的に適用し,再現性を重視した評価を実施した。実験の結果,ViT-Ti は TTA により最大で +1.8pt の精度向上を得た一方,ResNet-18 では RandAugment を中心としたデータ拡張で +2.1pt の改善が確認された。これらの知見を通じて,限られた計算資源でも堅牢なベースラインを構築するための指針を示す。
\end{abstract}

\begin{IEEEkeywords}
画像分類, 深層学習, Vision Transformer, Test-Time Augmentation, CIFAR-10
\end{IEEEkeywords}

\input{01_introduction/main}
\input{02_related_work/main}
\input{03_method/main}
\input{04_results/main}
\input{05_discussion/main}
\input{06_conclusion/main}
\input{appendix/main}

\bibliographystyle{ieeetr}
\bibliography{references}

\end{document}

% !TEX program = xelatex
\documentclass[conference]{IEEEtran}

% 日本語出力と XeLaTeX 対応設定
\usepackage{xeCJK}
\usepackage{fontspec}
\setCJKmainfont{HaranoAjiMincho}

% 数式・図表・参考文献などの標準パッケージ
\usepackage{amsmath, amssymb}
\usepackage{graphicx}
\usepackage{url}
\usepackage{hyperref}

\hypersetup{
  colorlinks=true,
  linkcolor=blue,
  citecolor=blue,
  urlcolor=blue
}

\begin{document}

\title{軽量 CNN と Vision Transformer の公平比較に向けた実験的評価}

\author{\IEEEauthorblockN{著者 太郎\\ }
\IEEEauthorblockA{所属機関\\ 連絡先: author@example.com}}

\maketitle

\begin{abstract}
本稿では,軽量な畳み込みニューラルネットワーク(CNN)と Vision Transformer(ViT)の画像分類性能を,CIFAR-10 を対象に公平に比較する実験プロトコルを提示する。学習率スケジューリング,データ拡張,Test-Time Augmentation (TTA) を統一的に適用し,再現性を重視した評価を実施した。実験の結果,ViT-Ti は TTA により最大で +1.8pt の精度向上を得た一方,ResNet-18 では RandAugment を中心としたデータ拡張で +2.1pt の改善が確認された。これらの知見を通じて,限られた計算資源でも堅牢なベースラインを構築するための指針を示す。
\end{abstract}

\begin{IEEEkeywords}
画像分類, 深層学習, Vision Transformer, Test-Time Augmentation, CIFAR-10
\end{IEEEkeywords}

\input{01_introduction/main}
\input{02_related_work/main}
\input{03_method/main}
\input{04_results/main}
\input{05_discussion/main}
\input{06_conclusion/main}
\input{appendix/main}

\bibliographystyle{ieeetr}
\bibliography{references}

\end{document}

% !TEX program = xelatex
\documentclass[conference]{IEEEtran}

% 日本語出力と XeLaTeX 対応設定
\usepackage{xeCJK}
\usepackage{fontspec}
\setCJKmainfont{HaranoAjiMincho}

% 数式・図表・参考文献などの標準パッケージ
\usepackage{amsmath, amssymb}
\usepackage{graphicx}
\usepackage{url}
\usepackage{hyperref}

\hypersetup{
  colorlinks=true,
  linkcolor=blue,
  citecolor=blue,
  urlcolor=blue
}

\begin{document}

\title{軽量 CNN と Vision Transformer の公平比較に向けた実験的評価}

\author{\IEEEauthorblockN{著者 太郎\\ }
\IEEEauthorblockA{所属機関\\ 連絡先: author@example.com}}

\maketitle

\begin{abstract}
本稿では,軽量な畳み込みニューラルネットワーク(CNN)と Vision Transformer(ViT)の画像分類性能を,CIFAR-10 を対象に公平に比較する実験プロトコルを提示する。学習率スケジューリング,データ拡張,Test-Time Augmentation (TTA) を統一的に適用し,再現性を重視した評価を実施した。実験の結果,ViT-Ti は TTA により最大で +1.8pt の精度向上を得た一方,ResNet-18 では RandAugment を中心としたデータ拡張で +2.1pt の改善が確認された。これらの知見を通じて,限られた計算資源でも堅牢なベースラインを構築するための指針を示す。
\end{abstract}

\begin{IEEEkeywords}
画像分類, 深層学習, Vision Transformer, Test-Time Augmentation, CIFAR-10
\end{IEEEkeywords}

\input{01_introduction/main}
\input{02_related_work/main}
\input{03_method/main}
\input{04_results/main}
\input{05_discussion/main}
\input{06_conclusion/main}
\input{appendix/main}

\bibliographystyle{ieeetr}
\bibliography{references}

\end{document}

% !TEX program = xelatex
\documentclass[conference]{IEEEtran}

% 日本語出力と XeLaTeX 対応設定
\usepackage{xeCJK}
\usepackage{fontspec}
\setCJKmainfont{HaranoAjiMincho}

% 数式・図表・参考文献などの標準パッケージ
\usepackage{amsmath, amssymb}
\usepackage{graphicx}
\usepackage{url}
\usepackage{hyperref}

\hypersetup{
  colorlinks=true,
  linkcolor=blue,
  citecolor=blue,
  urlcolor=blue
}

\begin{document}

\title{軽量 CNN と Vision Transformer の公平比較に向けた実験的評価}

\author{\IEEEauthorblockN{著者 太郎\\ }
\IEEEauthorblockA{所属機関\\ 連絡先: author@example.com}}

\maketitle

\begin{abstract}
本稿では,軽量な畳み込みニューラルネットワーク(CNN)と Vision Transformer(ViT)の画像分類性能を,CIFAR-10 を対象に公平に比較する実験プロトコルを提示する。学習率スケジューリング,データ拡張,Test-Time Augmentation (TTA) を統一的に適用し,再現性を重視した評価を実施した。実験の結果,ViT-Ti は TTA により最大で +1.8pt の精度向上を得た一方,ResNet-18 では RandAugment を中心としたデータ拡張で +2.1pt の改善が確認された。これらの知見を通じて,限られた計算資源でも堅牢なベースラインを構築するための指針を示す。
\end{abstract}

\begin{IEEEkeywords}
画像分類, 深層学習, Vision Transformer, Test-Time Augmentation, CIFAR-10
\end{IEEEkeywords}

\input{01_introduction/main}
\input{02_related_work/main}
\input{03_method/main}
\input{04_results/main}
\input{05_discussion/main}
\input{06_conclusion/main}
\input{appendix/main}

\bibliographystyle{ieeetr}
\bibliography{references}

\end{document}

% !TEX program = xelatex
\documentclass[conference]{IEEEtran}

% 日本語出力と XeLaTeX 対応設定
\usepackage{xeCJK}
\usepackage{fontspec}
\setCJKmainfont{HaranoAjiMincho}

% 数式・図表・参考文献などの標準パッケージ
\usepackage{amsmath, amssymb}
\usepackage{graphicx}
\usepackage{url}
\usepackage{hyperref}

\hypersetup{
  colorlinks=true,
  linkcolor=blue,
  citecolor=blue,
  urlcolor=blue
}

\begin{document}

\title{軽量 CNN と Vision Transformer の公平比較に向けた実験的評価}

\author{\IEEEauthorblockN{著者 太郎\\ }
\IEEEauthorblockA{所属機関\\ 連絡先: author@example.com}}

\maketitle

\begin{abstract}
本稿では,軽量な畳み込みニューラルネットワーク(CNN)と Vision Transformer(ViT)の画像分類性能を,CIFAR-10 を対象に公平に比較する実験プロトコルを提示する。学習率スケジューリング,データ拡張,Test-Time Augmentation (TTA) を統一的に適用し,再現性を重視した評価を実施した。実験の結果,ViT-Ti は TTA により最大で +1.8pt の精度向上を得た一方,ResNet-18 では RandAugment を中心としたデータ拡張で +2.1pt の改善が確認された。これらの知見を通じて,限られた計算資源でも堅牢なベースラインを構築するための指針を示す。
\end{abstract}

\begin{IEEEkeywords}
画像分類, 深層学習, Vision Transformer, Test-Time Augmentation, CIFAR-10
\end{IEEEkeywords}

\input{01_introduction/main}
\input{02_related_work/main}
\input{03_method/main}
\input{04_results/main}
\input{05_discussion/main}
\input{06_conclusion/main}
\input{appendix/main}

\bibliographystyle{ieeetr}
\bibliography{references}

\end{document}

% !TEX program = xelatex
\documentclass[conference]{IEEEtran}

% 日本語出力と XeLaTeX 対応設定
\usepackage{xeCJK}
\usepackage{fontspec}
\setCJKmainfont{HaranoAjiMincho}

% 数式・図表・参考文献などの標準パッケージ
\usepackage{amsmath, amssymb}
\usepackage{graphicx}
\usepackage{url}
\usepackage{hyperref}

\hypersetup{
  colorlinks=true,
  linkcolor=blue,
  citecolor=blue,
  urlcolor=blue
}

\begin{document}

\title{軽量 CNN と Vision Transformer の公平比較に向けた実験的評価}

\author{\IEEEauthorblockN{著者 太郎\\ }
\IEEEauthorblockA{所属機関\\ 連絡先: author@example.com}}

\maketitle

\begin{abstract}
本稿では,軽量な畳み込みニューラルネットワーク(CNN)と Vision Transformer(ViT)の画像分類性能を,CIFAR-10 を対象に公平に比較する実験プロトコルを提示する。学習率スケジューリング,データ拡張,Test-Time Augmentation (TTA) を統一的に適用し,再現性を重視した評価を実施した。実験の結果,ViT-Ti は TTA により最大で +1.8pt の精度向上を得た一方,ResNet-18 では RandAugment を中心としたデータ拡張で +2.1pt の改善が確認された。これらの知見を通じて,限られた計算資源でも堅牢なベースラインを構築するための指針を示す。
\end{abstract}

\begin{IEEEkeywords}
画像分類, 深層学習, Vision Transformer, Test-Time Augmentation, CIFAR-10
\end{IEEEkeywords}

\input{01_introduction/main}
\input{02_related_work/main}
\input{03_method/main}
\input{04_results/main}
\input{05_discussion/main}
\input{06_conclusion/main}
\input{appendix/main}

\bibliographystyle{ieeetr}
\bibliography{references}

\end{document}


\bibliographystyle{ieeetr}
\bibliography{references}

\end{document}

% !TEX program = xelatex
\documentclass[conference]{IEEEtran}

% 日本語出力と XeLaTeX 対応設定
\usepackage{xeCJK}
\usepackage{fontspec}
\setCJKmainfont{HaranoAjiMincho}

% 数式・図表・参考文献などの標準パッケージ
\usepackage{amsmath, amssymb}
\usepackage{graphicx}
\usepackage{url}
\usepackage{hyperref}

\hypersetup{
  colorlinks=true,
  linkcolor=blue,
  citecolor=blue,
  urlcolor=blue
}

\begin{document}

\title{軽量 CNN と Vision Transformer の公平比較に向けた実験的評価}

\author{\IEEEauthorblockN{著者 太郎\\ }
\IEEEauthorblockA{所属機関\\ 連絡先: author@example.com}}

\maketitle

\begin{abstract}
本稿では,軽量な畳み込みニューラルネットワーク(CNN)と Vision Transformer(ViT)の画像分類性能を,CIFAR-10 を対象に公平に比較する実験プロトコルを提示する。学習率スケジューリング,データ拡張,Test-Time Augmentation (TTA) を統一的に適用し,再現性を重視した評価を実施した。実験の結果,ViT-Ti は TTA により最大で +1.8pt の精度向上を得た一方,ResNet-18 では RandAugment を中心としたデータ拡張で +2.1pt の改善が確認された。これらの知見を通じて,限られた計算資源でも堅牢なベースラインを構築するための指針を示す。
\end{abstract}

\begin{IEEEkeywords}
画像分類, 深層学習, Vision Transformer, Test-Time Augmentation, CIFAR-10
\end{IEEEkeywords}

% !TEX program = xelatex
\documentclass[conference]{IEEEtran}

% 日本語出力と XeLaTeX 対応設定
\usepackage{xeCJK}
\usepackage{fontspec}
\setCJKmainfont{HaranoAjiMincho}

% 数式・図表・参考文献などの標準パッケージ
\usepackage{amsmath, amssymb}
\usepackage{graphicx}
\usepackage{url}
\usepackage{hyperref}

\hypersetup{
  colorlinks=true,
  linkcolor=blue,
  citecolor=blue,
  urlcolor=blue
}

\begin{document}

\title{軽量 CNN と Vision Transformer の公平比較に向けた実験的評価}

\author{\IEEEauthorblockN{著者 太郎\\ }
\IEEEauthorblockA{所属機関\\ 連絡先: author@example.com}}

\maketitle

\begin{abstract}
本稿では,軽量な畳み込みニューラルネットワーク(CNN)と Vision Transformer(ViT)の画像分類性能を,CIFAR-10 を対象に公平に比較する実験プロトコルを提示する。学習率スケジューリング,データ拡張,Test-Time Augmentation (TTA) を統一的に適用し,再現性を重視した評価を実施した。実験の結果,ViT-Ti は TTA により最大で +1.8pt の精度向上を得た一方,ResNet-18 では RandAugment を中心としたデータ拡張で +2.1pt の改善が確認された。これらの知見を通じて,限られた計算資源でも堅牢なベースラインを構築するための指針を示す。
\end{abstract}

\begin{IEEEkeywords}
画像分類, 深層学習, Vision Transformer, Test-Time Augmentation, CIFAR-10
\end{IEEEkeywords}

\input{01_introduction/main}
\input{02_related_work/main}
\input{03_method/main}
\input{04_results/main}
\input{05_discussion/main}
\input{06_conclusion/main}
\input{appendix/main}

\bibliographystyle{ieeetr}
\bibliography{references}

\end{document}

% !TEX program = xelatex
\documentclass[conference]{IEEEtran}

% 日本語出力と XeLaTeX 対応設定
\usepackage{xeCJK}
\usepackage{fontspec}
\setCJKmainfont{HaranoAjiMincho}

% 数式・図表・参考文献などの標準パッケージ
\usepackage{amsmath, amssymb}
\usepackage{graphicx}
\usepackage{url}
\usepackage{hyperref}

\hypersetup{
  colorlinks=true,
  linkcolor=blue,
  citecolor=blue,
  urlcolor=blue
}

\begin{document}

\title{軽量 CNN と Vision Transformer の公平比較に向けた実験的評価}

\author{\IEEEauthorblockN{著者 太郎\\ }
\IEEEauthorblockA{所属機関\\ 連絡先: author@example.com}}

\maketitle

\begin{abstract}
本稿では,軽量な畳み込みニューラルネットワーク(CNN)と Vision Transformer(ViT)の画像分類性能を,CIFAR-10 を対象に公平に比較する実験プロトコルを提示する。学習率スケジューリング,データ拡張,Test-Time Augmentation (TTA) を統一的に適用し,再現性を重視した評価を実施した。実験の結果,ViT-Ti は TTA により最大で +1.8pt の精度向上を得た一方,ResNet-18 では RandAugment を中心としたデータ拡張で +2.1pt の改善が確認された。これらの知見を通じて,限られた計算資源でも堅牢なベースラインを構築するための指針を示す。
\end{abstract}

\begin{IEEEkeywords}
画像分類, 深層学習, Vision Transformer, Test-Time Augmentation, CIFAR-10
\end{IEEEkeywords}

\input{01_introduction/main}
\input{02_related_work/main}
\input{03_method/main}
\input{04_results/main}
\input{05_discussion/main}
\input{06_conclusion/main}
\input{appendix/main}

\bibliographystyle{ieeetr}
\bibliography{references}

\end{document}

% !TEX program = xelatex
\documentclass[conference]{IEEEtran}

% 日本語出力と XeLaTeX 対応設定
\usepackage{xeCJK}
\usepackage{fontspec}
\setCJKmainfont{HaranoAjiMincho}

% 数式・図表・参考文献などの標準パッケージ
\usepackage{amsmath, amssymb}
\usepackage{graphicx}
\usepackage{url}
\usepackage{hyperref}

\hypersetup{
  colorlinks=true,
  linkcolor=blue,
  citecolor=blue,
  urlcolor=blue
}

\begin{document}

\title{軽量 CNN と Vision Transformer の公平比較に向けた実験的評価}

\author{\IEEEauthorblockN{著者 太郎\\ }
\IEEEauthorblockA{所属機関\\ 連絡先: author@example.com}}

\maketitle

\begin{abstract}
本稿では,軽量な畳み込みニューラルネットワーク(CNN)と Vision Transformer(ViT)の画像分類性能を,CIFAR-10 を対象に公平に比較する実験プロトコルを提示する。学習率スケジューリング,データ拡張,Test-Time Augmentation (TTA) を統一的に適用し,再現性を重視した評価を実施した。実験の結果,ViT-Ti は TTA により最大で +1.8pt の精度向上を得た一方,ResNet-18 では RandAugment を中心としたデータ拡張で +2.1pt の改善が確認された。これらの知見を通じて,限られた計算資源でも堅牢なベースラインを構築するための指針を示す。
\end{abstract}

\begin{IEEEkeywords}
画像分類, 深層学習, Vision Transformer, Test-Time Augmentation, CIFAR-10
\end{IEEEkeywords}

\input{01_introduction/main}
\input{02_related_work/main}
\input{03_method/main}
\input{04_results/main}
\input{05_discussion/main}
\input{06_conclusion/main}
\input{appendix/main}

\bibliographystyle{ieeetr}
\bibliography{references}

\end{document}

% !TEX program = xelatex
\documentclass[conference]{IEEEtran}

% 日本語出力と XeLaTeX 対応設定
\usepackage{xeCJK}
\usepackage{fontspec}
\setCJKmainfont{HaranoAjiMincho}

% 数式・図表・参考文献などの標準パッケージ
\usepackage{amsmath, amssymb}
\usepackage{graphicx}
\usepackage{url}
\usepackage{hyperref}

\hypersetup{
  colorlinks=true,
  linkcolor=blue,
  citecolor=blue,
  urlcolor=blue
}

\begin{document}

\title{軽量 CNN と Vision Transformer の公平比較に向けた実験的評価}

\author{\IEEEauthorblockN{著者 太郎\\ }
\IEEEauthorblockA{所属機関\\ 連絡先: author@example.com}}

\maketitle

\begin{abstract}
本稿では,軽量な畳み込みニューラルネットワーク(CNN)と Vision Transformer(ViT)の画像分類性能を,CIFAR-10 を対象に公平に比較する実験プロトコルを提示する。学習率スケジューリング,データ拡張,Test-Time Augmentation (TTA) を統一的に適用し,再現性を重視した評価を実施した。実験の結果,ViT-Ti は TTA により最大で +1.8pt の精度向上を得た一方,ResNet-18 では RandAugment を中心としたデータ拡張で +2.1pt の改善が確認された。これらの知見を通じて,限られた計算資源でも堅牢なベースラインを構築するための指針を示す。
\end{abstract}

\begin{IEEEkeywords}
画像分類, 深層学習, Vision Transformer, Test-Time Augmentation, CIFAR-10
\end{IEEEkeywords}

\input{01_introduction/main}
\input{02_related_work/main}
\input{03_method/main}
\input{04_results/main}
\input{05_discussion/main}
\input{06_conclusion/main}
\input{appendix/main}

\bibliographystyle{ieeetr}
\bibliography{references}

\end{document}

% !TEX program = xelatex
\documentclass[conference]{IEEEtran}

% 日本語出力と XeLaTeX 対応設定
\usepackage{xeCJK}
\usepackage{fontspec}
\setCJKmainfont{HaranoAjiMincho}

% 数式・図表・参考文献などの標準パッケージ
\usepackage{amsmath, amssymb}
\usepackage{graphicx}
\usepackage{url}
\usepackage{hyperref}

\hypersetup{
  colorlinks=true,
  linkcolor=blue,
  citecolor=blue,
  urlcolor=blue
}

\begin{document}

\title{軽量 CNN と Vision Transformer の公平比較に向けた実験的評価}

\author{\IEEEauthorblockN{著者 太郎\\ }
\IEEEauthorblockA{所属機関\\ 連絡先: author@example.com}}

\maketitle

\begin{abstract}
本稿では,軽量な畳み込みニューラルネットワーク(CNN)と Vision Transformer(ViT)の画像分類性能を,CIFAR-10 を対象に公平に比較する実験プロトコルを提示する。学習率スケジューリング,データ拡張,Test-Time Augmentation (TTA) を統一的に適用し,再現性を重視した評価を実施した。実験の結果,ViT-Ti は TTA により最大で +1.8pt の精度向上を得た一方,ResNet-18 では RandAugment を中心としたデータ拡張で +2.1pt の改善が確認された。これらの知見を通じて,限られた計算資源でも堅牢なベースラインを構築するための指針を示す。
\end{abstract}

\begin{IEEEkeywords}
画像分類, 深層学習, Vision Transformer, Test-Time Augmentation, CIFAR-10
\end{IEEEkeywords}

\input{01_introduction/main}
\input{02_related_work/main}
\input{03_method/main}
\input{04_results/main}
\input{05_discussion/main}
\input{06_conclusion/main}
\input{appendix/main}

\bibliographystyle{ieeetr}
\bibliography{references}

\end{document}

% !TEX program = xelatex
\documentclass[conference]{IEEEtran}

% 日本語出力と XeLaTeX 対応設定
\usepackage{xeCJK}
\usepackage{fontspec}
\setCJKmainfont{HaranoAjiMincho}

% 数式・図表・参考文献などの標準パッケージ
\usepackage{amsmath, amssymb}
\usepackage{graphicx}
\usepackage{url}
\usepackage{hyperref}

\hypersetup{
  colorlinks=true,
  linkcolor=blue,
  citecolor=blue,
  urlcolor=blue
}

\begin{document}

\title{軽量 CNN と Vision Transformer の公平比較に向けた実験的評価}

\author{\IEEEauthorblockN{著者 太郎\\ }
\IEEEauthorblockA{所属機関\\ 連絡先: author@example.com}}

\maketitle

\begin{abstract}
本稿では,軽量な畳み込みニューラルネットワーク(CNN)と Vision Transformer(ViT)の画像分類性能を,CIFAR-10 を対象に公平に比較する実験プロトコルを提示する。学習率スケジューリング,データ拡張,Test-Time Augmentation (TTA) を統一的に適用し,再現性を重視した評価を実施した。実験の結果,ViT-Ti は TTA により最大で +1.8pt の精度向上を得た一方,ResNet-18 では RandAugment を中心としたデータ拡張で +2.1pt の改善が確認された。これらの知見を通じて,限られた計算資源でも堅牢なベースラインを構築するための指針を示す。
\end{abstract}

\begin{IEEEkeywords}
画像分類, 深層学習, Vision Transformer, Test-Time Augmentation, CIFAR-10
\end{IEEEkeywords}

\input{01_introduction/main}
\input{02_related_work/main}
\input{03_method/main}
\input{04_results/main}
\input{05_discussion/main}
\input{06_conclusion/main}
\input{appendix/main}

\bibliographystyle{ieeetr}
\bibliography{references}

\end{document}

% !TEX program = xelatex
\documentclass[conference]{IEEEtran}

% 日本語出力と XeLaTeX 対応設定
\usepackage{xeCJK}
\usepackage{fontspec}
\setCJKmainfont{HaranoAjiMincho}

% 数式・図表・参考文献などの標準パッケージ
\usepackage{amsmath, amssymb}
\usepackage{graphicx}
\usepackage{url}
\usepackage{hyperref}

\hypersetup{
  colorlinks=true,
  linkcolor=blue,
  citecolor=blue,
  urlcolor=blue
}

\begin{document}

\title{軽量 CNN と Vision Transformer の公平比較に向けた実験的評価}

\author{\IEEEauthorblockN{著者 太郎\\ }
\IEEEauthorblockA{所属機関\\ 連絡先: author@example.com}}

\maketitle

\begin{abstract}
本稿では,軽量な畳み込みニューラルネットワーク(CNN)と Vision Transformer(ViT)の画像分類性能を,CIFAR-10 を対象に公平に比較する実験プロトコルを提示する。学習率スケジューリング,データ拡張,Test-Time Augmentation (TTA) を統一的に適用し,再現性を重視した評価を実施した。実験の結果,ViT-Ti は TTA により最大で +1.8pt の精度向上を得た一方,ResNet-18 では RandAugment を中心としたデータ拡張で +2.1pt の改善が確認された。これらの知見を通じて,限られた計算資源でも堅牢なベースラインを構築するための指針を示す。
\end{abstract}

\begin{IEEEkeywords}
画像分類, 深層学習, Vision Transformer, Test-Time Augmentation, CIFAR-10
\end{IEEEkeywords}

\input{01_introduction/main}
\input{02_related_work/main}
\input{03_method/main}
\input{04_results/main}
\input{05_discussion/main}
\input{06_conclusion/main}
\input{appendix/main}

\bibliographystyle{ieeetr}
\bibliography{references}

\end{document}


\bibliographystyle{ieeetr}
\bibliography{references}

\end{document}


\bibliographystyle{ieeetr}
\bibliography{references}

\end{document}

% !TEX program = xelatex
\documentclass[conference]{IEEEtran}

% 日本語出力と XeLaTeX 対応設定
\usepackage{xeCJK}
\usepackage{fontspec}
\setCJKmainfont{HaranoAjiMincho}

% 数式・図表・参考文献などの標準パッケージ
\usepackage{amsmath, amssymb}
\usepackage{graphicx}
\usepackage{url}
\usepackage{hyperref}

\hypersetup{
  colorlinks=true,
  linkcolor=blue,
  citecolor=blue,
  urlcolor=blue
}

\begin{document}

\title{軽量 CNN と Vision Transformer の公平比較に向けた実験的評価}

\author{\IEEEauthorblockN{著者 太郎\\ }
\IEEEauthorblockA{所属機関\\ 連絡先: author@example.com}}

\maketitle

\begin{abstract}
本稿では,軽量な畳み込みニューラルネットワーク(CNN)と Vision Transformer(ViT)の画像分類性能を,CIFAR-10 を対象に公平に比較する実験プロトコルを提示する。学習率スケジューリング,データ拡張,Test-Time Augmentation (TTA) を統一的に適用し,再現性を重視した評価を実施した。実験の結果,ViT-Ti は TTA により最大で +1.8pt の精度向上を得た一方,ResNet-18 では RandAugment を中心としたデータ拡張で +2.1pt の改善が確認された。これらの知見を通じて,限られた計算資源でも堅牢なベースラインを構築するための指針を示す。
\end{abstract}

\begin{IEEEkeywords}
画像分類, 深層学習, Vision Transformer, Test-Time Augmentation, CIFAR-10
\end{IEEEkeywords}

% !TEX program = xelatex
\documentclass[conference]{IEEEtran}

% 日本語出力と XeLaTeX 対応設定
\usepackage{xeCJK}
\usepackage{fontspec}
\setCJKmainfont{HaranoAjiMincho}

% 数式・図表・参考文献などの標準パッケージ
\usepackage{amsmath, amssymb}
\usepackage{graphicx}
\usepackage{url}
\usepackage{hyperref}

\hypersetup{
  colorlinks=true,
  linkcolor=blue,
  citecolor=blue,
  urlcolor=blue
}

\begin{document}

\title{軽量 CNN と Vision Transformer の公平比較に向けた実験的評価}

\author{\IEEEauthorblockN{著者 太郎\\ }
\IEEEauthorblockA{所属機関\\ 連絡先: author@example.com}}

\maketitle

\begin{abstract}
本稿では,軽量な畳み込みニューラルネットワーク(CNN)と Vision Transformer(ViT)の画像分類性能を,CIFAR-10 を対象に公平に比較する実験プロトコルを提示する。学習率スケジューリング,データ拡張,Test-Time Augmentation (TTA) を統一的に適用し,再現性を重視した評価を実施した。実験の結果,ViT-Ti は TTA により最大で +1.8pt の精度向上を得た一方,ResNet-18 では RandAugment を中心としたデータ拡張で +2.1pt の改善が確認された。これらの知見を通じて,限られた計算資源でも堅牢なベースラインを構築するための指針を示す。
\end{abstract}

\begin{IEEEkeywords}
画像分類, 深層学習, Vision Transformer, Test-Time Augmentation, CIFAR-10
\end{IEEEkeywords}

% !TEX program = xelatex
\documentclass[conference]{IEEEtran}

% 日本語出力と XeLaTeX 対応設定
\usepackage{xeCJK}
\usepackage{fontspec}
\setCJKmainfont{HaranoAjiMincho}

% 数式・図表・参考文献などの標準パッケージ
\usepackage{amsmath, amssymb}
\usepackage{graphicx}
\usepackage{url}
\usepackage{hyperref}

\hypersetup{
  colorlinks=true,
  linkcolor=blue,
  citecolor=blue,
  urlcolor=blue
}

\begin{document}

\title{軽量 CNN と Vision Transformer の公平比較に向けた実験的評価}

\author{\IEEEauthorblockN{著者 太郎\\ }
\IEEEauthorblockA{所属機関\\ 連絡先: author@example.com}}

\maketitle

\begin{abstract}
本稿では,軽量な畳み込みニューラルネットワーク(CNN)と Vision Transformer(ViT)の画像分類性能を,CIFAR-10 を対象に公平に比較する実験プロトコルを提示する。学習率スケジューリング,データ拡張,Test-Time Augmentation (TTA) を統一的に適用し,再現性を重視した評価を実施した。実験の結果,ViT-Ti は TTA により最大で +1.8pt の精度向上を得た一方,ResNet-18 では RandAugment を中心としたデータ拡張で +2.1pt の改善が確認された。これらの知見を通じて,限られた計算資源でも堅牢なベースラインを構築するための指針を示す。
\end{abstract}

\begin{IEEEkeywords}
画像分類, 深層学習, Vision Transformer, Test-Time Augmentation, CIFAR-10
\end{IEEEkeywords}

\input{01_introduction/main}
\input{02_related_work/main}
\input{03_method/main}
\input{04_results/main}
\input{05_discussion/main}
\input{06_conclusion/main}
\input{appendix/main}

\bibliographystyle{ieeetr}
\bibliography{references}

\end{document}

% !TEX program = xelatex
\documentclass[conference]{IEEEtran}

% 日本語出力と XeLaTeX 対応設定
\usepackage{xeCJK}
\usepackage{fontspec}
\setCJKmainfont{HaranoAjiMincho}

% 数式・図表・参考文献などの標準パッケージ
\usepackage{amsmath, amssymb}
\usepackage{graphicx}
\usepackage{url}
\usepackage{hyperref}

\hypersetup{
  colorlinks=true,
  linkcolor=blue,
  citecolor=blue,
  urlcolor=blue
}

\begin{document}

\title{軽量 CNN と Vision Transformer の公平比較に向けた実験的評価}

\author{\IEEEauthorblockN{著者 太郎\\ }
\IEEEauthorblockA{所属機関\\ 連絡先: author@example.com}}

\maketitle

\begin{abstract}
本稿では,軽量な畳み込みニューラルネットワーク(CNN)と Vision Transformer(ViT)の画像分類性能を,CIFAR-10 を対象に公平に比較する実験プロトコルを提示する。学習率スケジューリング,データ拡張,Test-Time Augmentation (TTA) を統一的に適用し,再現性を重視した評価を実施した。実験の結果,ViT-Ti は TTA により最大で +1.8pt の精度向上を得た一方,ResNet-18 では RandAugment を中心としたデータ拡張で +2.1pt の改善が確認された。これらの知見を通じて,限られた計算資源でも堅牢なベースラインを構築するための指針を示す。
\end{abstract}

\begin{IEEEkeywords}
画像分類, 深層学習, Vision Transformer, Test-Time Augmentation, CIFAR-10
\end{IEEEkeywords}

\input{01_introduction/main}
\input{02_related_work/main}
\input{03_method/main}
\input{04_results/main}
\input{05_discussion/main}
\input{06_conclusion/main}
\input{appendix/main}

\bibliographystyle{ieeetr}
\bibliography{references}

\end{document}

% !TEX program = xelatex
\documentclass[conference]{IEEEtran}

% 日本語出力と XeLaTeX 対応設定
\usepackage{xeCJK}
\usepackage{fontspec}
\setCJKmainfont{HaranoAjiMincho}

% 数式・図表・参考文献などの標準パッケージ
\usepackage{amsmath, amssymb}
\usepackage{graphicx}
\usepackage{url}
\usepackage{hyperref}

\hypersetup{
  colorlinks=true,
  linkcolor=blue,
  citecolor=blue,
  urlcolor=blue
}

\begin{document}

\title{軽量 CNN と Vision Transformer の公平比較に向けた実験的評価}

\author{\IEEEauthorblockN{著者 太郎\\ }
\IEEEauthorblockA{所属機関\\ 連絡先: author@example.com}}

\maketitle

\begin{abstract}
本稿では,軽量な畳み込みニューラルネットワーク(CNN)と Vision Transformer(ViT)の画像分類性能を,CIFAR-10 を対象に公平に比較する実験プロトコルを提示する。学習率スケジューリング,データ拡張,Test-Time Augmentation (TTA) を統一的に適用し,再現性を重視した評価を実施した。実験の結果,ViT-Ti は TTA により最大で +1.8pt の精度向上を得た一方,ResNet-18 では RandAugment を中心としたデータ拡張で +2.1pt の改善が確認された。これらの知見を通じて,限られた計算資源でも堅牢なベースラインを構築するための指針を示す。
\end{abstract}

\begin{IEEEkeywords}
画像分類, 深層学習, Vision Transformer, Test-Time Augmentation, CIFAR-10
\end{IEEEkeywords}

\input{01_introduction/main}
\input{02_related_work/main}
\input{03_method/main}
\input{04_results/main}
\input{05_discussion/main}
\input{06_conclusion/main}
\input{appendix/main}

\bibliographystyle{ieeetr}
\bibliography{references}

\end{document}

% !TEX program = xelatex
\documentclass[conference]{IEEEtran}

% 日本語出力と XeLaTeX 対応設定
\usepackage{xeCJK}
\usepackage{fontspec}
\setCJKmainfont{HaranoAjiMincho}

% 数式・図表・参考文献などの標準パッケージ
\usepackage{amsmath, amssymb}
\usepackage{graphicx}
\usepackage{url}
\usepackage{hyperref}

\hypersetup{
  colorlinks=true,
  linkcolor=blue,
  citecolor=blue,
  urlcolor=blue
}

\begin{document}

\title{軽量 CNN と Vision Transformer の公平比較に向けた実験的評価}

\author{\IEEEauthorblockN{著者 太郎\\ }
\IEEEauthorblockA{所属機関\\ 連絡先: author@example.com}}

\maketitle

\begin{abstract}
本稿では,軽量な畳み込みニューラルネットワーク(CNN)と Vision Transformer(ViT)の画像分類性能を,CIFAR-10 を対象に公平に比較する実験プロトコルを提示する。学習率スケジューリング,データ拡張,Test-Time Augmentation (TTA) を統一的に適用し,再現性を重視した評価を実施した。実験の結果,ViT-Ti は TTA により最大で +1.8pt の精度向上を得た一方,ResNet-18 では RandAugment を中心としたデータ拡張で +2.1pt の改善が確認された。これらの知見を通じて,限られた計算資源でも堅牢なベースラインを構築するための指針を示す。
\end{abstract}

\begin{IEEEkeywords}
画像分類, 深層学習, Vision Transformer, Test-Time Augmentation, CIFAR-10
\end{IEEEkeywords}

\input{01_introduction/main}
\input{02_related_work/main}
\input{03_method/main}
\input{04_results/main}
\input{05_discussion/main}
\input{06_conclusion/main}
\input{appendix/main}

\bibliographystyle{ieeetr}
\bibliography{references}

\end{document}

% !TEX program = xelatex
\documentclass[conference]{IEEEtran}

% 日本語出力と XeLaTeX 対応設定
\usepackage{xeCJK}
\usepackage{fontspec}
\setCJKmainfont{HaranoAjiMincho}

% 数式・図表・参考文献などの標準パッケージ
\usepackage{amsmath, amssymb}
\usepackage{graphicx}
\usepackage{url}
\usepackage{hyperref}

\hypersetup{
  colorlinks=true,
  linkcolor=blue,
  citecolor=blue,
  urlcolor=blue
}

\begin{document}

\title{軽量 CNN と Vision Transformer の公平比較に向けた実験的評価}

\author{\IEEEauthorblockN{著者 太郎\\ }
\IEEEauthorblockA{所属機関\\ 連絡先: author@example.com}}

\maketitle

\begin{abstract}
本稿では,軽量な畳み込みニューラルネットワーク(CNN)と Vision Transformer(ViT)の画像分類性能を,CIFAR-10 を対象に公平に比較する実験プロトコルを提示する。学習率スケジューリング,データ拡張,Test-Time Augmentation (TTA) を統一的に適用し,再現性を重視した評価を実施した。実験の結果,ViT-Ti は TTA により最大で +1.8pt の精度向上を得た一方,ResNet-18 では RandAugment を中心としたデータ拡張で +2.1pt の改善が確認された。これらの知見を通じて,限られた計算資源でも堅牢なベースラインを構築するための指針を示す。
\end{abstract}

\begin{IEEEkeywords}
画像分類, 深層学習, Vision Transformer, Test-Time Augmentation, CIFAR-10
\end{IEEEkeywords}

\input{01_introduction/main}
\input{02_related_work/main}
\input{03_method/main}
\input{04_results/main}
\input{05_discussion/main}
\input{06_conclusion/main}
\input{appendix/main}

\bibliographystyle{ieeetr}
\bibliography{references}

\end{document}

% !TEX program = xelatex
\documentclass[conference]{IEEEtran}

% 日本語出力と XeLaTeX 対応設定
\usepackage{xeCJK}
\usepackage{fontspec}
\setCJKmainfont{HaranoAjiMincho}

% 数式・図表・参考文献などの標準パッケージ
\usepackage{amsmath, amssymb}
\usepackage{graphicx}
\usepackage{url}
\usepackage{hyperref}

\hypersetup{
  colorlinks=true,
  linkcolor=blue,
  citecolor=blue,
  urlcolor=blue
}

\begin{document}

\title{軽量 CNN と Vision Transformer の公平比較に向けた実験的評価}

\author{\IEEEauthorblockN{著者 太郎\\ }
\IEEEauthorblockA{所属機関\\ 連絡先: author@example.com}}

\maketitle

\begin{abstract}
本稿では,軽量な畳み込みニューラルネットワーク(CNN)と Vision Transformer(ViT)の画像分類性能を,CIFAR-10 を対象に公平に比較する実験プロトコルを提示する。学習率スケジューリング,データ拡張,Test-Time Augmentation (TTA) を統一的に適用し,再現性を重視した評価を実施した。実験の結果,ViT-Ti は TTA により最大で +1.8pt の精度向上を得た一方,ResNet-18 では RandAugment を中心としたデータ拡張で +2.1pt の改善が確認された。これらの知見を通じて,限られた計算資源でも堅牢なベースラインを構築するための指針を示す。
\end{abstract}

\begin{IEEEkeywords}
画像分類, 深層学習, Vision Transformer, Test-Time Augmentation, CIFAR-10
\end{IEEEkeywords}

\input{01_introduction/main}
\input{02_related_work/main}
\input{03_method/main}
\input{04_results/main}
\input{05_discussion/main}
\input{06_conclusion/main}
\input{appendix/main}

\bibliographystyle{ieeetr}
\bibliography{references}

\end{document}

% !TEX program = xelatex
\documentclass[conference]{IEEEtran}

% 日本語出力と XeLaTeX 対応設定
\usepackage{xeCJK}
\usepackage{fontspec}
\setCJKmainfont{HaranoAjiMincho}

% 数式・図表・参考文献などの標準パッケージ
\usepackage{amsmath, amssymb}
\usepackage{graphicx}
\usepackage{url}
\usepackage{hyperref}

\hypersetup{
  colorlinks=true,
  linkcolor=blue,
  citecolor=blue,
  urlcolor=blue
}

\begin{document}

\title{軽量 CNN と Vision Transformer の公平比較に向けた実験的評価}

\author{\IEEEauthorblockN{著者 太郎\\ }
\IEEEauthorblockA{所属機関\\ 連絡先: author@example.com}}

\maketitle

\begin{abstract}
本稿では,軽量な畳み込みニューラルネットワーク(CNN)と Vision Transformer(ViT)の画像分類性能を,CIFAR-10 を対象に公平に比較する実験プロトコルを提示する。学習率スケジューリング,データ拡張,Test-Time Augmentation (TTA) を統一的に適用し,再現性を重視した評価を実施した。実験の結果,ViT-Ti は TTA により最大で +1.8pt の精度向上を得た一方,ResNet-18 では RandAugment を中心としたデータ拡張で +2.1pt の改善が確認された。これらの知見を通じて,限られた計算資源でも堅牢なベースラインを構築するための指針を示す。
\end{abstract}

\begin{IEEEkeywords}
画像分類, 深層学習, Vision Transformer, Test-Time Augmentation, CIFAR-10
\end{IEEEkeywords}

\input{01_introduction/main}
\input{02_related_work/main}
\input{03_method/main}
\input{04_results/main}
\input{05_discussion/main}
\input{06_conclusion/main}
\input{appendix/main}

\bibliographystyle{ieeetr}
\bibliography{references}

\end{document}


\bibliographystyle{ieeetr}
\bibliography{references}

\end{document}

% !TEX program = xelatex
\documentclass[conference]{IEEEtran}

% 日本語出力と XeLaTeX 対応設定
\usepackage{xeCJK}
\usepackage{fontspec}
\setCJKmainfont{HaranoAjiMincho}

% 数式・図表・参考文献などの標準パッケージ
\usepackage{amsmath, amssymb}
\usepackage{graphicx}
\usepackage{url}
\usepackage{hyperref}

\hypersetup{
  colorlinks=true,
  linkcolor=blue,
  citecolor=blue,
  urlcolor=blue
}

\begin{document}

\title{軽量 CNN と Vision Transformer の公平比較に向けた実験的評価}

\author{\IEEEauthorblockN{著者 太郎\\ }
\IEEEauthorblockA{所属機関\\ 連絡先: author@example.com}}

\maketitle

\begin{abstract}
本稿では,軽量な畳み込みニューラルネットワーク(CNN)と Vision Transformer(ViT)の画像分類性能を,CIFAR-10 を対象に公平に比較する実験プロトコルを提示する。学習率スケジューリング,データ拡張,Test-Time Augmentation (TTA) を統一的に適用し,再現性を重視した評価を実施した。実験の結果,ViT-Ti は TTA により最大で +1.8pt の精度向上を得た一方,ResNet-18 では RandAugment を中心としたデータ拡張で +2.1pt の改善が確認された。これらの知見を通じて,限られた計算資源でも堅牢なベースラインを構築するための指針を示す。
\end{abstract}

\begin{IEEEkeywords}
画像分類, 深層学習, Vision Transformer, Test-Time Augmentation, CIFAR-10
\end{IEEEkeywords}

% !TEX program = xelatex
\documentclass[conference]{IEEEtran}

% 日本語出力と XeLaTeX 対応設定
\usepackage{xeCJK}
\usepackage{fontspec}
\setCJKmainfont{HaranoAjiMincho}

% 数式・図表・参考文献などの標準パッケージ
\usepackage{amsmath, amssymb}
\usepackage{graphicx}
\usepackage{url}
\usepackage{hyperref}

\hypersetup{
  colorlinks=true,
  linkcolor=blue,
  citecolor=blue,
  urlcolor=blue
}

\begin{document}

\title{軽量 CNN と Vision Transformer の公平比較に向けた実験的評価}

\author{\IEEEauthorblockN{著者 太郎\\ }
\IEEEauthorblockA{所属機関\\ 連絡先: author@example.com}}

\maketitle

\begin{abstract}
本稿では,軽量な畳み込みニューラルネットワーク(CNN)と Vision Transformer(ViT)の画像分類性能を,CIFAR-10 を対象に公平に比較する実験プロトコルを提示する。学習率スケジューリング,データ拡張,Test-Time Augmentation (TTA) を統一的に適用し,再現性を重視した評価を実施した。実験の結果,ViT-Ti は TTA により最大で +1.8pt の精度向上を得た一方,ResNet-18 では RandAugment を中心としたデータ拡張で +2.1pt の改善が確認された。これらの知見を通じて,限られた計算資源でも堅牢なベースラインを構築するための指針を示す。
\end{abstract}

\begin{IEEEkeywords}
画像分類, 深層学習, Vision Transformer, Test-Time Augmentation, CIFAR-10
\end{IEEEkeywords}

\input{01_introduction/main}
\input{02_related_work/main}
\input{03_method/main}
\input{04_results/main}
\input{05_discussion/main}
\input{06_conclusion/main}
\input{appendix/main}

\bibliographystyle{ieeetr}
\bibliography{references}

\end{document}

% !TEX program = xelatex
\documentclass[conference]{IEEEtran}

% 日本語出力と XeLaTeX 対応設定
\usepackage{xeCJK}
\usepackage{fontspec}
\setCJKmainfont{HaranoAjiMincho}

% 数式・図表・参考文献などの標準パッケージ
\usepackage{amsmath, amssymb}
\usepackage{graphicx}
\usepackage{url}
\usepackage{hyperref}

\hypersetup{
  colorlinks=true,
  linkcolor=blue,
  citecolor=blue,
  urlcolor=blue
}

\begin{document}

\title{軽量 CNN と Vision Transformer の公平比較に向けた実験的評価}

\author{\IEEEauthorblockN{著者 太郎\\ }
\IEEEauthorblockA{所属機関\\ 連絡先: author@example.com}}

\maketitle

\begin{abstract}
本稿では,軽量な畳み込みニューラルネットワーク(CNN)と Vision Transformer(ViT)の画像分類性能を,CIFAR-10 を対象に公平に比較する実験プロトコルを提示する。学習率スケジューリング,データ拡張,Test-Time Augmentation (TTA) を統一的に適用し,再現性を重視した評価を実施した。実験の結果,ViT-Ti は TTA により最大で +1.8pt の精度向上を得た一方,ResNet-18 では RandAugment を中心としたデータ拡張で +2.1pt の改善が確認された。これらの知見を通じて,限られた計算資源でも堅牢なベースラインを構築するための指針を示す。
\end{abstract}

\begin{IEEEkeywords}
画像分類, 深層学習, Vision Transformer, Test-Time Augmentation, CIFAR-10
\end{IEEEkeywords}

\input{01_introduction/main}
\input{02_related_work/main}
\input{03_method/main}
\input{04_results/main}
\input{05_discussion/main}
\input{06_conclusion/main}
\input{appendix/main}

\bibliographystyle{ieeetr}
\bibliography{references}

\end{document}

% !TEX program = xelatex
\documentclass[conference]{IEEEtran}

% 日本語出力と XeLaTeX 対応設定
\usepackage{xeCJK}
\usepackage{fontspec}
\setCJKmainfont{HaranoAjiMincho}

% 数式・図表・参考文献などの標準パッケージ
\usepackage{amsmath, amssymb}
\usepackage{graphicx}
\usepackage{url}
\usepackage{hyperref}

\hypersetup{
  colorlinks=true,
  linkcolor=blue,
  citecolor=blue,
  urlcolor=blue
}

\begin{document}

\title{軽量 CNN と Vision Transformer の公平比較に向けた実験的評価}

\author{\IEEEauthorblockN{著者 太郎\\ }
\IEEEauthorblockA{所属機関\\ 連絡先: author@example.com}}

\maketitle

\begin{abstract}
本稿では,軽量な畳み込みニューラルネットワーク(CNN)と Vision Transformer(ViT)の画像分類性能を,CIFAR-10 を対象に公平に比較する実験プロトコルを提示する。学習率スケジューリング,データ拡張,Test-Time Augmentation (TTA) を統一的に適用し,再現性を重視した評価を実施した。実験の結果,ViT-Ti は TTA により最大で +1.8pt の精度向上を得た一方,ResNet-18 では RandAugment を中心としたデータ拡張で +2.1pt の改善が確認された。これらの知見を通じて,限られた計算資源でも堅牢なベースラインを構築するための指針を示す。
\end{abstract}

\begin{IEEEkeywords}
画像分類, 深層学習, Vision Transformer, Test-Time Augmentation, CIFAR-10
\end{IEEEkeywords}

\input{01_introduction/main}
\input{02_related_work/main}
\input{03_method/main}
\input{04_results/main}
\input{05_discussion/main}
\input{06_conclusion/main}
\input{appendix/main}

\bibliographystyle{ieeetr}
\bibliography{references}

\end{document}

% !TEX program = xelatex
\documentclass[conference]{IEEEtran}

% 日本語出力と XeLaTeX 対応設定
\usepackage{xeCJK}
\usepackage{fontspec}
\setCJKmainfont{HaranoAjiMincho}

% 数式・図表・参考文献などの標準パッケージ
\usepackage{amsmath, amssymb}
\usepackage{graphicx}
\usepackage{url}
\usepackage{hyperref}

\hypersetup{
  colorlinks=true,
  linkcolor=blue,
  citecolor=blue,
  urlcolor=blue
}

\begin{document}

\title{軽量 CNN と Vision Transformer の公平比較に向けた実験的評価}

\author{\IEEEauthorblockN{著者 太郎\\ }
\IEEEauthorblockA{所属機関\\ 連絡先: author@example.com}}

\maketitle

\begin{abstract}
本稿では,軽量な畳み込みニューラルネットワーク(CNN)と Vision Transformer(ViT)の画像分類性能を,CIFAR-10 を対象に公平に比較する実験プロトコルを提示する。学習率スケジューリング,データ拡張,Test-Time Augmentation (TTA) を統一的に適用し,再現性を重視した評価を実施した。実験の結果,ViT-Ti は TTA により最大で +1.8pt の精度向上を得た一方,ResNet-18 では RandAugment を中心としたデータ拡張で +2.1pt の改善が確認された。これらの知見を通じて,限られた計算資源でも堅牢なベースラインを構築するための指針を示す。
\end{abstract}

\begin{IEEEkeywords}
画像分類, 深層学習, Vision Transformer, Test-Time Augmentation, CIFAR-10
\end{IEEEkeywords}

\input{01_introduction/main}
\input{02_related_work/main}
\input{03_method/main}
\input{04_results/main}
\input{05_discussion/main}
\input{06_conclusion/main}
\input{appendix/main}

\bibliographystyle{ieeetr}
\bibliography{references}

\end{document}

% !TEX program = xelatex
\documentclass[conference]{IEEEtran}

% 日本語出力と XeLaTeX 対応設定
\usepackage{xeCJK}
\usepackage{fontspec}
\setCJKmainfont{HaranoAjiMincho}

% 数式・図表・参考文献などの標準パッケージ
\usepackage{amsmath, amssymb}
\usepackage{graphicx}
\usepackage{url}
\usepackage{hyperref}

\hypersetup{
  colorlinks=true,
  linkcolor=blue,
  citecolor=blue,
  urlcolor=blue
}

\begin{document}

\title{軽量 CNN と Vision Transformer の公平比較に向けた実験的評価}

\author{\IEEEauthorblockN{著者 太郎\\ }
\IEEEauthorblockA{所属機関\\ 連絡先: author@example.com}}

\maketitle

\begin{abstract}
本稿では,軽量な畳み込みニューラルネットワーク(CNN)と Vision Transformer(ViT)の画像分類性能を,CIFAR-10 を対象に公平に比較する実験プロトコルを提示する。学習率スケジューリング,データ拡張,Test-Time Augmentation (TTA) を統一的に適用し,再現性を重視した評価を実施した。実験の結果,ViT-Ti は TTA により最大で +1.8pt の精度向上を得た一方,ResNet-18 では RandAugment を中心としたデータ拡張で +2.1pt の改善が確認された。これらの知見を通じて,限られた計算資源でも堅牢なベースラインを構築するための指針を示す。
\end{abstract}

\begin{IEEEkeywords}
画像分類, 深層学習, Vision Transformer, Test-Time Augmentation, CIFAR-10
\end{IEEEkeywords}

\input{01_introduction/main}
\input{02_related_work/main}
\input{03_method/main}
\input{04_results/main}
\input{05_discussion/main}
\input{06_conclusion/main}
\input{appendix/main}

\bibliographystyle{ieeetr}
\bibliography{references}

\end{document}

% !TEX program = xelatex
\documentclass[conference]{IEEEtran}

% 日本語出力と XeLaTeX 対応設定
\usepackage{xeCJK}
\usepackage{fontspec}
\setCJKmainfont{HaranoAjiMincho}

% 数式・図表・参考文献などの標準パッケージ
\usepackage{amsmath, amssymb}
\usepackage{graphicx}
\usepackage{url}
\usepackage{hyperref}

\hypersetup{
  colorlinks=true,
  linkcolor=blue,
  citecolor=blue,
  urlcolor=blue
}

\begin{document}

\title{軽量 CNN と Vision Transformer の公平比較に向けた実験的評価}

\author{\IEEEauthorblockN{著者 太郎\\ }
\IEEEauthorblockA{所属機関\\ 連絡先: author@example.com}}

\maketitle

\begin{abstract}
本稿では,軽量な畳み込みニューラルネットワーク(CNN)と Vision Transformer(ViT)の画像分類性能を,CIFAR-10 を対象に公平に比較する実験プロトコルを提示する。学習率スケジューリング,データ拡張,Test-Time Augmentation (TTA) を統一的に適用し,再現性を重視した評価を実施した。実験の結果,ViT-Ti は TTA により最大で +1.8pt の精度向上を得た一方,ResNet-18 では RandAugment を中心としたデータ拡張で +2.1pt の改善が確認された。これらの知見を通じて,限られた計算資源でも堅牢なベースラインを構築するための指針を示す。
\end{abstract}

\begin{IEEEkeywords}
画像分類, 深層学習, Vision Transformer, Test-Time Augmentation, CIFAR-10
\end{IEEEkeywords}

\input{01_introduction/main}
\input{02_related_work/main}
\input{03_method/main}
\input{04_results/main}
\input{05_discussion/main}
\input{06_conclusion/main}
\input{appendix/main}

\bibliographystyle{ieeetr}
\bibliography{references}

\end{document}

% !TEX program = xelatex
\documentclass[conference]{IEEEtran}

% 日本語出力と XeLaTeX 対応設定
\usepackage{xeCJK}
\usepackage{fontspec}
\setCJKmainfont{HaranoAjiMincho}

% 数式・図表・参考文献などの標準パッケージ
\usepackage{amsmath, amssymb}
\usepackage{graphicx}
\usepackage{url}
\usepackage{hyperref}

\hypersetup{
  colorlinks=true,
  linkcolor=blue,
  citecolor=blue,
  urlcolor=blue
}

\begin{document}

\title{軽量 CNN と Vision Transformer の公平比較に向けた実験的評価}

\author{\IEEEauthorblockN{著者 太郎\\ }
\IEEEauthorblockA{所属機関\\ 連絡先: author@example.com}}

\maketitle

\begin{abstract}
本稿では,軽量な畳み込みニューラルネットワーク(CNN)と Vision Transformer(ViT)の画像分類性能を,CIFAR-10 を対象に公平に比較する実験プロトコルを提示する。学習率スケジューリング,データ拡張,Test-Time Augmentation (TTA) を統一的に適用し,再現性を重視した評価を実施した。実験の結果,ViT-Ti は TTA により最大で +1.8pt の精度向上を得た一方,ResNet-18 では RandAugment を中心としたデータ拡張で +2.1pt の改善が確認された。これらの知見を通じて,限られた計算資源でも堅牢なベースラインを構築するための指針を示す。
\end{abstract}

\begin{IEEEkeywords}
画像分類, 深層学習, Vision Transformer, Test-Time Augmentation, CIFAR-10
\end{IEEEkeywords}

\input{01_introduction/main}
\input{02_related_work/main}
\input{03_method/main}
\input{04_results/main}
\input{05_discussion/main}
\input{06_conclusion/main}
\input{appendix/main}

\bibliographystyle{ieeetr}
\bibliography{references}

\end{document}


\bibliographystyle{ieeetr}
\bibliography{references}

\end{document}

% !TEX program = xelatex
\documentclass[conference]{IEEEtran}

% 日本語出力と XeLaTeX 対応設定
\usepackage{xeCJK}
\usepackage{fontspec}
\setCJKmainfont{HaranoAjiMincho}

% 数式・図表・参考文献などの標準パッケージ
\usepackage{amsmath, amssymb}
\usepackage{graphicx}
\usepackage{url}
\usepackage{hyperref}

\hypersetup{
  colorlinks=true,
  linkcolor=blue,
  citecolor=blue,
  urlcolor=blue
}

\begin{document}

\title{軽量 CNN と Vision Transformer の公平比較に向けた実験的評価}

\author{\IEEEauthorblockN{著者 太郎\\ }
\IEEEauthorblockA{所属機関\\ 連絡先: author@example.com}}

\maketitle

\begin{abstract}
本稿では,軽量な畳み込みニューラルネットワーク(CNN)と Vision Transformer(ViT)の画像分類性能を,CIFAR-10 を対象に公平に比較する実験プロトコルを提示する。学習率スケジューリング,データ拡張,Test-Time Augmentation (TTA) を統一的に適用し,再現性を重視した評価を実施した。実験の結果,ViT-Ti は TTA により最大で +1.8pt の精度向上を得た一方,ResNet-18 では RandAugment を中心としたデータ拡張で +2.1pt の改善が確認された。これらの知見を通じて,限られた計算資源でも堅牢なベースラインを構築するための指針を示す。
\end{abstract}

\begin{IEEEkeywords}
画像分類, 深層学習, Vision Transformer, Test-Time Augmentation, CIFAR-10
\end{IEEEkeywords}

% !TEX program = xelatex
\documentclass[conference]{IEEEtran}

% 日本語出力と XeLaTeX 対応設定
\usepackage{xeCJK}
\usepackage{fontspec}
\setCJKmainfont{HaranoAjiMincho}

% 数式・図表・参考文献などの標準パッケージ
\usepackage{amsmath, amssymb}
\usepackage{graphicx}
\usepackage{url}
\usepackage{hyperref}

\hypersetup{
  colorlinks=true,
  linkcolor=blue,
  citecolor=blue,
  urlcolor=blue
}

\begin{document}

\title{軽量 CNN と Vision Transformer の公平比較に向けた実験的評価}

\author{\IEEEauthorblockN{著者 太郎\\ }
\IEEEauthorblockA{所属機関\\ 連絡先: author@example.com}}

\maketitle

\begin{abstract}
本稿では,軽量な畳み込みニューラルネットワーク(CNN)と Vision Transformer(ViT)の画像分類性能を,CIFAR-10 を対象に公平に比較する実験プロトコルを提示する。学習率スケジューリング,データ拡張,Test-Time Augmentation (TTA) を統一的に適用し,再現性を重視した評価を実施した。実験の結果,ViT-Ti は TTA により最大で +1.8pt の精度向上を得た一方,ResNet-18 では RandAugment を中心としたデータ拡張で +2.1pt の改善が確認された。これらの知見を通じて,限られた計算資源でも堅牢なベースラインを構築するための指針を示す。
\end{abstract}

\begin{IEEEkeywords}
画像分類, 深層学習, Vision Transformer, Test-Time Augmentation, CIFAR-10
\end{IEEEkeywords}

\input{01_introduction/main}
\input{02_related_work/main}
\input{03_method/main}
\input{04_results/main}
\input{05_discussion/main}
\input{06_conclusion/main}
\input{appendix/main}

\bibliographystyle{ieeetr}
\bibliography{references}

\end{document}

% !TEX program = xelatex
\documentclass[conference]{IEEEtran}

% 日本語出力と XeLaTeX 対応設定
\usepackage{xeCJK}
\usepackage{fontspec}
\setCJKmainfont{HaranoAjiMincho}

% 数式・図表・参考文献などの標準パッケージ
\usepackage{amsmath, amssymb}
\usepackage{graphicx}
\usepackage{url}
\usepackage{hyperref}

\hypersetup{
  colorlinks=true,
  linkcolor=blue,
  citecolor=blue,
  urlcolor=blue
}

\begin{document}

\title{軽量 CNN と Vision Transformer の公平比較に向けた実験的評価}

\author{\IEEEauthorblockN{著者 太郎\\ }
\IEEEauthorblockA{所属機関\\ 連絡先: author@example.com}}

\maketitle

\begin{abstract}
本稿では,軽量な畳み込みニューラルネットワーク(CNN)と Vision Transformer(ViT)の画像分類性能を,CIFAR-10 を対象に公平に比較する実験プロトコルを提示する。学習率スケジューリング,データ拡張,Test-Time Augmentation (TTA) を統一的に適用し,再現性を重視した評価を実施した。実験の結果,ViT-Ti は TTA により最大で +1.8pt の精度向上を得た一方,ResNet-18 では RandAugment を中心としたデータ拡張で +2.1pt の改善が確認された。これらの知見を通じて,限られた計算資源でも堅牢なベースラインを構築するための指針を示す。
\end{abstract}

\begin{IEEEkeywords}
画像分類, 深層学習, Vision Transformer, Test-Time Augmentation, CIFAR-10
\end{IEEEkeywords}

\input{01_introduction/main}
\input{02_related_work/main}
\input{03_method/main}
\input{04_results/main}
\input{05_discussion/main}
\input{06_conclusion/main}
\input{appendix/main}

\bibliographystyle{ieeetr}
\bibliography{references}

\end{document}

% !TEX program = xelatex
\documentclass[conference]{IEEEtran}

% 日本語出力と XeLaTeX 対応設定
\usepackage{xeCJK}
\usepackage{fontspec}
\setCJKmainfont{HaranoAjiMincho}

% 数式・図表・参考文献などの標準パッケージ
\usepackage{amsmath, amssymb}
\usepackage{graphicx}
\usepackage{url}
\usepackage{hyperref}

\hypersetup{
  colorlinks=true,
  linkcolor=blue,
  citecolor=blue,
  urlcolor=blue
}

\begin{document}

\title{軽量 CNN と Vision Transformer の公平比較に向けた実験的評価}

\author{\IEEEauthorblockN{著者 太郎\\ }
\IEEEauthorblockA{所属機関\\ 連絡先: author@example.com}}

\maketitle

\begin{abstract}
本稿では,軽量な畳み込みニューラルネットワーク(CNN)と Vision Transformer(ViT)の画像分類性能を,CIFAR-10 を対象に公平に比較する実験プロトコルを提示する。学習率スケジューリング,データ拡張,Test-Time Augmentation (TTA) を統一的に適用し,再現性を重視した評価を実施した。実験の結果,ViT-Ti は TTA により最大で +1.8pt の精度向上を得た一方,ResNet-18 では RandAugment を中心としたデータ拡張で +2.1pt の改善が確認された。これらの知見を通じて,限られた計算資源でも堅牢なベースラインを構築するための指針を示す。
\end{abstract}

\begin{IEEEkeywords}
画像分類, 深層学習, Vision Transformer, Test-Time Augmentation, CIFAR-10
\end{IEEEkeywords}

\input{01_introduction/main}
\input{02_related_work/main}
\input{03_method/main}
\input{04_results/main}
\input{05_discussion/main}
\input{06_conclusion/main}
\input{appendix/main}

\bibliographystyle{ieeetr}
\bibliography{references}

\end{document}

% !TEX program = xelatex
\documentclass[conference]{IEEEtran}

% 日本語出力と XeLaTeX 対応設定
\usepackage{xeCJK}
\usepackage{fontspec}
\setCJKmainfont{HaranoAjiMincho}

% 数式・図表・参考文献などの標準パッケージ
\usepackage{amsmath, amssymb}
\usepackage{graphicx}
\usepackage{url}
\usepackage{hyperref}

\hypersetup{
  colorlinks=true,
  linkcolor=blue,
  citecolor=blue,
  urlcolor=blue
}

\begin{document}

\title{軽量 CNN と Vision Transformer の公平比較に向けた実験的評価}

\author{\IEEEauthorblockN{著者 太郎\\ }
\IEEEauthorblockA{所属機関\\ 連絡先: author@example.com}}

\maketitle

\begin{abstract}
本稿では,軽量な畳み込みニューラルネットワーク(CNN)と Vision Transformer(ViT)の画像分類性能を,CIFAR-10 を対象に公平に比較する実験プロトコルを提示する。学習率スケジューリング,データ拡張,Test-Time Augmentation (TTA) を統一的に適用し,再現性を重視した評価を実施した。実験の結果,ViT-Ti は TTA により最大で +1.8pt の精度向上を得た一方,ResNet-18 では RandAugment を中心としたデータ拡張で +2.1pt の改善が確認された。これらの知見を通じて,限られた計算資源でも堅牢なベースラインを構築するための指針を示す。
\end{abstract}

\begin{IEEEkeywords}
画像分類, 深層学習, Vision Transformer, Test-Time Augmentation, CIFAR-10
\end{IEEEkeywords}

\input{01_introduction/main}
\input{02_related_work/main}
\input{03_method/main}
\input{04_results/main}
\input{05_discussion/main}
\input{06_conclusion/main}
\input{appendix/main}

\bibliographystyle{ieeetr}
\bibliography{references}

\end{document}

% !TEX program = xelatex
\documentclass[conference]{IEEEtran}

% 日本語出力と XeLaTeX 対応設定
\usepackage{xeCJK}
\usepackage{fontspec}
\setCJKmainfont{HaranoAjiMincho}

% 数式・図表・参考文献などの標準パッケージ
\usepackage{amsmath, amssymb}
\usepackage{graphicx}
\usepackage{url}
\usepackage{hyperref}

\hypersetup{
  colorlinks=true,
  linkcolor=blue,
  citecolor=blue,
  urlcolor=blue
}

\begin{document}

\title{軽量 CNN と Vision Transformer の公平比較に向けた実験的評価}

\author{\IEEEauthorblockN{著者 太郎\\ }
\IEEEauthorblockA{所属機関\\ 連絡先: author@example.com}}

\maketitle

\begin{abstract}
本稿では,軽量な畳み込みニューラルネットワーク(CNN)と Vision Transformer(ViT)の画像分類性能を,CIFAR-10 を対象に公平に比較する実験プロトコルを提示する。学習率スケジューリング,データ拡張,Test-Time Augmentation (TTA) を統一的に適用し,再現性を重視した評価を実施した。実験の結果,ViT-Ti は TTA により最大で +1.8pt の精度向上を得た一方,ResNet-18 では RandAugment を中心としたデータ拡張で +2.1pt の改善が確認された。これらの知見を通じて,限られた計算資源でも堅牢なベースラインを構築するための指針を示す。
\end{abstract}

\begin{IEEEkeywords}
画像分類, 深層学習, Vision Transformer, Test-Time Augmentation, CIFAR-10
\end{IEEEkeywords}

\input{01_introduction/main}
\input{02_related_work/main}
\input{03_method/main}
\input{04_results/main}
\input{05_discussion/main}
\input{06_conclusion/main}
\input{appendix/main}

\bibliographystyle{ieeetr}
\bibliography{references}

\end{document}

% !TEX program = xelatex
\documentclass[conference]{IEEEtran}

% 日本語出力と XeLaTeX 対応設定
\usepackage{xeCJK}
\usepackage{fontspec}
\setCJKmainfont{HaranoAjiMincho}

% 数式・図表・参考文献などの標準パッケージ
\usepackage{amsmath, amssymb}
\usepackage{graphicx}
\usepackage{url}
\usepackage{hyperref}

\hypersetup{
  colorlinks=true,
  linkcolor=blue,
  citecolor=blue,
  urlcolor=blue
}

\begin{document}

\title{軽量 CNN と Vision Transformer の公平比較に向けた実験的評価}

\author{\IEEEauthorblockN{著者 太郎\\ }
\IEEEauthorblockA{所属機関\\ 連絡先: author@example.com}}

\maketitle

\begin{abstract}
本稿では,軽量な畳み込みニューラルネットワーク(CNN)と Vision Transformer(ViT)の画像分類性能を,CIFAR-10 を対象に公平に比較する実験プロトコルを提示する。学習率スケジューリング,データ拡張,Test-Time Augmentation (TTA) を統一的に適用し,再現性を重視した評価を実施した。実験の結果,ViT-Ti は TTA により最大で +1.8pt の精度向上を得た一方,ResNet-18 では RandAugment を中心としたデータ拡張で +2.1pt の改善が確認された。これらの知見を通じて,限られた計算資源でも堅牢なベースラインを構築するための指針を示す。
\end{abstract}

\begin{IEEEkeywords}
画像分類, 深層学習, Vision Transformer, Test-Time Augmentation, CIFAR-10
\end{IEEEkeywords}

\input{01_introduction/main}
\input{02_related_work/main}
\input{03_method/main}
\input{04_results/main}
\input{05_discussion/main}
\input{06_conclusion/main}
\input{appendix/main}

\bibliographystyle{ieeetr}
\bibliography{references}

\end{document}

% !TEX program = xelatex
\documentclass[conference]{IEEEtran}

% 日本語出力と XeLaTeX 対応設定
\usepackage{xeCJK}
\usepackage{fontspec}
\setCJKmainfont{HaranoAjiMincho}

% 数式・図表・参考文献などの標準パッケージ
\usepackage{amsmath, amssymb}
\usepackage{graphicx}
\usepackage{url}
\usepackage{hyperref}

\hypersetup{
  colorlinks=true,
  linkcolor=blue,
  citecolor=blue,
  urlcolor=blue
}

\begin{document}

\title{軽量 CNN と Vision Transformer の公平比較に向けた実験的評価}

\author{\IEEEauthorblockN{著者 太郎\\ }
\IEEEauthorblockA{所属機関\\ 連絡先: author@example.com}}

\maketitle

\begin{abstract}
本稿では,軽量な畳み込みニューラルネットワーク(CNN)と Vision Transformer(ViT)の画像分類性能を,CIFAR-10 を対象に公平に比較する実験プロトコルを提示する。学習率スケジューリング,データ拡張,Test-Time Augmentation (TTA) を統一的に適用し,再現性を重視した評価を実施した。実験の結果,ViT-Ti は TTA により最大で +1.8pt の精度向上を得た一方,ResNet-18 では RandAugment を中心としたデータ拡張で +2.1pt の改善が確認された。これらの知見を通じて,限られた計算資源でも堅牢なベースラインを構築するための指針を示す。
\end{abstract}

\begin{IEEEkeywords}
画像分類, 深層学習, Vision Transformer, Test-Time Augmentation, CIFAR-10
\end{IEEEkeywords}

\input{01_introduction/main}
\input{02_related_work/main}
\input{03_method/main}
\input{04_results/main}
\input{05_discussion/main}
\input{06_conclusion/main}
\input{appendix/main}

\bibliographystyle{ieeetr}
\bibliography{references}

\end{document}


\bibliographystyle{ieeetr}
\bibliography{references}

\end{document}

% !TEX program = xelatex
\documentclass[conference]{IEEEtran}

% 日本語出力と XeLaTeX 対応設定
\usepackage{xeCJK}
\usepackage{fontspec}
\setCJKmainfont{HaranoAjiMincho}

% 数式・図表・参考文献などの標準パッケージ
\usepackage{amsmath, amssymb}
\usepackage{graphicx}
\usepackage{url}
\usepackage{hyperref}

\hypersetup{
  colorlinks=true,
  linkcolor=blue,
  citecolor=blue,
  urlcolor=blue
}

\begin{document}

\title{軽量 CNN と Vision Transformer の公平比較に向けた実験的評価}

\author{\IEEEauthorblockN{著者 太郎\\ }
\IEEEauthorblockA{所属機関\\ 連絡先: author@example.com}}

\maketitle

\begin{abstract}
本稿では,軽量な畳み込みニューラルネットワーク(CNN)と Vision Transformer(ViT)の画像分類性能を,CIFAR-10 を対象に公平に比較する実験プロトコルを提示する。学習率スケジューリング,データ拡張,Test-Time Augmentation (TTA) を統一的に適用し,再現性を重視した評価を実施した。実験の結果,ViT-Ti は TTA により最大で +1.8pt の精度向上を得た一方,ResNet-18 では RandAugment を中心としたデータ拡張で +2.1pt の改善が確認された。これらの知見を通じて,限られた計算資源でも堅牢なベースラインを構築するための指針を示す。
\end{abstract}

\begin{IEEEkeywords}
画像分類, 深層学習, Vision Transformer, Test-Time Augmentation, CIFAR-10
\end{IEEEkeywords}

% !TEX program = xelatex
\documentclass[conference]{IEEEtran}

% 日本語出力と XeLaTeX 対応設定
\usepackage{xeCJK}
\usepackage{fontspec}
\setCJKmainfont{HaranoAjiMincho}

% 数式・図表・参考文献などの標準パッケージ
\usepackage{amsmath, amssymb}
\usepackage{graphicx}
\usepackage{url}
\usepackage{hyperref}

\hypersetup{
  colorlinks=true,
  linkcolor=blue,
  citecolor=blue,
  urlcolor=blue
}

\begin{document}

\title{軽量 CNN と Vision Transformer の公平比較に向けた実験的評価}

\author{\IEEEauthorblockN{著者 太郎\\ }
\IEEEauthorblockA{所属機関\\ 連絡先: author@example.com}}

\maketitle

\begin{abstract}
本稿では,軽量な畳み込みニューラルネットワーク(CNN)と Vision Transformer(ViT)の画像分類性能を,CIFAR-10 を対象に公平に比較する実験プロトコルを提示する。学習率スケジューリング,データ拡張,Test-Time Augmentation (TTA) を統一的に適用し,再現性を重視した評価を実施した。実験の結果,ViT-Ti は TTA により最大で +1.8pt の精度向上を得た一方,ResNet-18 では RandAugment を中心としたデータ拡張で +2.1pt の改善が確認された。これらの知見を通じて,限られた計算資源でも堅牢なベースラインを構築するための指針を示す。
\end{abstract}

\begin{IEEEkeywords}
画像分類, 深層学習, Vision Transformer, Test-Time Augmentation, CIFAR-10
\end{IEEEkeywords}

\input{01_introduction/main}
\input{02_related_work/main}
\input{03_method/main}
\input{04_results/main}
\input{05_discussion/main}
\input{06_conclusion/main}
\input{appendix/main}

\bibliographystyle{ieeetr}
\bibliography{references}

\end{document}

% !TEX program = xelatex
\documentclass[conference]{IEEEtran}

% 日本語出力と XeLaTeX 対応設定
\usepackage{xeCJK}
\usepackage{fontspec}
\setCJKmainfont{HaranoAjiMincho}

% 数式・図表・参考文献などの標準パッケージ
\usepackage{amsmath, amssymb}
\usepackage{graphicx}
\usepackage{url}
\usepackage{hyperref}

\hypersetup{
  colorlinks=true,
  linkcolor=blue,
  citecolor=blue,
  urlcolor=blue
}

\begin{document}

\title{軽量 CNN と Vision Transformer の公平比較に向けた実験的評価}

\author{\IEEEauthorblockN{著者 太郎\\ }
\IEEEauthorblockA{所属機関\\ 連絡先: author@example.com}}

\maketitle

\begin{abstract}
本稿では,軽量な畳み込みニューラルネットワーク(CNN)と Vision Transformer(ViT)の画像分類性能を,CIFAR-10 を対象に公平に比較する実験プロトコルを提示する。学習率スケジューリング,データ拡張,Test-Time Augmentation (TTA) を統一的に適用し,再現性を重視した評価を実施した。実験の結果,ViT-Ti は TTA により最大で +1.8pt の精度向上を得た一方,ResNet-18 では RandAugment を中心としたデータ拡張で +2.1pt の改善が確認された。これらの知見を通じて,限られた計算資源でも堅牢なベースラインを構築するための指針を示す。
\end{abstract}

\begin{IEEEkeywords}
画像分類, 深層学習, Vision Transformer, Test-Time Augmentation, CIFAR-10
\end{IEEEkeywords}

\input{01_introduction/main}
\input{02_related_work/main}
\input{03_method/main}
\input{04_results/main}
\input{05_discussion/main}
\input{06_conclusion/main}
\input{appendix/main}

\bibliographystyle{ieeetr}
\bibliography{references}

\end{document}

% !TEX program = xelatex
\documentclass[conference]{IEEEtran}

% 日本語出力と XeLaTeX 対応設定
\usepackage{xeCJK}
\usepackage{fontspec}
\setCJKmainfont{HaranoAjiMincho}

% 数式・図表・参考文献などの標準パッケージ
\usepackage{amsmath, amssymb}
\usepackage{graphicx}
\usepackage{url}
\usepackage{hyperref}

\hypersetup{
  colorlinks=true,
  linkcolor=blue,
  citecolor=blue,
  urlcolor=blue
}

\begin{document}

\title{軽量 CNN と Vision Transformer の公平比較に向けた実験的評価}

\author{\IEEEauthorblockN{著者 太郎\\ }
\IEEEauthorblockA{所属機関\\ 連絡先: author@example.com}}

\maketitle

\begin{abstract}
本稿では,軽量な畳み込みニューラルネットワーク(CNN)と Vision Transformer(ViT)の画像分類性能を,CIFAR-10 を対象に公平に比較する実験プロトコルを提示する。学習率スケジューリング,データ拡張,Test-Time Augmentation (TTA) を統一的に適用し,再現性を重視した評価を実施した。実験の結果,ViT-Ti は TTA により最大で +1.8pt の精度向上を得た一方,ResNet-18 では RandAugment を中心としたデータ拡張で +2.1pt の改善が確認された。これらの知見を通じて,限られた計算資源でも堅牢なベースラインを構築するための指針を示す。
\end{abstract}

\begin{IEEEkeywords}
画像分類, 深層学習, Vision Transformer, Test-Time Augmentation, CIFAR-10
\end{IEEEkeywords}

\input{01_introduction/main}
\input{02_related_work/main}
\input{03_method/main}
\input{04_results/main}
\input{05_discussion/main}
\input{06_conclusion/main}
\input{appendix/main}

\bibliographystyle{ieeetr}
\bibliography{references}

\end{document}

% !TEX program = xelatex
\documentclass[conference]{IEEEtran}

% 日本語出力と XeLaTeX 対応設定
\usepackage{xeCJK}
\usepackage{fontspec}
\setCJKmainfont{HaranoAjiMincho}

% 数式・図表・参考文献などの標準パッケージ
\usepackage{amsmath, amssymb}
\usepackage{graphicx}
\usepackage{url}
\usepackage{hyperref}

\hypersetup{
  colorlinks=true,
  linkcolor=blue,
  citecolor=blue,
  urlcolor=blue
}

\begin{document}

\title{軽量 CNN と Vision Transformer の公平比較に向けた実験的評価}

\author{\IEEEauthorblockN{著者 太郎\\ }
\IEEEauthorblockA{所属機関\\ 連絡先: author@example.com}}

\maketitle

\begin{abstract}
本稿では,軽量な畳み込みニューラルネットワーク(CNN)と Vision Transformer(ViT)の画像分類性能を,CIFAR-10 を対象に公平に比較する実験プロトコルを提示する。学習率スケジューリング,データ拡張,Test-Time Augmentation (TTA) を統一的に適用し,再現性を重視した評価を実施した。実験の結果,ViT-Ti は TTA により最大で +1.8pt の精度向上を得た一方,ResNet-18 では RandAugment を中心としたデータ拡張で +2.1pt の改善が確認された。これらの知見を通じて,限られた計算資源でも堅牢なベースラインを構築するための指針を示す。
\end{abstract}

\begin{IEEEkeywords}
画像分類, 深層学習, Vision Transformer, Test-Time Augmentation, CIFAR-10
\end{IEEEkeywords}

\input{01_introduction/main}
\input{02_related_work/main}
\input{03_method/main}
\input{04_results/main}
\input{05_discussion/main}
\input{06_conclusion/main}
\input{appendix/main}

\bibliographystyle{ieeetr}
\bibliography{references}

\end{document}

% !TEX program = xelatex
\documentclass[conference]{IEEEtran}

% 日本語出力と XeLaTeX 対応設定
\usepackage{xeCJK}
\usepackage{fontspec}
\setCJKmainfont{HaranoAjiMincho}

% 数式・図表・参考文献などの標準パッケージ
\usepackage{amsmath, amssymb}
\usepackage{graphicx}
\usepackage{url}
\usepackage{hyperref}

\hypersetup{
  colorlinks=true,
  linkcolor=blue,
  citecolor=blue,
  urlcolor=blue
}

\begin{document}

\title{軽量 CNN と Vision Transformer の公平比較に向けた実験的評価}

\author{\IEEEauthorblockN{著者 太郎\\ }
\IEEEauthorblockA{所属機関\\ 連絡先: author@example.com}}

\maketitle

\begin{abstract}
本稿では,軽量な畳み込みニューラルネットワーク(CNN)と Vision Transformer(ViT)の画像分類性能を,CIFAR-10 を対象に公平に比較する実験プロトコルを提示する。学習率スケジューリング,データ拡張,Test-Time Augmentation (TTA) を統一的に適用し,再現性を重視した評価を実施した。実験の結果,ViT-Ti は TTA により最大で +1.8pt の精度向上を得た一方,ResNet-18 では RandAugment を中心としたデータ拡張で +2.1pt の改善が確認された。これらの知見を通じて,限られた計算資源でも堅牢なベースラインを構築するための指針を示す。
\end{abstract}

\begin{IEEEkeywords}
画像分類, 深層学習, Vision Transformer, Test-Time Augmentation, CIFAR-10
\end{IEEEkeywords}

\input{01_introduction/main}
\input{02_related_work/main}
\input{03_method/main}
\input{04_results/main}
\input{05_discussion/main}
\input{06_conclusion/main}
\input{appendix/main}

\bibliographystyle{ieeetr}
\bibliography{references}

\end{document}

% !TEX program = xelatex
\documentclass[conference]{IEEEtran}

% 日本語出力と XeLaTeX 対応設定
\usepackage{xeCJK}
\usepackage{fontspec}
\setCJKmainfont{HaranoAjiMincho}

% 数式・図表・参考文献などの標準パッケージ
\usepackage{amsmath, amssymb}
\usepackage{graphicx}
\usepackage{url}
\usepackage{hyperref}

\hypersetup{
  colorlinks=true,
  linkcolor=blue,
  citecolor=blue,
  urlcolor=blue
}

\begin{document}

\title{軽量 CNN と Vision Transformer の公平比較に向けた実験的評価}

\author{\IEEEauthorblockN{著者 太郎\\ }
\IEEEauthorblockA{所属機関\\ 連絡先: author@example.com}}

\maketitle

\begin{abstract}
本稿では,軽量な畳み込みニューラルネットワーク(CNN)と Vision Transformer(ViT)の画像分類性能を,CIFAR-10 を対象に公平に比較する実験プロトコルを提示する。学習率スケジューリング,データ拡張,Test-Time Augmentation (TTA) を統一的に適用し,再現性を重視した評価を実施した。実験の結果,ViT-Ti は TTA により最大で +1.8pt の精度向上を得た一方,ResNet-18 では RandAugment を中心としたデータ拡張で +2.1pt の改善が確認された。これらの知見を通じて,限られた計算資源でも堅牢なベースラインを構築するための指針を示す。
\end{abstract}

\begin{IEEEkeywords}
画像分類, 深層学習, Vision Transformer, Test-Time Augmentation, CIFAR-10
\end{IEEEkeywords}

\input{01_introduction/main}
\input{02_related_work/main}
\input{03_method/main}
\input{04_results/main}
\input{05_discussion/main}
\input{06_conclusion/main}
\input{appendix/main}

\bibliographystyle{ieeetr}
\bibliography{references}

\end{document}

% !TEX program = xelatex
\documentclass[conference]{IEEEtran}

% 日本語出力と XeLaTeX 対応設定
\usepackage{xeCJK}
\usepackage{fontspec}
\setCJKmainfont{HaranoAjiMincho}

% 数式・図表・参考文献などの標準パッケージ
\usepackage{amsmath, amssymb}
\usepackage{graphicx}
\usepackage{url}
\usepackage{hyperref}

\hypersetup{
  colorlinks=true,
  linkcolor=blue,
  citecolor=blue,
  urlcolor=blue
}

\begin{document}

\title{軽量 CNN と Vision Transformer の公平比較に向けた実験的評価}

\author{\IEEEauthorblockN{著者 太郎\\ }
\IEEEauthorblockA{所属機関\\ 連絡先: author@example.com}}

\maketitle

\begin{abstract}
本稿では,軽量な畳み込みニューラルネットワーク(CNN)と Vision Transformer(ViT)の画像分類性能を,CIFAR-10 を対象に公平に比較する実験プロトコルを提示する。学習率スケジューリング,データ拡張,Test-Time Augmentation (TTA) を統一的に適用し,再現性を重視した評価を実施した。実験の結果,ViT-Ti は TTA により最大で +1.8pt の精度向上を得た一方,ResNet-18 では RandAugment を中心としたデータ拡張で +2.1pt の改善が確認された。これらの知見を通じて,限られた計算資源でも堅牢なベースラインを構築するための指針を示す。
\end{abstract}

\begin{IEEEkeywords}
画像分類, 深層学習, Vision Transformer, Test-Time Augmentation, CIFAR-10
\end{IEEEkeywords}

\input{01_introduction/main}
\input{02_related_work/main}
\input{03_method/main}
\input{04_results/main}
\input{05_discussion/main}
\input{06_conclusion/main}
\input{appendix/main}

\bibliographystyle{ieeetr}
\bibliography{references}

\end{document}


\bibliographystyle{ieeetr}
\bibliography{references}

\end{document}

% !TEX program = xelatex
\documentclass[conference]{IEEEtran}

% 日本語出力と XeLaTeX 対応設定
\usepackage{xeCJK}
\usepackage{fontspec}
\setCJKmainfont{HaranoAjiMincho}

% 数式・図表・参考文献などの標準パッケージ
\usepackage{amsmath, amssymb}
\usepackage{graphicx}
\usepackage{url}
\usepackage{hyperref}

\hypersetup{
  colorlinks=true,
  linkcolor=blue,
  citecolor=blue,
  urlcolor=blue
}

\begin{document}

\title{軽量 CNN と Vision Transformer の公平比較に向けた実験的評価}

\author{\IEEEauthorblockN{著者 太郎\\ }
\IEEEauthorblockA{所属機関\\ 連絡先: author@example.com}}

\maketitle

\begin{abstract}
本稿では,軽量な畳み込みニューラルネットワーク(CNN)と Vision Transformer(ViT)の画像分類性能を,CIFAR-10 を対象に公平に比較する実験プロトコルを提示する。学習率スケジューリング,データ拡張,Test-Time Augmentation (TTA) を統一的に適用し,再現性を重視した評価を実施した。実験の結果,ViT-Ti は TTA により最大で +1.8pt の精度向上を得た一方,ResNet-18 では RandAugment を中心としたデータ拡張で +2.1pt の改善が確認された。これらの知見を通じて,限られた計算資源でも堅牢なベースラインを構築するための指針を示す。
\end{abstract}

\begin{IEEEkeywords}
画像分類, 深層学習, Vision Transformer, Test-Time Augmentation, CIFAR-10
\end{IEEEkeywords}

% !TEX program = xelatex
\documentclass[conference]{IEEEtran}

% 日本語出力と XeLaTeX 対応設定
\usepackage{xeCJK}
\usepackage{fontspec}
\setCJKmainfont{HaranoAjiMincho}

% 数式・図表・参考文献などの標準パッケージ
\usepackage{amsmath, amssymb}
\usepackage{graphicx}
\usepackage{url}
\usepackage{hyperref}

\hypersetup{
  colorlinks=true,
  linkcolor=blue,
  citecolor=blue,
  urlcolor=blue
}

\begin{document}

\title{軽量 CNN と Vision Transformer の公平比較に向けた実験的評価}

\author{\IEEEauthorblockN{著者 太郎\\ }
\IEEEauthorblockA{所属機関\\ 連絡先: author@example.com}}

\maketitle

\begin{abstract}
本稿では,軽量な畳み込みニューラルネットワーク(CNN)と Vision Transformer(ViT)の画像分類性能を,CIFAR-10 を対象に公平に比較する実験プロトコルを提示する。学習率スケジューリング,データ拡張,Test-Time Augmentation (TTA) を統一的に適用し,再現性を重視した評価を実施した。実験の結果,ViT-Ti は TTA により最大で +1.8pt の精度向上を得た一方,ResNet-18 では RandAugment を中心としたデータ拡張で +2.1pt の改善が確認された。これらの知見を通じて,限られた計算資源でも堅牢なベースラインを構築するための指針を示す。
\end{abstract}

\begin{IEEEkeywords}
画像分類, 深層学習, Vision Transformer, Test-Time Augmentation, CIFAR-10
\end{IEEEkeywords}

\input{01_introduction/main}
\input{02_related_work/main}
\input{03_method/main}
\input{04_results/main}
\input{05_discussion/main}
\input{06_conclusion/main}
\input{appendix/main}

\bibliographystyle{ieeetr}
\bibliography{references}

\end{document}

% !TEX program = xelatex
\documentclass[conference]{IEEEtran}

% 日本語出力と XeLaTeX 対応設定
\usepackage{xeCJK}
\usepackage{fontspec}
\setCJKmainfont{HaranoAjiMincho}

% 数式・図表・参考文献などの標準パッケージ
\usepackage{amsmath, amssymb}
\usepackage{graphicx}
\usepackage{url}
\usepackage{hyperref}

\hypersetup{
  colorlinks=true,
  linkcolor=blue,
  citecolor=blue,
  urlcolor=blue
}

\begin{document}

\title{軽量 CNN と Vision Transformer の公平比較に向けた実験的評価}

\author{\IEEEauthorblockN{著者 太郎\\ }
\IEEEauthorblockA{所属機関\\ 連絡先: author@example.com}}

\maketitle

\begin{abstract}
本稿では,軽量な畳み込みニューラルネットワーク(CNN)と Vision Transformer(ViT)の画像分類性能を,CIFAR-10 を対象に公平に比較する実験プロトコルを提示する。学習率スケジューリング,データ拡張,Test-Time Augmentation (TTA) を統一的に適用し,再現性を重視した評価を実施した。実験の結果,ViT-Ti は TTA により最大で +1.8pt の精度向上を得た一方,ResNet-18 では RandAugment を中心としたデータ拡張で +2.1pt の改善が確認された。これらの知見を通じて,限られた計算資源でも堅牢なベースラインを構築するための指針を示す。
\end{abstract}

\begin{IEEEkeywords}
画像分類, 深層学習, Vision Transformer, Test-Time Augmentation, CIFAR-10
\end{IEEEkeywords}

\input{01_introduction/main}
\input{02_related_work/main}
\input{03_method/main}
\input{04_results/main}
\input{05_discussion/main}
\input{06_conclusion/main}
\input{appendix/main}

\bibliographystyle{ieeetr}
\bibliography{references}

\end{document}

% !TEX program = xelatex
\documentclass[conference]{IEEEtran}

% 日本語出力と XeLaTeX 対応設定
\usepackage{xeCJK}
\usepackage{fontspec}
\setCJKmainfont{HaranoAjiMincho}

% 数式・図表・参考文献などの標準パッケージ
\usepackage{amsmath, amssymb}
\usepackage{graphicx}
\usepackage{url}
\usepackage{hyperref}

\hypersetup{
  colorlinks=true,
  linkcolor=blue,
  citecolor=blue,
  urlcolor=blue
}

\begin{document}

\title{軽量 CNN と Vision Transformer の公平比較に向けた実験的評価}

\author{\IEEEauthorblockN{著者 太郎\\ }
\IEEEauthorblockA{所属機関\\ 連絡先: author@example.com}}

\maketitle

\begin{abstract}
本稿では,軽量な畳み込みニューラルネットワーク(CNN)と Vision Transformer(ViT)の画像分類性能を,CIFAR-10 を対象に公平に比較する実験プロトコルを提示する。学習率スケジューリング,データ拡張,Test-Time Augmentation (TTA) を統一的に適用し,再現性を重視した評価を実施した。実験の結果,ViT-Ti は TTA により最大で +1.8pt の精度向上を得た一方,ResNet-18 では RandAugment を中心としたデータ拡張で +2.1pt の改善が確認された。これらの知見を通じて,限られた計算資源でも堅牢なベースラインを構築するための指針を示す。
\end{abstract}

\begin{IEEEkeywords}
画像分類, 深層学習, Vision Transformer, Test-Time Augmentation, CIFAR-10
\end{IEEEkeywords}

\input{01_introduction/main}
\input{02_related_work/main}
\input{03_method/main}
\input{04_results/main}
\input{05_discussion/main}
\input{06_conclusion/main}
\input{appendix/main}

\bibliographystyle{ieeetr}
\bibliography{references}

\end{document}

% !TEX program = xelatex
\documentclass[conference]{IEEEtran}

% 日本語出力と XeLaTeX 対応設定
\usepackage{xeCJK}
\usepackage{fontspec}
\setCJKmainfont{HaranoAjiMincho}

% 数式・図表・参考文献などの標準パッケージ
\usepackage{amsmath, amssymb}
\usepackage{graphicx}
\usepackage{url}
\usepackage{hyperref}

\hypersetup{
  colorlinks=true,
  linkcolor=blue,
  citecolor=blue,
  urlcolor=blue
}

\begin{document}

\title{軽量 CNN と Vision Transformer の公平比較に向けた実験的評価}

\author{\IEEEauthorblockN{著者 太郎\\ }
\IEEEauthorblockA{所属機関\\ 連絡先: author@example.com}}

\maketitle

\begin{abstract}
本稿では,軽量な畳み込みニューラルネットワーク(CNN)と Vision Transformer(ViT)の画像分類性能を,CIFAR-10 を対象に公平に比較する実験プロトコルを提示する。学習率スケジューリング,データ拡張,Test-Time Augmentation (TTA) を統一的に適用し,再現性を重視した評価を実施した。実験の結果,ViT-Ti は TTA により最大で +1.8pt の精度向上を得た一方,ResNet-18 では RandAugment を中心としたデータ拡張で +2.1pt の改善が確認された。これらの知見を通じて,限られた計算資源でも堅牢なベースラインを構築するための指針を示す。
\end{abstract}

\begin{IEEEkeywords}
画像分類, 深層学習, Vision Transformer, Test-Time Augmentation, CIFAR-10
\end{IEEEkeywords}

\input{01_introduction/main}
\input{02_related_work/main}
\input{03_method/main}
\input{04_results/main}
\input{05_discussion/main}
\input{06_conclusion/main}
\input{appendix/main}

\bibliographystyle{ieeetr}
\bibliography{references}

\end{document}

% !TEX program = xelatex
\documentclass[conference]{IEEEtran}

% 日本語出力と XeLaTeX 対応設定
\usepackage{xeCJK}
\usepackage{fontspec}
\setCJKmainfont{HaranoAjiMincho}

% 数式・図表・参考文献などの標準パッケージ
\usepackage{amsmath, amssymb}
\usepackage{graphicx}
\usepackage{url}
\usepackage{hyperref}

\hypersetup{
  colorlinks=true,
  linkcolor=blue,
  citecolor=blue,
  urlcolor=blue
}

\begin{document}

\title{軽量 CNN と Vision Transformer の公平比較に向けた実験的評価}

\author{\IEEEauthorblockN{著者 太郎\\ }
\IEEEauthorblockA{所属機関\\ 連絡先: author@example.com}}

\maketitle

\begin{abstract}
本稿では,軽量な畳み込みニューラルネットワーク(CNN)と Vision Transformer(ViT)の画像分類性能を,CIFAR-10 を対象に公平に比較する実験プロトコルを提示する。学習率スケジューリング,データ拡張,Test-Time Augmentation (TTA) を統一的に適用し,再現性を重視した評価を実施した。実験の結果,ViT-Ti は TTA により最大で +1.8pt の精度向上を得た一方,ResNet-18 では RandAugment を中心としたデータ拡張で +2.1pt の改善が確認された。これらの知見を通じて,限られた計算資源でも堅牢なベースラインを構築するための指針を示す。
\end{abstract}

\begin{IEEEkeywords}
画像分類, 深層学習, Vision Transformer, Test-Time Augmentation, CIFAR-10
\end{IEEEkeywords}

\input{01_introduction/main}
\input{02_related_work/main}
\input{03_method/main}
\input{04_results/main}
\input{05_discussion/main}
\input{06_conclusion/main}
\input{appendix/main}

\bibliographystyle{ieeetr}
\bibliography{references}

\end{document}

% !TEX program = xelatex
\documentclass[conference]{IEEEtran}

% 日本語出力と XeLaTeX 対応設定
\usepackage{xeCJK}
\usepackage{fontspec}
\setCJKmainfont{HaranoAjiMincho}

% 数式・図表・参考文献などの標準パッケージ
\usepackage{amsmath, amssymb}
\usepackage{graphicx}
\usepackage{url}
\usepackage{hyperref}

\hypersetup{
  colorlinks=true,
  linkcolor=blue,
  citecolor=blue,
  urlcolor=blue
}

\begin{document}

\title{軽量 CNN と Vision Transformer の公平比較に向けた実験的評価}

\author{\IEEEauthorblockN{著者 太郎\\ }
\IEEEauthorblockA{所属機関\\ 連絡先: author@example.com}}

\maketitle

\begin{abstract}
本稿では,軽量な畳み込みニューラルネットワーク(CNN)と Vision Transformer(ViT)の画像分類性能を,CIFAR-10 を対象に公平に比較する実験プロトコルを提示する。学習率スケジューリング,データ拡張,Test-Time Augmentation (TTA) を統一的に適用し,再現性を重視した評価を実施した。実験の結果,ViT-Ti は TTA により最大で +1.8pt の精度向上を得た一方,ResNet-18 では RandAugment を中心としたデータ拡張で +2.1pt の改善が確認された。これらの知見を通じて,限られた計算資源でも堅牢なベースラインを構築するための指針を示す。
\end{abstract}

\begin{IEEEkeywords}
画像分類, 深層学習, Vision Transformer, Test-Time Augmentation, CIFAR-10
\end{IEEEkeywords}

\input{01_introduction/main}
\input{02_related_work/main}
\input{03_method/main}
\input{04_results/main}
\input{05_discussion/main}
\input{06_conclusion/main}
\input{appendix/main}

\bibliographystyle{ieeetr}
\bibliography{references}

\end{document}

% !TEX program = xelatex
\documentclass[conference]{IEEEtran}

% 日本語出力と XeLaTeX 対応設定
\usepackage{xeCJK}
\usepackage{fontspec}
\setCJKmainfont{HaranoAjiMincho}

% 数式・図表・参考文献などの標準パッケージ
\usepackage{amsmath, amssymb}
\usepackage{graphicx}
\usepackage{url}
\usepackage{hyperref}

\hypersetup{
  colorlinks=true,
  linkcolor=blue,
  citecolor=blue,
  urlcolor=blue
}

\begin{document}

\title{軽量 CNN と Vision Transformer の公平比較に向けた実験的評価}

\author{\IEEEauthorblockN{著者 太郎\\ }
\IEEEauthorblockA{所属機関\\ 連絡先: author@example.com}}

\maketitle

\begin{abstract}
本稿では,軽量な畳み込みニューラルネットワーク(CNN)と Vision Transformer(ViT)の画像分類性能を,CIFAR-10 を対象に公平に比較する実験プロトコルを提示する。学習率スケジューリング,データ拡張,Test-Time Augmentation (TTA) を統一的に適用し,再現性を重視した評価を実施した。実験の結果,ViT-Ti は TTA により最大で +1.8pt の精度向上を得た一方,ResNet-18 では RandAugment を中心としたデータ拡張で +2.1pt の改善が確認された。これらの知見を通じて,限られた計算資源でも堅牢なベースラインを構築するための指針を示す。
\end{abstract}

\begin{IEEEkeywords}
画像分類, 深層学習, Vision Transformer, Test-Time Augmentation, CIFAR-10
\end{IEEEkeywords}

\input{01_introduction/main}
\input{02_related_work/main}
\input{03_method/main}
\input{04_results/main}
\input{05_discussion/main}
\input{06_conclusion/main}
\input{appendix/main}

\bibliographystyle{ieeetr}
\bibliography{references}

\end{document}


\bibliographystyle{ieeetr}
\bibliography{references}

\end{document}

% !TEX program = xelatex
\documentclass[conference]{IEEEtran}

% 日本語出力と XeLaTeX 対応設定
\usepackage{xeCJK}
\usepackage{fontspec}
\setCJKmainfont{HaranoAjiMincho}

% 数式・図表・参考文献などの標準パッケージ
\usepackage{amsmath, amssymb}
\usepackage{graphicx}
\usepackage{url}
\usepackage{hyperref}

\hypersetup{
  colorlinks=true,
  linkcolor=blue,
  citecolor=blue,
  urlcolor=blue
}

\begin{document}

\title{軽量 CNN と Vision Transformer の公平比較に向けた実験的評価}

\author{\IEEEauthorblockN{著者 太郎\\ }
\IEEEauthorblockA{所属機関\\ 連絡先: author@example.com}}

\maketitle

\begin{abstract}
本稿では,軽量な畳み込みニューラルネットワーク(CNN)と Vision Transformer(ViT)の画像分類性能を,CIFAR-10 を対象に公平に比較する実験プロトコルを提示する。学習率スケジューリング,データ拡張,Test-Time Augmentation (TTA) を統一的に適用し,再現性を重視した評価を実施した。実験の結果,ViT-Ti は TTA により最大で +1.8pt の精度向上を得た一方,ResNet-18 では RandAugment を中心としたデータ拡張で +2.1pt の改善が確認された。これらの知見を通じて,限られた計算資源でも堅牢なベースラインを構築するための指針を示す。
\end{abstract}

\begin{IEEEkeywords}
画像分類, 深層学習, Vision Transformer, Test-Time Augmentation, CIFAR-10
\end{IEEEkeywords}

% !TEX program = xelatex
\documentclass[conference]{IEEEtran}

% 日本語出力と XeLaTeX 対応設定
\usepackage{xeCJK}
\usepackage{fontspec}
\setCJKmainfont{HaranoAjiMincho}

% 数式・図表・参考文献などの標準パッケージ
\usepackage{amsmath, amssymb}
\usepackage{graphicx}
\usepackage{url}
\usepackage{hyperref}

\hypersetup{
  colorlinks=true,
  linkcolor=blue,
  citecolor=blue,
  urlcolor=blue
}

\begin{document}

\title{軽量 CNN と Vision Transformer の公平比較に向けた実験的評価}

\author{\IEEEauthorblockN{著者 太郎\\ }
\IEEEauthorblockA{所属機関\\ 連絡先: author@example.com}}

\maketitle

\begin{abstract}
本稿では,軽量な畳み込みニューラルネットワーク(CNN)と Vision Transformer(ViT)の画像分類性能を,CIFAR-10 を対象に公平に比較する実験プロトコルを提示する。学習率スケジューリング,データ拡張,Test-Time Augmentation (TTA) を統一的に適用し,再現性を重視した評価を実施した。実験の結果,ViT-Ti は TTA により最大で +1.8pt の精度向上を得た一方,ResNet-18 では RandAugment を中心としたデータ拡張で +2.1pt の改善が確認された。これらの知見を通じて,限られた計算資源でも堅牢なベースラインを構築するための指針を示す。
\end{abstract}

\begin{IEEEkeywords}
画像分類, 深層学習, Vision Transformer, Test-Time Augmentation, CIFAR-10
\end{IEEEkeywords}

\input{01_introduction/main}
\input{02_related_work/main}
\input{03_method/main}
\input{04_results/main}
\input{05_discussion/main}
\input{06_conclusion/main}
\input{appendix/main}

\bibliographystyle{ieeetr}
\bibliography{references}

\end{document}

% !TEX program = xelatex
\documentclass[conference]{IEEEtran}

% 日本語出力と XeLaTeX 対応設定
\usepackage{xeCJK}
\usepackage{fontspec}
\setCJKmainfont{HaranoAjiMincho}

% 数式・図表・参考文献などの標準パッケージ
\usepackage{amsmath, amssymb}
\usepackage{graphicx}
\usepackage{url}
\usepackage{hyperref}

\hypersetup{
  colorlinks=true,
  linkcolor=blue,
  citecolor=blue,
  urlcolor=blue
}

\begin{document}

\title{軽量 CNN と Vision Transformer の公平比較に向けた実験的評価}

\author{\IEEEauthorblockN{著者 太郎\\ }
\IEEEauthorblockA{所属機関\\ 連絡先: author@example.com}}

\maketitle

\begin{abstract}
本稿では,軽量な畳み込みニューラルネットワーク(CNN)と Vision Transformer(ViT)の画像分類性能を,CIFAR-10 を対象に公平に比較する実験プロトコルを提示する。学習率スケジューリング,データ拡張,Test-Time Augmentation (TTA) を統一的に適用し,再現性を重視した評価を実施した。実験の結果,ViT-Ti は TTA により最大で +1.8pt の精度向上を得た一方,ResNet-18 では RandAugment を中心としたデータ拡張で +2.1pt の改善が確認された。これらの知見を通じて,限られた計算資源でも堅牢なベースラインを構築するための指針を示す。
\end{abstract}

\begin{IEEEkeywords}
画像分類, 深層学習, Vision Transformer, Test-Time Augmentation, CIFAR-10
\end{IEEEkeywords}

\input{01_introduction/main}
\input{02_related_work/main}
\input{03_method/main}
\input{04_results/main}
\input{05_discussion/main}
\input{06_conclusion/main}
\input{appendix/main}

\bibliographystyle{ieeetr}
\bibliography{references}

\end{document}

% !TEX program = xelatex
\documentclass[conference]{IEEEtran}

% 日本語出力と XeLaTeX 対応設定
\usepackage{xeCJK}
\usepackage{fontspec}
\setCJKmainfont{HaranoAjiMincho}

% 数式・図表・参考文献などの標準パッケージ
\usepackage{amsmath, amssymb}
\usepackage{graphicx}
\usepackage{url}
\usepackage{hyperref}

\hypersetup{
  colorlinks=true,
  linkcolor=blue,
  citecolor=blue,
  urlcolor=blue
}

\begin{document}

\title{軽量 CNN と Vision Transformer の公平比較に向けた実験的評価}

\author{\IEEEauthorblockN{著者 太郎\\ }
\IEEEauthorblockA{所属機関\\ 連絡先: author@example.com}}

\maketitle

\begin{abstract}
本稿では,軽量な畳み込みニューラルネットワーク(CNN)と Vision Transformer(ViT)の画像分類性能を,CIFAR-10 を対象に公平に比較する実験プロトコルを提示する。学習率スケジューリング,データ拡張,Test-Time Augmentation (TTA) を統一的に適用し,再現性を重視した評価を実施した。実験の結果,ViT-Ti は TTA により最大で +1.8pt の精度向上を得た一方,ResNet-18 では RandAugment を中心としたデータ拡張で +2.1pt の改善が確認された。これらの知見を通じて,限られた計算資源でも堅牢なベースラインを構築するための指針を示す。
\end{abstract}

\begin{IEEEkeywords}
画像分類, 深層学習, Vision Transformer, Test-Time Augmentation, CIFAR-10
\end{IEEEkeywords}

\input{01_introduction/main}
\input{02_related_work/main}
\input{03_method/main}
\input{04_results/main}
\input{05_discussion/main}
\input{06_conclusion/main}
\input{appendix/main}

\bibliographystyle{ieeetr}
\bibliography{references}

\end{document}

% !TEX program = xelatex
\documentclass[conference]{IEEEtran}

% 日本語出力と XeLaTeX 対応設定
\usepackage{xeCJK}
\usepackage{fontspec}
\setCJKmainfont{HaranoAjiMincho}

% 数式・図表・参考文献などの標準パッケージ
\usepackage{amsmath, amssymb}
\usepackage{graphicx}
\usepackage{url}
\usepackage{hyperref}

\hypersetup{
  colorlinks=true,
  linkcolor=blue,
  citecolor=blue,
  urlcolor=blue
}

\begin{document}

\title{軽量 CNN と Vision Transformer の公平比較に向けた実験的評価}

\author{\IEEEauthorblockN{著者 太郎\\ }
\IEEEauthorblockA{所属機関\\ 連絡先: author@example.com}}

\maketitle

\begin{abstract}
本稿では,軽量な畳み込みニューラルネットワーク(CNN)と Vision Transformer(ViT)の画像分類性能を,CIFAR-10 を対象に公平に比較する実験プロトコルを提示する。学習率スケジューリング,データ拡張,Test-Time Augmentation (TTA) を統一的に適用し,再現性を重視した評価を実施した。実験の結果,ViT-Ti は TTA により最大で +1.8pt の精度向上を得た一方,ResNet-18 では RandAugment を中心としたデータ拡張で +2.1pt の改善が確認された。これらの知見を通じて,限られた計算資源でも堅牢なベースラインを構築するための指針を示す。
\end{abstract}

\begin{IEEEkeywords}
画像分類, 深層学習, Vision Transformer, Test-Time Augmentation, CIFAR-10
\end{IEEEkeywords}

\input{01_introduction/main}
\input{02_related_work/main}
\input{03_method/main}
\input{04_results/main}
\input{05_discussion/main}
\input{06_conclusion/main}
\input{appendix/main}

\bibliographystyle{ieeetr}
\bibliography{references}

\end{document}

% !TEX program = xelatex
\documentclass[conference]{IEEEtran}

% 日本語出力と XeLaTeX 対応設定
\usepackage{xeCJK}
\usepackage{fontspec}
\setCJKmainfont{HaranoAjiMincho}

% 数式・図表・参考文献などの標準パッケージ
\usepackage{amsmath, amssymb}
\usepackage{graphicx}
\usepackage{url}
\usepackage{hyperref}

\hypersetup{
  colorlinks=true,
  linkcolor=blue,
  citecolor=blue,
  urlcolor=blue
}

\begin{document}

\title{軽量 CNN と Vision Transformer の公平比較に向けた実験的評価}

\author{\IEEEauthorblockN{著者 太郎\\ }
\IEEEauthorblockA{所属機関\\ 連絡先: author@example.com}}

\maketitle

\begin{abstract}
本稿では,軽量な畳み込みニューラルネットワーク(CNN)と Vision Transformer(ViT)の画像分類性能を,CIFAR-10 を対象に公平に比較する実験プロトコルを提示する。学習率スケジューリング,データ拡張,Test-Time Augmentation (TTA) を統一的に適用し,再現性を重視した評価を実施した。実験の結果,ViT-Ti は TTA により最大で +1.8pt の精度向上を得た一方,ResNet-18 では RandAugment を中心としたデータ拡張で +2.1pt の改善が確認された。これらの知見を通じて,限られた計算資源でも堅牢なベースラインを構築するための指針を示す。
\end{abstract}

\begin{IEEEkeywords}
画像分類, 深層学習, Vision Transformer, Test-Time Augmentation, CIFAR-10
\end{IEEEkeywords}

\input{01_introduction/main}
\input{02_related_work/main}
\input{03_method/main}
\input{04_results/main}
\input{05_discussion/main}
\input{06_conclusion/main}
\input{appendix/main}

\bibliographystyle{ieeetr}
\bibliography{references}

\end{document}

% !TEX program = xelatex
\documentclass[conference]{IEEEtran}

% 日本語出力と XeLaTeX 対応設定
\usepackage{xeCJK}
\usepackage{fontspec}
\setCJKmainfont{HaranoAjiMincho}

% 数式・図表・参考文献などの標準パッケージ
\usepackage{amsmath, amssymb}
\usepackage{graphicx}
\usepackage{url}
\usepackage{hyperref}

\hypersetup{
  colorlinks=true,
  linkcolor=blue,
  citecolor=blue,
  urlcolor=blue
}

\begin{document}

\title{軽量 CNN と Vision Transformer の公平比較に向けた実験的評価}

\author{\IEEEauthorblockN{著者 太郎\\ }
\IEEEauthorblockA{所属機関\\ 連絡先: author@example.com}}

\maketitle

\begin{abstract}
本稿では,軽量な畳み込みニューラルネットワーク(CNN)と Vision Transformer(ViT)の画像分類性能を,CIFAR-10 を対象に公平に比較する実験プロトコルを提示する。学習率スケジューリング,データ拡張,Test-Time Augmentation (TTA) を統一的に適用し,再現性を重視した評価を実施した。実験の結果,ViT-Ti は TTA により最大で +1.8pt の精度向上を得た一方,ResNet-18 では RandAugment を中心としたデータ拡張で +2.1pt の改善が確認された。これらの知見を通じて,限られた計算資源でも堅牢なベースラインを構築するための指針を示す。
\end{abstract}

\begin{IEEEkeywords}
画像分類, 深層学習, Vision Transformer, Test-Time Augmentation, CIFAR-10
\end{IEEEkeywords}

\input{01_introduction/main}
\input{02_related_work/main}
\input{03_method/main}
\input{04_results/main}
\input{05_discussion/main}
\input{06_conclusion/main}
\input{appendix/main}

\bibliographystyle{ieeetr}
\bibliography{references}

\end{document}

% !TEX program = xelatex
\documentclass[conference]{IEEEtran}

% 日本語出力と XeLaTeX 対応設定
\usepackage{xeCJK}
\usepackage{fontspec}
\setCJKmainfont{HaranoAjiMincho}

% 数式・図表・参考文献などの標準パッケージ
\usepackage{amsmath, amssymb}
\usepackage{graphicx}
\usepackage{url}
\usepackage{hyperref}

\hypersetup{
  colorlinks=true,
  linkcolor=blue,
  citecolor=blue,
  urlcolor=blue
}

\begin{document}

\title{軽量 CNN と Vision Transformer の公平比較に向けた実験的評価}

\author{\IEEEauthorblockN{著者 太郎\\ }
\IEEEauthorblockA{所属機関\\ 連絡先: author@example.com}}

\maketitle

\begin{abstract}
本稿では,軽量な畳み込みニューラルネットワーク(CNN)と Vision Transformer(ViT)の画像分類性能を,CIFAR-10 を対象に公平に比較する実験プロトコルを提示する。学習率スケジューリング,データ拡張,Test-Time Augmentation (TTA) を統一的に適用し,再現性を重視した評価を実施した。実験の結果,ViT-Ti は TTA により最大で +1.8pt の精度向上を得た一方,ResNet-18 では RandAugment を中心としたデータ拡張で +2.1pt の改善が確認された。これらの知見を通じて,限られた計算資源でも堅牢なベースラインを構築するための指針を示す。
\end{abstract}

\begin{IEEEkeywords}
画像分類, 深層学習, Vision Transformer, Test-Time Augmentation, CIFAR-10
\end{IEEEkeywords}

\input{01_introduction/main}
\input{02_related_work/main}
\input{03_method/main}
\input{04_results/main}
\input{05_discussion/main}
\input{06_conclusion/main}
\input{appendix/main}

\bibliographystyle{ieeetr}
\bibliography{references}

\end{document}


\bibliographystyle{ieeetr}
\bibliography{references}

\end{document}

% !TEX program = xelatex
\documentclass[conference]{IEEEtran}

% 日本語出力と XeLaTeX 対応設定
\usepackage{xeCJK}
\usepackage{fontspec}
\setCJKmainfont{HaranoAjiMincho}

% 数式・図表・参考文献などの標準パッケージ
\usepackage{amsmath, amssymb}
\usepackage{graphicx}
\usepackage{url}
\usepackage{hyperref}

\hypersetup{
  colorlinks=true,
  linkcolor=blue,
  citecolor=blue,
  urlcolor=blue
}

\begin{document}

\title{軽量 CNN と Vision Transformer の公平比較に向けた実験的評価}

\author{\IEEEauthorblockN{著者 太郎\\ }
\IEEEauthorblockA{所属機関\\ 連絡先: author@example.com}}

\maketitle

\begin{abstract}
本稿では,軽量な畳み込みニューラルネットワーク(CNN)と Vision Transformer(ViT)の画像分類性能を,CIFAR-10 を対象に公平に比較する実験プロトコルを提示する。学習率スケジューリング,データ拡張,Test-Time Augmentation (TTA) を統一的に適用し,再現性を重視した評価を実施した。実験の結果,ViT-Ti は TTA により最大で +1.8pt の精度向上を得た一方,ResNet-18 では RandAugment を中心としたデータ拡張で +2.1pt の改善が確認された。これらの知見を通じて,限られた計算資源でも堅牢なベースラインを構築するための指針を示す。
\end{abstract}

\begin{IEEEkeywords}
画像分類, 深層学習, Vision Transformer, Test-Time Augmentation, CIFAR-10
\end{IEEEkeywords}

% !TEX program = xelatex
\documentclass[conference]{IEEEtran}

% 日本語出力と XeLaTeX 対応設定
\usepackage{xeCJK}
\usepackage{fontspec}
\setCJKmainfont{HaranoAjiMincho}

% 数式・図表・参考文献などの標準パッケージ
\usepackage{amsmath, amssymb}
\usepackage{graphicx}
\usepackage{url}
\usepackage{hyperref}

\hypersetup{
  colorlinks=true,
  linkcolor=blue,
  citecolor=blue,
  urlcolor=blue
}

\begin{document}

\title{軽量 CNN と Vision Transformer の公平比較に向けた実験的評価}

\author{\IEEEauthorblockN{著者 太郎\\ }
\IEEEauthorblockA{所属機関\\ 連絡先: author@example.com}}

\maketitle

\begin{abstract}
本稿では,軽量な畳み込みニューラルネットワーク(CNN)と Vision Transformer(ViT)の画像分類性能を,CIFAR-10 を対象に公平に比較する実験プロトコルを提示する。学習率スケジューリング,データ拡張,Test-Time Augmentation (TTA) を統一的に適用し,再現性を重視した評価を実施した。実験の結果,ViT-Ti は TTA により最大で +1.8pt の精度向上を得た一方,ResNet-18 では RandAugment を中心としたデータ拡張で +2.1pt の改善が確認された。これらの知見を通じて,限られた計算資源でも堅牢なベースラインを構築するための指針を示す。
\end{abstract}

\begin{IEEEkeywords}
画像分類, 深層学習, Vision Transformer, Test-Time Augmentation, CIFAR-10
\end{IEEEkeywords}

\input{01_introduction/main}
\input{02_related_work/main}
\input{03_method/main}
\input{04_results/main}
\input{05_discussion/main}
\input{06_conclusion/main}
\input{appendix/main}

\bibliographystyle{ieeetr}
\bibliography{references}

\end{document}

% !TEX program = xelatex
\documentclass[conference]{IEEEtran}

% 日本語出力と XeLaTeX 対応設定
\usepackage{xeCJK}
\usepackage{fontspec}
\setCJKmainfont{HaranoAjiMincho}

% 数式・図表・参考文献などの標準パッケージ
\usepackage{amsmath, amssymb}
\usepackage{graphicx}
\usepackage{url}
\usepackage{hyperref}

\hypersetup{
  colorlinks=true,
  linkcolor=blue,
  citecolor=blue,
  urlcolor=blue
}

\begin{document}

\title{軽量 CNN と Vision Transformer の公平比較に向けた実験的評価}

\author{\IEEEauthorblockN{著者 太郎\\ }
\IEEEauthorblockA{所属機関\\ 連絡先: author@example.com}}

\maketitle

\begin{abstract}
本稿では,軽量な畳み込みニューラルネットワーク(CNN)と Vision Transformer(ViT)の画像分類性能を,CIFAR-10 を対象に公平に比較する実験プロトコルを提示する。学習率スケジューリング,データ拡張,Test-Time Augmentation (TTA) を統一的に適用し,再現性を重視した評価を実施した。実験の結果,ViT-Ti は TTA により最大で +1.8pt の精度向上を得た一方,ResNet-18 では RandAugment を中心としたデータ拡張で +2.1pt の改善が確認された。これらの知見を通じて,限られた計算資源でも堅牢なベースラインを構築するための指針を示す。
\end{abstract}

\begin{IEEEkeywords}
画像分類, 深層学習, Vision Transformer, Test-Time Augmentation, CIFAR-10
\end{IEEEkeywords}

\input{01_introduction/main}
\input{02_related_work/main}
\input{03_method/main}
\input{04_results/main}
\input{05_discussion/main}
\input{06_conclusion/main}
\input{appendix/main}

\bibliographystyle{ieeetr}
\bibliography{references}

\end{document}

% !TEX program = xelatex
\documentclass[conference]{IEEEtran}

% 日本語出力と XeLaTeX 対応設定
\usepackage{xeCJK}
\usepackage{fontspec}
\setCJKmainfont{HaranoAjiMincho}

% 数式・図表・参考文献などの標準パッケージ
\usepackage{amsmath, amssymb}
\usepackage{graphicx}
\usepackage{url}
\usepackage{hyperref}

\hypersetup{
  colorlinks=true,
  linkcolor=blue,
  citecolor=blue,
  urlcolor=blue
}

\begin{document}

\title{軽量 CNN と Vision Transformer の公平比較に向けた実験的評価}

\author{\IEEEauthorblockN{著者 太郎\\ }
\IEEEauthorblockA{所属機関\\ 連絡先: author@example.com}}

\maketitle

\begin{abstract}
本稿では,軽量な畳み込みニューラルネットワーク(CNN)と Vision Transformer(ViT)の画像分類性能を,CIFAR-10 を対象に公平に比較する実験プロトコルを提示する。学習率スケジューリング,データ拡張,Test-Time Augmentation (TTA) を統一的に適用し,再現性を重視した評価を実施した。実験の結果,ViT-Ti は TTA により最大で +1.8pt の精度向上を得た一方,ResNet-18 では RandAugment を中心としたデータ拡張で +2.1pt の改善が確認された。これらの知見を通じて,限られた計算資源でも堅牢なベースラインを構築するための指針を示す。
\end{abstract}

\begin{IEEEkeywords}
画像分類, 深層学習, Vision Transformer, Test-Time Augmentation, CIFAR-10
\end{IEEEkeywords}

\input{01_introduction/main}
\input{02_related_work/main}
\input{03_method/main}
\input{04_results/main}
\input{05_discussion/main}
\input{06_conclusion/main}
\input{appendix/main}

\bibliographystyle{ieeetr}
\bibliography{references}

\end{document}

% !TEX program = xelatex
\documentclass[conference]{IEEEtran}

% 日本語出力と XeLaTeX 対応設定
\usepackage{xeCJK}
\usepackage{fontspec}
\setCJKmainfont{HaranoAjiMincho}

% 数式・図表・参考文献などの標準パッケージ
\usepackage{amsmath, amssymb}
\usepackage{graphicx}
\usepackage{url}
\usepackage{hyperref}

\hypersetup{
  colorlinks=true,
  linkcolor=blue,
  citecolor=blue,
  urlcolor=blue
}

\begin{document}

\title{軽量 CNN と Vision Transformer の公平比較に向けた実験的評価}

\author{\IEEEauthorblockN{著者 太郎\\ }
\IEEEauthorblockA{所属機関\\ 連絡先: author@example.com}}

\maketitle

\begin{abstract}
本稿では,軽量な畳み込みニューラルネットワーク(CNN)と Vision Transformer(ViT)の画像分類性能を,CIFAR-10 を対象に公平に比較する実験プロトコルを提示する。学習率スケジューリング,データ拡張,Test-Time Augmentation (TTA) を統一的に適用し,再現性を重視した評価を実施した。実験の結果,ViT-Ti は TTA により最大で +1.8pt の精度向上を得た一方,ResNet-18 では RandAugment を中心としたデータ拡張で +2.1pt の改善が確認された。これらの知見を通じて,限られた計算資源でも堅牢なベースラインを構築するための指針を示す。
\end{abstract}

\begin{IEEEkeywords}
画像分類, 深層学習, Vision Transformer, Test-Time Augmentation, CIFAR-10
\end{IEEEkeywords}

\input{01_introduction/main}
\input{02_related_work/main}
\input{03_method/main}
\input{04_results/main}
\input{05_discussion/main}
\input{06_conclusion/main}
\input{appendix/main}

\bibliographystyle{ieeetr}
\bibliography{references}

\end{document}

% !TEX program = xelatex
\documentclass[conference]{IEEEtran}

% 日本語出力と XeLaTeX 対応設定
\usepackage{xeCJK}
\usepackage{fontspec}
\setCJKmainfont{HaranoAjiMincho}

% 数式・図表・参考文献などの標準パッケージ
\usepackage{amsmath, amssymb}
\usepackage{graphicx}
\usepackage{url}
\usepackage{hyperref}

\hypersetup{
  colorlinks=true,
  linkcolor=blue,
  citecolor=blue,
  urlcolor=blue
}

\begin{document}

\title{軽量 CNN と Vision Transformer の公平比較に向けた実験的評価}

\author{\IEEEauthorblockN{著者 太郎\\ }
\IEEEauthorblockA{所属機関\\ 連絡先: author@example.com}}

\maketitle

\begin{abstract}
本稿では,軽量な畳み込みニューラルネットワーク(CNN)と Vision Transformer(ViT)の画像分類性能を,CIFAR-10 を対象に公平に比較する実験プロトコルを提示する。学習率スケジューリング,データ拡張,Test-Time Augmentation (TTA) を統一的に適用し,再現性を重視した評価を実施した。実験の結果,ViT-Ti は TTA により最大で +1.8pt の精度向上を得た一方,ResNet-18 では RandAugment を中心としたデータ拡張で +2.1pt の改善が確認された。これらの知見を通じて,限られた計算資源でも堅牢なベースラインを構築するための指針を示す。
\end{abstract}

\begin{IEEEkeywords}
画像分類, 深層学習, Vision Transformer, Test-Time Augmentation, CIFAR-10
\end{IEEEkeywords}

\input{01_introduction/main}
\input{02_related_work/main}
\input{03_method/main}
\input{04_results/main}
\input{05_discussion/main}
\input{06_conclusion/main}
\input{appendix/main}

\bibliographystyle{ieeetr}
\bibliography{references}

\end{document}

% !TEX program = xelatex
\documentclass[conference]{IEEEtran}

% 日本語出力と XeLaTeX 対応設定
\usepackage{xeCJK}
\usepackage{fontspec}
\setCJKmainfont{HaranoAjiMincho}

% 数式・図表・参考文献などの標準パッケージ
\usepackage{amsmath, amssymb}
\usepackage{graphicx}
\usepackage{url}
\usepackage{hyperref}

\hypersetup{
  colorlinks=true,
  linkcolor=blue,
  citecolor=blue,
  urlcolor=blue
}

\begin{document}

\title{軽量 CNN と Vision Transformer の公平比較に向けた実験的評価}

\author{\IEEEauthorblockN{著者 太郎\\ }
\IEEEauthorblockA{所属機関\\ 連絡先: author@example.com}}

\maketitle

\begin{abstract}
本稿では,軽量な畳み込みニューラルネットワーク(CNN)と Vision Transformer(ViT)の画像分類性能を,CIFAR-10 を対象に公平に比較する実験プロトコルを提示する。学習率スケジューリング,データ拡張,Test-Time Augmentation (TTA) を統一的に適用し,再現性を重視した評価を実施した。実験の結果,ViT-Ti は TTA により最大で +1.8pt の精度向上を得た一方,ResNet-18 では RandAugment を中心としたデータ拡張で +2.1pt の改善が確認された。これらの知見を通じて,限られた計算資源でも堅牢なベースラインを構築するための指針を示す。
\end{abstract}

\begin{IEEEkeywords}
画像分類, 深層学習, Vision Transformer, Test-Time Augmentation, CIFAR-10
\end{IEEEkeywords}

\input{01_introduction/main}
\input{02_related_work/main}
\input{03_method/main}
\input{04_results/main}
\input{05_discussion/main}
\input{06_conclusion/main}
\input{appendix/main}

\bibliographystyle{ieeetr}
\bibliography{references}

\end{document}

% !TEX program = xelatex
\documentclass[conference]{IEEEtran}

% 日本語出力と XeLaTeX 対応設定
\usepackage{xeCJK}
\usepackage{fontspec}
\setCJKmainfont{HaranoAjiMincho}

% 数式・図表・参考文献などの標準パッケージ
\usepackage{amsmath, amssymb}
\usepackage{graphicx}
\usepackage{url}
\usepackage{hyperref}

\hypersetup{
  colorlinks=true,
  linkcolor=blue,
  citecolor=blue,
  urlcolor=blue
}

\begin{document}

\title{軽量 CNN と Vision Transformer の公平比較に向けた実験的評価}

\author{\IEEEauthorblockN{著者 太郎\\ }
\IEEEauthorblockA{所属機関\\ 連絡先: author@example.com}}

\maketitle

\begin{abstract}
本稿では,軽量な畳み込みニューラルネットワーク(CNN)と Vision Transformer(ViT)の画像分類性能を,CIFAR-10 を対象に公平に比較する実験プロトコルを提示する。学習率スケジューリング,データ拡張,Test-Time Augmentation (TTA) を統一的に適用し,再現性を重視した評価を実施した。実験の結果,ViT-Ti は TTA により最大で +1.8pt の精度向上を得た一方,ResNet-18 では RandAugment を中心としたデータ拡張で +2.1pt の改善が確認された。これらの知見を通じて,限られた計算資源でも堅牢なベースラインを構築するための指針を示す。
\end{abstract}

\begin{IEEEkeywords}
画像分類, 深層学習, Vision Transformer, Test-Time Augmentation, CIFAR-10
\end{IEEEkeywords}

\input{01_introduction/main}
\input{02_related_work/main}
\input{03_method/main}
\input{04_results/main}
\input{05_discussion/main}
\input{06_conclusion/main}
\input{appendix/main}

\bibliographystyle{ieeetr}
\bibliography{references}

\end{document}


\bibliographystyle{ieeetr}
\bibliography{references}

\end{document}


\bibliographystyle{ieeetr}
\bibliography{references}

\end{document}

% !TEX program = xelatex
\documentclass[conference]{IEEEtran}

% 日本語出力と XeLaTeX 対応設定
\usepackage{xeCJK}
\usepackage{fontspec}
\setCJKmainfont{HaranoAjiMincho}

% 数式・図表・参考文献などの標準パッケージ
\usepackage{amsmath, amssymb}
\usepackage{graphicx}
\usepackage{url}
\usepackage{hyperref}

\hypersetup{
  colorlinks=true,
  linkcolor=blue,
  citecolor=blue,
  urlcolor=blue
}

\begin{document}

\title{軽量 CNN と Vision Transformer の公平比較に向けた実験的評価}

\author{\IEEEauthorblockN{著者 太郎\\ }
\IEEEauthorblockA{所属機関\\ 連絡先: author@example.com}}

\maketitle

\begin{abstract}
本稿では,軽量な畳み込みニューラルネットワーク(CNN)と Vision Transformer(ViT)の画像分類性能を,CIFAR-10 を対象に公平に比較する実験プロトコルを提示する。学習率スケジューリング,データ拡張,Test-Time Augmentation (TTA) を統一的に適用し,再現性を重視した評価を実施した。実験の結果,ViT-Ti は TTA により最大で +1.8pt の精度向上を得た一方,ResNet-18 では RandAugment を中心としたデータ拡張で +2.1pt の改善が確認された。これらの知見を通じて,限られた計算資源でも堅牢なベースラインを構築するための指針を示す。
\end{abstract}

\begin{IEEEkeywords}
画像分類, 深層学習, Vision Transformer, Test-Time Augmentation, CIFAR-10
\end{IEEEkeywords}

% !TEX program = xelatex
\documentclass[conference]{IEEEtran}

% 日本語出力と XeLaTeX 対応設定
\usepackage{xeCJK}
\usepackage{fontspec}
\setCJKmainfont{HaranoAjiMincho}

% 数式・図表・参考文献などの標準パッケージ
\usepackage{amsmath, amssymb}
\usepackage{graphicx}
\usepackage{url}
\usepackage{hyperref}

\hypersetup{
  colorlinks=true,
  linkcolor=blue,
  citecolor=blue,
  urlcolor=blue
}

\begin{document}

\title{軽量 CNN と Vision Transformer の公平比較に向けた実験的評価}

\author{\IEEEauthorblockN{著者 太郎\\ }
\IEEEauthorblockA{所属機関\\ 連絡先: author@example.com}}

\maketitle

\begin{abstract}
本稿では,軽量な畳み込みニューラルネットワーク(CNN)と Vision Transformer(ViT)の画像分類性能を,CIFAR-10 を対象に公平に比較する実験プロトコルを提示する。学習率スケジューリング,データ拡張,Test-Time Augmentation (TTA) を統一的に適用し,再現性を重視した評価を実施した。実験の結果,ViT-Ti は TTA により最大で +1.8pt の精度向上を得た一方,ResNet-18 では RandAugment を中心としたデータ拡張で +2.1pt の改善が確認された。これらの知見を通じて,限られた計算資源でも堅牢なベースラインを構築するための指針を示す。
\end{abstract}

\begin{IEEEkeywords}
画像分類, 深層学習, Vision Transformer, Test-Time Augmentation, CIFAR-10
\end{IEEEkeywords}

% !TEX program = xelatex
\documentclass[conference]{IEEEtran}

% 日本語出力と XeLaTeX 対応設定
\usepackage{xeCJK}
\usepackage{fontspec}
\setCJKmainfont{HaranoAjiMincho}

% 数式・図表・参考文献などの標準パッケージ
\usepackage{amsmath, amssymb}
\usepackage{graphicx}
\usepackage{url}
\usepackage{hyperref}

\hypersetup{
  colorlinks=true,
  linkcolor=blue,
  citecolor=blue,
  urlcolor=blue
}

\begin{document}

\title{軽量 CNN と Vision Transformer の公平比較に向けた実験的評価}

\author{\IEEEauthorblockN{著者 太郎\\ }
\IEEEauthorblockA{所属機関\\ 連絡先: author@example.com}}

\maketitle

\begin{abstract}
本稿では,軽量な畳み込みニューラルネットワーク(CNN)と Vision Transformer(ViT)の画像分類性能を,CIFAR-10 を対象に公平に比較する実験プロトコルを提示する。学習率スケジューリング,データ拡張,Test-Time Augmentation (TTA) を統一的に適用し,再現性を重視した評価を実施した。実験の結果,ViT-Ti は TTA により最大で +1.8pt の精度向上を得た一方,ResNet-18 では RandAugment を中心としたデータ拡張で +2.1pt の改善が確認された。これらの知見を通じて,限られた計算資源でも堅牢なベースラインを構築するための指針を示す。
\end{abstract}

\begin{IEEEkeywords}
画像分類, 深層学習, Vision Transformer, Test-Time Augmentation, CIFAR-10
\end{IEEEkeywords}

\input{01_introduction/main}
\input{02_related_work/main}
\input{03_method/main}
\input{04_results/main}
\input{05_discussion/main}
\input{06_conclusion/main}
\input{appendix/main}

\bibliographystyle{ieeetr}
\bibliography{references}

\end{document}

% !TEX program = xelatex
\documentclass[conference]{IEEEtran}

% 日本語出力と XeLaTeX 対応設定
\usepackage{xeCJK}
\usepackage{fontspec}
\setCJKmainfont{HaranoAjiMincho}

% 数式・図表・参考文献などの標準パッケージ
\usepackage{amsmath, amssymb}
\usepackage{graphicx}
\usepackage{url}
\usepackage{hyperref}

\hypersetup{
  colorlinks=true,
  linkcolor=blue,
  citecolor=blue,
  urlcolor=blue
}

\begin{document}

\title{軽量 CNN と Vision Transformer の公平比較に向けた実験的評価}

\author{\IEEEauthorblockN{著者 太郎\\ }
\IEEEauthorblockA{所属機関\\ 連絡先: author@example.com}}

\maketitle

\begin{abstract}
本稿では,軽量な畳み込みニューラルネットワーク(CNN)と Vision Transformer(ViT)の画像分類性能を,CIFAR-10 を対象に公平に比較する実験プロトコルを提示する。学習率スケジューリング,データ拡張,Test-Time Augmentation (TTA) を統一的に適用し,再現性を重視した評価を実施した。実験の結果,ViT-Ti は TTA により最大で +1.8pt の精度向上を得た一方,ResNet-18 では RandAugment を中心としたデータ拡張で +2.1pt の改善が確認された。これらの知見を通じて,限られた計算資源でも堅牢なベースラインを構築するための指針を示す。
\end{abstract}

\begin{IEEEkeywords}
画像分類, 深層学習, Vision Transformer, Test-Time Augmentation, CIFAR-10
\end{IEEEkeywords}

\input{01_introduction/main}
\input{02_related_work/main}
\input{03_method/main}
\input{04_results/main}
\input{05_discussion/main}
\input{06_conclusion/main}
\input{appendix/main}

\bibliographystyle{ieeetr}
\bibliography{references}

\end{document}

% !TEX program = xelatex
\documentclass[conference]{IEEEtran}

% 日本語出力と XeLaTeX 対応設定
\usepackage{xeCJK}
\usepackage{fontspec}
\setCJKmainfont{HaranoAjiMincho}

% 数式・図表・参考文献などの標準パッケージ
\usepackage{amsmath, amssymb}
\usepackage{graphicx}
\usepackage{url}
\usepackage{hyperref}

\hypersetup{
  colorlinks=true,
  linkcolor=blue,
  citecolor=blue,
  urlcolor=blue
}

\begin{document}

\title{軽量 CNN と Vision Transformer の公平比較に向けた実験的評価}

\author{\IEEEauthorblockN{著者 太郎\\ }
\IEEEauthorblockA{所属機関\\ 連絡先: author@example.com}}

\maketitle

\begin{abstract}
本稿では,軽量な畳み込みニューラルネットワーク(CNN)と Vision Transformer(ViT)の画像分類性能を,CIFAR-10 を対象に公平に比較する実験プロトコルを提示する。学習率スケジューリング,データ拡張,Test-Time Augmentation (TTA) を統一的に適用し,再現性を重視した評価を実施した。実験の結果,ViT-Ti は TTA により最大で +1.8pt の精度向上を得た一方,ResNet-18 では RandAugment を中心としたデータ拡張で +2.1pt の改善が確認された。これらの知見を通じて,限られた計算資源でも堅牢なベースラインを構築するための指針を示す。
\end{abstract}

\begin{IEEEkeywords}
画像分類, 深層学習, Vision Transformer, Test-Time Augmentation, CIFAR-10
\end{IEEEkeywords}

\input{01_introduction/main}
\input{02_related_work/main}
\input{03_method/main}
\input{04_results/main}
\input{05_discussion/main}
\input{06_conclusion/main}
\input{appendix/main}

\bibliographystyle{ieeetr}
\bibliography{references}

\end{document}

% !TEX program = xelatex
\documentclass[conference]{IEEEtran}

% 日本語出力と XeLaTeX 対応設定
\usepackage{xeCJK}
\usepackage{fontspec}
\setCJKmainfont{HaranoAjiMincho}

% 数式・図表・参考文献などの標準パッケージ
\usepackage{amsmath, amssymb}
\usepackage{graphicx}
\usepackage{url}
\usepackage{hyperref}

\hypersetup{
  colorlinks=true,
  linkcolor=blue,
  citecolor=blue,
  urlcolor=blue
}

\begin{document}

\title{軽量 CNN と Vision Transformer の公平比較に向けた実験的評価}

\author{\IEEEauthorblockN{著者 太郎\\ }
\IEEEauthorblockA{所属機関\\ 連絡先: author@example.com}}

\maketitle

\begin{abstract}
本稿では,軽量な畳み込みニューラルネットワーク(CNN)と Vision Transformer(ViT)の画像分類性能を,CIFAR-10 を対象に公平に比較する実験プロトコルを提示する。学習率スケジューリング,データ拡張,Test-Time Augmentation (TTA) を統一的に適用し,再現性を重視した評価を実施した。実験の結果,ViT-Ti は TTA により最大で +1.8pt の精度向上を得た一方,ResNet-18 では RandAugment を中心としたデータ拡張で +2.1pt の改善が確認された。これらの知見を通じて,限られた計算資源でも堅牢なベースラインを構築するための指針を示す。
\end{abstract}

\begin{IEEEkeywords}
画像分類, 深層学習, Vision Transformer, Test-Time Augmentation, CIFAR-10
\end{IEEEkeywords}

\input{01_introduction/main}
\input{02_related_work/main}
\input{03_method/main}
\input{04_results/main}
\input{05_discussion/main}
\input{06_conclusion/main}
\input{appendix/main}

\bibliographystyle{ieeetr}
\bibliography{references}

\end{document}

% !TEX program = xelatex
\documentclass[conference]{IEEEtran}

% 日本語出力と XeLaTeX 対応設定
\usepackage{xeCJK}
\usepackage{fontspec}
\setCJKmainfont{HaranoAjiMincho}

% 数式・図表・参考文献などの標準パッケージ
\usepackage{amsmath, amssymb}
\usepackage{graphicx}
\usepackage{url}
\usepackage{hyperref}

\hypersetup{
  colorlinks=true,
  linkcolor=blue,
  citecolor=blue,
  urlcolor=blue
}

\begin{document}

\title{軽量 CNN と Vision Transformer の公平比較に向けた実験的評価}

\author{\IEEEauthorblockN{著者 太郎\\ }
\IEEEauthorblockA{所属機関\\ 連絡先: author@example.com}}

\maketitle

\begin{abstract}
本稿では,軽量な畳み込みニューラルネットワーク(CNN)と Vision Transformer(ViT)の画像分類性能を,CIFAR-10 を対象に公平に比較する実験プロトコルを提示する。学習率スケジューリング,データ拡張,Test-Time Augmentation (TTA) を統一的に適用し,再現性を重視した評価を実施した。実験の結果,ViT-Ti は TTA により最大で +1.8pt の精度向上を得た一方,ResNet-18 では RandAugment を中心としたデータ拡張で +2.1pt の改善が確認された。これらの知見を通じて,限られた計算資源でも堅牢なベースラインを構築するための指針を示す。
\end{abstract}

\begin{IEEEkeywords}
画像分類, 深層学習, Vision Transformer, Test-Time Augmentation, CIFAR-10
\end{IEEEkeywords}

\input{01_introduction/main}
\input{02_related_work/main}
\input{03_method/main}
\input{04_results/main}
\input{05_discussion/main}
\input{06_conclusion/main}
\input{appendix/main}

\bibliographystyle{ieeetr}
\bibliography{references}

\end{document}

% !TEX program = xelatex
\documentclass[conference]{IEEEtran}

% 日本語出力と XeLaTeX 対応設定
\usepackage{xeCJK}
\usepackage{fontspec}
\setCJKmainfont{HaranoAjiMincho}

% 数式・図表・参考文献などの標準パッケージ
\usepackage{amsmath, amssymb}
\usepackage{graphicx}
\usepackage{url}
\usepackage{hyperref}

\hypersetup{
  colorlinks=true,
  linkcolor=blue,
  citecolor=blue,
  urlcolor=blue
}

\begin{document}

\title{軽量 CNN と Vision Transformer の公平比較に向けた実験的評価}

\author{\IEEEauthorblockN{著者 太郎\\ }
\IEEEauthorblockA{所属機関\\ 連絡先: author@example.com}}

\maketitle

\begin{abstract}
本稿では,軽量な畳み込みニューラルネットワーク(CNN)と Vision Transformer(ViT)の画像分類性能を,CIFAR-10 を対象に公平に比較する実験プロトコルを提示する。学習率スケジューリング,データ拡張,Test-Time Augmentation (TTA) を統一的に適用し,再現性を重視した評価を実施した。実験の結果,ViT-Ti は TTA により最大で +1.8pt の精度向上を得た一方,ResNet-18 では RandAugment を中心としたデータ拡張で +2.1pt の改善が確認された。これらの知見を通じて,限られた計算資源でも堅牢なベースラインを構築するための指針を示す。
\end{abstract}

\begin{IEEEkeywords}
画像分類, 深層学習, Vision Transformer, Test-Time Augmentation, CIFAR-10
\end{IEEEkeywords}

\input{01_introduction/main}
\input{02_related_work/main}
\input{03_method/main}
\input{04_results/main}
\input{05_discussion/main}
\input{06_conclusion/main}
\input{appendix/main}

\bibliographystyle{ieeetr}
\bibliography{references}

\end{document}

% !TEX program = xelatex
\documentclass[conference]{IEEEtran}

% 日本語出力と XeLaTeX 対応設定
\usepackage{xeCJK}
\usepackage{fontspec}
\setCJKmainfont{HaranoAjiMincho}

% 数式・図表・参考文献などの標準パッケージ
\usepackage{amsmath, amssymb}
\usepackage{graphicx}
\usepackage{url}
\usepackage{hyperref}

\hypersetup{
  colorlinks=true,
  linkcolor=blue,
  citecolor=blue,
  urlcolor=blue
}

\begin{document}

\title{軽量 CNN と Vision Transformer の公平比較に向けた実験的評価}

\author{\IEEEauthorblockN{著者 太郎\\ }
\IEEEauthorblockA{所属機関\\ 連絡先: author@example.com}}

\maketitle

\begin{abstract}
本稿では,軽量な畳み込みニューラルネットワーク(CNN)と Vision Transformer(ViT)の画像分類性能を,CIFAR-10 を対象に公平に比較する実験プロトコルを提示する。学習率スケジューリング,データ拡張,Test-Time Augmentation (TTA) を統一的に適用し,再現性を重視した評価を実施した。実験の結果,ViT-Ti は TTA により最大で +1.8pt の精度向上を得た一方,ResNet-18 では RandAugment を中心としたデータ拡張で +2.1pt の改善が確認された。これらの知見を通じて,限られた計算資源でも堅牢なベースラインを構築するための指針を示す。
\end{abstract}

\begin{IEEEkeywords}
画像分類, 深層学習, Vision Transformer, Test-Time Augmentation, CIFAR-10
\end{IEEEkeywords}

\input{01_introduction/main}
\input{02_related_work/main}
\input{03_method/main}
\input{04_results/main}
\input{05_discussion/main}
\input{06_conclusion/main}
\input{appendix/main}

\bibliographystyle{ieeetr}
\bibliography{references}

\end{document}


\bibliographystyle{ieeetr}
\bibliography{references}

\end{document}

% !TEX program = xelatex
\documentclass[conference]{IEEEtran}

% 日本語出力と XeLaTeX 対応設定
\usepackage{xeCJK}
\usepackage{fontspec}
\setCJKmainfont{HaranoAjiMincho}

% 数式・図表・参考文献などの標準パッケージ
\usepackage{amsmath, amssymb}
\usepackage{graphicx}
\usepackage{url}
\usepackage{hyperref}

\hypersetup{
  colorlinks=true,
  linkcolor=blue,
  citecolor=blue,
  urlcolor=blue
}

\begin{document}

\title{軽量 CNN と Vision Transformer の公平比較に向けた実験的評価}

\author{\IEEEauthorblockN{著者 太郎\\ }
\IEEEauthorblockA{所属機関\\ 連絡先: author@example.com}}

\maketitle

\begin{abstract}
本稿では,軽量な畳み込みニューラルネットワーク(CNN)と Vision Transformer(ViT)の画像分類性能を,CIFAR-10 を対象に公平に比較する実験プロトコルを提示する。学習率スケジューリング,データ拡張,Test-Time Augmentation (TTA) を統一的に適用し,再現性を重視した評価を実施した。実験の結果,ViT-Ti は TTA により最大で +1.8pt の精度向上を得た一方,ResNet-18 では RandAugment を中心としたデータ拡張で +2.1pt の改善が確認された。これらの知見を通じて,限られた計算資源でも堅牢なベースラインを構築するための指針を示す。
\end{abstract}

\begin{IEEEkeywords}
画像分類, 深層学習, Vision Transformer, Test-Time Augmentation, CIFAR-10
\end{IEEEkeywords}

% !TEX program = xelatex
\documentclass[conference]{IEEEtran}

% 日本語出力と XeLaTeX 対応設定
\usepackage{xeCJK}
\usepackage{fontspec}
\setCJKmainfont{HaranoAjiMincho}

% 数式・図表・参考文献などの標準パッケージ
\usepackage{amsmath, amssymb}
\usepackage{graphicx}
\usepackage{url}
\usepackage{hyperref}

\hypersetup{
  colorlinks=true,
  linkcolor=blue,
  citecolor=blue,
  urlcolor=blue
}

\begin{document}

\title{軽量 CNN と Vision Transformer の公平比較に向けた実験的評価}

\author{\IEEEauthorblockN{著者 太郎\\ }
\IEEEauthorblockA{所属機関\\ 連絡先: author@example.com}}

\maketitle

\begin{abstract}
本稿では,軽量な畳み込みニューラルネットワーク(CNN)と Vision Transformer(ViT)の画像分類性能を,CIFAR-10 を対象に公平に比較する実験プロトコルを提示する。学習率スケジューリング,データ拡張,Test-Time Augmentation (TTA) を統一的に適用し,再現性を重視した評価を実施した。実験の結果,ViT-Ti は TTA により最大で +1.8pt の精度向上を得た一方,ResNet-18 では RandAugment を中心としたデータ拡張で +2.1pt の改善が確認された。これらの知見を通じて,限られた計算資源でも堅牢なベースラインを構築するための指針を示す。
\end{abstract}

\begin{IEEEkeywords}
画像分類, 深層学習, Vision Transformer, Test-Time Augmentation, CIFAR-10
\end{IEEEkeywords}

\input{01_introduction/main}
\input{02_related_work/main}
\input{03_method/main}
\input{04_results/main}
\input{05_discussion/main}
\input{06_conclusion/main}
\input{appendix/main}

\bibliographystyle{ieeetr}
\bibliography{references}

\end{document}

% !TEX program = xelatex
\documentclass[conference]{IEEEtran}

% 日本語出力と XeLaTeX 対応設定
\usepackage{xeCJK}
\usepackage{fontspec}
\setCJKmainfont{HaranoAjiMincho}

% 数式・図表・参考文献などの標準パッケージ
\usepackage{amsmath, amssymb}
\usepackage{graphicx}
\usepackage{url}
\usepackage{hyperref}

\hypersetup{
  colorlinks=true,
  linkcolor=blue,
  citecolor=blue,
  urlcolor=blue
}

\begin{document}

\title{軽量 CNN と Vision Transformer の公平比較に向けた実験的評価}

\author{\IEEEauthorblockN{著者 太郎\\ }
\IEEEauthorblockA{所属機関\\ 連絡先: author@example.com}}

\maketitle

\begin{abstract}
本稿では,軽量な畳み込みニューラルネットワーク(CNN)と Vision Transformer(ViT)の画像分類性能を,CIFAR-10 を対象に公平に比較する実験プロトコルを提示する。学習率スケジューリング,データ拡張,Test-Time Augmentation (TTA) を統一的に適用し,再現性を重視した評価を実施した。実験の結果,ViT-Ti は TTA により最大で +1.8pt の精度向上を得た一方,ResNet-18 では RandAugment を中心としたデータ拡張で +2.1pt の改善が確認された。これらの知見を通じて,限られた計算資源でも堅牢なベースラインを構築するための指針を示す。
\end{abstract}

\begin{IEEEkeywords}
画像分類, 深層学習, Vision Transformer, Test-Time Augmentation, CIFAR-10
\end{IEEEkeywords}

\input{01_introduction/main}
\input{02_related_work/main}
\input{03_method/main}
\input{04_results/main}
\input{05_discussion/main}
\input{06_conclusion/main}
\input{appendix/main}

\bibliographystyle{ieeetr}
\bibliography{references}

\end{document}

% !TEX program = xelatex
\documentclass[conference]{IEEEtran}

% 日本語出力と XeLaTeX 対応設定
\usepackage{xeCJK}
\usepackage{fontspec}
\setCJKmainfont{HaranoAjiMincho}

% 数式・図表・参考文献などの標準パッケージ
\usepackage{amsmath, amssymb}
\usepackage{graphicx}
\usepackage{url}
\usepackage{hyperref}

\hypersetup{
  colorlinks=true,
  linkcolor=blue,
  citecolor=blue,
  urlcolor=blue
}

\begin{document}

\title{軽量 CNN と Vision Transformer の公平比較に向けた実験的評価}

\author{\IEEEauthorblockN{著者 太郎\\ }
\IEEEauthorblockA{所属機関\\ 連絡先: author@example.com}}

\maketitle

\begin{abstract}
本稿では,軽量な畳み込みニューラルネットワーク(CNN)と Vision Transformer(ViT)の画像分類性能を,CIFAR-10 を対象に公平に比較する実験プロトコルを提示する。学習率スケジューリング,データ拡張,Test-Time Augmentation (TTA) を統一的に適用し,再現性を重視した評価を実施した。実験の結果,ViT-Ti は TTA により最大で +1.8pt の精度向上を得た一方,ResNet-18 では RandAugment を中心としたデータ拡張で +2.1pt の改善が確認された。これらの知見を通じて,限られた計算資源でも堅牢なベースラインを構築するための指針を示す。
\end{abstract}

\begin{IEEEkeywords}
画像分類, 深層学習, Vision Transformer, Test-Time Augmentation, CIFAR-10
\end{IEEEkeywords}

\input{01_introduction/main}
\input{02_related_work/main}
\input{03_method/main}
\input{04_results/main}
\input{05_discussion/main}
\input{06_conclusion/main}
\input{appendix/main}

\bibliographystyle{ieeetr}
\bibliography{references}

\end{document}

% !TEX program = xelatex
\documentclass[conference]{IEEEtran}

% 日本語出力と XeLaTeX 対応設定
\usepackage{xeCJK}
\usepackage{fontspec}
\setCJKmainfont{HaranoAjiMincho}

% 数式・図表・参考文献などの標準パッケージ
\usepackage{amsmath, amssymb}
\usepackage{graphicx}
\usepackage{url}
\usepackage{hyperref}

\hypersetup{
  colorlinks=true,
  linkcolor=blue,
  citecolor=blue,
  urlcolor=blue
}

\begin{document}

\title{軽量 CNN と Vision Transformer の公平比較に向けた実験的評価}

\author{\IEEEauthorblockN{著者 太郎\\ }
\IEEEauthorblockA{所属機関\\ 連絡先: author@example.com}}

\maketitle

\begin{abstract}
本稿では,軽量な畳み込みニューラルネットワーク(CNN)と Vision Transformer(ViT)の画像分類性能を,CIFAR-10 を対象に公平に比較する実験プロトコルを提示する。学習率スケジューリング,データ拡張,Test-Time Augmentation (TTA) を統一的に適用し,再現性を重視した評価を実施した。実験の結果,ViT-Ti は TTA により最大で +1.8pt の精度向上を得た一方,ResNet-18 では RandAugment を中心としたデータ拡張で +2.1pt の改善が確認された。これらの知見を通じて,限られた計算資源でも堅牢なベースラインを構築するための指針を示す。
\end{abstract}

\begin{IEEEkeywords}
画像分類, 深層学習, Vision Transformer, Test-Time Augmentation, CIFAR-10
\end{IEEEkeywords}

\input{01_introduction/main}
\input{02_related_work/main}
\input{03_method/main}
\input{04_results/main}
\input{05_discussion/main}
\input{06_conclusion/main}
\input{appendix/main}

\bibliographystyle{ieeetr}
\bibliography{references}

\end{document}

% !TEX program = xelatex
\documentclass[conference]{IEEEtran}

% 日本語出力と XeLaTeX 対応設定
\usepackage{xeCJK}
\usepackage{fontspec}
\setCJKmainfont{HaranoAjiMincho}

% 数式・図表・参考文献などの標準パッケージ
\usepackage{amsmath, amssymb}
\usepackage{graphicx}
\usepackage{url}
\usepackage{hyperref}

\hypersetup{
  colorlinks=true,
  linkcolor=blue,
  citecolor=blue,
  urlcolor=blue
}

\begin{document}

\title{軽量 CNN と Vision Transformer の公平比較に向けた実験的評価}

\author{\IEEEauthorblockN{著者 太郎\\ }
\IEEEauthorblockA{所属機関\\ 連絡先: author@example.com}}

\maketitle

\begin{abstract}
本稿では,軽量な畳み込みニューラルネットワーク(CNN)と Vision Transformer(ViT)の画像分類性能を,CIFAR-10 を対象に公平に比較する実験プロトコルを提示する。学習率スケジューリング,データ拡張,Test-Time Augmentation (TTA) を統一的に適用し,再現性を重視した評価を実施した。実験の結果,ViT-Ti は TTA により最大で +1.8pt の精度向上を得た一方,ResNet-18 では RandAugment を中心としたデータ拡張で +2.1pt の改善が確認された。これらの知見を通じて,限られた計算資源でも堅牢なベースラインを構築するための指針を示す。
\end{abstract}

\begin{IEEEkeywords}
画像分類, 深層学習, Vision Transformer, Test-Time Augmentation, CIFAR-10
\end{IEEEkeywords}

\input{01_introduction/main}
\input{02_related_work/main}
\input{03_method/main}
\input{04_results/main}
\input{05_discussion/main}
\input{06_conclusion/main}
\input{appendix/main}

\bibliographystyle{ieeetr}
\bibliography{references}

\end{document}

% !TEX program = xelatex
\documentclass[conference]{IEEEtran}

% 日本語出力と XeLaTeX 対応設定
\usepackage{xeCJK}
\usepackage{fontspec}
\setCJKmainfont{HaranoAjiMincho}

% 数式・図表・参考文献などの標準パッケージ
\usepackage{amsmath, amssymb}
\usepackage{graphicx}
\usepackage{url}
\usepackage{hyperref}

\hypersetup{
  colorlinks=true,
  linkcolor=blue,
  citecolor=blue,
  urlcolor=blue
}

\begin{document}

\title{軽量 CNN と Vision Transformer の公平比較に向けた実験的評価}

\author{\IEEEauthorblockN{著者 太郎\\ }
\IEEEauthorblockA{所属機関\\ 連絡先: author@example.com}}

\maketitle

\begin{abstract}
本稿では,軽量な畳み込みニューラルネットワーク(CNN)と Vision Transformer(ViT)の画像分類性能を,CIFAR-10 を対象に公平に比較する実験プロトコルを提示する。学習率スケジューリング,データ拡張,Test-Time Augmentation (TTA) を統一的に適用し,再現性を重視した評価を実施した。実験の結果,ViT-Ti は TTA により最大で +1.8pt の精度向上を得た一方,ResNet-18 では RandAugment を中心としたデータ拡張で +2.1pt の改善が確認された。これらの知見を通じて,限られた計算資源でも堅牢なベースラインを構築するための指針を示す。
\end{abstract}

\begin{IEEEkeywords}
画像分類, 深層学習, Vision Transformer, Test-Time Augmentation, CIFAR-10
\end{IEEEkeywords}

\input{01_introduction/main}
\input{02_related_work/main}
\input{03_method/main}
\input{04_results/main}
\input{05_discussion/main}
\input{06_conclusion/main}
\input{appendix/main}

\bibliographystyle{ieeetr}
\bibliography{references}

\end{document}

% !TEX program = xelatex
\documentclass[conference]{IEEEtran}

% 日本語出力と XeLaTeX 対応設定
\usepackage{xeCJK}
\usepackage{fontspec}
\setCJKmainfont{HaranoAjiMincho}

% 数式・図表・参考文献などの標準パッケージ
\usepackage{amsmath, amssymb}
\usepackage{graphicx}
\usepackage{url}
\usepackage{hyperref}

\hypersetup{
  colorlinks=true,
  linkcolor=blue,
  citecolor=blue,
  urlcolor=blue
}

\begin{document}

\title{軽量 CNN と Vision Transformer の公平比較に向けた実験的評価}

\author{\IEEEauthorblockN{著者 太郎\\ }
\IEEEauthorblockA{所属機関\\ 連絡先: author@example.com}}

\maketitle

\begin{abstract}
本稿では,軽量な畳み込みニューラルネットワーク(CNN)と Vision Transformer(ViT)の画像分類性能を,CIFAR-10 を対象に公平に比較する実験プロトコルを提示する。学習率スケジューリング,データ拡張,Test-Time Augmentation (TTA) を統一的に適用し,再現性を重視した評価を実施した。実験の結果,ViT-Ti は TTA により最大で +1.8pt の精度向上を得た一方,ResNet-18 では RandAugment を中心としたデータ拡張で +2.1pt の改善が確認された。これらの知見を通じて,限られた計算資源でも堅牢なベースラインを構築するための指針を示す。
\end{abstract}

\begin{IEEEkeywords}
画像分類, 深層学習, Vision Transformer, Test-Time Augmentation, CIFAR-10
\end{IEEEkeywords}

\input{01_introduction/main}
\input{02_related_work/main}
\input{03_method/main}
\input{04_results/main}
\input{05_discussion/main}
\input{06_conclusion/main}
\input{appendix/main}

\bibliographystyle{ieeetr}
\bibliography{references}

\end{document}


\bibliographystyle{ieeetr}
\bibliography{references}

\end{document}

% !TEX program = xelatex
\documentclass[conference]{IEEEtran}

% 日本語出力と XeLaTeX 対応設定
\usepackage{xeCJK}
\usepackage{fontspec}
\setCJKmainfont{HaranoAjiMincho}

% 数式・図表・参考文献などの標準パッケージ
\usepackage{amsmath, amssymb}
\usepackage{graphicx}
\usepackage{url}
\usepackage{hyperref}

\hypersetup{
  colorlinks=true,
  linkcolor=blue,
  citecolor=blue,
  urlcolor=blue
}

\begin{document}

\title{軽量 CNN と Vision Transformer の公平比較に向けた実験的評価}

\author{\IEEEauthorblockN{著者 太郎\\ }
\IEEEauthorblockA{所属機関\\ 連絡先: author@example.com}}

\maketitle

\begin{abstract}
本稿では,軽量な畳み込みニューラルネットワーク(CNN)と Vision Transformer(ViT)の画像分類性能を,CIFAR-10 を対象に公平に比較する実験プロトコルを提示する。学習率スケジューリング,データ拡張,Test-Time Augmentation (TTA) を統一的に適用し,再現性を重視した評価を実施した。実験の結果,ViT-Ti は TTA により最大で +1.8pt の精度向上を得た一方,ResNet-18 では RandAugment を中心としたデータ拡張で +2.1pt の改善が確認された。これらの知見を通じて,限られた計算資源でも堅牢なベースラインを構築するための指針を示す。
\end{abstract}

\begin{IEEEkeywords}
画像分類, 深層学習, Vision Transformer, Test-Time Augmentation, CIFAR-10
\end{IEEEkeywords}

% !TEX program = xelatex
\documentclass[conference]{IEEEtran}

% 日本語出力と XeLaTeX 対応設定
\usepackage{xeCJK}
\usepackage{fontspec}
\setCJKmainfont{HaranoAjiMincho}

% 数式・図表・参考文献などの標準パッケージ
\usepackage{amsmath, amssymb}
\usepackage{graphicx}
\usepackage{url}
\usepackage{hyperref}

\hypersetup{
  colorlinks=true,
  linkcolor=blue,
  citecolor=blue,
  urlcolor=blue
}

\begin{document}

\title{軽量 CNN と Vision Transformer の公平比較に向けた実験的評価}

\author{\IEEEauthorblockN{著者 太郎\\ }
\IEEEauthorblockA{所属機関\\ 連絡先: author@example.com}}

\maketitle

\begin{abstract}
本稿では,軽量な畳み込みニューラルネットワーク(CNN)と Vision Transformer(ViT)の画像分類性能を,CIFAR-10 を対象に公平に比較する実験プロトコルを提示する。学習率スケジューリング,データ拡張,Test-Time Augmentation (TTA) を統一的に適用し,再現性を重視した評価を実施した。実験の結果,ViT-Ti は TTA により最大で +1.8pt の精度向上を得た一方,ResNet-18 では RandAugment を中心としたデータ拡張で +2.1pt の改善が確認された。これらの知見を通じて,限られた計算資源でも堅牢なベースラインを構築するための指針を示す。
\end{abstract}

\begin{IEEEkeywords}
画像分類, 深層学習, Vision Transformer, Test-Time Augmentation, CIFAR-10
\end{IEEEkeywords}

\input{01_introduction/main}
\input{02_related_work/main}
\input{03_method/main}
\input{04_results/main}
\input{05_discussion/main}
\input{06_conclusion/main}
\input{appendix/main}

\bibliographystyle{ieeetr}
\bibliography{references}

\end{document}

% !TEX program = xelatex
\documentclass[conference]{IEEEtran}

% 日本語出力と XeLaTeX 対応設定
\usepackage{xeCJK}
\usepackage{fontspec}
\setCJKmainfont{HaranoAjiMincho}

% 数式・図表・参考文献などの標準パッケージ
\usepackage{amsmath, amssymb}
\usepackage{graphicx}
\usepackage{url}
\usepackage{hyperref}

\hypersetup{
  colorlinks=true,
  linkcolor=blue,
  citecolor=blue,
  urlcolor=blue
}

\begin{document}

\title{軽量 CNN と Vision Transformer の公平比較に向けた実験的評価}

\author{\IEEEauthorblockN{著者 太郎\\ }
\IEEEauthorblockA{所属機関\\ 連絡先: author@example.com}}

\maketitle

\begin{abstract}
本稿では,軽量な畳み込みニューラルネットワーク(CNN)と Vision Transformer(ViT)の画像分類性能を,CIFAR-10 を対象に公平に比較する実験プロトコルを提示する。学習率スケジューリング,データ拡張,Test-Time Augmentation (TTA) を統一的に適用し,再現性を重視した評価を実施した。実験の結果,ViT-Ti は TTA により最大で +1.8pt の精度向上を得た一方,ResNet-18 では RandAugment を中心としたデータ拡張で +2.1pt の改善が確認された。これらの知見を通じて,限られた計算資源でも堅牢なベースラインを構築するための指針を示す。
\end{abstract}

\begin{IEEEkeywords}
画像分類, 深層学習, Vision Transformer, Test-Time Augmentation, CIFAR-10
\end{IEEEkeywords}

\input{01_introduction/main}
\input{02_related_work/main}
\input{03_method/main}
\input{04_results/main}
\input{05_discussion/main}
\input{06_conclusion/main}
\input{appendix/main}

\bibliographystyle{ieeetr}
\bibliography{references}

\end{document}

% !TEX program = xelatex
\documentclass[conference]{IEEEtran}

% 日本語出力と XeLaTeX 対応設定
\usepackage{xeCJK}
\usepackage{fontspec}
\setCJKmainfont{HaranoAjiMincho}

% 数式・図表・参考文献などの標準パッケージ
\usepackage{amsmath, amssymb}
\usepackage{graphicx}
\usepackage{url}
\usepackage{hyperref}

\hypersetup{
  colorlinks=true,
  linkcolor=blue,
  citecolor=blue,
  urlcolor=blue
}

\begin{document}

\title{軽量 CNN と Vision Transformer の公平比較に向けた実験的評価}

\author{\IEEEauthorblockN{著者 太郎\\ }
\IEEEauthorblockA{所属機関\\ 連絡先: author@example.com}}

\maketitle

\begin{abstract}
本稿では,軽量な畳み込みニューラルネットワーク(CNN)と Vision Transformer(ViT)の画像分類性能を,CIFAR-10 を対象に公平に比較する実験プロトコルを提示する。学習率スケジューリング,データ拡張,Test-Time Augmentation (TTA) を統一的に適用し,再現性を重視した評価を実施した。実験の結果,ViT-Ti は TTA により最大で +1.8pt の精度向上を得た一方,ResNet-18 では RandAugment を中心としたデータ拡張で +2.1pt の改善が確認された。これらの知見を通じて,限られた計算資源でも堅牢なベースラインを構築するための指針を示す。
\end{abstract}

\begin{IEEEkeywords}
画像分類, 深層学習, Vision Transformer, Test-Time Augmentation, CIFAR-10
\end{IEEEkeywords}

\input{01_introduction/main}
\input{02_related_work/main}
\input{03_method/main}
\input{04_results/main}
\input{05_discussion/main}
\input{06_conclusion/main}
\input{appendix/main}

\bibliographystyle{ieeetr}
\bibliography{references}

\end{document}

% !TEX program = xelatex
\documentclass[conference]{IEEEtran}

% 日本語出力と XeLaTeX 対応設定
\usepackage{xeCJK}
\usepackage{fontspec}
\setCJKmainfont{HaranoAjiMincho}

% 数式・図表・参考文献などの標準パッケージ
\usepackage{amsmath, amssymb}
\usepackage{graphicx}
\usepackage{url}
\usepackage{hyperref}

\hypersetup{
  colorlinks=true,
  linkcolor=blue,
  citecolor=blue,
  urlcolor=blue
}

\begin{document}

\title{軽量 CNN と Vision Transformer の公平比較に向けた実験的評価}

\author{\IEEEauthorblockN{著者 太郎\\ }
\IEEEauthorblockA{所属機関\\ 連絡先: author@example.com}}

\maketitle

\begin{abstract}
本稿では,軽量な畳み込みニューラルネットワーク(CNN)と Vision Transformer(ViT)の画像分類性能を,CIFAR-10 を対象に公平に比較する実験プロトコルを提示する。学習率スケジューリング,データ拡張,Test-Time Augmentation (TTA) を統一的に適用し,再現性を重視した評価を実施した。実験の結果,ViT-Ti は TTA により最大で +1.8pt の精度向上を得た一方,ResNet-18 では RandAugment を中心としたデータ拡張で +2.1pt の改善が確認された。これらの知見を通じて,限られた計算資源でも堅牢なベースラインを構築するための指針を示す。
\end{abstract}

\begin{IEEEkeywords}
画像分類, 深層学習, Vision Transformer, Test-Time Augmentation, CIFAR-10
\end{IEEEkeywords}

\input{01_introduction/main}
\input{02_related_work/main}
\input{03_method/main}
\input{04_results/main}
\input{05_discussion/main}
\input{06_conclusion/main}
\input{appendix/main}

\bibliographystyle{ieeetr}
\bibliography{references}

\end{document}

% !TEX program = xelatex
\documentclass[conference]{IEEEtran}

% 日本語出力と XeLaTeX 対応設定
\usepackage{xeCJK}
\usepackage{fontspec}
\setCJKmainfont{HaranoAjiMincho}

% 数式・図表・参考文献などの標準パッケージ
\usepackage{amsmath, amssymb}
\usepackage{graphicx}
\usepackage{url}
\usepackage{hyperref}

\hypersetup{
  colorlinks=true,
  linkcolor=blue,
  citecolor=blue,
  urlcolor=blue
}

\begin{document}

\title{軽量 CNN と Vision Transformer の公平比較に向けた実験的評価}

\author{\IEEEauthorblockN{著者 太郎\\ }
\IEEEauthorblockA{所属機関\\ 連絡先: author@example.com}}

\maketitle

\begin{abstract}
本稿では,軽量な畳み込みニューラルネットワーク(CNN)と Vision Transformer(ViT)の画像分類性能を,CIFAR-10 を対象に公平に比較する実験プロトコルを提示する。学習率スケジューリング,データ拡張,Test-Time Augmentation (TTA) を統一的に適用し,再現性を重視した評価を実施した。実験の結果,ViT-Ti は TTA により最大で +1.8pt の精度向上を得た一方,ResNet-18 では RandAugment を中心としたデータ拡張で +2.1pt の改善が確認された。これらの知見を通じて,限られた計算資源でも堅牢なベースラインを構築するための指針を示す。
\end{abstract}

\begin{IEEEkeywords}
画像分類, 深層学習, Vision Transformer, Test-Time Augmentation, CIFAR-10
\end{IEEEkeywords}

\input{01_introduction/main}
\input{02_related_work/main}
\input{03_method/main}
\input{04_results/main}
\input{05_discussion/main}
\input{06_conclusion/main}
\input{appendix/main}

\bibliographystyle{ieeetr}
\bibliography{references}

\end{document}

% !TEX program = xelatex
\documentclass[conference]{IEEEtran}

% 日本語出力と XeLaTeX 対応設定
\usepackage{xeCJK}
\usepackage{fontspec}
\setCJKmainfont{HaranoAjiMincho}

% 数式・図表・参考文献などの標準パッケージ
\usepackage{amsmath, amssymb}
\usepackage{graphicx}
\usepackage{url}
\usepackage{hyperref}

\hypersetup{
  colorlinks=true,
  linkcolor=blue,
  citecolor=blue,
  urlcolor=blue
}

\begin{document}

\title{軽量 CNN と Vision Transformer の公平比較に向けた実験的評価}

\author{\IEEEauthorblockN{著者 太郎\\ }
\IEEEauthorblockA{所属機関\\ 連絡先: author@example.com}}

\maketitle

\begin{abstract}
本稿では,軽量な畳み込みニューラルネットワーク(CNN)と Vision Transformer(ViT)の画像分類性能を,CIFAR-10 を対象に公平に比較する実験プロトコルを提示する。学習率スケジューリング,データ拡張,Test-Time Augmentation (TTA) を統一的に適用し,再現性を重視した評価を実施した。実験の結果,ViT-Ti は TTA により最大で +1.8pt の精度向上を得た一方,ResNet-18 では RandAugment を中心としたデータ拡張で +2.1pt の改善が確認された。これらの知見を通じて,限られた計算資源でも堅牢なベースラインを構築するための指針を示す。
\end{abstract}

\begin{IEEEkeywords}
画像分類, 深層学習, Vision Transformer, Test-Time Augmentation, CIFAR-10
\end{IEEEkeywords}

\input{01_introduction/main}
\input{02_related_work/main}
\input{03_method/main}
\input{04_results/main}
\input{05_discussion/main}
\input{06_conclusion/main}
\input{appendix/main}

\bibliographystyle{ieeetr}
\bibliography{references}

\end{document}

% !TEX program = xelatex
\documentclass[conference]{IEEEtran}

% 日本語出力と XeLaTeX 対応設定
\usepackage{xeCJK}
\usepackage{fontspec}
\setCJKmainfont{HaranoAjiMincho}

% 数式・図表・参考文献などの標準パッケージ
\usepackage{amsmath, amssymb}
\usepackage{graphicx}
\usepackage{url}
\usepackage{hyperref}

\hypersetup{
  colorlinks=true,
  linkcolor=blue,
  citecolor=blue,
  urlcolor=blue
}

\begin{document}

\title{軽量 CNN と Vision Transformer の公平比較に向けた実験的評価}

\author{\IEEEauthorblockN{著者 太郎\\ }
\IEEEauthorblockA{所属機関\\ 連絡先: author@example.com}}

\maketitle

\begin{abstract}
本稿では,軽量な畳み込みニューラルネットワーク(CNN)と Vision Transformer(ViT)の画像分類性能を,CIFAR-10 を対象に公平に比較する実験プロトコルを提示する。学習率スケジューリング,データ拡張,Test-Time Augmentation (TTA) を統一的に適用し,再現性を重視した評価を実施した。実験の結果,ViT-Ti は TTA により最大で +1.8pt の精度向上を得た一方,ResNet-18 では RandAugment を中心としたデータ拡張で +2.1pt の改善が確認された。これらの知見を通じて,限られた計算資源でも堅牢なベースラインを構築するための指針を示す。
\end{abstract}

\begin{IEEEkeywords}
画像分類, 深層学習, Vision Transformer, Test-Time Augmentation, CIFAR-10
\end{IEEEkeywords}

\input{01_introduction/main}
\input{02_related_work/main}
\input{03_method/main}
\input{04_results/main}
\input{05_discussion/main}
\input{06_conclusion/main}
\input{appendix/main}

\bibliographystyle{ieeetr}
\bibliography{references}

\end{document}


\bibliographystyle{ieeetr}
\bibliography{references}

\end{document}

% !TEX program = xelatex
\documentclass[conference]{IEEEtran}

% 日本語出力と XeLaTeX 対応設定
\usepackage{xeCJK}
\usepackage{fontspec}
\setCJKmainfont{HaranoAjiMincho}

% 数式・図表・参考文献などの標準パッケージ
\usepackage{amsmath, amssymb}
\usepackage{graphicx}
\usepackage{url}
\usepackage{hyperref}

\hypersetup{
  colorlinks=true,
  linkcolor=blue,
  citecolor=blue,
  urlcolor=blue
}

\begin{document}

\title{軽量 CNN と Vision Transformer の公平比較に向けた実験的評価}

\author{\IEEEauthorblockN{著者 太郎\\ }
\IEEEauthorblockA{所属機関\\ 連絡先: author@example.com}}

\maketitle

\begin{abstract}
本稿では,軽量な畳み込みニューラルネットワーク(CNN)と Vision Transformer(ViT)の画像分類性能を,CIFAR-10 を対象に公平に比較する実験プロトコルを提示する。学習率スケジューリング,データ拡張,Test-Time Augmentation (TTA) を統一的に適用し,再現性を重視した評価を実施した。実験の結果,ViT-Ti は TTA により最大で +1.8pt の精度向上を得た一方,ResNet-18 では RandAugment を中心としたデータ拡張で +2.1pt の改善が確認された。これらの知見を通じて,限られた計算資源でも堅牢なベースラインを構築するための指針を示す。
\end{abstract}

\begin{IEEEkeywords}
画像分類, 深層学習, Vision Transformer, Test-Time Augmentation, CIFAR-10
\end{IEEEkeywords}

% !TEX program = xelatex
\documentclass[conference]{IEEEtran}

% 日本語出力と XeLaTeX 対応設定
\usepackage{xeCJK}
\usepackage{fontspec}
\setCJKmainfont{HaranoAjiMincho}

% 数式・図表・参考文献などの標準パッケージ
\usepackage{amsmath, amssymb}
\usepackage{graphicx}
\usepackage{url}
\usepackage{hyperref}

\hypersetup{
  colorlinks=true,
  linkcolor=blue,
  citecolor=blue,
  urlcolor=blue
}

\begin{document}

\title{軽量 CNN と Vision Transformer の公平比較に向けた実験的評価}

\author{\IEEEauthorblockN{著者 太郎\\ }
\IEEEauthorblockA{所属機関\\ 連絡先: author@example.com}}

\maketitle

\begin{abstract}
本稿では,軽量な畳み込みニューラルネットワーク(CNN)と Vision Transformer(ViT)の画像分類性能を,CIFAR-10 を対象に公平に比較する実験プロトコルを提示する。学習率スケジューリング,データ拡張,Test-Time Augmentation (TTA) を統一的に適用し,再現性を重視した評価を実施した。実験の結果,ViT-Ti は TTA により最大で +1.8pt の精度向上を得た一方,ResNet-18 では RandAugment を中心としたデータ拡張で +2.1pt の改善が確認された。これらの知見を通じて,限られた計算資源でも堅牢なベースラインを構築するための指針を示す。
\end{abstract}

\begin{IEEEkeywords}
画像分類, 深層学習, Vision Transformer, Test-Time Augmentation, CIFAR-10
\end{IEEEkeywords}

\input{01_introduction/main}
\input{02_related_work/main}
\input{03_method/main}
\input{04_results/main}
\input{05_discussion/main}
\input{06_conclusion/main}
\input{appendix/main}

\bibliographystyle{ieeetr}
\bibliography{references}

\end{document}

% !TEX program = xelatex
\documentclass[conference]{IEEEtran}

% 日本語出力と XeLaTeX 対応設定
\usepackage{xeCJK}
\usepackage{fontspec}
\setCJKmainfont{HaranoAjiMincho}

% 数式・図表・参考文献などの標準パッケージ
\usepackage{amsmath, amssymb}
\usepackage{graphicx}
\usepackage{url}
\usepackage{hyperref}

\hypersetup{
  colorlinks=true,
  linkcolor=blue,
  citecolor=blue,
  urlcolor=blue
}

\begin{document}

\title{軽量 CNN と Vision Transformer の公平比較に向けた実験的評価}

\author{\IEEEauthorblockN{著者 太郎\\ }
\IEEEauthorblockA{所属機関\\ 連絡先: author@example.com}}

\maketitle

\begin{abstract}
本稿では,軽量な畳み込みニューラルネットワーク(CNN)と Vision Transformer(ViT)の画像分類性能を,CIFAR-10 を対象に公平に比較する実験プロトコルを提示する。学習率スケジューリング,データ拡張,Test-Time Augmentation (TTA) を統一的に適用し,再現性を重視した評価を実施した。実験の結果,ViT-Ti は TTA により最大で +1.8pt の精度向上を得た一方,ResNet-18 では RandAugment を中心としたデータ拡張で +2.1pt の改善が確認された。これらの知見を通じて,限られた計算資源でも堅牢なベースラインを構築するための指針を示す。
\end{abstract}

\begin{IEEEkeywords}
画像分類, 深層学習, Vision Transformer, Test-Time Augmentation, CIFAR-10
\end{IEEEkeywords}

\input{01_introduction/main}
\input{02_related_work/main}
\input{03_method/main}
\input{04_results/main}
\input{05_discussion/main}
\input{06_conclusion/main}
\input{appendix/main}

\bibliographystyle{ieeetr}
\bibliography{references}

\end{document}

% !TEX program = xelatex
\documentclass[conference]{IEEEtran}

% 日本語出力と XeLaTeX 対応設定
\usepackage{xeCJK}
\usepackage{fontspec}
\setCJKmainfont{HaranoAjiMincho}

% 数式・図表・参考文献などの標準パッケージ
\usepackage{amsmath, amssymb}
\usepackage{graphicx}
\usepackage{url}
\usepackage{hyperref}

\hypersetup{
  colorlinks=true,
  linkcolor=blue,
  citecolor=blue,
  urlcolor=blue
}

\begin{document}

\title{軽量 CNN と Vision Transformer の公平比較に向けた実験的評価}

\author{\IEEEauthorblockN{著者 太郎\\ }
\IEEEauthorblockA{所属機関\\ 連絡先: author@example.com}}

\maketitle

\begin{abstract}
本稿では,軽量な畳み込みニューラルネットワーク(CNN)と Vision Transformer(ViT)の画像分類性能を,CIFAR-10 を対象に公平に比較する実験プロトコルを提示する。学習率スケジューリング,データ拡張,Test-Time Augmentation (TTA) を統一的に適用し,再現性を重視した評価を実施した。実験の結果,ViT-Ti は TTA により最大で +1.8pt の精度向上を得た一方,ResNet-18 では RandAugment を中心としたデータ拡張で +2.1pt の改善が確認された。これらの知見を通じて,限られた計算資源でも堅牢なベースラインを構築するための指針を示す。
\end{abstract}

\begin{IEEEkeywords}
画像分類, 深層学習, Vision Transformer, Test-Time Augmentation, CIFAR-10
\end{IEEEkeywords}

\input{01_introduction/main}
\input{02_related_work/main}
\input{03_method/main}
\input{04_results/main}
\input{05_discussion/main}
\input{06_conclusion/main}
\input{appendix/main}

\bibliographystyle{ieeetr}
\bibliography{references}

\end{document}

% !TEX program = xelatex
\documentclass[conference]{IEEEtran}

% 日本語出力と XeLaTeX 対応設定
\usepackage{xeCJK}
\usepackage{fontspec}
\setCJKmainfont{HaranoAjiMincho}

% 数式・図表・参考文献などの標準パッケージ
\usepackage{amsmath, amssymb}
\usepackage{graphicx}
\usepackage{url}
\usepackage{hyperref}

\hypersetup{
  colorlinks=true,
  linkcolor=blue,
  citecolor=blue,
  urlcolor=blue
}

\begin{document}

\title{軽量 CNN と Vision Transformer の公平比較に向けた実験的評価}

\author{\IEEEauthorblockN{著者 太郎\\ }
\IEEEauthorblockA{所属機関\\ 連絡先: author@example.com}}

\maketitle

\begin{abstract}
本稿では,軽量な畳み込みニューラルネットワーク(CNN)と Vision Transformer(ViT)の画像分類性能を,CIFAR-10 を対象に公平に比較する実験プロトコルを提示する。学習率スケジューリング,データ拡張,Test-Time Augmentation (TTA) を統一的に適用し,再現性を重視した評価を実施した。実験の結果,ViT-Ti は TTA により最大で +1.8pt の精度向上を得た一方,ResNet-18 では RandAugment を中心としたデータ拡張で +2.1pt の改善が確認された。これらの知見を通じて,限られた計算資源でも堅牢なベースラインを構築するための指針を示す。
\end{abstract}

\begin{IEEEkeywords}
画像分類, 深層学習, Vision Transformer, Test-Time Augmentation, CIFAR-10
\end{IEEEkeywords}

\input{01_introduction/main}
\input{02_related_work/main}
\input{03_method/main}
\input{04_results/main}
\input{05_discussion/main}
\input{06_conclusion/main}
\input{appendix/main}

\bibliographystyle{ieeetr}
\bibliography{references}

\end{document}

% !TEX program = xelatex
\documentclass[conference]{IEEEtran}

% 日本語出力と XeLaTeX 対応設定
\usepackage{xeCJK}
\usepackage{fontspec}
\setCJKmainfont{HaranoAjiMincho}

% 数式・図表・参考文献などの標準パッケージ
\usepackage{amsmath, amssymb}
\usepackage{graphicx}
\usepackage{url}
\usepackage{hyperref}

\hypersetup{
  colorlinks=true,
  linkcolor=blue,
  citecolor=blue,
  urlcolor=blue
}

\begin{document}

\title{軽量 CNN と Vision Transformer の公平比較に向けた実験的評価}

\author{\IEEEauthorblockN{著者 太郎\\ }
\IEEEauthorblockA{所属機関\\ 連絡先: author@example.com}}

\maketitle

\begin{abstract}
本稿では,軽量な畳み込みニューラルネットワーク(CNN)と Vision Transformer(ViT)の画像分類性能を,CIFAR-10 を対象に公平に比較する実験プロトコルを提示する。学習率スケジューリング,データ拡張,Test-Time Augmentation (TTA) を統一的に適用し,再現性を重視した評価を実施した。実験の結果,ViT-Ti は TTA により最大で +1.8pt の精度向上を得た一方,ResNet-18 では RandAugment を中心としたデータ拡張で +2.1pt の改善が確認された。これらの知見を通じて,限られた計算資源でも堅牢なベースラインを構築するための指針を示す。
\end{abstract}

\begin{IEEEkeywords}
画像分類, 深層学習, Vision Transformer, Test-Time Augmentation, CIFAR-10
\end{IEEEkeywords}

\input{01_introduction/main}
\input{02_related_work/main}
\input{03_method/main}
\input{04_results/main}
\input{05_discussion/main}
\input{06_conclusion/main}
\input{appendix/main}

\bibliographystyle{ieeetr}
\bibliography{references}

\end{document}

% !TEX program = xelatex
\documentclass[conference]{IEEEtran}

% 日本語出力と XeLaTeX 対応設定
\usepackage{xeCJK}
\usepackage{fontspec}
\setCJKmainfont{HaranoAjiMincho}

% 数式・図表・参考文献などの標準パッケージ
\usepackage{amsmath, amssymb}
\usepackage{graphicx}
\usepackage{url}
\usepackage{hyperref}

\hypersetup{
  colorlinks=true,
  linkcolor=blue,
  citecolor=blue,
  urlcolor=blue
}

\begin{document}

\title{軽量 CNN と Vision Transformer の公平比較に向けた実験的評価}

\author{\IEEEauthorblockN{著者 太郎\\ }
\IEEEauthorblockA{所属機関\\ 連絡先: author@example.com}}

\maketitle

\begin{abstract}
本稿では,軽量な畳み込みニューラルネットワーク(CNN)と Vision Transformer(ViT)の画像分類性能を,CIFAR-10 を対象に公平に比較する実験プロトコルを提示する。学習率スケジューリング,データ拡張,Test-Time Augmentation (TTA) を統一的に適用し,再現性を重視した評価を実施した。実験の結果,ViT-Ti は TTA により最大で +1.8pt の精度向上を得た一方,ResNet-18 では RandAugment を中心としたデータ拡張で +2.1pt の改善が確認された。これらの知見を通じて,限られた計算資源でも堅牢なベースラインを構築するための指針を示す。
\end{abstract}

\begin{IEEEkeywords}
画像分類, 深層学習, Vision Transformer, Test-Time Augmentation, CIFAR-10
\end{IEEEkeywords}

\input{01_introduction/main}
\input{02_related_work/main}
\input{03_method/main}
\input{04_results/main}
\input{05_discussion/main}
\input{06_conclusion/main}
\input{appendix/main}

\bibliographystyle{ieeetr}
\bibliography{references}

\end{document}

% !TEX program = xelatex
\documentclass[conference]{IEEEtran}

% 日本語出力と XeLaTeX 対応設定
\usepackage{xeCJK}
\usepackage{fontspec}
\setCJKmainfont{HaranoAjiMincho}

% 数式・図表・参考文献などの標準パッケージ
\usepackage{amsmath, amssymb}
\usepackage{graphicx}
\usepackage{url}
\usepackage{hyperref}

\hypersetup{
  colorlinks=true,
  linkcolor=blue,
  citecolor=blue,
  urlcolor=blue
}

\begin{document}

\title{軽量 CNN と Vision Transformer の公平比較に向けた実験的評価}

\author{\IEEEauthorblockN{著者 太郎\\ }
\IEEEauthorblockA{所属機関\\ 連絡先: author@example.com}}

\maketitle

\begin{abstract}
本稿では,軽量な畳み込みニューラルネットワーク(CNN)と Vision Transformer(ViT)の画像分類性能を,CIFAR-10 を対象に公平に比較する実験プロトコルを提示する。学習率スケジューリング,データ拡張,Test-Time Augmentation (TTA) を統一的に適用し,再現性を重視した評価を実施した。実験の結果,ViT-Ti は TTA により最大で +1.8pt の精度向上を得た一方,ResNet-18 では RandAugment を中心としたデータ拡張で +2.1pt の改善が確認された。これらの知見を通じて,限られた計算資源でも堅牢なベースラインを構築するための指針を示す。
\end{abstract}

\begin{IEEEkeywords}
画像分類, 深層学習, Vision Transformer, Test-Time Augmentation, CIFAR-10
\end{IEEEkeywords}

\input{01_introduction/main}
\input{02_related_work/main}
\input{03_method/main}
\input{04_results/main}
\input{05_discussion/main}
\input{06_conclusion/main}
\input{appendix/main}

\bibliographystyle{ieeetr}
\bibliography{references}

\end{document}


\bibliographystyle{ieeetr}
\bibliography{references}

\end{document}

% !TEX program = xelatex
\documentclass[conference]{IEEEtran}

% 日本語出力と XeLaTeX 対応設定
\usepackage{xeCJK}
\usepackage{fontspec}
\setCJKmainfont{HaranoAjiMincho}

% 数式・図表・参考文献などの標準パッケージ
\usepackage{amsmath, amssymb}
\usepackage{graphicx}
\usepackage{url}
\usepackage{hyperref}

\hypersetup{
  colorlinks=true,
  linkcolor=blue,
  citecolor=blue,
  urlcolor=blue
}

\begin{document}

\title{軽量 CNN と Vision Transformer の公平比較に向けた実験的評価}

\author{\IEEEauthorblockN{著者 太郎\\ }
\IEEEauthorblockA{所属機関\\ 連絡先: author@example.com}}

\maketitle

\begin{abstract}
本稿では,軽量な畳み込みニューラルネットワーク(CNN)と Vision Transformer(ViT)の画像分類性能を,CIFAR-10 を対象に公平に比較する実験プロトコルを提示する。学習率スケジューリング,データ拡張,Test-Time Augmentation (TTA) を統一的に適用し,再現性を重視した評価を実施した。実験の結果,ViT-Ti は TTA により最大で +1.8pt の精度向上を得た一方,ResNet-18 では RandAugment を中心としたデータ拡張で +2.1pt の改善が確認された。これらの知見を通じて,限られた計算資源でも堅牢なベースラインを構築するための指針を示す。
\end{abstract}

\begin{IEEEkeywords}
画像分類, 深層学習, Vision Transformer, Test-Time Augmentation, CIFAR-10
\end{IEEEkeywords}

% !TEX program = xelatex
\documentclass[conference]{IEEEtran}

% 日本語出力と XeLaTeX 対応設定
\usepackage{xeCJK}
\usepackage{fontspec}
\setCJKmainfont{HaranoAjiMincho}

% 数式・図表・参考文献などの標準パッケージ
\usepackage{amsmath, amssymb}
\usepackage{graphicx}
\usepackage{url}
\usepackage{hyperref}

\hypersetup{
  colorlinks=true,
  linkcolor=blue,
  citecolor=blue,
  urlcolor=blue
}

\begin{document}

\title{軽量 CNN と Vision Transformer の公平比較に向けた実験的評価}

\author{\IEEEauthorblockN{著者 太郎\\ }
\IEEEauthorblockA{所属機関\\ 連絡先: author@example.com}}

\maketitle

\begin{abstract}
本稿では,軽量な畳み込みニューラルネットワーク(CNN)と Vision Transformer(ViT)の画像分類性能を,CIFAR-10 を対象に公平に比較する実験プロトコルを提示する。学習率スケジューリング,データ拡張,Test-Time Augmentation (TTA) を統一的に適用し,再現性を重視した評価を実施した。実験の結果,ViT-Ti は TTA により最大で +1.8pt の精度向上を得た一方,ResNet-18 では RandAugment を中心としたデータ拡張で +2.1pt の改善が確認された。これらの知見を通じて,限られた計算資源でも堅牢なベースラインを構築するための指針を示す。
\end{abstract}

\begin{IEEEkeywords}
画像分類, 深層学習, Vision Transformer, Test-Time Augmentation, CIFAR-10
\end{IEEEkeywords}

\input{01_introduction/main}
\input{02_related_work/main}
\input{03_method/main}
\input{04_results/main}
\input{05_discussion/main}
\input{06_conclusion/main}
\input{appendix/main}

\bibliographystyle{ieeetr}
\bibliography{references}

\end{document}

% !TEX program = xelatex
\documentclass[conference]{IEEEtran}

% 日本語出力と XeLaTeX 対応設定
\usepackage{xeCJK}
\usepackage{fontspec}
\setCJKmainfont{HaranoAjiMincho}

% 数式・図表・参考文献などの標準パッケージ
\usepackage{amsmath, amssymb}
\usepackage{graphicx}
\usepackage{url}
\usepackage{hyperref}

\hypersetup{
  colorlinks=true,
  linkcolor=blue,
  citecolor=blue,
  urlcolor=blue
}

\begin{document}

\title{軽量 CNN と Vision Transformer の公平比較に向けた実験的評価}

\author{\IEEEauthorblockN{著者 太郎\\ }
\IEEEauthorblockA{所属機関\\ 連絡先: author@example.com}}

\maketitle

\begin{abstract}
本稿では,軽量な畳み込みニューラルネットワーク(CNN)と Vision Transformer(ViT)の画像分類性能を,CIFAR-10 を対象に公平に比較する実験プロトコルを提示する。学習率スケジューリング,データ拡張,Test-Time Augmentation (TTA) を統一的に適用し,再現性を重視した評価を実施した。実験の結果,ViT-Ti は TTA により最大で +1.8pt の精度向上を得た一方,ResNet-18 では RandAugment を中心としたデータ拡張で +2.1pt の改善が確認された。これらの知見を通じて,限られた計算資源でも堅牢なベースラインを構築するための指針を示す。
\end{abstract}

\begin{IEEEkeywords}
画像分類, 深層学習, Vision Transformer, Test-Time Augmentation, CIFAR-10
\end{IEEEkeywords}

\input{01_introduction/main}
\input{02_related_work/main}
\input{03_method/main}
\input{04_results/main}
\input{05_discussion/main}
\input{06_conclusion/main}
\input{appendix/main}

\bibliographystyle{ieeetr}
\bibliography{references}

\end{document}

% !TEX program = xelatex
\documentclass[conference]{IEEEtran}

% 日本語出力と XeLaTeX 対応設定
\usepackage{xeCJK}
\usepackage{fontspec}
\setCJKmainfont{HaranoAjiMincho}

% 数式・図表・参考文献などの標準パッケージ
\usepackage{amsmath, amssymb}
\usepackage{graphicx}
\usepackage{url}
\usepackage{hyperref}

\hypersetup{
  colorlinks=true,
  linkcolor=blue,
  citecolor=blue,
  urlcolor=blue
}

\begin{document}

\title{軽量 CNN と Vision Transformer の公平比較に向けた実験的評価}

\author{\IEEEauthorblockN{著者 太郎\\ }
\IEEEauthorblockA{所属機関\\ 連絡先: author@example.com}}

\maketitle

\begin{abstract}
本稿では,軽量な畳み込みニューラルネットワーク(CNN)と Vision Transformer(ViT)の画像分類性能を,CIFAR-10 を対象に公平に比較する実験プロトコルを提示する。学習率スケジューリング,データ拡張,Test-Time Augmentation (TTA) を統一的に適用し,再現性を重視した評価を実施した。実験の結果,ViT-Ti は TTA により最大で +1.8pt の精度向上を得た一方,ResNet-18 では RandAugment を中心としたデータ拡張で +2.1pt の改善が確認された。これらの知見を通じて,限られた計算資源でも堅牢なベースラインを構築するための指針を示す。
\end{abstract}

\begin{IEEEkeywords}
画像分類, 深層学習, Vision Transformer, Test-Time Augmentation, CIFAR-10
\end{IEEEkeywords}

\input{01_introduction/main}
\input{02_related_work/main}
\input{03_method/main}
\input{04_results/main}
\input{05_discussion/main}
\input{06_conclusion/main}
\input{appendix/main}

\bibliographystyle{ieeetr}
\bibliography{references}

\end{document}

% !TEX program = xelatex
\documentclass[conference]{IEEEtran}

% 日本語出力と XeLaTeX 対応設定
\usepackage{xeCJK}
\usepackage{fontspec}
\setCJKmainfont{HaranoAjiMincho}

% 数式・図表・参考文献などの標準パッケージ
\usepackage{amsmath, amssymb}
\usepackage{graphicx}
\usepackage{url}
\usepackage{hyperref}

\hypersetup{
  colorlinks=true,
  linkcolor=blue,
  citecolor=blue,
  urlcolor=blue
}

\begin{document}

\title{軽量 CNN と Vision Transformer の公平比較に向けた実験的評価}

\author{\IEEEauthorblockN{著者 太郎\\ }
\IEEEauthorblockA{所属機関\\ 連絡先: author@example.com}}

\maketitle

\begin{abstract}
本稿では,軽量な畳み込みニューラルネットワーク(CNN)と Vision Transformer(ViT)の画像分類性能を,CIFAR-10 を対象に公平に比較する実験プロトコルを提示する。学習率スケジューリング,データ拡張,Test-Time Augmentation (TTA) を統一的に適用し,再現性を重視した評価を実施した。実験の結果,ViT-Ti は TTA により最大で +1.8pt の精度向上を得た一方,ResNet-18 では RandAugment を中心としたデータ拡張で +2.1pt の改善が確認された。これらの知見を通じて,限られた計算資源でも堅牢なベースラインを構築するための指針を示す。
\end{abstract}

\begin{IEEEkeywords}
画像分類, 深層学習, Vision Transformer, Test-Time Augmentation, CIFAR-10
\end{IEEEkeywords}

\input{01_introduction/main}
\input{02_related_work/main}
\input{03_method/main}
\input{04_results/main}
\input{05_discussion/main}
\input{06_conclusion/main}
\input{appendix/main}

\bibliographystyle{ieeetr}
\bibliography{references}

\end{document}

% !TEX program = xelatex
\documentclass[conference]{IEEEtran}

% 日本語出力と XeLaTeX 対応設定
\usepackage{xeCJK}
\usepackage{fontspec}
\setCJKmainfont{HaranoAjiMincho}

% 数式・図表・参考文献などの標準パッケージ
\usepackage{amsmath, amssymb}
\usepackage{graphicx}
\usepackage{url}
\usepackage{hyperref}

\hypersetup{
  colorlinks=true,
  linkcolor=blue,
  citecolor=blue,
  urlcolor=blue
}

\begin{document}

\title{軽量 CNN と Vision Transformer の公平比較に向けた実験的評価}

\author{\IEEEauthorblockN{著者 太郎\\ }
\IEEEauthorblockA{所属機関\\ 連絡先: author@example.com}}

\maketitle

\begin{abstract}
本稿では,軽量な畳み込みニューラルネットワーク(CNN)と Vision Transformer(ViT)の画像分類性能を,CIFAR-10 を対象に公平に比較する実験プロトコルを提示する。学習率スケジューリング,データ拡張,Test-Time Augmentation (TTA) を統一的に適用し,再現性を重視した評価を実施した。実験の結果,ViT-Ti は TTA により最大で +1.8pt の精度向上を得た一方,ResNet-18 では RandAugment を中心としたデータ拡張で +2.1pt の改善が確認された。これらの知見を通じて,限られた計算資源でも堅牢なベースラインを構築するための指針を示す。
\end{abstract}

\begin{IEEEkeywords}
画像分類, 深層学習, Vision Transformer, Test-Time Augmentation, CIFAR-10
\end{IEEEkeywords}

\input{01_introduction/main}
\input{02_related_work/main}
\input{03_method/main}
\input{04_results/main}
\input{05_discussion/main}
\input{06_conclusion/main}
\input{appendix/main}

\bibliographystyle{ieeetr}
\bibliography{references}

\end{document}

% !TEX program = xelatex
\documentclass[conference]{IEEEtran}

% 日本語出力と XeLaTeX 対応設定
\usepackage{xeCJK}
\usepackage{fontspec}
\setCJKmainfont{HaranoAjiMincho}

% 数式・図表・参考文献などの標準パッケージ
\usepackage{amsmath, amssymb}
\usepackage{graphicx}
\usepackage{url}
\usepackage{hyperref}

\hypersetup{
  colorlinks=true,
  linkcolor=blue,
  citecolor=blue,
  urlcolor=blue
}

\begin{document}

\title{軽量 CNN と Vision Transformer の公平比較に向けた実験的評価}

\author{\IEEEauthorblockN{著者 太郎\\ }
\IEEEauthorblockA{所属機関\\ 連絡先: author@example.com}}

\maketitle

\begin{abstract}
本稿では,軽量な畳み込みニューラルネットワーク(CNN)と Vision Transformer(ViT)の画像分類性能を,CIFAR-10 を対象に公平に比較する実験プロトコルを提示する。学習率スケジューリング,データ拡張,Test-Time Augmentation (TTA) を統一的に適用し,再現性を重視した評価を実施した。実験の結果,ViT-Ti は TTA により最大で +1.8pt の精度向上を得た一方,ResNet-18 では RandAugment を中心としたデータ拡張で +2.1pt の改善が確認された。これらの知見を通じて,限られた計算資源でも堅牢なベースラインを構築するための指針を示す。
\end{abstract}

\begin{IEEEkeywords}
画像分類, 深層学習, Vision Transformer, Test-Time Augmentation, CIFAR-10
\end{IEEEkeywords}

\input{01_introduction/main}
\input{02_related_work/main}
\input{03_method/main}
\input{04_results/main}
\input{05_discussion/main}
\input{06_conclusion/main}
\input{appendix/main}

\bibliographystyle{ieeetr}
\bibliography{references}

\end{document}

% !TEX program = xelatex
\documentclass[conference]{IEEEtran}

% 日本語出力と XeLaTeX 対応設定
\usepackage{xeCJK}
\usepackage{fontspec}
\setCJKmainfont{HaranoAjiMincho}

% 数式・図表・参考文献などの標準パッケージ
\usepackage{amsmath, amssymb}
\usepackage{graphicx}
\usepackage{url}
\usepackage{hyperref}

\hypersetup{
  colorlinks=true,
  linkcolor=blue,
  citecolor=blue,
  urlcolor=blue
}

\begin{document}

\title{軽量 CNN と Vision Transformer の公平比較に向けた実験的評価}

\author{\IEEEauthorblockN{著者 太郎\\ }
\IEEEauthorblockA{所属機関\\ 連絡先: author@example.com}}

\maketitle

\begin{abstract}
本稿では,軽量な畳み込みニューラルネットワーク(CNN)と Vision Transformer(ViT)の画像分類性能を,CIFAR-10 を対象に公平に比較する実験プロトコルを提示する。学習率スケジューリング,データ拡張,Test-Time Augmentation (TTA) を統一的に適用し,再現性を重視した評価を実施した。実験の結果,ViT-Ti は TTA により最大で +1.8pt の精度向上を得た一方,ResNet-18 では RandAugment を中心としたデータ拡張で +2.1pt の改善が確認された。これらの知見を通じて,限られた計算資源でも堅牢なベースラインを構築するための指針を示す。
\end{abstract}

\begin{IEEEkeywords}
画像分類, 深層学習, Vision Transformer, Test-Time Augmentation, CIFAR-10
\end{IEEEkeywords}

\input{01_introduction/main}
\input{02_related_work/main}
\input{03_method/main}
\input{04_results/main}
\input{05_discussion/main}
\input{06_conclusion/main}
\input{appendix/main}

\bibliographystyle{ieeetr}
\bibliography{references}

\end{document}


\bibliographystyle{ieeetr}
\bibliography{references}

\end{document}

% !TEX program = xelatex
\documentclass[conference]{IEEEtran}

% 日本語出力と XeLaTeX 対応設定
\usepackage{xeCJK}
\usepackage{fontspec}
\setCJKmainfont{HaranoAjiMincho}

% 数式・図表・参考文献などの標準パッケージ
\usepackage{amsmath, amssymb}
\usepackage{graphicx}
\usepackage{url}
\usepackage{hyperref}

\hypersetup{
  colorlinks=true,
  linkcolor=blue,
  citecolor=blue,
  urlcolor=blue
}

\begin{document}

\title{軽量 CNN と Vision Transformer の公平比較に向けた実験的評価}

\author{\IEEEauthorblockN{著者 太郎\\ }
\IEEEauthorblockA{所属機関\\ 連絡先: author@example.com}}

\maketitle

\begin{abstract}
本稿では,軽量な畳み込みニューラルネットワーク(CNN)と Vision Transformer(ViT)の画像分類性能を,CIFAR-10 を対象に公平に比較する実験プロトコルを提示する。学習率スケジューリング,データ拡張,Test-Time Augmentation (TTA) を統一的に適用し,再現性を重視した評価を実施した。実験の結果,ViT-Ti は TTA により最大で +1.8pt の精度向上を得た一方,ResNet-18 では RandAugment を中心としたデータ拡張で +2.1pt の改善が確認された。これらの知見を通じて,限られた計算資源でも堅牢なベースラインを構築するための指針を示す。
\end{abstract}

\begin{IEEEkeywords}
画像分類, 深層学習, Vision Transformer, Test-Time Augmentation, CIFAR-10
\end{IEEEkeywords}

% !TEX program = xelatex
\documentclass[conference]{IEEEtran}

% 日本語出力と XeLaTeX 対応設定
\usepackage{xeCJK}
\usepackage{fontspec}
\setCJKmainfont{HaranoAjiMincho}

% 数式・図表・参考文献などの標準パッケージ
\usepackage{amsmath, amssymb}
\usepackage{graphicx}
\usepackage{url}
\usepackage{hyperref}

\hypersetup{
  colorlinks=true,
  linkcolor=blue,
  citecolor=blue,
  urlcolor=blue
}

\begin{document}

\title{軽量 CNN と Vision Transformer の公平比較に向けた実験的評価}

\author{\IEEEauthorblockN{著者 太郎\\ }
\IEEEauthorblockA{所属機関\\ 連絡先: author@example.com}}

\maketitle

\begin{abstract}
本稿では,軽量な畳み込みニューラルネットワーク(CNN)と Vision Transformer(ViT)の画像分類性能を,CIFAR-10 を対象に公平に比較する実験プロトコルを提示する。学習率スケジューリング,データ拡張,Test-Time Augmentation (TTA) を統一的に適用し,再現性を重視した評価を実施した。実験の結果,ViT-Ti は TTA により最大で +1.8pt の精度向上を得た一方,ResNet-18 では RandAugment を中心としたデータ拡張で +2.1pt の改善が確認された。これらの知見を通じて,限られた計算資源でも堅牢なベースラインを構築するための指針を示す。
\end{abstract}

\begin{IEEEkeywords}
画像分類, 深層学習, Vision Transformer, Test-Time Augmentation, CIFAR-10
\end{IEEEkeywords}

\input{01_introduction/main}
\input{02_related_work/main}
\input{03_method/main}
\input{04_results/main}
\input{05_discussion/main}
\input{06_conclusion/main}
\input{appendix/main}

\bibliographystyle{ieeetr}
\bibliography{references}

\end{document}

% !TEX program = xelatex
\documentclass[conference]{IEEEtran}

% 日本語出力と XeLaTeX 対応設定
\usepackage{xeCJK}
\usepackage{fontspec}
\setCJKmainfont{HaranoAjiMincho}

% 数式・図表・参考文献などの標準パッケージ
\usepackage{amsmath, amssymb}
\usepackage{graphicx}
\usepackage{url}
\usepackage{hyperref}

\hypersetup{
  colorlinks=true,
  linkcolor=blue,
  citecolor=blue,
  urlcolor=blue
}

\begin{document}

\title{軽量 CNN と Vision Transformer の公平比較に向けた実験的評価}

\author{\IEEEauthorblockN{著者 太郎\\ }
\IEEEauthorblockA{所属機関\\ 連絡先: author@example.com}}

\maketitle

\begin{abstract}
本稿では,軽量な畳み込みニューラルネットワーク(CNN)と Vision Transformer(ViT)の画像分類性能を,CIFAR-10 を対象に公平に比較する実験プロトコルを提示する。学習率スケジューリング,データ拡張,Test-Time Augmentation (TTA) を統一的に適用し,再現性を重視した評価を実施した。実験の結果,ViT-Ti は TTA により最大で +1.8pt の精度向上を得た一方,ResNet-18 では RandAugment を中心としたデータ拡張で +2.1pt の改善が確認された。これらの知見を通じて,限られた計算資源でも堅牢なベースラインを構築するための指針を示す。
\end{abstract}

\begin{IEEEkeywords}
画像分類, 深層学習, Vision Transformer, Test-Time Augmentation, CIFAR-10
\end{IEEEkeywords}

\input{01_introduction/main}
\input{02_related_work/main}
\input{03_method/main}
\input{04_results/main}
\input{05_discussion/main}
\input{06_conclusion/main}
\input{appendix/main}

\bibliographystyle{ieeetr}
\bibliography{references}

\end{document}

% !TEX program = xelatex
\documentclass[conference]{IEEEtran}

% 日本語出力と XeLaTeX 対応設定
\usepackage{xeCJK}
\usepackage{fontspec}
\setCJKmainfont{HaranoAjiMincho}

% 数式・図表・参考文献などの標準パッケージ
\usepackage{amsmath, amssymb}
\usepackage{graphicx}
\usepackage{url}
\usepackage{hyperref}

\hypersetup{
  colorlinks=true,
  linkcolor=blue,
  citecolor=blue,
  urlcolor=blue
}

\begin{document}

\title{軽量 CNN と Vision Transformer の公平比較に向けた実験的評価}

\author{\IEEEauthorblockN{著者 太郎\\ }
\IEEEauthorblockA{所属機関\\ 連絡先: author@example.com}}

\maketitle

\begin{abstract}
本稿では,軽量な畳み込みニューラルネットワーク(CNN)と Vision Transformer(ViT)の画像分類性能を,CIFAR-10 を対象に公平に比較する実験プロトコルを提示する。学習率スケジューリング,データ拡張,Test-Time Augmentation (TTA) を統一的に適用し,再現性を重視した評価を実施した。実験の結果,ViT-Ti は TTA により最大で +1.8pt の精度向上を得た一方,ResNet-18 では RandAugment を中心としたデータ拡張で +2.1pt の改善が確認された。これらの知見を通じて,限られた計算資源でも堅牢なベースラインを構築するための指針を示す。
\end{abstract}

\begin{IEEEkeywords}
画像分類, 深層学習, Vision Transformer, Test-Time Augmentation, CIFAR-10
\end{IEEEkeywords}

\input{01_introduction/main}
\input{02_related_work/main}
\input{03_method/main}
\input{04_results/main}
\input{05_discussion/main}
\input{06_conclusion/main}
\input{appendix/main}

\bibliographystyle{ieeetr}
\bibliography{references}

\end{document}

% !TEX program = xelatex
\documentclass[conference]{IEEEtran}

% 日本語出力と XeLaTeX 対応設定
\usepackage{xeCJK}
\usepackage{fontspec}
\setCJKmainfont{HaranoAjiMincho}

% 数式・図表・参考文献などの標準パッケージ
\usepackage{amsmath, amssymb}
\usepackage{graphicx}
\usepackage{url}
\usepackage{hyperref}

\hypersetup{
  colorlinks=true,
  linkcolor=blue,
  citecolor=blue,
  urlcolor=blue
}

\begin{document}

\title{軽量 CNN と Vision Transformer の公平比較に向けた実験的評価}

\author{\IEEEauthorblockN{著者 太郎\\ }
\IEEEauthorblockA{所属機関\\ 連絡先: author@example.com}}

\maketitle

\begin{abstract}
本稿では,軽量な畳み込みニューラルネットワーク(CNN)と Vision Transformer(ViT)の画像分類性能を,CIFAR-10 を対象に公平に比較する実験プロトコルを提示する。学習率スケジューリング,データ拡張,Test-Time Augmentation (TTA) を統一的に適用し,再現性を重視した評価を実施した。実験の結果,ViT-Ti は TTA により最大で +1.8pt の精度向上を得た一方,ResNet-18 では RandAugment を中心としたデータ拡張で +2.1pt の改善が確認された。これらの知見を通じて,限られた計算資源でも堅牢なベースラインを構築するための指針を示す。
\end{abstract}

\begin{IEEEkeywords}
画像分類, 深層学習, Vision Transformer, Test-Time Augmentation, CIFAR-10
\end{IEEEkeywords}

\input{01_introduction/main}
\input{02_related_work/main}
\input{03_method/main}
\input{04_results/main}
\input{05_discussion/main}
\input{06_conclusion/main}
\input{appendix/main}

\bibliographystyle{ieeetr}
\bibliography{references}

\end{document}

% !TEX program = xelatex
\documentclass[conference]{IEEEtran}

% 日本語出力と XeLaTeX 対応設定
\usepackage{xeCJK}
\usepackage{fontspec}
\setCJKmainfont{HaranoAjiMincho}

% 数式・図表・参考文献などの標準パッケージ
\usepackage{amsmath, amssymb}
\usepackage{graphicx}
\usepackage{url}
\usepackage{hyperref}

\hypersetup{
  colorlinks=true,
  linkcolor=blue,
  citecolor=blue,
  urlcolor=blue
}

\begin{document}

\title{軽量 CNN と Vision Transformer の公平比較に向けた実験的評価}

\author{\IEEEauthorblockN{著者 太郎\\ }
\IEEEauthorblockA{所属機関\\ 連絡先: author@example.com}}

\maketitle

\begin{abstract}
本稿では,軽量な畳み込みニューラルネットワーク(CNN)と Vision Transformer(ViT)の画像分類性能を,CIFAR-10 を対象に公平に比較する実験プロトコルを提示する。学習率スケジューリング,データ拡張,Test-Time Augmentation (TTA) を統一的に適用し,再現性を重視した評価を実施した。実験の結果,ViT-Ti は TTA により最大で +1.8pt の精度向上を得た一方,ResNet-18 では RandAugment を中心としたデータ拡張で +2.1pt の改善が確認された。これらの知見を通じて,限られた計算資源でも堅牢なベースラインを構築するための指針を示す。
\end{abstract}

\begin{IEEEkeywords}
画像分類, 深層学習, Vision Transformer, Test-Time Augmentation, CIFAR-10
\end{IEEEkeywords}

\input{01_introduction/main}
\input{02_related_work/main}
\input{03_method/main}
\input{04_results/main}
\input{05_discussion/main}
\input{06_conclusion/main}
\input{appendix/main}

\bibliographystyle{ieeetr}
\bibliography{references}

\end{document}

% !TEX program = xelatex
\documentclass[conference]{IEEEtran}

% 日本語出力と XeLaTeX 対応設定
\usepackage{xeCJK}
\usepackage{fontspec}
\setCJKmainfont{HaranoAjiMincho}

% 数式・図表・参考文献などの標準パッケージ
\usepackage{amsmath, amssymb}
\usepackage{graphicx}
\usepackage{url}
\usepackage{hyperref}

\hypersetup{
  colorlinks=true,
  linkcolor=blue,
  citecolor=blue,
  urlcolor=blue
}

\begin{document}

\title{軽量 CNN と Vision Transformer の公平比較に向けた実験的評価}

\author{\IEEEauthorblockN{著者 太郎\\ }
\IEEEauthorblockA{所属機関\\ 連絡先: author@example.com}}

\maketitle

\begin{abstract}
本稿では,軽量な畳み込みニューラルネットワーク(CNN)と Vision Transformer(ViT)の画像分類性能を,CIFAR-10 を対象に公平に比較する実験プロトコルを提示する。学習率スケジューリング,データ拡張,Test-Time Augmentation (TTA) を統一的に適用し,再現性を重視した評価を実施した。実験の結果,ViT-Ti は TTA により最大で +1.8pt の精度向上を得た一方,ResNet-18 では RandAugment を中心としたデータ拡張で +2.1pt の改善が確認された。これらの知見を通じて,限られた計算資源でも堅牢なベースラインを構築するための指針を示す。
\end{abstract}

\begin{IEEEkeywords}
画像分類, 深層学習, Vision Transformer, Test-Time Augmentation, CIFAR-10
\end{IEEEkeywords}

\input{01_introduction/main}
\input{02_related_work/main}
\input{03_method/main}
\input{04_results/main}
\input{05_discussion/main}
\input{06_conclusion/main}
\input{appendix/main}

\bibliographystyle{ieeetr}
\bibliography{references}

\end{document}

% !TEX program = xelatex
\documentclass[conference]{IEEEtran}

% 日本語出力と XeLaTeX 対応設定
\usepackage{xeCJK}
\usepackage{fontspec}
\setCJKmainfont{HaranoAjiMincho}

% 数式・図表・参考文献などの標準パッケージ
\usepackage{amsmath, amssymb}
\usepackage{graphicx}
\usepackage{url}
\usepackage{hyperref}

\hypersetup{
  colorlinks=true,
  linkcolor=blue,
  citecolor=blue,
  urlcolor=blue
}

\begin{document}

\title{軽量 CNN と Vision Transformer の公平比較に向けた実験的評価}

\author{\IEEEauthorblockN{著者 太郎\\ }
\IEEEauthorblockA{所属機関\\ 連絡先: author@example.com}}

\maketitle

\begin{abstract}
本稿では,軽量な畳み込みニューラルネットワーク(CNN)と Vision Transformer(ViT)の画像分類性能を,CIFAR-10 を対象に公平に比較する実験プロトコルを提示する。学習率スケジューリング,データ拡張,Test-Time Augmentation (TTA) を統一的に適用し,再現性を重視した評価を実施した。実験の結果,ViT-Ti は TTA により最大で +1.8pt の精度向上を得た一方,ResNet-18 では RandAugment を中心としたデータ拡張で +2.1pt の改善が確認された。これらの知見を通じて,限られた計算資源でも堅牢なベースラインを構築するための指針を示す。
\end{abstract}

\begin{IEEEkeywords}
画像分類, 深層学習, Vision Transformer, Test-Time Augmentation, CIFAR-10
\end{IEEEkeywords}

\input{01_introduction/main}
\input{02_related_work/main}
\input{03_method/main}
\input{04_results/main}
\input{05_discussion/main}
\input{06_conclusion/main}
\input{appendix/main}

\bibliographystyle{ieeetr}
\bibliography{references}

\end{document}


\bibliographystyle{ieeetr}
\bibliography{references}

\end{document}

% !TEX program = xelatex
\documentclass[conference]{IEEEtran}

% 日本語出力と XeLaTeX 対応設定
\usepackage{xeCJK}
\usepackage{fontspec}
\setCJKmainfont{HaranoAjiMincho}

% 数式・図表・参考文献などの標準パッケージ
\usepackage{amsmath, amssymb}
\usepackage{graphicx}
\usepackage{url}
\usepackage{hyperref}

\hypersetup{
  colorlinks=true,
  linkcolor=blue,
  citecolor=blue,
  urlcolor=blue
}

\begin{document}

\title{軽量 CNN と Vision Transformer の公平比較に向けた実験的評価}

\author{\IEEEauthorblockN{著者 太郎\\ }
\IEEEauthorblockA{所属機関\\ 連絡先: author@example.com}}

\maketitle

\begin{abstract}
本稿では,軽量な畳み込みニューラルネットワーク(CNN)と Vision Transformer(ViT)の画像分類性能を,CIFAR-10 を対象に公平に比較する実験プロトコルを提示する。学習率スケジューリング,データ拡張,Test-Time Augmentation (TTA) を統一的に適用し,再現性を重視した評価を実施した。実験の結果,ViT-Ti は TTA により最大で +1.8pt の精度向上を得た一方,ResNet-18 では RandAugment を中心としたデータ拡張で +2.1pt の改善が確認された。これらの知見を通じて,限られた計算資源でも堅牢なベースラインを構築するための指針を示す。
\end{abstract}

\begin{IEEEkeywords}
画像分類, 深層学習, Vision Transformer, Test-Time Augmentation, CIFAR-10
\end{IEEEkeywords}

% !TEX program = xelatex
\documentclass[conference]{IEEEtran}

% 日本語出力と XeLaTeX 対応設定
\usepackage{xeCJK}
\usepackage{fontspec}
\setCJKmainfont{HaranoAjiMincho}

% 数式・図表・参考文献などの標準パッケージ
\usepackage{amsmath, amssymb}
\usepackage{graphicx}
\usepackage{url}
\usepackage{hyperref}

\hypersetup{
  colorlinks=true,
  linkcolor=blue,
  citecolor=blue,
  urlcolor=blue
}

\begin{document}

\title{軽量 CNN と Vision Transformer の公平比較に向けた実験的評価}

\author{\IEEEauthorblockN{著者 太郎\\ }
\IEEEauthorblockA{所属機関\\ 連絡先: author@example.com}}

\maketitle

\begin{abstract}
本稿では,軽量な畳み込みニューラルネットワーク(CNN)と Vision Transformer(ViT)の画像分類性能を,CIFAR-10 を対象に公平に比較する実験プロトコルを提示する。学習率スケジューリング,データ拡張,Test-Time Augmentation (TTA) を統一的に適用し,再現性を重視した評価を実施した。実験の結果,ViT-Ti は TTA により最大で +1.8pt の精度向上を得た一方,ResNet-18 では RandAugment を中心としたデータ拡張で +2.1pt の改善が確認された。これらの知見を通じて,限られた計算資源でも堅牢なベースラインを構築するための指針を示す。
\end{abstract}

\begin{IEEEkeywords}
画像分類, 深層学習, Vision Transformer, Test-Time Augmentation, CIFAR-10
\end{IEEEkeywords}

\input{01_introduction/main}
\input{02_related_work/main}
\input{03_method/main}
\input{04_results/main}
\input{05_discussion/main}
\input{06_conclusion/main}
\input{appendix/main}

\bibliographystyle{ieeetr}
\bibliography{references}

\end{document}

% !TEX program = xelatex
\documentclass[conference]{IEEEtran}

% 日本語出力と XeLaTeX 対応設定
\usepackage{xeCJK}
\usepackage{fontspec}
\setCJKmainfont{HaranoAjiMincho}

% 数式・図表・参考文献などの標準パッケージ
\usepackage{amsmath, amssymb}
\usepackage{graphicx}
\usepackage{url}
\usepackage{hyperref}

\hypersetup{
  colorlinks=true,
  linkcolor=blue,
  citecolor=blue,
  urlcolor=blue
}

\begin{document}

\title{軽量 CNN と Vision Transformer の公平比較に向けた実験的評価}

\author{\IEEEauthorblockN{著者 太郎\\ }
\IEEEauthorblockA{所属機関\\ 連絡先: author@example.com}}

\maketitle

\begin{abstract}
本稿では,軽量な畳み込みニューラルネットワーク(CNN)と Vision Transformer(ViT)の画像分類性能を,CIFAR-10 を対象に公平に比較する実験プロトコルを提示する。学習率スケジューリング,データ拡張,Test-Time Augmentation (TTA) を統一的に適用し,再現性を重視した評価を実施した。実験の結果,ViT-Ti は TTA により最大で +1.8pt の精度向上を得た一方,ResNet-18 では RandAugment を中心としたデータ拡張で +2.1pt の改善が確認された。これらの知見を通じて,限られた計算資源でも堅牢なベースラインを構築するための指針を示す。
\end{abstract}

\begin{IEEEkeywords}
画像分類, 深層学習, Vision Transformer, Test-Time Augmentation, CIFAR-10
\end{IEEEkeywords}

\input{01_introduction/main}
\input{02_related_work/main}
\input{03_method/main}
\input{04_results/main}
\input{05_discussion/main}
\input{06_conclusion/main}
\input{appendix/main}

\bibliographystyle{ieeetr}
\bibliography{references}

\end{document}

% !TEX program = xelatex
\documentclass[conference]{IEEEtran}

% 日本語出力と XeLaTeX 対応設定
\usepackage{xeCJK}
\usepackage{fontspec}
\setCJKmainfont{HaranoAjiMincho}

% 数式・図表・参考文献などの標準パッケージ
\usepackage{amsmath, amssymb}
\usepackage{graphicx}
\usepackage{url}
\usepackage{hyperref}

\hypersetup{
  colorlinks=true,
  linkcolor=blue,
  citecolor=blue,
  urlcolor=blue
}

\begin{document}

\title{軽量 CNN と Vision Transformer の公平比較に向けた実験的評価}

\author{\IEEEauthorblockN{著者 太郎\\ }
\IEEEauthorblockA{所属機関\\ 連絡先: author@example.com}}

\maketitle

\begin{abstract}
本稿では,軽量な畳み込みニューラルネットワーク(CNN)と Vision Transformer(ViT)の画像分類性能を,CIFAR-10 を対象に公平に比較する実験プロトコルを提示する。学習率スケジューリング,データ拡張,Test-Time Augmentation (TTA) を統一的に適用し,再現性を重視した評価を実施した。実験の結果,ViT-Ti は TTA により最大で +1.8pt の精度向上を得た一方,ResNet-18 では RandAugment を中心としたデータ拡張で +2.1pt の改善が確認された。これらの知見を通じて,限られた計算資源でも堅牢なベースラインを構築するための指針を示す。
\end{abstract}

\begin{IEEEkeywords}
画像分類, 深層学習, Vision Transformer, Test-Time Augmentation, CIFAR-10
\end{IEEEkeywords}

\input{01_introduction/main}
\input{02_related_work/main}
\input{03_method/main}
\input{04_results/main}
\input{05_discussion/main}
\input{06_conclusion/main}
\input{appendix/main}

\bibliographystyle{ieeetr}
\bibliography{references}

\end{document}

% !TEX program = xelatex
\documentclass[conference]{IEEEtran}

% 日本語出力と XeLaTeX 対応設定
\usepackage{xeCJK}
\usepackage{fontspec}
\setCJKmainfont{HaranoAjiMincho}

% 数式・図表・参考文献などの標準パッケージ
\usepackage{amsmath, amssymb}
\usepackage{graphicx}
\usepackage{url}
\usepackage{hyperref}

\hypersetup{
  colorlinks=true,
  linkcolor=blue,
  citecolor=blue,
  urlcolor=blue
}

\begin{document}

\title{軽量 CNN と Vision Transformer の公平比較に向けた実験的評価}

\author{\IEEEauthorblockN{著者 太郎\\ }
\IEEEauthorblockA{所属機関\\ 連絡先: author@example.com}}

\maketitle

\begin{abstract}
本稿では,軽量な畳み込みニューラルネットワーク(CNN)と Vision Transformer(ViT)の画像分類性能を,CIFAR-10 を対象に公平に比較する実験プロトコルを提示する。学習率スケジューリング,データ拡張,Test-Time Augmentation (TTA) を統一的に適用し,再現性を重視した評価を実施した。実験の結果,ViT-Ti は TTA により最大で +1.8pt の精度向上を得た一方,ResNet-18 では RandAugment を中心としたデータ拡張で +2.1pt の改善が確認された。これらの知見を通じて,限られた計算資源でも堅牢なベースラインを構築するための指針を示す。
\end{abstract}

\begin{IEEEkeywords}
画像分類, 深層学習, Vision Transformer, Test-Time Augmentation, CIFAR-10
\end{IEEEkeywords}

\input{01_introduction/main}
\input{02_related_work/main}
\input{03_method/main}
\input{04_results/main}
\input{05_discussion/main}
\input{06_conclusion/main}
\input{appendix/main}

\bibliographystyle{ieeetr}
\bibliography{references}

\end{document}

% !TEX program = xelatex
\documentclass[conference]{IEEEtran}

% 日本語出力と XeLaTeX 対応設定
\usepackage{xeCJK}
\usepackage{fontspec}
\setCJKmainfont{HaranoAjiMincho}

% 数式・図表・参考文献などの標準パッケージ
\usepackage{amsmath, amssymb}
\usepackage{graphicx}
\usepackage{url}
\usepackage{hyperref}

\hypersetup{
  colorlinks=true,
  linkcolor=blue,
  citecolor=blue,
  urlcolor=blue
}

\begin{document}

\title{軽量 CNN と Vision Transformer の公平比較に向けた実験的評価}

\author{\IEEEauthorblockN{著者 太郎\\ }
\IEEEauthorblockA{所属機関\\ 連絡先: author@example.com}}

\maketitle

\begin{abstract}
本稿では,軽量な畳み込みニューラルネットワーク(CNN)と Vision Transformer(ViT)の画像分類性能を,CIFAR-10 を対象に公平に比較する実験プロトコルを提示する。学習率スケジューリング,データ拡張,Test-Time Augmentation (TTA) を統一的に適用し,再現性を重視した評価を実施した。実験の結果,ViT-Ti は TTA により最大で +1.8pt の精度向上を得た一方,ResNet-18 では RandAugment を中心としたデータ拡張で +2.1pt の改善が確認された。これらの知見を通じて,限られた計算資源でも堅牢なベースラインを構築するための指針を示す。
\end{abstract}

\begin{IEEEkeywords}
画像分類, 深層学習, Vision Transformer, Test-Time Augmentation, CIFAR-10
\end{IEEEkeywords}

\input{01_introduction/main}
\input{02_related_work/main}
\input{03_method/main}
\input{04_results/main}
\input{05_discussion/main}
\input{06_conclusion/main}
\input{appendix/main}

\bibliographystyle{ieeetr}
\bibliography{references}

\end{document}

% !TEX program = xelatex
\documentclass[conference]{IEEEtran}

% 日本語出力と XeLaTeX 対応設定
\usepackage{xeCJK}
\usepackage{fontspec}
\setCJKmainfont{HaranoAjiMincho}

% 数式・図表・参考文献などの標準パッケージ
\usepackage{amsmath, amssymb}
\usepackage{graphicx}
\usepackage{url}
\usepackage{hyperref}

\hypersetup{
  colorlinks=true,
  linkcolor=blue,
  citecolor=blue,
  urlcolor=blue
}

\begin{document}

\title{軽量 CNN と Vision Transformer の公平比較に向けた実験的評価}

\author{\IEEEauthorblockN{著者 太郎\\ }
\IEEEauthorblockA{所属機関\\ 連絡先: author@example.com}}

\maketitle

\begin{abstract}
本稿では,軽量な畳み込みニューラルネットワーク(CNN)と Vision Transformer(ViT)の画像分類性能を,CIFAR-10 を対象に公平に比較する実験プロトコルを提示する。学習率スケジューリング,データ拡張,Test-Time Augmentation (TTA) を統一的に適用し,再現性を重視した評価を実施した。実験の結果,ViT-Ti は TTA により最大で +1.8pt の精度向上を得た一方,ResNet-18 では RandAugment を中心としたデータ拡張で +2.1pt の改善が確認された。これらの知見を通じて,限られた計算資源でも堅牢なベースラインを構築するための指針を示す。
\end{abstract}

\begin{IEEEkeywords}
画像分類, 深層学習, Vision Transformer, Test-Time Augmentation, CIFAR-10
\end{IEEEkeywords}

\input{01_introduction/main}
\input{02_related_work/main}
\input{03_method/main}
\input{04_results/main}
\input{05_discussion/main}
\input{06_conclusion/main}
\input{appendix/main}

\bibliographystyle{ieeetr}
\bibliography{references}

\end{document}

% !TEX program = xelatex
\documentclass[conference]{IEEEtran}

% 日本語出力と XeLaTeX 対応設定
\usepackage{xeCJK}
\usepackage{fontspec}
\setCJKmainfont{HaranoAjiMincho}

% 数式・図表・参考文献などの標準パッケージ
\usepackage{amsmath, amssymb}
\usepackage{graphicx}
\usepackage{url}
\usepackage{hyperref}

\hypersetup{
  colorlinks=true,
  linkcolor=blue,
  citecolor=blue,
  urlcolor=blue
}

\begin{document}

\title{軽量 CNN と Vision Transformer の公平比較に向けた実験的評価}

\author{\IEEEauthorblockN{著者 太郎\\ }
\IEEEauthorblockA{所属機関\\ 連絡先: author@example.com}}

\maketitle

\begin{abstract}
本稿では,軽量な畳み込みニューラルネットワーク(CNN)と Vision Transformer(ViT)の画像分類性能を,CIFAR-10 を対象に公平に比較する実験プロトコルを提示する。学習率スケジューリング,データ拡張,Test-Time Augmentation (TTA) を統一的に適用し,再現性を重視した評価を実施した。実験の結果,ViT-Ti は TTA により最大で +1.8pt の精度向上を得た一方,ResNet-18 では RandAugment を中心としたデータ拡張で +2.1pt の改善が確認された。これらの知見を通じて,限られた計算資源でも堅牢なベースラインを構築するための指針を示す。
\end{abstract}

\begin{IEEEkeywords}
画像分類, 深層学習, Vision Transformer, Test-Time Augmentation, CIFAR-10
\end{IEEEkeywords}

\input{01_introduction/main}
\input{02_related_work/main}
\input{03_method/main}
\input{04_results/main}
\input{05_discussion/main}
\input{06_conclusion/main}
\input{appendix/main}

\bibliographystyle{ieeetr}
\bibliography{references}

\end{document}


\bibliographystyle{ieeetr}
\bibliography{references}

\end{document}


\bibliographystyle{ieeetr}
\bibliography{references}

\end{document}


\bibliographystyle{ieeetr}
\bibliography{references}

\end{document}
