\section{はじめに}
画像分類タスクは,監視システムや医用画像解析,製造ライン検査など多様な応用領域で中核技術として利用されている。AlexNet によるブレイクスルー以降,深層学習モデルは大規模データセットでの性能向上を牽引し続けており,その性能差を適切に評価する手法が求められている。\cite{krizhevsky2012imagenet}

しかし,研究コミュニティにおける再現性の欠如や評価条件のばらつきは依然として課題である。特に,軽量な畳み込みニューラルネットワーク(CNN)と Vision Transformer (ViT) 系モデルの比較では,ハイパーパラメータや学習スケジュールが統一されないまま結果が報告されるケースが多く,公平な比較を阻害している。\cite{henderson2018deep}

本研究では,ResNet-18 と DeiT-Ti を対象に,軽量構成かつ短時間学習の条件下で性能を公平に評価するプロトコルを構築する。統一したデータ前処理,学習率スケジューリング,Test-Time Augmentation (TTA) を適用し,少数エポックでも再現性の高いベースラインを提供することが本稿の目的である。論文の構成は以下の通りである。第\ref{sec:related}章で関連研究を概観し,第\ref{sec:method}章で実験設定とプロトコルを示す。第\ref{sec:results}章では実験結果を報告し,第\ref{sec:discussion}章で考察を述べた後,第\ref{sec:conclusion}章で結論と今後の課題をまとめる。
